%
%
% UCSD Doctoral Dissertation Template
% -----------------------------------
% http://ucsd-thesis.googlecode.com
%
%


%% REQUIRED FIELDS -- Replace with the values appropriate to you

% No symbols, formulas, superscripts, or Greek letters are allowed
% in your title.
\title{Mechanizing Refinement Types}

\author{Michael H. Borkowski}
\degreeyear{\the\year}

% Master's Degree theses will NOT be formatted properly with this file.
\degreetitle{Doctor of Philosophy}

\field{Computer Science}
%\specialization{Programming Languages}  % If you have a specialization, add it here

\chair{Professor Ranjit Jhala}
% Uncomment the next line iff you have a Co-Chair
% \cochair{Professor Cochair Semimaster}
%
% Or, uncomment the next line iff you have two equal Co-Chairs.
%\cochairs{Professor Chair Masterish}{Professor Chair Masterish}

%  The rest of the committee members  must be alphabetized by last name.
\othermembers{
Professor Samuel R. Buss\\
Professor Cormac Flanagan\\
Professor Nadia Polikarpova\\
Professor Victor Vianu\\
}
\numberofmembers{5} % |chair| + |cochair| + |othermembers|


%% START THE FRONTMATTER
%
\begin{frontmatter}

%% TITLE PAGES
%
%  This command generates the title, copyright, and signature pages.
%
\makefrontmatter

%% DEDICATION
%
%  You have three choices here:
%    1. Use the ``dedication'' environment.
%       Put in the text you want, and everything will be formated for
%       you. You'll get a perfectly respectable dedication page.
%
%
%    2. Use the ``mydedication'' environment.  If you don't like the
%       formatting of option 1, use this environment and format things
%       however you wish.
%
%    3. If you don't want a dedication, it's not required.
%
%
\begin{dedication}
  To Kiyoshi, Daikichi, and Ziggy
\end{dedication}


% \begin{mydedication} % You are responsible for formatting here.
%   \vspace{1in}
%   \begin{flushleft}
% 	To me.
%   \end{flushleft}
%
%   \vspace{2in}
%   \begin{center}
% 	And you.
%   \end{center}
%
%   \vspace{2in}
%   \begin{flushright}
% 	Which equals us.
%   \end{flushright}
% \end{mydedication}



%% EPIGRAPH
%
%  The same choices that applied to the dedication apply here.
%
%\begin{epigraph} % The style file will position the text for you.
%  \emph{A careful quotation\\
%  conveys brilliance.}\\
%  ---Smarty Pants
%\end{epigraph}

% \begin{myepigraph} % You position the text yourself.
%   \vfil
%   \begin{center}
%     {\bf Think! It ain't illegal yet.}
%
% 	\emph{---George Clinton}
%   \end{center}
% \end{myepigraph}


%% SETUP THE TABLE OF CONTENTS
%
\tableofcontents
\listoffigures  % Comment if you don't have any figures
\listoftables   % Comment if you don't have any tables



%% ACKNOWLEDGEMENTS
%
%  While technically optional, you probably have someone to thank.
%  Also, a paragraph acknowledging all coauthors and publishers (if
%  you have any) is required in the acknowledgements page and as the
%  last paragraph of text at the end of each respective chapter. See
%  the OGS Formatting Manual for more information.
%
\begin{acknowledgements}
    %
    First, I would like to thank Ranjit Jhala for all of his support as
    my advisor. 
    %
    I am greatly indebted to Ranjit for taking me on as his student when I 
    was a complete beginner at type theory and software verification
    research.
    %
    He provided the original motivation for my work in his 2019 
    graduate class on \lh, and I've been hooked on theorem proving ever since.
    % 
    I appreciate Ranjit's insights and feedback during our 
    meetings and, most of all, his continuing confidence in my work 
    throughout four conference rejections motivated me to keep improving 
    and adding to our work.
    %

    I would also like to thank my collaborator and coauthor Niki Vazou
    for all of her patient help, support, and ideas. I couldn't have 
    done this research without her support either!
    %
    I want to thank each of the members of my committee, Nadia Polikarpova,
    Victor Vianu, Sam Buss, and Cormac Flanagan for their support through 
    this process and for the opportunity to TA
    for some of their classes as well.

    I want to thank my wife Ashley and our sons Kiyoshi, Daikichi,
    and Zygmunt for their patience and support for the many hours that
    I spent 
    away from them working on the mechanizations and on this dissertation.

    I want to thank my fellow PL students for many helpful conversations, 
    and especially Saketh Kasibatla, Kyle Thompson, and Cole Kurashige
    for helpful conversations about \coq and theorem proving.
    %
    I also thank James Parker for a helpful discussion 
    about data propositions and the anonymous reviewers across five 
    conferences for their useful comments and suggestions. 
    %
    I owe a debt of gratitude to Joe Politz and Sorin Lerner for detailed
    comments and feedback on an early version of my POPL 24 talk.
    %

    \section*{Work adapted in this dissertation}

    Chapters 1-3, 5-8, and the conclusion are adapted from 
    ``Mechanizing Refinement Types'' in the proceedings of the 
    $51^{st}$ ACM SIGPLAN Symposium on Principles of Programming
    Languages (POPL 2024), by Michael Borkowski, Niki Vazou, and
    Ranjit Jhala.
    
    Chapter 4 is adapted from unpublished material that was originally
    prepared for the same  ``Mechanizing Refinement Types''
    by Michael Borkowski, Niki Vazou, and Ranjit Jhala but did not 
    appear in the final published version.

    Chapter 9 describes unpublished work done in collaboration with Ranjit Jhala.

    Appendix A contains unpublished work done to accompany a
    future archival version of ``Mechanizing Refinement Types.''

    The dissertation author was the primary investigator and author of these works.


\end{acknowledgements}


%% VITA
%
%  A brief vita is required in a doctoral thesis. See the OGS
%  Formatting Manual for more information.
%
\begin{vitapage}
\begin{vita}
  \item[2016] B.~A. in Computer Science \emph{magna cum laude}, Amherst College
  \item[2019] M.~S. in Computer Science, University of California San Diego
  \item[2024] Ph.~D. in Computer Science, University of California San Diego
\end{vita}
\begin{publications}
  \item M.~H. Borkowski, N. Vazou, and R. Jhala, ``Mechanizing Refinement Types'', \emph{POPL}, 2024.
\end{publications}
\end{vitapage}


%% ABSTRACT
%
%  Doctoral dissertation abstracts should not exceed 350 words.
%   The abstract may continue to a second page if necessary.
%
\begin{abstract}
    Practical checkers based on refinement types
    use the combination of implicit semantic subtyping
    and parametric polymorphism to simplify the specification
    and automate the verification of sophisticated properties
    of programs.
    %
    However, a formal metatheoretic accounting
    of the \emph{soundness} of refinement type
    systems using this combination has proved elusive.
    %
    We present \sysrf, a core refinement calculus
    that combines semantic subtyping and parametric
    polymorphism.
    %
    We develop a metatheory for this calculus
    and prove soundness of the type system.
    %
    We give two mechanizations
    of our metatheory.
    %
    First, we introduce \textit{data propositions}, 
    a novel feature that enables encoding derivation 
    trees for inductively defined judgments as refined 
    data types, and use them to show that \lh's refinement
    types can be used \emph{for} mechanization.
    %
    Second, we mechanize our results in \coq, which 
    comes with stronger soundness guarantees than \lh, 
    thereby laying the foundations for mechanizing the 
    metatheory \emph{of} \lh.
    %
    Finally, we present an extension \sysrfd, which adds
    lists and a length measure. We extend the metatheory
    to prove the soundness of the extended type system
    and give another mechanization in \coq.
    %
    % TODO: if i add any bidirectional material add another
    % sentence
    
\end{abstract}


\end{frontmatter}
