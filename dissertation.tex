%
%
% UCSD Doctoral Dissertation Template
% -----------------------------------
% https://github.com/ucsd-thesis/ucsd-thesis
%
%
% ----------------------------------------------------------------------
% WARNING: 
%
%   This template has not endorced by OGS or any other official entity.
%   The official formatting guide can be obtained from OGS.
%   It can be found on the web here:
%   http://grad.ucsd.edu/_files/academic-affairs/Dissertations_Theses_Formatting_Manual.pdf
%
%   No guaranty is made that this LaTeX class conforms to the official UCSD guidelines.
%   Make sure that you check the final document against the Formatting Manual.
%  
%   That being said, this class has been routinely used for successful 
%   publication of doctoral theses.  
%
%   The ucsd.cls class files are only valid for doctoral dissertations.
%
%
% ----------------------------------------------------------------------
% GETTING STARTED:
%
%   Lots of information can be found on the project wiki:
%   http://code.google.com/p/ucsd-thesis/wiki/GettingStarted
%
%
%   To make a pdf from this template use the command:
%     pdflatex template
%
%
%   To get started please read the comments in this template file 
%   and make changes as appropriate.
%
%   If you successfully submit a thesis with this package please let us
%   know.
%
%
% ----------------------------------------------------------------------
% KNOWN ISSUES:
%
%   Currently only the 12pt size conforms to the UCSD requirements.
%   The 10pt and 11pt options make the footnote fonts too small.
%
%
% ----------------------------------------------------------------------
% HELP/CONTACT:
%
%   If you need help try the ucsd-thesis google group:
%   http://groups.google.com/group/ucsd-thesis
%
%
% ----------------------------------------------------------------------
% BUGS:
%
%   Please report all bugs at:
%   https://github.com/ucsd-thesis/ucsd-thesis/issues
%
%
% ----------------------------------------------------------------------
% More control of the formatting of your thesis can be achieved through
% modifications of the included LaTeX class files:
%
%   * ucsd.cls    -- Class file
%   * uct10.clo   -- Configuration files for font sizes 10pt, 11pt, 12pt
%     uct11.clo                            
%     uct12.clo
%
% ----------------------------------------------------------------------



% Setup the documentclass 
% default options: 12pt, oneside, final
%
% fonts: 10pt, 11pt, 12pt -- are valid for UCSD dissertations.
% sides: oneside, twoside -- note that two-sided theses are not accepted 
%                            by OGS.
% mode: draft, final      -- draft mode switches to single spacing, 
%                            removes hyperlinks, and places a black box
%                            at every overfull hbox (check these before
%                            submission).
% chapterheads            -- Include this if you want your chapters to read:
%                              Chapter 1
%                              Title of Chapter
%
%                            instead of
%                              1 Title of Chapter
\documentclass[12pt,chapterheads]{ucsd}



% Include all packages you need here.  
% Some standard options are suggested below.
%
% See the project wiki for information on how to use 
% these packages. Other useful packages are also listed there.
%
%   http://code.google.com/p/ucsd-thesis/wiki/GettingStarted



%% AMS PACKAGES - Chances are you will want some or all 
%    of these if writing a dissertation that includes equations.
\usepackage{amsmath, amscd, amssymb, amsthm}

%% BONUS MATH
\usepackage{mathtools} 

%% MARGIN REQUIREMENTS IN TITLES - Hyphenation in a Section Title does not always respect margin settings in Latex.  To force no hyphentation, uncomment the package below.
%  \usepackage[raggedright]{titlesec} 

%% GRAPHICX - This is the standard package for 
%    including graphics for latex/pdflatex.
\usepackage{scrextend}
\usepackage{pslatex}
\usepackage{graphicx}

%% CAPTION
% This overrides some of the ugliness in ucsd.cls and
% allows the text to be double-spaced while letting figures,
% tables, and footnotes to be single-spaced--all OGS requirements.
% NOTE: Must appear after graphics and ams math
\makeatletter
\gdef\@ptsize{2}% 12pt documents
\let\@currsize\normalsize
\makeatother
\usepackage{setspace}
\doublespace
\usepackage[font=small, width=0.9\textwidth]{caption}
\usepackage[capposition=bottom]{floatrow} %force captions below figure per OGS requirement

%% SUBFIG - Use this to place multiple images in a
%    single figure.  Subfig will handle placement and
%    proper captioning (e.g. Figure 1.2(a))
% \usepackage{subfig}

%% TIMES FONT - replacements for Computer Modern
%%   This package will replace the default font with a
%%   Times-Roman font with math support.
% \usepackage[T1]{fontenc}
% \usepackage{mathptmx}

%% INDEX
%   Uncomment the following two lines to create an index: 
% \usepackage{makeidx}
% \makeindex
%   You will need to uncomment the \printindex line near the
%   bibliography to display the index.  Use the command
% \index{keyword} 
%   within the text to create an entry in the index for keyword.
%   To compile a LaTeX document with an index the 'makeindex'
%   command will need to be run.  See the wiki for more details.

%% CITATIONS
% Sets citation format
% and fixes up citations madness
\usepackage{microtype}  % avoids citations that hang into the margin


%% FOOTNOTE-MAGIC
% Enables footnotes in tables, re-referencing the same footnote multiple times.
\usepackage{footnote}
\makesavenoteenv{tabular}
\makesavenoteenv{table}


%% TABLE FORMATTING MADNESS
% Enable all sorts of fun table tricks
\usepackage{rotating}  % Enables the sideways environment (NCPW)
\usepackage{array}  % Enables "m" tabular environment http://ctan.org/pkg/array
\usepackage{booktabs}  % Enables \toprule  http://ctan.org/pkg/array

%% DEPENDENCIES for MECHANIZING REFINEMENT TPYES

\usepackage{booktabs}   %% For formal tables:
                        %% http://ctan.org/pkg/booktabs
\usepackage{subcaption} %% For complex figures with subfigures/subcaptions
                        %% http://ctan.org/pkg/subcaption


\usepackage[square,numbers]{natbib}
\usepackage{float}
\usepackage[english]{babel}
\usepackage[normalem]{ulem}
\usepackage[T1]{fontenc}
\usepackage[scaled=0.85]{beramono}

\usepackage{framed}
\usepackage{xspace}
\usepackage{fancyvrb}
\usepackage{listings}
\usepackage{array}%, bigfoot}
\usepackage{url}
\usepackage[inline]{enumitem}
\usepackage{pifont}

\usepackage{stmaryrd}
\usepackage{xcolor}
\usepackage{xspace}
\usepackage{mathpartir}

\usepackage{comment}

\usepackage{tikz}
\usetikzlibrary{arrows.meta,automata,shapes,patterns}
\usetikzlibrary{matrix,positioning}
\usepackage{semantic}

\usepackage{commands_full}
\usepackage[shortcuts]{extdash}
\usepackage{enumitem}


%% HYPERLINKS
%   To create a PDF with hyperlinks, you need to include the hyperref package.
%   THIS HAS TO BE THE LAST PACKAGE INCLUDED!
%   Note that the options plainpages=false and pdfpagelabels exist
%   to fix indexing associated with having both (ii) and (2) as pages.
%   Also, all links must be black according to OGS.
%   See: http://www.tex.ac.uk/cgi-bin/texfaq2html?label=hyperdupdest
%   Note: This may not work correctly with all DVI viewers (i.e. Yap breaks).
%   NOTE: hyperref will NOT work in draft mode, as noted above.
\usepackage[colorlinks=true, pdfstartview=FitV, 
            linkcolor=black, citecolor=black, 
            urlcolor=black, plainpages=false,
            pdfpagelabels]{hyperref}
\hypersetup{ pdfauthor = {Michael H. Borkowski}, 
             pdftitle = {Mechanizing Refinement Types}, 
             pdfkeywords = {Refinement Types}, 
             pdfcreator = {pdfLaTeX with hyperref package}, 
             pdfproducer = {pdfLaTeX} }
\hypersetup{colorlinks=true,allcolors=blue}
\urlstyle{same}
% \usepackage{bookmark}

\usepackage{cleveref}





%% MY DEFINITIONS

% additional definitions lifted from acmart.cls
\newtheorem{theorem}{Theorem}[chapter]
\newtheorem{proposition}[theorem]{Proposition}
\newtheorem{lemma}[theorem]{Lemma}
\newtheorem{corollary}[theorem]{Corollary}
% mine:
\newtheorem{assumption}[theorem]{Assumption}
\newtheorem{requirement}[theorem]{Requirement}

\newif\iffullvers % toggle true or false based on full or conference version
\fullverstrue % show the full text == Full Version
%\fullversfalse % hide text and show abbreviated parts == POPL'24 Version

\iffullvers
  \includecomment{fullversion}
  \excludecomment{conference}
\else
  \includecomment{conference}
  \excludecomment{fullversion} % anything wrapped in a {wrap} environment will be excluded
\fi



\lstset{
    keywordstyle=\color{blue}
  , basicstyle=\ttfamily
  , commentstyle={}
  , columns=flexible
  %, numbers=left
  , showstringspaces=false
  }


%% TIKZ DEPENDENCY CHART SETTINGS

  \newlength{\hatchspread}
  \newlength{\hatchthickness}
  \newlength{\hatchshift}
  \newcommand{\hatchcolor}{}
  % declaring the keys in tikz
  \tikzset{hatchspread/.code={\setlength{\hatchspread}{#1}},
           hatchthickness/.code={\setlength{\hatchthickness}{#1}},
           hatchshift/.code={\setlength{\hatchshift}{#1}},% must be >= 0
           hatchcolor/.code={\renewcommand{\hatchcolor}{#1}}}
  % setting the default values
  \tikzset{hatchspread=3pt,
           hatchthickness=0.4pt,
           hatchshift=0pt,% must be >= 0
           hatchcolor=black}
  % declaring the pattern
  \pgfdeclarepatternformonly[\hatchspread,\hatchthickness,\hatchshift,\hatchcolor]% variables
     {custom north west lines}% name
     {\pgfqpoint{\dimexpr-2\hatchthickness}{\dimexpr-2\hatchthickness}}% lower left corner
     {\pgfqpoint{\dimexpr\hatchspread+2\hatchthickness}{\dimexpr\hatchspread+2\hatchthickness}}% upper right corner
     {\pgfqpoint{\dimexpr\hatchspread}{\dimexpr\hatchspread}}% tile size
     {% shape description
      \pgfsetlinewidth{\hatchthickness}
      \pgfpathmoveto{\pgfqpoint{0pt}{\dimexpr\hatchspread+\hatchshift}}
      \pgfpathlineto{\pgfqpoint{\dimexpr\hatchspread+0.15pt+\hatchshift}{-0.15pt}}
      \ifdim \hatchshift > 0pt
        \pgfpathmoveto{\pgfqpoint{0pt}{\hatchshift}}
        \pgfpathlineto{\pgfqpoint{\dimexpr0.15pt+\hatchshift}{-0.15pt}}
      \fi
      \pgfsetstrokecolor{\hatchcolor}
  %    \pgfsetdash{{1pt}{1pt}}{0pt}% dashing cannot work correctly in all situation this way
      \pgfusepath{stroke}
     }
  
  % declaring the pattern
  \pgfdeclarepatternformonly[\hatchspread,\hatchthickness,\hatchcolor]% variables
    {custom horizontal lines} % name
    {\pgfpointorigin}{\pgfqpoint{100pt}{1pt}}{\pgfqpoint{100pt}{\dimexpr\hatchspread}}%
    {%
      \pgfsetlinewidth{\hatchthickness}%
      \pgfpathmoveto{\pgfqpoint{0pt}{0.5pt}}%
      \pgfpathlineto{\pgfqpoint{100pt}{0.5pt}}%
      \pgfsetstrokecolor{\hatchcolor}
      \pgfusepath{stroke}%
    }%
  
  \pgfdeclarepatternformonly[\hatchspread,\hatchthickness,\hatchcolor]% variables
    {custom vertical lines}% name
    {\pgfpointorigin}{\pgfqpoint{1pt}{100pt}}{\pgfqpoint{\dimexpr\hatchspread}{100pt}}%
    {%
      \pgfsetlinewidth{\hatchthickness}%
      \pgfpathmoveto{\pgfqpoint{0.5pt}{0pt}}%
      \pgfpathlineto{\pgfqpoint{0.5pt}{100pt}}%
      \pgfsetstrokecolor{\hatchcolor}
      \pgfusepath{stroke}%
    }%
  
  \pgfdeclarepatternformonly[\hatchspread,\hatchthickness,\hatchcolor]% variables
    {custom crosshatch dots}
    {\pgfqpoint{-1pt}{-1pt}}{\pgfqpoint{2.5pt}{2.5pt}}{\pgfqpoint{\dimexpr\hatchspread+\hatchspread}{\dimexpr\hatchspread+\hatchspread}}%
    {%
      \pgfpathcircle{\pgfqpoint{0pt}{0pt}}{\hatchthickness}%
      %\pgfpathcircle{\pgfqpoint{\dimexpr\hatchspread}{\dimexpr\hatchspread}}{\hatchthickness}%
      \pgfsetstrokecolor{\hatchcolor}%
      \pgfusepath{stroke}%
      \pgfusepath{fill}%
    }%
  
  \pgfdeclarepatternformonly[\hatchcolor]%
    {custom checkerboard}
    {\pgfpointorigin}{\pgfqpoint{4mm}{4mm}}{\pgfqpoint{4mm}{4mm}}%
    {%
      \pgfpathrectangle{\pgfpointorigin}{\pgfqpoint{2mm}{2mm}}%
      \pgfsetfillcolor{\hatchcolor}%
      \pgfusepath{fill}%
      \pgfpathrectangle{\pgfqpoint{2mm}{2mm}}{\pgfqpoint{2mm}{2mm}}%
      \pgfsetfillcolor{\hatchcolor}%
      \pgfusepath{fill}%
    }%
      

\makeatother
\newcommand{\cL}{{\cal L}}

\usepackage{listings}

% uncomment next line to restore colors
% \def\withcolor{}


\ifdefined\withcolor
  \definecolor{haskellstr}{rgb}{0.2, 0.2, 0.6}
  \definecolor{haskellred}{rgb}{1.0, 0.0, 0.0}
  \definecolor{gray_ulisses}{gray}{0.55}
  \definecolor{green_ulises}{rgb}{0.2,0.75,0}
  \definecolor{haskelltypes}{rgb}{0.71,0.33,0.14}
  \definecolor{logiccolor}{rgb}{0.0, 0.0, 1.0}
\else
  \definecolor{haskellstr}{rgb}{0.2, 0.2, 0.6}
  \definecolor{haskellred}{gray}{0.1}
  \definecolor{gray_ulisses}{gray}{0.55}
  \definecolor{green_ulisses}{gray}{0.1}
  \definecolor{haskelltypes}{gray}{0.1}
  \definecolor{logiccolor}{gray}{0.1}
\fi

\definecolor{haskellblue}{rgb}{0.0, 0.0, 1.0}
\definecolor{haskell_green}{rgb}{0.0, 0.5, 0.0}
\definecolor{blue_violet}{rgb}{0.54, 0.17, 0.89}
\definecolor{castanho_ulisses}{rgb}{0.43, 0.21, 0.1}
\definecolor{preto_ulisses}{rgb}{0.21,0.00,0.80}

\definecolor{subtleOpHighlight}{rgb}{0.4, 0.2, 0.0}

% RJ: I don't like the colors. Too confusing.
%
% \definecolor{lcolor}{rgb}{0.0, 0.0, 1.0}
% \definecolor{lappcolor}{rgb}{1.0, 0.0, 0.0}
% \definecolor{lappascolor}{rgb}{0.0, 1.0, 0.0}

\definecolor{lcolor}{gray}{0.0}
\definecolor{lappcolor}{gray}{0.0}
\definecolor{lappascolor}{gray}{0.0}

\def\codesize{\normalsize}
% \def\codesize{\footnotesize}
\def\incodesize{\normalsize}


% \newcommand\showfocus[1]{\color{lappcolor}{#1}}
% \newcommand\showfocus[1]{\underbar{#1}}
\newcommand\showfocus[1]{\color{purple}{\textbf{#1}}}
\newcommand\showlogic[1]{\color{logiccolor}{#1}}

\lstdefinelanguage{HaskellUlisses} {
  aboveskip=\smallskipamount,
  belowskip=\smallskipamount,
  basicstyle=\ttfamily\codesize,
  moredelim=[is][\showfocus]{\#}{\#},
  moredelim=[is][\showlogic]{!}{!},
  sensitive=true,
  morecomment=[l][\color{gray_ulisses}\ttfamily\itshape\codesize]{--},
  % morecomment=[l][\color{green}]{--},
  % morecomment=[s][\color{gray_ulisses}\ttfamily\itshape\codesize]{\{-}{-\}},
  %morecomment=[l][\ttfamily\itshape\codesize]{--},
  %morecomment=[s][\ttfamily\itshape\codesize]{\{-}{-\}},
  morestring=[b]",
  %% escapeinside={(*}{*)},
  stringstyle=\color{haskellstr},
  showstringspaces=false,
  numberstyle=\codesize,
  numberblanklines=true,
  showspaces=false,
  breaklines=true,
  showtabs=false,
  %% whitespace hackery
  %%lineskip=-2pt,
  % aboveskip=0pt,
  % belowskip=0pt,
  literate={
           {<!}{{{\color{lcolor}<!}}}2
%           {==!}{{{\color{lcolor}==!}}}3
           {`}{{{$^{\backprime}{}$}}}1
           % {QED}{{{\color{lcolor}QED}}}3
           % {***}{{{\color{lcolor}***}}}3
           {?}{{{\color{lcolor}?}}}1
           {<=}{{$\leq\;\;$}}1
           {/=}{{$\neq$}}1
           {bot}{{$\bot$}}1
           {top}{{$\top$}}1
           {theta}{{$\theta$}}1
           {gf}{{{\color{lappascolor}f}}}1
           {rmap}{{{\color{lappcolor}map}}}3
           {gmap}{{{\color{lappascolor}map}}}3
           {r.}{{{\color{lappcolor}.}}}1
           {g.}{{{\color{lappascolor}.}}}1
           {r++}{{{\showfocus{++}}}}2
           {g++}{{{\color{lappascolor}++}}}2
           {>>}{{{\color{haskellblue}>>}}}2
           {>>=}{{{\color{haskellblue}>>=}}}3
           {</>}{{{\color{haskellblue}</>}}}3
           {<*>}{{{\color{haskellblue}<*>}}}3
           {++}{{{\color{haskellblue}++}}}3
           %{g>>=}{{{\color{lappascolor}>>=}}}3
           {gfib}{{{\color{lappascolor}fib}}}3
           {rfib}{{{\color{lappcolor}fib}}}3
           {r++}{{{\color{lappcolor}++}}}2
           % {>>=}{$\ebind$}2
           % {env}{{$\Gamma$}}1
           {|-}{{$\vdash\;$}}1
          % {>=}{{$\geq$}}1
          % {<}{{$<$}}1
          % {>}{{$>$}}1
           {<=!}{{{\color{lcolor}<=!}}}3
          % {\\}{{$\lambda$}}1
           {!=}{{$\neq$}}1
           % {forall}{{$\forall$}}1
           % {->}{{$\rightarrow$}}2
           {~>}{{$\imparrow$}}2
           {<-}{{$\leftarrow$}}2
           {->}{{$\rightarrow\;\;$}}2                    % added by Michael
           % {->}{{$\rightarrow\;\;$}}2                    % added by Michael
           {:=}{{$\defeq$}}2                          % added by Michael
           {Gamma}{{$\Gamma$}}1                        % added by Michael
           {dollar}{{$\texttt{\$}$}}1
% Testing messing around with these
%----------------------------------------
           %% {++}{{\color{subtleOpHighlight}{++}}}2
           %% {=.}{{\color{subtleOpHighlight}{=.}}}2
           %% {::}{{\color{subtleOpHighlight}{::}}}2
%           {=}{{\color{subtleOpHighlight}{=}}}1
%----------------------------------------
           {Set_mem}{{$\in$}}1
           {Set_cup}{{$\cup$}}1
           {Set_cap}{{$\cap$}}1
           {Set_emp}{{$\emptyset$}}1
           {Set_sub}{{$\subseteq$}}1
           {<=>}{{$\Leftrightarrow$}}3
           {=>}{{$\Rightarrow$}}2
           {1->}{{$\rightarrow$}}1
           {1=>}{{$\Rightarrow$}}1
           {||-}{{$\vdash$}}1
           {|->}{{$\mapsto$}}2
           {<:}{{$\preceq$}}1
           {Inarritu}{Inarritu}8},
%           {Inarritu}{I$\tilde{\text{n}}\acute{\text{a}}$rritu}8,
  emph=
  {[1]
    succ,incr,two,incrMany,three,id,new,get,set,return,grant,revoke,main,canRead,safeRead,
    test1,test2,
    gnt,rev,next,foo,bar,baz,pure,client,done,clynt,fin,
    FilePath,IOError,abs,acos,acosh,all,and,any,appendFile,approxRational,asTypeOf,asin,
    asinh,atan,atan2,atanh,basicIORun,break,catch,ceiling,chr,compare,concat,concatMap,
    cos,cosh,curry,cycle,decodeFloat,denominator,digitToInt,div,divMod,drop,
    dropWhile,either,elem,encodeFloat,enumFrom,enumFromThen,enumFromThenTo,enumFromTo,
    error,even,exponent,fail,mapMaybe,filter,flip,floatDigits,floatRadix,floatRange,floor,
    foldl,foldl1,foldr1,fromDouble,fromEnum,fromInt,fromInteger,fromIntegral,
    fromRational,fst,gcd,put,tick,tock,tocker,ticker,getChar,getContents,getLine,head,inRange,index,init,intToDigit,
    interact,ioError,isAlpha,isAlphaNum,isAscii,isControl,isDenormalized,isDigit,isHexDigit,
    isIEEE,isInfinite,isLower,isNaN,isNegativeZero,isOctDigit,isPrint,isSpace,isUpper,iterate,
    last,lcm,length,lex,lexDigits,lexLitChar,lines,log,logBase,lookup,mapM,mapM_,max,
    maxBound,posMax,negMax,maximum,maybe,min,minBound,minimum,mod,negate,notElem,null,numerator,odd,
    or,ord,pi,pred,primExitWith,print,product,putChar,putStr,putStrLn,quot,
    quotRem,range,rangeSize,read,readDec,readFile,readFloat,readHex,readIO,readInt,readList,readLitChar,
    readLn,readOct,readParen,readSigned,reads,readsPrec,realToFrac,recip,rem,repeat,replicate,return,
    reverse,round,scaleFloat,scanl,scanl1,scanr,scanr1,seq,sequence,sequence_,show,showChar,showInt,
    showList,showLitChar,showParen,showSigned,showString,shows,showsPrec,significand,signum,sin,
    sinh,snd,span,splitAt,sqrt,subtract,tail,take,takeWhile,tan,tanh,threadToIOResult,toEnum,
    toInteger,toLower,toRational,toUpper,truncate,uncurry,undefined,unlines,until,unwords,unzip,
    unzip3,userError,words,writeFile,zip,zip3,zipWith,zipWith3,listArray,doParse,empty,for,initTo,
        assert,compose,checkGE,maxEvens,empty,create,get,set,initialize,idVec,fastFib,fibMemo,
        ex1,ex2,ex3,inc,dec,isPos,positives,find,flatten, expand,exAll,
        ind,evenLen,lenAppend,exDistOr,allDistAnd,len,size,union,singleton,initUpto,trim,
        insertSort,decsort,qsort,reverse,append,upperCase, ifM, whileM, get, decrM, diff,
        project, select, sel, elts, keys, dkeys, dfun, addKey, pTrue, emptyRD, rFalse,
                dom, rng, isI, isD, isS, movie1, movie2,  toI, toS, toD, good_titles, runState, ret,
                update, getCtr, setCtr, ctr, rdCtr, wrCtr, ifTest, whileTest, posCtr, zeroCtr, decr, decCtr,
                pread , pwrite , plookup , pcontents, pcreateF , pcreateFP, pcreateD, active, caps, pset, eqP,
                write, contents, alloc, derivP, copyP, createDir, store, copyRec, copySpec,
                preservation, progress,subFV, safety, soundness, impossible, lemValStep, isVal,
                forM_, when, flookup, fread, createDir, pcreateFile, isFile, copyFrame, ?
  },
  emphstyle={[1]\color{haskellblue}},
  emph=
  {[2]Show,Eq,Iso,VerifiedOrd,Ord,Num,UpClosed,Comp,Wit,Witness,Inductive,match, end, Definition, Meet,Flip,TRUE,
      Peano,Nat,Prime, ArrayN,Pos,SInt,Neg,IntGE,Plus,List,PAnd, POr, POrL, POrR,
        Bool,Char,Double,Either,Float,IO,Integer,Int,Maybe,Up,Mono,Identity,
        Ordering,Rational,Ratio,ReadS,ShowS,File,Token,ST,String,Str,Word8,
        InPacket,Tree,Vec,NullTerm,IncrList,DecrList,
        UniqList,BST,MinHeap,MaxHeap,World,RIO,IO,HIO,Post,Pre, OptEq,
        Privilege, Chain, ChainTy, Range, Dict, RD, Dom, Set, P, Univ, Schema, MovieSchema, RT,
        TDom, TRange, MoviesTable, RTSubEqFlds, RTEqFlds, Disjoint, Union, Ret, Seq, Trans, Map,
        Pure, Then, Else, Exit, Inv, OneState, Priv, Path, FH, Stable,
      Monoid, VerifiedMonoid, VerifiedComMonoid, Plus_2_2_eq_4, Plus_2_2, Nat_up, Int_up,
      AppendNilId, AppendAssoc,MapFusion,
      Plus_comm, Par, Term,HasTyPr,HasTyEv,IsSubTyPr,IsSubTyEv,StepPr,StepEv, StepsPr,StepsEv,Env,Prim,Expr,Proposition,DataProp,VName,ProofOf,ProofOfN,Type,Names,Kind,
      HasTy,WfType,IsSubTy,Step,Steps,EvalsTo,Subtype,
      Formula, Assignment, Body, Accel, Real, Accel', RVar, VerifiedCommutativeMonoid, CommutativeMonoid
  },
  emphstyle={[2]\color{blue_violet}},
  emph=
  {[3]
    case,class,newtype,data,deriving,do,else,if,unpack,import,in,infixl,infixr,instance,let,
    module,of,primitive,then,refinement,type,where,forall,bound,expr,vname, env, prim, bool,
    % otherwise,
    measure,reflect,predicate, instance, class,
    exists, Qed, Theorem, Proof, option, Lemma
  },
  emphstyle={[3]\color{castanho_ulisses}\textbf},
  emph=
  {[4]
    quot,rem,div,mod,elem,notElem,seq
  },
  emphstyle={[4]\color{preto_ulisses}},
  emph=
  {[5]
    EQ,GT,LT,Left,Right,SBase,TPrim,TSub,TAbs,TAbsEx,TAbsCQ,
    TRefn,TInt,EApp,EPrim,EEq,EVar,PInt,Empty,Cons,EVar,ELam,EApp,TFunc,
    Refl,AddStep, Add
    %, False, True, Just, Nothing
  },
  emphstyle={[5]\color{haskell_green}},
  emph=
  {[6]
      axiomatize, measure, inline
  },
  emphstyle={[6]\color{lcolor}\ttfamily\itshape},
  emph=
  {[6]
      hasTyEvPr,isSubTyEvPr
  }
}

%%%ORIG
%%%\lstnewenvironment{code}
%%%{\textbf{Haskell Code} \hspace{1cm} \hrulefill \lstset{language=HaskellUlisses}}
%%%{\hrule\smallskip}

%V1
%\lstnewenvironment{code}
%{\smallskip \lstset{language=HaskellUlisses}}
%{\smallskip}

\lstnewenvironment{code}
{\lstset{language=HaskellUlisses}}
{}

\lstnewenvironment{mcode}
{\lstset{language=HaskellUlisses,columns=fullflexible,keepspaces,mathescape}}
{}

\lstnewenvironment{mcodef}
{\lstset{language=HaskellUlisses,columns=fullflexible,keepspaces,mathescape,frame=single}}
{}

\lstMakeShortInline[language=HaskellUlisses,mathescape,keepspaces,mathescape,basicstyle=\ttfamily\codesize,breakatwhitespace]@

\lstdefinelanguage{Pseudo} {
  basicstyle=\ttfamily\codesize,
  sensitive=true,
  mathescape=true,
  morecomment=[l][\color{gray_ulisses}\ttfamily\codesize]{--},
  morecomment=[s][\color{gray_ulisses}\ttfamily\codesize]{\{-}{-\}},
  morestring=[b]",
  showstringspaces=false,
  numberstyle=\codesize,
  numberblanklines=true,
  showspaces=false,
  breaklines=true,
  showtabs=false
}

\lstdefinelanguage{java} {
    keywordstyle=[1],
    keywordstyle=[2]\color{ForestGreen},
    keywordstyle=[3]\color{Bittersweet},
    keywordstyle=[4]\color{RoyalPurple},
    morekeywords={region,private,synchronized}
}


%%% PSEUDOCODE LISTING Python Style

\ifdefined\withcolor
        \definecolor{typecol}{rgb}{0.0,0.5,0.0}
        \definecolor{funcol}{rgb}{0.0,0.1,0.9}
        % BROWN \definecolor{typecol}{rgb}{0.71,0.33,0.14}
        % PURPLE \definecolor{funcol}{rgb}{0.8,0.0,0.6}
\else
        \definecolor{typecol}{gray}{0.0}
        \definecolor{funcol}{gray}{0.0}
\fi

\lstdefinelanguage{pseudo2}{
  language=Python,
  basicstyle=\ttfamily\normalsize,
  mathescape=true,
  morekeywords={type,def,do,let,unpack},
  emph={[1] \Expr,\Pred, HP, FP, Int},
  emphstyle={[1]\itshape\color{typecol}},
  emph={[2] \pbe,normalize},
  emphstyle={[2]\itshape\color{funcol}},
  % numbers=left,
  emph={[3] repeat,until},
  emphstyle={[3]\textbf}
}

\lstnewenvironment{code2}{\lstset{language=pseudo2}}{}

\lstnewenvironment{fscode}
{\lstset{language=HaskellUlisses,basicstyle=\ttfamily\small,mathescape=true}}
{}

%% \lstnewenvironment{code}
%% {\lstset{language=HaskellUlisses,mathescape=true}}
%% {}

\lstnewenvironment{codebox}
{\lstset{language=HaskellUlisses,frame=tlbr,basicstyle=\ttfamily\codesize,mathescape=true}}
{}

\def \ha {\lstinline[language=HaskellUlisses,basicstyle=\ttfamily\incodesize,mathescape=true]}




\begin{document}



%% CAN THIS BE MOVED ABOVE begin?
\pgfdeclarelayer{background}
\pgfdeclarelayer{middleground}
\pgfdeclarelayer{foreground}
\pgfsetlayers{background,middleground,main,foreground}


%% FRONT MATTER
%
%  All of the front matter.
%  This includes the title, degree, dedication, vita, abstract, etc..
%  Modify the file template_frontmatter.tex to change these pages.
%
%
% UCSD Doctoral Dissertation Template
% -----------------------------------
% http://ucsd-thesis.googlecode.com
%
%


%% REQUIRED FIELDS -- Replace with the values appropriate to you

% No symbols, formulas, superscripts, or Greek letters are allowed
% in your title.
\title{Mechanizing Refinement Types}

\author{Michael H. Borkowski}
\degreeyear{\the\year}

% Master's Degree theses will NOT be formatted properly with this file.
\degreetitle{Doctor of Philosophy}

\field{Computer Science}
%\specialization{Programming Languages}  % If you have a specialization, add it here

\chair{Professor Ranjit Jhala}
% Uncomment the next line iff you have a Co-Chair
% \cochair{Professor Cochair Semimaster}
%
% Or, uncomment the next line iff you have two equal Co-Chairs.
%\cochairs{Professor Chair Masterish}{Professor Chair Masterish}

%  The rest of the committee members  must be alphabetized by last name.
\othermembers{
Professor Samuel R. Buss\\
Professor Cormac Flanagan\\
Professor Nadia Polikarpova\\
Professor Victor Vianu\\
}
\numberofmembers{5} % |chair| + |cochair| + |othermembers|


%% START THE FRONTMATTER
%
\begin{frontmatter}

%% TITLE PAGES
%
%  This command generates the title, copyright, and signature pages.
%
\makefrontmatter

%% DEDICATION
%
%  You have three choices here:
%    1. Use the ``dedication'' environment.
%       Put in the text you want, and everything will be formated for
%       you. You'll get a perfectly respectable dedication page.
%
%
%    2. Use the ``mydedication'' environment.  If you don't like the
%       formatting of option 1, use this environment and format things
%       however you wish.
%
%    3. If you don't want a dedication, it's not required.
%
%
\begin{dedication}
  To Kiyoshi, Daikichi, and Ziggy
\end{dedication}


% \begin{mydedication} % You are responsible for formatting here.
%   \vspace{1in}
%   \begin{flushleft}
% 	To me.
%   \end{flushleft}
%
%   \vspace{2in}
%   \begin{center}
% 	And you.
%   \end{center}
%
%   \vspace{2in}
%   \begin{flushright}
% 	Which equals us.
%   \end{flushright}
% \end{mydedication}



%% EPIGRAPH
%
%  The same choices that applied to the dedication apply here.
%
%\begin{epigraph} % The style file will position the text for you.
%  \emph{A careful quotation\\
%  conveys brilliance.}\\
%  ---Smarty Pants
%\end{epigraph}

% \begin{myepigraph} % You position the text yourself.
%   \vfil
%   \begin{center}
%     {\bf Think! It ain't illegal yet.}
%
% 	\emph{---George Clinton}
%   \end{center}
% \end{myepigraph}


%% SETUP THE TABLE OF CONTENTS
%
\tableofcontents
\listoffigures  % Comment if you don't have any figures
\listoftables   % Comment if you don't have any tables



%% ACKNOWLEDGEMENTS
%
%  While technically optional, you probably have someone to thank.
%  Also, a paragraph acknowledging all coauthors and publishers (if
%  you have any) is required in the acknowledgements page and as the
%  last paragraph of text at the end of each respective chapter. See
%  the OGS Formatting Manual for more information.
%
\begin{acknowledgements}
    %
    First, I would like to thank Ranjit Jhala for all of his support as
    my advisor. 
    %
    I am greatly indebted to Ranjit for taking me on as his student when I 
    was a complete beginner at type theory and software verification
    research.
    %
    He provided the original motivation for my work in his 2019 
    graduate class on \lh, and I've been hooked on theorem proving ever since.
    % 
    I appreciate Ranjit's insights and feedback during our 
    meetings and, most of all, his continuing confidence in my work 
    throughout four conference rejections motivated me to keep improving 
    and adding to our work.
    %

    I would also like to thank my collaborator and coauthor Niki Vazou
    for all of her patient help, support, and ideas. I couldn't have 
    done this research without her support either!
    %
    I want to thank each of the members of my committee, Nadia Polikarpova,
    Victor Vianu, Sam Buss, and Cormac Flanagan for their support through 
    this process and for the opportunity to TA
    for some of their classes as well.

    I want to thank my wife Ashley and our sons Kiyoshi, Daikichi,
    and Zygmunt for their patience and support for the many hours that
    I spent 
    away from them working on the mechanizations and on this dissertation.

    I want to thank my fellow PL students for many helpful conversations, 
    and especially Saketh Kasibatla, Kyle Thompson, and Cole Kurashige
    for helpful conversations about \coq and theorem proving.
    %
    I also thank James Parker for a helpful discussion 
    about data propositions and the anonymous reviewers across five 
    conferences for their useful comments and suggestions. 
    %
    I owe a debt of gratitude to Joe Politz and Sorin Lerner for detailed
    comments and feedback on an early version of my POPL 24 talk.
    %

    \section*{Work adapted in this dissertation}

    Chapters 1-3, 5-8, and the conclusion are adapted from 
    ``Mechanizing Refinement Types'' in the proceedings of the 
    $51^{st}$ ACM SIGPLAN Symposium on Principles of Programming
    Languages (POPL 2024), by Michael Borkowski, Niki Vazou, and
    Ranjit Jhala.
    
    Chapter 4 is adapted from unpublished material that was originally
    prepared for the same  ``Mechanizing Refinement Types''
    by Michael Borkowski, Niki Vazou, and Ranjit Jhala but did not 
    appear in the final published version.

    Chapter 9 describes unpublished work done in collaboration with Ranjit Jhala.

    %Appendix A contains unpublished work done to accompany a
    %future archival version of ``Mechanizing Refinement Types.''

    The dissertation author was the primary investigator and author of these works.


\end{acknowledgements}


%% VITA
%
%  A brief vita is required in a doctoral thesis. See the OGS
%  Formatting Manual for more information.
%
\begin{vitapage}
\begin{vita}
  \item[2016] B.~A. in Computer Science \emph{magna cum laude}, Amherst College
  \item[2019] M.~S. in Computer Science, University of California San Diego
  \item[2024] Ph.~D. in Computer Science, University of California San Diego
\end{vita}
\begin{publications}
  \item M.~H. Borkowski, N. Vazou, and R. Jhala, ``Mechanizing Refinement Types'', \emph{POPL}, 2024.
\end{publications}
\end{vitapage}


%% ABSTRACT
%
%  Doctoral dissertation abstracts should not exceed 350 words.
%   The abstract may continue to a second page if necessary.
%
\begin{abstract}
    Practical checkers based on refinement types
    use the combination of implicit semantic subtyping
    and parametric polymorphism to simplify the specification
    and automate the verification of sophisticated properties
    of programs.
    %
    However, a formal metatheoretic accounting
    of the \emph{soundness} of refinement type
    systems using this combination has proved elusive.
    %
    We present \sysrf, a core refinement calculus
    that combines semantic subtyping and parametric
    polymorphism.
    %
    We develop a metatheory for this calculus
    and prove soundness of the type system.
    %
    We give two mechanizations
    of our metatheory.
    %
    First, we introduce \textit{data propositions}, 
    a novel feature that enables encoding derivation 
    trees for inductively defined judgments as refined 
    data types, and use them to show that \lh's refinement
    types can be used \emph{for} mechanization.
    %
    Second, we mechanize our results in \coq, which 
    comes with stronger soundness guarantees than \lh, 
    thereby laying the foundations for mechanizing the 
    metatheory \emph{of} \lh.
    %
    Finally, we present an extension \sysrfd, which adds
    lists and a length measure. We extend the metatheory
    to prove the soundness of the extended type system
    and give another mechanization in \coq.
    %
    % TODO: if i add any bidirectional material add another
    % sentence
    
\end{abstract}


\end{frontmatter}






%% DISSERTATION

% A common strategy here is to include files for each of the chapters. I.e.,
% Place the chapters is separate files: 
%   chapter1.tex, chapter2.tex
% Then use the commands:
%   \include{chapter1}
%   \include{chapter2}
%
% Of course, if you prefer, you can just start with
%   \chapter{My First Chapter Name}
% and start typing away.  

% Ch 1: Introduction
\chapter{Introduction}
\label{ch:intro}

Refinements constrain types with logical predicates
to specify new concepts.
%
For example, the refinement type
$\tpos \defeq \breft{\tint}{\vv}{0 < v}$
describes \emph{positive} integers
and $\tnat \doteq \breft{\tint}{\vv}{0 \leq v}$
specifies natural numbers.
%
Refinement types have been successfully
used to specify various properties
like secrecy \citep{FournetCCS11},
resource usage \citep{Knoth20}, or
information flow \citep{STORM}
that can then be verified in programs
developed in various programming
languages like Haskell~\citep{Vazou14},
Scala~\citep{kuncak-stainless}, and
Racket~\citep{RefinedRacket}.

%
The success of refinement types
relies on the combination of two
essential features.
%
First, \textit{implicit} semantic
subtyping uses semantic (SMT-based)
reasoning to automatically convert
the types of expressions without
hassling the programmer for explicit type
casts.
%
For example, consider a positive expression
$e : \tpos$ and a function expecting natural
numbers $f:\funcftype{\tnat}{\tint}$.
%
To type check the application $f\ e$,
the refinement type system will implicitly
convert the type of $e$ from \tpos to \tnat,
because $0 < v \Rightarrow 0 \leq v$
holds semantically.
%
Importantly, refinement types propagate
semantic subtyping inside type constructors
to, for example, treat function arguments in
a contravariant manner.
%
Second, \textit{parametric polymorphism}
allows the propagation of the refined types
through polymorphic function interfaces,
without the need for extra reasoning.
%
As a trivial example, once we have
established that $e$ is positive,
parametric polymorphism should let
us conclude that $g\ e : \tpos$ if,
for example, $g$ is the identity function
$g:\funcftype{a}{a}$.
As a more interesting example, in~\S~\ref{sec:overview:primes}     % TOFIX
we combine semantic subtyping and polymorphism
to verify a safe-indexing array of prime numbers.

The engineering of practical refinement
type checkers has galloped far ahead of
the development of their metatheoretical
foundations.
%
In fact, semantic subtyping is very tricky
as it is mutually defined with typing,
leading to metatheoretic proofs with
circular dependencies (\Cref{fig:dependencies}).
%
Unsurprisingly, the addition of polymorphism
poses further challenges.
%
As \citet{SekiyamaIG17} observe, a na\"ive definition
of type instantiation can lose potentially contradicting
refinements leading to unsoundness.
%
Existing formalizations of refinement types
drop semantic subtyping~\citep{SekiyamaIG17,kuncak-stainless}
or polymorphism~\citep{flanagan06, newfstar},
or have problematic metatheory~\citep{Belo11}.

\section{Outline}
\label{sec:outline}

In this dissertation we formalize \sysrf, a core calculus
with a refinement type system that combines semantic
subtyping with polymorphism, via six concrete contributions.
%
But first, we begin in \Cref{ch:overview} with an overview 
of refinement types, giving examples of their applications
and discussing their essential features. 
%
We conclude the
chapter in \S~\ref{sec:overview:soundness}          % TOFIX
by outlining the challenges we encountered in attempting to
prove the soundness of \sysrf and how we addressed each of them.

\mypara{1. Reconciliation}
%
In \Cref{ch:language} we introduce            % Lang && Static Sem
our first contribution: a language
that combines refinements and polymorphism
in a way that ensures the metatheory remains sound
without sacrificing the expressiveness needed
for practical verification.
%
To this end, \sysrf introduces a kind
system that distinguishes the type
variables that can be soundly refined
(without the risk of losing refinements
at instantiation) from the rest,
which are then left unrefined.
%
In addition our design includes
a form of existential typing \cite{Knowles09}
which is essential to \emph{synthesize} the types
-- in the sense of bidirectional typing -- for applications
and let-binders in a compositional manner 
(\S~\ref{subsec:overview:exists}, \S~\ref{sec:lang:static}).             % TOFIX

\mypara{2. Foundation}                      % Soundness
%
Our second contribution,
described in \Cref{ch:soundness},
is to establish the foundations
of \sysrf by proving soundness,
which says that well-typed expressions cannot get stuck
and belong to the denotation of their type 
(\S~\ref{sec:soundness:denotational}, \S~\ref{sec:soundness:safety}).                                   % TOFIX
%which says that if $e$ has a type, then
%either $e$ is a value or it can
%step to another term of the same type.
%
%Our second contribution is to establish the foundations
%of \sysrf by defining a notion of type denotations and using
%it to prove two key properties: first, \emph{denotational}
%soundness which says that if an expression $e$ has type $t$,
%then $e$ belongs in the denotation of the type $t$, and second,
%\emph{evaluation} soundness that either $e$ is a value or it can
%step to another term of the same type.
%
The combination of semantic subtyping, polymorphism,
and existentials makes the soundness proof challenging
with circular dependencies that do not arise in standard
(unrefined) calculi.
%
The mechanization was simplified by the use of
two essential ingredients.
%
First, we use an unrefined \emph{base} language
\sysf, a classic System F \cite{TAPL}, in rules
where refinements are not required, cutting two potential
circularities in the static judgments (\Cref{fig:dependencies}).
Second, we define an \emph{implication interface}
that abstractly specifies the properties of implication
required to prove type soundess, and show how this interface
can be implemented via denotational semantics 
(\S~\ref{sec:typing:implication}).                          % TOFIX

\mypara{3. Reification}                     % Data Props
%
Our third contribution,
presented in \Cref{ch:data-props}, is to introduce 
\textit{data propositions}, a novel feature in \lh
that enables the encoding of derivation trees for inductively
defined judgments as refined data types, by first reifying
the propositions and evidence as plain Haskell data, and then
using refinements to connect the two.
%
Hence, data propositions let us write plain Haskell functions
over refined data to provide explicit, 
constructive proofs (\S~\ref{sec:data-props}).              % TOFIX
%
Without data propositions reasoning about
potentially non-terminating computations was not possible in \lh,
thereby precluding even simple
metatheoretic developments such as the soundness
of \sysf let alone \sysrf.

\mypara{4. Mechanization}                   % Mechanization
%
\Cref{ch:implementation} describes another contribution:
we mechanized the metatheory
of \sysrf \emph{twice}: using \lh and \coq.
%
We formalized \sysrf in \lh (\S~\ref{sec:implementation:lh}) % TOFIX
to evaluate the feasibility of such
substantial metatheoretical formalizations.
%Our \lh formalization uses data propositions and recursive
%Haskell functions on derivation
%trees to produce explicit witnesses that correspond
%to proofs of our soundness theorems~\cite{Vazou18}.
%
Our proof is non-trivial, requiring 9,400 lines
of code, 30 minutes to verify, and various modifications
in the  internals of \lh.
%
We translated the same proof 
to \coq (\S~\ref{sec:implementation:coq})             % TOFIX
to compare the two alternatives.
%
Certain definitions, concretely the type denotations,
not admissible by \lh's positivity checker,
were possible to define in \coq using Equations~\cite{10.1145/3341690}.
%
Further, the \coq development is
much faster (about 60 seconds to verify),
but it is also more difficult to manipulate various
partial and mutual recursive definitions of the formalization.
%
Finally, \coq comes with stronger foundational
soundness guarantees than \lh.
%
While the metatheory of \coq is well studied,
\sysrf lays the foundation for the mechanized metatheory of \lh.

%Although it was possible, and perhaps simpler,
%to re-implement our mechanization with a
%specialized proof assistant and tactics,
%and although a fully verified \coq mechanization
%provides a high level of confidence in the
%correctness and soundness of our proof,
%to the community at large,
%our contribution here is to show that a large
%\coq development is also feasible
%purely as a (refined) Haskell program with
%a similar proof structure.
%
%Indeed, we show that substantial meta-theoretical
%formalizations over arbitrary computations
%are feasible via data propositions via
%\lh-style refinement typing (\S~\ref{sec:implementation}).

\mypara{5. Data Types}                     % Lists
%
Our next contribution,
described in \Cref{ch:lists},
is to add data types to \sysrf in the form of
lists that are equipped with a length measure.
%
For instance, the programmer can write a safe polymorphic 
tail function that requires its input to be a non-empty
list (length at least one).
%
The addition of lists to our language, which we denote
by \sysrfd,
adds new complexity to our metatheory.
%
A case switch operator enables path-sensitive reasoning when
destructing lists and leads to challenging new cases for our
soundness theorems.

\mypara{6. Bidirectional Typing Algorithm} % Bidirectional Typing
%
Our final contribution,
presented in \Cref{ch:bidirectional},
is to present a bidirectional typing algortihm, which emits
\emph{verification conditions (VCs)}. If we restrict the syntax
of our refinement predicates to a decidable logic, then these VCs 
can be checked or rejected by an SMT solver. Thus, like \lh,
\sysrfd can be decidably typechecked without placing severe 
restrictions on the user. 


% Section 1.x, Related Work 
\section{Related Work}
\label{sec:related}

We discuss the most closely related work
on the metatheory of unrefined and refined
type systems.


% \subsection{Metatheory for Refined Systems}

\mypara{Hybrid \& Contract Systems}
%
\citet{flanagan06} formalizes on paper
a monomorphic lambda calculus with
refinement types that differs from our \sysrf
in two ways.
%
%Our soundness formalization follows
First, in \citet{flanagan06}'s type checking
is hybrid: the developed system
is undecidable and inserts runtime
casts when subtyping cannot
be statically decided.
%
Second, the original system lacks polymorphism.
%
\citet{SekiyamaIG17} extended hybrid types
with polymorphism, but unlike \sysrf,
their system does not support semantic subtyping.
%
For example, consider a divide by zero-error.
The refined types for \ha{div} and 0 could
be given by
$\mathsf{div} :: \funcftype{\tint}{\breft{\tint}{n}{n \neq 0} \rightarrow {\tint}}$
and $0 :: \breft{\tint}{n}{n=0}$.
This system will compile \ha{div 1 0}
by inserting a cast on 0:
$\langle\breft{\tint}{n}{n=0} \Rightarrow \breft{\tint}{n}{n\neq 0}\rangle$,
causing a definite runtime failure
that could have easily been prevented
statically.
%
Having removed semantic subtyping,
the metatheory of~\citet{SekiyamaIG17}
is highly simplified.
%
% \mypara{Decidable Systems}
%
Static refinement type systems
(as summarized by~\citet{sprite})
usually restrict the definition
of predicates to quantifier-free
first-order formulae that can be
\emph{decided} by SMT solvers.
%
This restriction is not preserved
by evaluation that can substitute
variables with any value, thus
allowing expressions that cannot
be encoded in decidable logics, like
lambdas, to seep into the predicates of types.
%
In contrast, we allow predicates to be
any language term (including lambdas)
to prove soundness via preservation
and progress: our meta-theoretical
results trivially apply to systems that,
for efficiency of implementation, restrict
their source languages.
%
Finally, none of the above systems (hybrid,
contracts or static refinement types) come
with a machine checked soundness proof.

% \newtext{
  \mypara{Semantic Subtyping}
  %
  Semantic subtyping is not a unique feature of refinement types. 
  For example, \citet{Castagna02} use the set theoretic models of types 
  to decide subtyping. 
  \citet{Castagna05} present an algorithm that decides semantic subtyping 
  for a core calculus with functional types. 
  Like \sysrf, \citet{Castagna05} introduce a denotational interpretation of types 
  to break the circularity between the typing and subtyping relations.
  Unlike \sysrf, their system does not have polymorphism and, crucially, has no notion of dependency
  (no refinement type-style binder of arguments).
  %
  Moreover, their subtyping algorithm is different than our refinement based algorithm:
  it is neither type directed nor efficient (\ie it requires backtracking), and
  cannot be automated by an external SMT solver. 
% }

\mypara{Mechanizations of Refinement Types}
%
%Our soundness formalization follows
%the axiomatized implication relation
%of \citet{LehmannTanter} that decides
%subtyping (our rule \sBase) without
%formally connecting implication and
%expression evaluation.
%
\citet{LehmannTanter}'s \coq formalization
of a monomorphic, refined calculus
differs from \sysrf
in two ways.
%
First, their axiomatized implication, 
which is similar to our implication interface,
allows them to 
restrict the language of refinements
to decidable logics but provides no formal connection 
between subtyping and evaluation.
%
Instead, we also provide the denotational 
implementation of the implication interface, 
thus establish denotation soundness.
%We allow refinements to be arbitrary
%program terms and intend, in the future,
%to connect our axioms to SMT solvers
%or other oracles.
%
Second, \sysrf includes polymorphism,
existentials, and selfification
which are critical for % path- and
context-sensitive refinement typing,
but make the metatheory more challenging.
%
% \mypara{System FR}
%
\citet{kuncak-stainless} present System FR,
a polymorphic, refined language
with a mechanized metatheory
of about 30K lines of \coq.
%
Compared to our system, their notion
of subtyping is not semantic, but relies on
a reducibility relation.
%
For example, even though System FR
will deduce that \tpos is a subtype of
\tint, it will fail to derive that
$\tint \rightarrow \tpos$ is subtype
of  $\tpos \rightarrow \tint$
as reduction-based subtyping
cannot reason about contra-variance.
%
Because of this more restrictive notion
of subtyping, their mechanization
requires neither the indirection of
denotational soundness nor 
an implication proving oracle.
%
Further, System FR's support
for polymorphism is limited
in that it disallows refinements
on type variables, thereby
precluding many practically
useful specifications. %\RJ{check!}
%
% \newtext{
  Recently, \citet{10.1145/3546196.3550162}
  formalized a 
  refinement type system 
  as an embedding of refinement types in Agda.
  This system is verified 
  in a few thousand lines of Agda.
  %
  This formalism differs significantly from ours in that
  as an embedding it is built on top of a rich theorem prover 
  and cannot be used to refine some existing programming language.
  %
  Further, 
  it does not support higher-order
  functions, polymorphism, 
  semantic subtyping, neither be automated by an external solver 
  since soundness reduces to Agda's soundness.
% }
%
Finally, \citet{explicit} mechanize refinement types
with explicit proof terms in 15K lines of \lean code. 
They use a categorical, denotational semantics 
soundness statement, but their calculus 
by design supports neither semantic subtyping
nor polymorphism. 

\mypara{Metatheory in \lh}
\texttt{LWeb}~\cite{lweb} also used \lh 
to prove metatheory, %concretely 
the non-interference of $\lambda_{\text{LWeb}}$,
a core calculus that extends the LIO formalism with database access. 
The \texttt{LWeb} proof did not use refined data propositions, 
which were not present at development time,
%the time of its development, 
and thus it has two major weaknesses compared to our 
present development. 
First, \texttt{LWeb}
\textit{assumes} termination of $\lambda_{\text{LWeb}}$'s evaluation function; 
without refined data propositions metatheory can be developed only over 
terminating functions. 
This was not a critical limitation since non-interference was only 
proved for terminating programs. 
However, in our proof the requirement that evaluation of \sysrf terminates 
would be too strict. 
In our encoding with refined data propositions
such an assumption was not required. 
Second, the \texttt{LWeb} development is not constructive: 
the structure of an assumed evaluation tree is logically inspected 
instead of the more natural case splitting permitted only with 
refined data propositions. 
This constructive way to develop metatheories is more compact 
(\eg there is no need to logically inspect 
%the structure of the 
derivation trees) and akin to the standard 
meta-theoretic developments of constructive tools like 
\coq and \isabelle. 


% Ch 2: Refinement Types
\chapter{Refinement Types}
\label{ch:overview}
We start with an informal overview of
the usefulness of refinement types and of our
refined core calculus \sysrf,
which we later present formally (\Cref{ch:language})
and prove sound (\Cref{ch:soundness}).
%
Concretely,
we present the goals of refinement types (\S~\ref{sec:overview:goal})
and how they are achieved via three essential features:
semantic subtyping, existential types, and polymorphism
(\S~\ref{sec:overview:essence}).
We explain how the typing judgements are designed to accommodate these features
(\S~\ref{sec:overview:design})
and how we addressed the challenges these features impose in the mechanization
of the soundness proof (\S~\ref{sec:overview:soundness}).
Our examples here are presented with the syntax of \lh,
but can be encoded in \sysrf.

\section{The goal of Refinement Types} 
\label{sec:overview:goal}

Refinement types refine the types of an existing programming language
with logical predicates to define abstractions not expressible by the
underlying type system, which can then be used for static
%
(1) error prevention and (2) functional correctness.


\begin{figure}[t]
% \begin{code}
%   type Array a = Int -> a
%   {-@ type ArrayN a N = {i:Nat | i < N} -> a @-}

%   {-@ new :: n:Nat -> a -> ArrayN a n @-}
%   new :: Int -> a -> Array a
%   new n x = \i -> if 0 <= i && i < n then x else error "Out of Bounds"

%   {-@ set :: n:Nat -> i:{Nat | i < n} -> a -> ArrayN a n -> ArrayN a n @-}
%   set :: Int -> Int -> a -> Array a  -> Array a
%   set n i x a = \j -> if i == j then x else a j

%   {-@ get :: n:Nat -> i:{Nat | i < n} -> ArrayN a n -> a @-}
%   get :: Int -> Int -> Array a -> a
%   get n i a = a i
% \end{code}
\begin{code}
  type ArrayN a N = {i:Nat | i < N} -> a

  new :: n:Nat -> a -> ArrayN a n
  new n x = \i -> if 0 <= i && i < n then x else error "Out of Bounds"

  set :: n:Nat -> i:{Nat | i < n} -> a -> ArrayN a n -> ArrayN a n
  set n i x a = \j -> if i == j then x else a j

  get :: n:Nat -> i:{Nat | i < n} -> ArrayN a n -> a
  get n i a = a i
\end{code}
\vspace{0.0cm}
\caption{Functional Arrays with refinement types that ensure safe indexing.}
\vspace{0.0cm}
\label{fig:array}
\end{figure}


\mypara{Error Prevention}
Figure \ref{fig:array} presents the interface of a fixed size array
that is encoded in the core calculus \sysrf as a function.
%
The function @new n x@ returns an array that contains
@x@ when indexed with an integer between @0@ and @n@
and otherwise throws an ``out of bounds'' error.
To statically ensure that this error will never occur,
@new@ returns the refined array @ArrayN a n@, \ie
a function whose domain is restricted to integers less than @n@.
%
The @set@ and @get@ operators manipulate the refined arrays
on the index @i:{Nat | i < n}@, \ie refined to be in-bounds of the array.
%
With this refined interface, out-of-bounds indexing is statically ruled out:
\begin{code}
  array10 :: ArrayN Int 10
  array10 = new 10 0

  good = get 10 4  array10 -- OK
  bad  = get 10 42 array10 -- Refinement Type Error
\end{code}

\mypara{Functional Correctness}
\label{sec:overview:primes}
%
Refinement types are also used to ensure that the program has the intended behavior.
To achieve this, we use uninterpreted functions to \textit{specify} behaviors
and rely on the type system to propagate them.
For example, below using the uninterpreted function @isPrime@ we \textit{specify}
that some integers are primes, as denoted by the \emph{uninterpreted} predicate @isPrime@.
%
\begin{code}
  measure isPrime :: Int -> Bool
  type Prime = {v:Int | isPrime v}
\end{code}
%
Refinement types are not ideally suited to verifying properties
like primality checking, which requires reasoning beyond SMT
decidable fragments.
%
However, \emph{assuming} that a function establishes primality,
refinements can be used to easily track and propagate the invariant:
%
\begin{code}
  assume checkPrime :: x:Int -> {v:Bool | v <=> isPrime x}

  nextPrime :: Nat -> Prime
  nextPrime x = if checkPrime x then x else nextPrime (x+1)
\end{code}
%  nextPrime x = if checkPrime x then x else nextPrime (x+1)
% fix f = f (fix f)
%
The path-sensitivity of refinement types (Rule~\tIf of \Cref{fig:typing})
ensures that the function @nextPrime@ returns only
values that pass the primality check.

\mypara{Note on recursion}
%
Our core calculus does not explicitly support recursion.
But it can be extended with primitive constants (as long as they satisfy 
the consistency condition in ~\Cref{lem:prim-typing} below).
So, to encode inductive definitions, like @nextPrime@ in our system, we
use the fixpoint constant @fix@:
\begin{mcode}
  fix :: (a -> a) -> a
  nextPrime = fix @(Nat->Prime) (\f x->if checkPrime x then x else f (x+1))
\end{mcode}
Importantly, our calculus is fully polymorphic, in the sense that
type variables can be instantiated with refined types.
So, the type variable of @fix@ can be instantiated with
the refined type @Nat ->Prime@ to get the desired type of @nextPrime@.
Here, for emphasis, we make this instantiation explicit,
but in real systems, like
\href{http://goto.ucsd.edu:8090/index.html#?demo=permalink%2F1687971696_2572.hs}{\lh},
the refined type application is inferred.

\mypara{Primes Array Example}
%
As a bigger example, consider an example where refinements are used for both
error prevention and functional correctness.
%
The function @primes n@ generates an array with the first @n@ prime numbers:
%
\begin{code}
  primes :: n:Nat -> ArrayN Prime n
  primes n = (fix go) 1 0 (new n (nextPrime 1))
    where go f i p acc = if i < n
                               then let p' = nextPrime (p+1) in
                                      go f (i+1) p' (set n i p' acc)
                               else acc
\end{code}
%
Since @primes@ typechecks under the safe array interface of \Cref{fig:array},
no out-of-bounds errors will occur.
%
At the same time, all elements of the array are @set@ by a result @nextPrime@
and thus @primes@ returns an array of prime numbers.

\section{The Essence of Refinement Types} \label{sec:overview:essence}
%
The practicality of refinement types, as illustrated in the examples above,
is due to the combination of three essential features:
%
\begin{enumerate}[leftmargin=*]

\item \textbf{\textit{Semantic Subtyping:}}
The user does not need to provide any explicit type casts, because
subtyping is implicit and semantic.
%
For example, to type check @get 10 4 array10@ (from~\S~\ref{sec:overview:goal}),
the type of @4 :: {v:Int | v == 4}@ is implicitly converted to @{v:Int | 0 <= v < 10}@

\item \textbf{\textit{Decidability:}}
%
The semantic casts are reduced to logical implications checked by an SMT solver.
Refinement types are designed to generate decidable logical implications,
to ensure predictable verification and also permit type inference~\cite{LiquidPLDI08}
that makes verification practical, \eg the @primes@ definition requires zero annotations.

\item \textbf{\textit{Polymorphism:}}
Polymorphism on refinement types permits instantiation of type variables with any refined type.
For example, the same array interface can be used to describe primes,
functions with positive domains, and any other concept encoded as a refinement type.
\end{enumerate}

\section{The Design of Refinement Types}
\label{sec:overview:design}

Next, we develop a minimal calculus \sysrf that shows how
Refinement type systems enjoy the three essential features
of \S~\ref{subsec:overview:essence}.
%
\sysrf has four judgements that relate expressions ($e$),
types ($t$), kinds ($k$), predicates ($p$), and environments ($\tcenv$):
%
(1)~typing (\hastype{\tcenv}{e}{t}),
%
(2)~subtyping (\isSubType{\tcenv}{t_1}{t_2}),
%
(3)~well-formedness (\isWellFormed{\tcenv}{\stype}{\skind}),
and
(4)~implication checking (\imply{\tcenv}{p_1}{p_2}).
%
In \S~\ref{sec:lang:static} we define the judgements in detail.
%
Here, we present the design decisions that ensure the three
essential features of refinement types.


\subsection{Semantic Subtyping}
%
\label{subsec:overview:subtyping}
Refinement types rely on implicit semantic subtyping,
that is, type conversion (from subtypes) happens without
any explicit casts and is checked semantically via logical
validity.
For example, in the application @get 10 4 array10@ (of Fig.~\ref{fig:array}),
the type of @4@ was implicitly converted.
%
To see how, consider an environment $\tcenv$ that contains
the array interface.
%
Let
%
$
    \tcenv \subseteq \{\texttt{get}:n:\tint \rightarrow i:\breft{\tint}{\vv}{ \vv < n} \rightarrow \texttt{ArrayN}\ a\ n \rightarrow a\}
$
%
For brevity, we ignore the requirement that $i$ and $n$ are natural numbers
and, as in Fig.~\ref{fig:array}, we use $\texttt{ArrayN}\ a\ n$ as shorthand for
$\breft{\tint}{\vv}{ \vv < n} \rightarrow a$.
%
The application $(\texttt{get}\ 10)\ 4$ will type check as below,
using the \tSub rule to implicitly convert the
type of the argument and the \sBase rule to
check that $4$ is a valid index
by checking the validity of the formula $\forall \vv.\ \vv = 4 \Rightarrow \vv < 10$.

$${\footnotesize
\inferrule%*[Right=\tApp]
{
    \inferrule%*[Right=\tVar]
    { \dots }{
    \hastype{\tcenv}{\texttt{get}\ 10}{{\color{olive}\breft{\tint}{\vv}{ \vv < 10}} \rightarrow
    \dots
    % \texttt{ArrayN}\ a\ 10 \rightarrow a
    }
    }
%    \qquad
%    \qquad
   \
   \inferrule%*[Right=\tSub]
    {
        \inferrule%*[Right=\tVar]
        { }
        {
            \hastype{\tcenv}{4}{ {\color{teal} \breft{\tint}{\vv}{ \vv = 4 }}}
        }% {\tPrim}
        \
        \inferrule%*[Right=\sBase]
        { \forall \vv. {\color{teal} \vv = 4} \Rightarrow  {\color{olive} \vv  < 10 }}{
              \isSubType{\tcenv}{{\color{teal}\breft{\tint}{\vv}{ \vv = 4 }}}{{\color{olive}\breft{\tint}{\vv}{ \vv < 10}}}
        }{\sBase}
     }{
        \hastype{\tcenv}{4}{{\color{olive}\breft{\tint}{\vv}{ \vv < 10}}}
    }{\tSub}
 }{
    \hastype{\tcenv}{\app{\texttt{get}\ 10}{\;\;4}}{\texttt{ArrayN}\ a\ 10 \rightarrow a}
}% {\tApp}
}
$$

Importantly, most refinement type systems
use syntax-directed rules to destruct
subtyping obligations into basic (semantic)
implications.
%
For example, in~\Cref{fig:subtyping}
the rule \sFunc states that functions
are covariant on the result and contravariant
on the arguments.
%
Thus, a refinement type system can,
without any %annotations or
casts,
decide that $a_{20} : \texttt{ArrayN}\ a\ 20$
is a suitable argument for the higher order function
$\texttt{get}\ 10\ 4 : \funcftype{\texttt{ArrayN}\ a\ 10}{a}$
and type check the expression $\texttt{get}\ 10\ 4\ a_{20}$.


\subsection{Decidability}
\label{subsec:overview:exists}
%
As illustrated in the previous type derivation,
refinement type checking essentially
generates a set of verification conditions (VCs)
whose validity implies type safety.
Importantly, the refinement type checking rules
are designed to generate VCs in the logical language 
used by the user-provided specifications.
%
In general, let $\mathcal{L}$ be a logical language that contains
equality and conjunction. If all the user-specified predicates 
belong to $\mathcal{L}$,
then the VCs will be in $\mathcal{L}$ as well.
In practice (\eg in Liquid Haskell~\cite{Seidel14} and Flux~\cite{Flux}), 
$\mathcal{L}$ is the
qualifier-free logic of equality, uninterpreted functions, 
and linear arithmetic (QF-EUFLIA).

To achieve this logical-language preservation, special care is taken
in type checking function declarations and applications.


\mypara{Function Declarations}
Function declarations are checked using the refinement type rule 
for let bindings (Rule~\tLet also in~\Cref{fig:typing}).
%(We note that recursive definitions are encoded using a
%fixpoint primitive that we assume satisfies~\Cref{lem:prim-typing}
%and do not require a special rule.)
$$
\inferrule
{\hastype{\tcenv}{\sexpr_f}{\stype_f} \\
{\isWellFormed{\tcenv}{\stype}{\skind}}\\
\hastype{\bind{f}{\stype_f},\tcenv}{{\sexpr}}
     {\stype}
 }
{\hastype{\tcenv}{\eletin{f}{\sexpr_f}{\sexpr}}{\stype}}
{\tLet}
$$
The type checking must infer the type $\stype_f$ of the function,
but that could be user-annotated
(\eg $\sexpr_f$ could be $\tyann{\sexpr_f'}{\stype_f}$).

Importantly, the body $e$ is checked without knowledge of the definition of $f$.
%
The exact encoding of the body of the function definitions
(for example, as done in Dafny~\cite{Dafny} or Prusti~\cite{Prusti})
requires the use of $\forall$-quantifiers in the SMT solver,
thus potentially leading to undecidability.
%
Instead, refinement types only use the refinement
type specifications of functions, providing a fast
but incomplete verification technique.
%
For example, given only the specifications of @get@ and @set@,
and not their exact definitions, it is \emph{not} possible to show
that @get@ after @set@ returns the value that was set.

\begin{code}
  getSet :: n:Int -> i:{Nat|i<n} -> x:a -> ArrayN a n -> {v:a|x == v}
  getSet n i x a = get n i (set n i x a) -- Refinement Type Error
\end{code}

\mypara{Function Application}
% \NV{this is mostly as before, can we instead use an array related example?}
For decidable type checking, refinement types use an existential type~\cite{Knowles09} to check
dependent function application, \ie the \rulename{TApp-Exists} rule below, instead
of the standard type-theoretic \rulename{TApp-Exact} rule.
%
$$
{
\inferrule%*[Right=\rulename{TApp-Exact}]
{
    \hastype{\tcenv}{f}{\functype{x}{\stype_x}{t}} \qquad \hastype{\tcenv}{e}{\stype_x}
 }{
    \hastype{\tcenv}{\app{f}{e}}{\stype\subst{}{x}{e}}
}{\rulename{TApp-Exact}} \qquad\quad
\inferrule%*[Right=\rulename{TApp-Exists}]
{
    \hastype{\tcenv}{f}{\functype{x}{\stype_x}{t}} \qquad \hastype{\tcenv}{e}{\stype_x}
 }{
    \hastype{\tcenv}{\app{f}{e}}{\existype{x}{\stype_x}{\stype}}
}{\rulename{TApp-Exists}}
}
$$

To understand the difference, consider
some expression $e$ of type $\tpos$ and the identity function $f$
%
\begin{align*}
    e&:\tpos & f&:\functype{x}{\tint}{\breft{\tint}{v}{v = x}}
\end{align*}
%
The application $f\ e$ is typed as $\breft{\tint}{v}{v = e}$ with the \rulename{TApp-Exact} rule,
which has two problems.
%
First, the information that $e$ is positive is lost.
%
To regain this information the system needs to re-analyze the expression $e$
breaking compositional reasoning.
%
Second, the arbitrary expression $e$ enters the refinement logic
potentially breaking decidability.
%
Using the \rulename{TApp-Exists} rule, both of these problems are addressed.
Typing first uses subtyping on $f$ to track the actual type of the argument,
thus weakening the type of $f$ to $f:\functype{x}{\tpos}{\breft{\tint}{v}{v = x}}$.
%
With this, the type of $f\ e$ becomes $\existype{x}{\tpos}{\breft{\tint}{v}{v = x}}$
preserving the information that the application argument is positive,
while the variable $x$ cannot break any carefully crafted decidability guarantees.

\citet{Knowles09} introduce the existential application rule
and show that it preserves the decidability and completeness of the refinement type system.
%
An alternative approach for decidable and compositional type checking
is to ensure that all the application arguments
are variables by ANF transforming the original
program~\cite{Flanagan93}.
%
ANF is more amicable to \emph{implementation}
as it does not require the definition of one
more type form.
%
However, ANF is more problematic for the
\emph{metatheory}, as ANF is not preserved
by evaluation.
%
Additionally, existentials let us \emph{synthesize}
types for let-binders in a bidirectional style: when
typing $\eletin{x}{e_1}{e_2}$, the existential lets
us eliminate $x$ from the type synthesized for
$e_2$, yielding a precise, algorithmic
system \cite{CosmanICFP17}.
%
Thus, we choose to use existential types in \sysrf.


\subsection{Polymorphism}
\label{subsec:overview:polymorphism}

Polymorphism is a precious type
abstraction~\cite{10.1145/99370.99404},
but combined with refinements, it can lead to
imprecise or, worse, unsound systems.
%
As an example, below we present the
function \ha{max} with four potential
type signatures.
%
$$\begin{array}{rrrcl}
& \text{Definition} &  \texttt{max} &  = & \lambda x\ y . \texttt{if } x < y \texttt{ then } y \texttt{ else } x \\[0.05in]
\text{Attempt 1:}& \textit{Monomorphism} & \texttt{max} & :: &
\functype{x}{\tint}{\functype{y}{\tint}{\breft{\tint}{\vv}{x \leq \vv \wedge y \leq \vv}}}\\
\text{Attempt 2:}& \textit{Unrefined Polymorphism} & \texttt{max} &  :: &
\functype{x}{\al}{\functype{y}{\al}{\al}}\\
\text{Attempt 3:}& \textit{Refined Polymorphism} & \texttt{max} &  :: &
\functype{x}{\al}{\functype{y}{\al}{\breft{\al}{\vv}{x \leq \vv \wedge y \leq \vv}}}\\
\sysrf\text{:} & \textit{Kinded Polymorphism} & \texttt{max} &  :: &
\polytype{\al}{\skbase}\functype{x}{\al}{\functype{y}{\al}{\breft{\al}{\vv}{x \leq \vv \wedge y \leq \vv}}}
\end{array}$$
As a first attempt, we give \ha{max} a monomorphic type,
stating that the result of  \ha{max} is an integer greater than
or equal to each of its arguments.
This type is insufficient because it forgets any information known for
\ha{max}'s arguments. For example, if both arguments are positive,
the system cannot decide that \ha{max x y} is also positive.
%
To preserve the argument information we give \ha{max} a polymorphic
type, as a second attempt. Now the system can deduce that
\ha{max x y} is positive, but forgets that it is also
greater than or equal to both \ha{x} and \ha{y}.
In a third attempt, we naively combine the benefits of
polymorphism with refinements to give \ha{max} a very precise type
that is sufficient to propagate the arguments' properties (positivity)
and \ha{max} behavior (inequality).

Unfortunately, refinements on arbitrary type
variables are dangerous for two reasons.
%
First, the type of \ha{max} implies
that the system allows comparison
of any values (including functions).
%
Second, if refinements on type variables
are allowed, then, for soundness~\cite{Belo11},
all the types that substitute variables should be refined.
%
For example, as detailed in~\S 6~of~\cite{sprite}, if a type variable
is refined with \tfalse (\ie $\breft{\al}{\vv}{\tfalse}$)
and gets instantiated with an unrefined function
type ($\functype{x}{t_x}{t}$),
then the \tfalse refinement
is lost and the system becomes unsound.

\mypara{Base Kind when Refined}
%
To preserve the benefits of refinements
on type variables, without the complications
of refining function types, we introduce
a kind system that separates the type
variables that can be refined from the
ones that cannot.
%
To do so, we extend the standard well-formedness rule of refinement types
to also perform kind checking (\isWellFormed{\tcenv}{\stype}{\skind}).
%
Variables with the base kind $\skbase$
can be refined, compared, and only
substituted by base, refined types.
%
The other type variables have kind $\skstar$
and can only be trivially refined with \ttrue.
With this kind system, we have a simple
and convenient way to encode comparable values
and we can give \ha{max}
a polymorphic and precise type that
naturally rejects non-comparable
(\eg function) arguments.


This simple kind system could be further stratified,
\ie if some base types did not support comparison,
and it could be implemented via typeclass constraints,
if our system contained data types.
%% NEW sentence:
A first step towards data types is presented in Chapter
\Cref{ch:lists}, where we add polymorphic lists to \sysrf,
which can be refined (for instance, to restrict the length
of the list), but cannot be compared like base types. This 
latter restriction is needed because lists themselves can
contain incomparable terms like functions.
%%


\section{The Soundness of Refinement Types}
\label{sec:overview:soundness}

In this work we establish two soundness theorems for refinement types
that precisely relate typing judgments $\hastype{\tcenv}{e}{t}$
with the high-level goals of error prevention (type safety) and
functional correctness (denotational soundness).

\mypara{1. Type Safety} ensures that well-typed programs 
do not get stuck at runtime.
It says that  if an expression has a type ($\hastype{\emptyset}{e}{t}$)
and evaluates to another expression ($\evalsTo{\sexpr}{\sexpr'}$),
then either evaluation reached a value 
or it can take another step (${\sexpr'}\step{\sexpr''}$).
In \sysrf, we use the primitive \eerr to denote program errors 
(such as out-of-bounds indexing of~\Cref{fig:array}).
The \eerr primitive neither is a value nor takes a step.
Thus, if an expression type checks, via type safety, 
we know that \eerr will not be reached at runtime.
\Cref{lem:soundness} formally defines type safety 
and it is proved via the preservation and progress lemmas.
%
Type safety ensures that programs will not get stuck, but does not ensure that they
satisfy their functional specifications. This is ensured by the second soundness theorem.
\NV{But how to we know that evaluation preserves semantics? Revisit after we add the denotations}
\NV{CHECK RJ's comment}

\mypara{2. Denotational Soundness} states that if an expression has a type ($\hastype{\emptyset}{e}{t}$),
then it belongs in the denotations of this type ($e \in \denote{t}$).
For example, the denotation of the type @{i:Nat | i <= 42}@ is the set of integers
between 0 and 42.
%
In~\S~\ref{sec:lang:static} we inductively define the denotations of each type and
\Cref{lem:denote-sound-first} formally encodes denotational soundness.
%% NEW 
Combining Lemmas \Cref{lem:soundness} and \Cref{lem:denote-sound-first},
we see that the a refined \sysrf type that can be checked for a specific program
is preserved under evaluation, and thus the semantics of the program is 
preserved under evaluation.
%%

This dissertation, for the first time, mechanizes the soundness of refinement types with
semantic subtyping, existential types, and polymorphism.
This mechanization turned out to be challenging for three main reasons:

\tikzstyle{block} = [rectangle, draw,
    text width=6em, text centered, rounded corners, minimum height=3em]
\begin{figure}
    \begin{tikzpicture}[
        colored/.style={fill=#1!20,draw=#1!50!black!50}
      ]
        \node [block] (init) (T){\hastype{\tcenv}{e}{t}\\ {\tiny TYPING}};
        \node [block] (S) [right=1.5cm  of T]{\isSubType{\tcenv}{t}{t} {\tiny SUBTYPING}};
        \node [block] (W) [left=1.5cm  of T]{\isWellFormed{\tcenv}{\stype}{\skind} {\tiny WELL-FORMEDNESS}};
        \node [block] (I) [below=1cm  of T]{\imply{\tcenv}{p}{p} {\tiny IMPLICATION}};
        \draw[-latex]
           ([yshift=-3mm]T.north east) -- node[right,above]{1}([yshift=-3mm]S.north west);
        \draw[-latex]
           ([yshift=+3mm]S.south west) -- node[right,below]{2}([yshift=+3mm]T.south east);
        \draw[-latex]
           ([yshift=-3mm]T.north west) -- node[above]{3}([yshift=-3mm]W.north east);
        \draw[-latex, dashed, blue]
           ([yshift=+3mm]W.south east) -- node[below]{4}([yshift=+3mm]T.south west);
        \draw[-latex, dashed, red]
           (I) -- node[right]{6} (T);
        \draw[-latex]
           (S) |- node[midway,left, yshift=1.7ex, xshift=-7ex]{5}  (I);
    \end{tikzpicture}
\caption{Dependencies of Typing Judgements in Refinement Types. (Dashed lines do not exist in our formalism.) }
\label{fig:dependencies}
\end{figure}

\mypara{Challenge 1: Circularities}
%
\Cref{fig:dependencies} presents the dependencies of the four typing judgements in refinement types.
As we saw in the example of~\S~\ref{subsec:overview:subtyping} 
(and can be confirmed in the rules defined in~\S~\ref{sec:lang:static}),
typing depends on subtyping (arrow 1) which in turn depends on implication checking (arrow 5).
Subtyping depends on typing (arrow 2; because of rule \sWitn of \Cref{fig:subtyping}),
so typing and subtyping have a circular dependency we cannot break.
%
Typing also depends on well-formedness (arrow 3) that checks that
types, especially the ones inferred by the system, are well-formed:
all the variables appearing in the refinements are bound in the type environment
and refinements are of boolean type.
To check the type of the refinements the system could use typing
thus introducing one more dependency (arrow 4) and yet another circle.
We break this dependency by using an unrefined calculus (system \sysf)
that erases refinements, to check that refinements are well-typed booleans.
%
The final potential circle is introduced when implication depends on typing (arrow 6).
In~\S~\ref{sec:typing:implication:denotational} we define implication via type denotations,
but as observed by~\citet{Greenberg13}, in this case, special care should be taken
so that the system is monotonic and thus well-defined.
To avoid this dangerous circularity we again use typing of \sysf (and not \sysrf) to define
denotations and thus implication.

% \newtext{
In summary, circularities in typing judgements are problematic for two reasons:
\begin{enumerate}[leftmargin=*]
\item Circularities increase the complexity of proof mechanization. 
Concretely, because typing and subtyping have a circular dependency, 
the metatheoretical lemmas (substitution, weakening, narrowing, \etc) 
require versions for both typing and subtyping, which are proved by mutual induction. 
If well-formedness was also included in this circularity (arrow 4),
 then the complexity of the proofs would greatly increase, 
 but would not necessarily be impossible.
\item Second, circularities are problematic because they can lead to non well-defined systems. 
Concretely, \citet{Greenberg13} describes an older refinement type system 
in which typing appeared in the left hand side of subtyping and, as such, 
it was non-monotonic and thus not well-defined. 
This situation corresponds to the red arrow 6 in \cref{fig:dependencies}, 
which would make the proof impossible due to the typing judgment occurring in 
a negative position in the implication judgment.  
\end{enumerate}
% }


\mypara{Challenge 2: Implications}
%
The second mechanization challenge was the encoding of implication.
In the bibliography of refinement types, implication has been defined in three ways:
\begin{enumerate}[leftmargin=*]
\item Using \emph{denotations} (of types as sets of terms) defined via operational semantics~\cite{Vazou18, flanagan06}.
This encoding is more convenient when proving the soundness of the system, since implication
and thus subtyping and typing, directly connect with operational semantics, making the proof of soundness more direct.
However, the implementation of this encoding of implication is not realistic, since it is not decidable.
\item Using \emph{logical implication}~\cite{LiquidPLDI08, Gordon2010PrinciplesAA}.
The encoding of the implication as a logical implication is the closest to the implementation of a refinement system,
where an SMT is used to check logical implications. Yet, to prove soundness, a claim should be made that
logical implication checked by the SMT correctly approximates the runtime semantics of the system
(\ie presented in rule~\iLog of~\S~\ref{sec:typing:implication:logical}) which has never been mechanized.
\item By \emph{axiomatization}~\cite{LehmannTanter}.
A final approach is to leave the implication uninterpreted and axiomatize it with all the properties required to prove soundness.
This approach is the easiest to mechanize, but it is dangerous, since in the past the axioms assumed for implication
were inconsistent, thus soundness was ``proved with flawed premises'' (as quoted from Table 1 of~\cite{SekiyamaIG17}).
\end{enumerate}
Our mechanization follows a combination of the first and the third approach.
We specify the interface of implication (via \Cref{lem:implication} of~\S~\ref{sec:typing:implication:interface} which is encoded as
an inductive data type in the proof mechanization)
to articulate the exact properties required by the soundness proof.
Then, in~\S~\ref{sec:typing:implication:denotational}, we implement the implication interface using
the denotational semantics of the system.
This encoding has two major benefits.
First, the denotational implementation ensures that our interface is consistent.
Second, the development of the interface leaves room for the implementation
of alternative implication ``oracles'', \eg closer to SMT solvers.
Even though we did not mechanize this alternative implementation, in~\S~\ref{sec:typing:implication:logical} we present how
logical implications are derived from the implication judgement.


\mypara{Challenge 3: Proof Complexity}
%
All the three essential features of refinement types add complexity 
to the mechanization of the soundness proof.
Polymorphism requires the extension of well-formedness to kind checking.
%
Semantic subtyping makes type checking not syntax-directed
(thus inversion is not trivial, cf.~\S~\ref{sec:soundness:inversion}) and dependent upon subtyping.
%
In turn, the existential types required for decidability make subtyping dependent upon type checking.
%
Due to this mutual dependency, the standard metatheoretical lemmas (substitution, weakening, narrowing, \etc)
require versions for both typing and subtyping, which are proved by mutual induction.
Thus, the combination of the three essential for refinement types features
makes the metatheoretical development more complex and prone to unsoundness.
%
Once, we have carefully broken the various circularities
and eliminated potential sources of unsoundness, we get
unsurprising, albeit strenuous, proofs of the soundness
of refinement typing.


% Ch 3, 3.1-3.2: System F and RF
\chapter{The Languages \sysf and \sysrf}
\label{ch:language}

%For brevity, clarity and also 
To cut the circularities in the metatheory,
%(in rule \wtRefn in \S~\ref{sec:typing:wf}), 
we formalize 
refinements using two calculi. 
% 
The first is the \emph{base} language 
\sysf: a classic System F \cite{TAPL} 
with call-by-value semantics extended 
with primitive \tint and \tbool types 
and operations.
%
The second is the \emph{refined} language
\sysrf which extends \sysf with refinements.
%
By using the first calculus to express the typing judgments
for our refinements,
we avoid making 
the well-formedness (in rule \wtRefn in \S~\ref{sec:typing:wf})
and the implication (in type denotations of~\Cref{fig:den}) judgments
mutually dependent with the typing judgments. 
%
We use the $\greybox{\mbox{grey}}$ highlights 
for the extensions to  
$\sysf$ required % needed to support refinements 
for $\sysrf$.

\begin{conference}
\begin{figure}[t!]
  {\small
    \begin{tabular}{rrcll}
\emphbf{Primitives} 
  & \sconst & $\bnfdef$ & $\ttrue$ $\spmid$ $\tfalse$ $\spmid$  
  $0, 1, 2, \ldots$ 
  $\spmid$  $\wedge, \neg$ 
  $\spmid$  $\leq, \sconst\!\!\leq, =,  \sconst\!\!=$                             
  \\
\emphbf{Values}
  & \sval   & $\bnfdef$ & \sconst  
  $\spmid  x, y, \ldots
  \spmid   \vabs{x}{e}
  \spmid   \tabs{\alpha}{k}{e}
  $
  &\\
\emphbf{Terms}
  & \sexpr  & $\bnfdef$ & 
  $
  \sval \spmid \app{e_1}{e_2} \spmid \tyapp{e}{t} \spmid \eletin{x}{e_1}{e_2} 
  \spmid \tyann{e}{t} \spmid \eif{e}{e_1}{e_2} \spmid \eerr
  $ 
\end{tabular}
}
\caption{Syntax of Primitives, Values, and Expressions.}
\label{fig:syn:terms}
\vspace{-0.4cm}
\end{figure}
\end{conference}


\begin{fullversion}
\begin{figure}[t!]
%  \scalebox{0.80}{
    \begin{tabular}{rrcll}
\emphbf{Primitives} 
  & \sconst & $\bnfdef$ & $\ttrue$ $\spmid$ $\tfalse$ $\spmid$  $0, 1, 2, \ldots$    & \emph{booleans and integers} \\
%  &         & $\spmid$  & $0, 1, 2, \ldots$                     & \emph{integers} \\ 
  &         & $\spmid$  & $\wedge, \vee, \neg, \leftrightarrow$ & \emph{boolean ops.} \\
  &         & $\spmid$  & $\leq, =$                             & \emph{polymorphic comparisons} \\ [0.05in] 

\emphbf{Values}
  & \sval   & $\bnfdef$ & \sconst               & \emph{primitives} \\ 
  &         & $\spmid$  & x, y, \ldots          & \emph{variables} \\
  &         & $\spmid$  & \vabs{x}{e}           & \emph{abstractions} \\
  &         & $\spmid$  & \tabs{\alpha}{k}{e}   & \emph{type abstractions} \\[0.05in]

\emphbf{Terms}
  & \sexpr  & $\bnfdef$ & \sval                 & \emph{values} \\ 
  &         & $\spmid$  & \app{e_1}{e_2}        & \emph{applications} \\
  &         & $\spmid$  & \tyapp{e}{t}          & \emph{type applications} \\
  &         & $\spmid$  & \eletin{x}{e_1}{e_2}  & \emph{let-binders} \\
  &         & $\spmid$  & \tyann{e}{t}          & \emph{annotations} \\
  &         & $\spmid$  & \eif{e_0}{e_1}{e_2}   & \emph{conditionals} \\
  &         & $\spmid$  & \eerr                 & \emph{runtime errors} \\
\end{tabular}
%  }
  \vspace{-0.0cm}
  \caption{Syntax of Primitives, Values, and Expressions.}
\label{fig:syn:terms}
\vspace{-0.0cm}
\end{figure}
\end{fullversion}



\begin{figure}%[b!]
%{\small
  \begin{tabular}{rrcll}
  \emphbf{Kinds} 
    & \skind & $\bnfdef$ & $\skbase \spmid \skstar$ & \emph{base and star kind} \\[0.05in]
%    &        & $\spmid$  & \skstar & \emph{star kind} \\[0.05in]
  
  \emphbf{Predicates}  
    & \spred & $\bnfdef$ & $\greybox{\{ e \;|\; \exists\, \tcenv.\, \hasftype{\tcenv}{\sexpr}{\tbool} \}}$ & \greytextbox{\emph{boolean-typed terms}} 
    \\[0.05in] 
  
  \emphbf{Base Types} 
    & \sbase & $\bnfdef$ & $\tbool \spmid \tint \spmid \tvar$ 
    & \emph{bool, ints, and type variables} 
    \\[0.05in] 

  \emphbf{Types}
%  & \stype & $\bnfdef$ & $\sbase\greybox{\!\breft{}{\vv}{\spred}}$  
%             $\spmid$   $\greybox{x\!}\!\functype{}{t_x}{t}$ 
%             $\spmid$   \greybox{\existype{x}{t_x}{t}} 
%            $\spmid$   $\polytype{\tvar}{\skind}{t}$  & \\ [0.05in]    
% RESTORE IF THERE IS SPACE     
   & \stype & $\bnfdef$ & $\sbase\greybox{\!\breft{}{\vv}{\spred}}$  & \emph{\greytextbox{refined} base type} \\ 
   &        & $\spmid$  & $\greybox{x\!}\!\functype{}{t_x}{t}$         & \emph{function type}     \\        
   &        & $\spmid$  & \greybox{\existype{x}{t_x}{t}} & \greytextbox{\emph{existential type}}  \\        
   &        & $\spmid$  & $\polytype{\tvar}{\skind}{t}$  & \emph{polymorphic type}  \\ [0.05in]        

  \emphbf{Environments}
    & $\tcenv$ & $\bnfdef$ & 
    $\varnothing \spmid \tcenv, \bind{x}{t} \spmid \tcenv, \bind{\tvar}{\skind}$                  
    & \emph{variable and type bindings} 
  \end{tabular}
%}
\vspace{-0.0cm}
  \caption{Syntax of Types. % and Environments.
           The grey boxes are the extensions 
           to $\sysf$ needed by $\sysrf$.
           We use $\sftype$ for $\sysf$-only types.}
  \label{fig:syn:types}
  \label{fig:syn:reft}
  \label{fig:syn:env}
  \vspace{-0.00cm}
\end{figure}

\section{Syntax} \label{sec:lang:syntax}

We start by describing the syntax of terms and types 
in the two calculi.  

\mypara{Constants, Values and Terms}
%
\Cref{fig:syn:terms} summarizes the 
syntax of terms in both calculi.
%
%Terms are stratified into primitive \emph{constants} 
%and \emph{values}.
%
The \emph{primitives} $\sconst$
include \tint and \tbool constants, 
boolean operations, 
the polymorphic comparison and equality, 
and their curried versions.
%
%For clarity we will write both the 
%polymorphic equality and the monomorphic 
%equality for $\tint$ as $=$ in our refinements
%and elide the type application.
%\NV{Why do we need both int and polymorphic equality? 
%Isn't inequalities also polymorphic? This is what I assumed in the overview max example}
%
\emph{Values} $\sval$ are 
%those expressions 
%which cannot be evaluated any further, 
%including primitive 
constants, binders 
and $\lambda$- and type- abstractions.
%
Finally, the \emph{terms} $\sexpr$ comprise values,
value- and type- applications, let-binders, 
annotated expressions, conditionals, and runtime errors.
The types in annotations are, potentially wrong, specifications 
written by the user and checked by the type checker.  


\mypara{Kinds \& Types}       
%
\Cref{fig:syn:types} shows the syntax of the types,
with the grey boxes indicating the extensions to $\sysf$ 
required by $\sysrf$.
%
In \sysrf, only base types % \tbool, \tint, and \tvar
can be refined: we do not permit refinements 
for functions and polymorphic types. 
%
\sysrf enforces this restriction using two kinds
which denote types that may ($\skbase$) or may not ($\skstar$)
be refined.
%
The (unrefined) \emph{base} types $\sbase$ comprise 
$\tint$, $\tbool$, and type variables $\tvar$. 
%
The simplest type is of the form
$\breft{\sbase}{\vv}{\spred}$ 
comprising a base type $\sbase$ 
and a \emph{refinement} that restricts 
$\sbase$ to the subset of values 
$\vv$ that satisfy $\spred$ \ie 
for which $\spred$ evaluates to $\ttrue$.
%
We use refined base types to build up
dependent function types (where the input 
parameter $x$ can appear in the output type's 
refinement), existential and polymorphic 
types.
%
%Note that $\sbase$ by itself denotes the trivially 
%refined type $\breft{\sbase}{\vv}{\ttrue}$.
In the sequel, we write $\sbase$ to abbreviate 
$\breft{\sbase}{\vv}{\ttrue}$
%, particularly for type variables, 
and call types refined with only \ttrue 
``trivially refined'' types. 

\mypara{Refinement Erasure}
%
The reduction semantics of our polymorphic 
primitives are defined 
using an \emph{erasure} function that 
returns the unrefined, \sysf version of a refined \sysrf type:
\[
  \forgetreft{ \breft{\sbase}{\vv}{\spred} } \defeq \sbase, \quad 
  \forgetreft{ \functype{x}{t_x}{t} } \defeq \forgetreft{t_x} \rightarrow \forgetreft{t}, \quad
  \forgetreft{ \existype{x}{t_x}{t} } \defeq \forgetreft{t}, \quad{\rm and} \quad
  \forgetreft{ \polytype{\al}{k}{t} } \defeq \polytype{\al}{k}{\forgetreft{t}}
\]


\mypara{Environments}
%
\Cref{fig:syn:env} describes 
the syntax of typing environments 
$\tcenv$ which contain both term 
variables bound to types and type 
variables bound to kinds. 
%
These variables may appear in types 
bound later in the environment.
%
In our formalism, environments grow 
from right to left.
%

\mypara{Note on Variable Representation}
Our metatheory requires that all variables 
bound in the environment are distinct. 
%
Our mechanization enforces this invariant 
via the locally nameless representation~\cite{Aydemir05}: 
free and bound variables are distinct objects 
in the syntax, as are type and term variables.
%
All free variables have unique names which
never conflict with bound variables represented as 
de Bruijn indices. This eliminates
the possibility of capture in substitution
and the need to perform alpha-renaming during
substitution. 
The locally nameless representation avoids %the need for 
technical manipulations such as index shifting by using 
names instead of indices for  free variables 
(we discuss alternatives in~\S~\ref{sec:related}).
%
To simplify the presentation of the syntax and rules, we 
use names for bound variables to make the dependent nature
of the function arrow clear.

\section{Dynamic Semantics} \label{sec:lang:dynamic}
\begin{figure}
%  {\small
  \begin{mathpar} %%%%%%%%% SMALL-STEP SEMANTICS %%%%%%%%%%
    \judgementHead{Operational Semantics}{$\sexpr \step \sexpr'$}\\
        \inferrule%*[Right=\ePrim]
        {  }
        {\app{\sconst}{\sval} \step \delta(\sconst,\sval)}
        {\ePrim} 
        \quad
        \inferrule%*[Right=\eTPrim]
        {  }
        {\tyapp{\sconst}{\stype} \step \delta_T(\sconst,\forgetreft{\stype})}
        {\eTPrim} 
        \quad
        \inferrule%*[Right=\eAnn]
        {\sexpr \step \sexpr'}{\tyann{\sexpr}{\stype} \step \tyann{\sexpr'}{\stype}}
        {\eAnn}
        \quad
        \inferrule%*[Right=\eAnnV]
        { }
        {\tyann{\sval}{\stype} \step \sval}
        {\eAnnV}
        \\
        \inferrule%*[Right=\eApp]
        {\sexpr \step \sexpr'}
        {\app{\sexpr}{\sexpr_1} \step \app{\sexpr'}{\sexpr_1}}
        {\eApp} 
        \quad
        \inferrule%*[Right=\eAppV]
        {\sexpr \step \sexpr'}
        {\app{\sval}{\sexpr} \step \app{\sval}{\sexpr'}}
        {\eAppV} 
        \quad
        \inferrule%*[Right=\eAppAbs]
        { }
          {\app{(\vabs{x}{\sexpr})}{\sval} \step \subst{\sexpr}{x}{\sval}}
          {\eAppAbs} 
          \\
          \inferrule%*[Right=\eTAppAbs]
          { }
          {\tyapp{(\tabs{\tvar}{\skind}{\sexpr})}{\stype} \step \subst{\sexpr}{\tvar}{\stype}}
          {\eTAppAbs} 
        \quad
        \inferrule%*[Right=\eTApp]
        {\sexpr \step \sexpr'}
        {\tyapp{\sexpr}{t} \step \tyapp{\sexpr'}{t}}
        {\eTApp} 
        \\
        \inferrule% *[Right=\eLet]
          { \sexpr_x \step \sexpr'_x}
          {\eletin{x}{\sexpr_x}{\sexpr} \step \eletin{x}{\sexpr'_x}{\sexpr}}
          {\eLet} 
          \quad
        \inferrule% *[Right=\eLetV]
        { }
        {\eletin{x}{\sval}{\sexpr} \step \subst{\sexpr}{x}{\sval}}
        {\eLetV} 
        \\
        \inferrule%*[Right=\eIf]
          {\sexpr \step \sexpr'}
          {\eif{\sexpr}{\sexpr_1}{\sexpr_2} \step \eif{\sexpr'}{\sexpr_1}{\sexpr_2}}
          {\eIf} 
        \\
        \inferrule% *[Right=\eIfT]
          {   }
          {\eif{\ttrue}{\sexpr_1}{\sexpr_2} \step \sexpr_1}
          {\eIfT} 
          \quad
        \inferrule% *[Right=\eIfF]
          { }
          {\eif{\tfalse}{\sexpr_1}{\sexpr_2} \step \sexpr_2}
          {\eIfF} 
        \end{mathpar}        
%    }
\caption{The small-step semantics.} 
\label{fig:e}
\label{fig:opsem}
\end{figure}


\Cref{fig:opsem} summarizes the substitution-based, 
call-by-value, contextual, small-step semantics 
for both calculi.
%
We specify the reduction semantics 
of the primitives using the functions 
$\delta$ and $\delta_T$.

\mypara{Substitution} 
%
% \NV{Put the definitions in Subst Figure one after the other, now that there is space?}
The key difference with standard formulations
is the notion of substitution for type variables 
at (polymorphic) type-application sites as shown 
in rule $\eTAppAbs$.
%
Type 
substitution is defined on the left of~\Cref{fig:type-subst}
and it is standard except 
for the last line which defines the substitution 
of a type variable $\al$ in a refined type variable 
${\breft{\al}{x}{p}}$ with a 
(potentially refined)
type $\stype_\al$.
%
To do this substitution, we combine $p$ with the type $\stype_\al$ 
by using $\strengthen{\stype_\al}{p}{x}$ 
which essentially conjoins the refinement $p$ 
to the top-level refinement of a base-kinded 
$\stype_\al$. 
%
For existential types, $\mathsf{refine}$ 
\emph{pushes} the refinement through the 
existential quantifier. 
%
Function and quantified types are left unchanged 
as they cannot instantiate a \emph{refined} 
type variable (which must be of base kind).

% Given the $\mathsf{strengthen}$ function we can define 
% \begin{equation*}
%   \subst{\breft{\al}{\vv}{p}}{\al}{\stype} \defeq \strengthen{\stype}{\subst{p}{\al}{\stype}}
% \end{equation*}

% In order to define substitution of a type variable
% by an arbitrary type in $\subst{\breft{\al}{\vv}{p}}{\al}{\stype}$, 
% we would simply write $\breft{\stype}{\vv}{\subst{p}{\al}{\stype}}$ 
% if we could stack refinements arbitrarily.
% %
% We have already covered the case of $\stype$ being a refinement type
% above. In the case of an existential type, we can 
% ``push'' the refinement $\breft{\xspace}{\vv}{\subst{p}{\al}{\stype}}$
% through the existential quantifier and define:
% \begin{equation*}
%   \subst{\breft{\al}{\vv}{p}}{\al}{\existype{z}{t_z}{t}} \defeq
%   \existype{z}{t_z}{ \subst{\breft{\al}{\vv}{p}}{\al}{t} }.
% \end{equation*}
% %
% These are the only two cases we need to elaborate: $\stype$ must have
% base kind unless $\al$ is unrefined (\ie, $p = \ttrue$), so
% $\stype$ cannot be a function type or a polymorphic type in a valid
% substitution into a refined type variable. 
% %
% For an unrefined type variable we can define simply
% $\subst{\al}{\al}{t} \defeq t$.

% To facilitate a unified definition for the substitution of a
% type variable by an arbitrary type, we define an auxiliary 
% function $\mathsf{strengthen}$ that describes how we ``push''
% a predicate into a base kinded type:
% \begin{align*}
%   \strengthen{\breft{\al}{z}{q}}{p}    & \defeq \breft{\al}{z}{p \wedge q} \\
%   \strengthen{\functype{x}{t_x}{t}}{p} & \defeq \functype{x}{t_x}{t} \\
%   \strengthen{\existype{z}{t_z}{t}}{p} & \defeq \existype{z}{t_z}{\strengthen{t}{p}} \\
%   \strengthen{\polytype{\al}{k}{t}}{p} & \defeq \polytype{\al}{k}{t}
% \end{align*}
% Given the $\mathsf{strengthen}$ function we can define 
% \begin{equation*}
%   \subst{\breft{\al}{\vv}{p}}{\al}{\stype} \defeq \strengthen{\stype}{\subst{p}{\al}{\stype}}
% \end{equation*}
% % \RJ{define and use a SUBST macro}
% % \RJ{i cannot even understand defs below}



\begin{figure}%[t!]
  %{\small
  %\begin{array}{r@{\hskip 0.05in}c@{\hskip 0.05in}l}
  \begin{tabular}{r@{\hskip 0.05in}c@{\hskip 0.05in}l}
  $\subst{\breft{\tvarb}{x}{p}}{\al}{t_\al}$       & $\defeq$ & $\breft{\tvarb}{x}{\subst{p}{\al}{t_\al}}, \al \not = \tvarb$ \\ 
  $\subst{(\functype{x}{t_x}{t})}{\al}{t_\al}$     & $\defeq$ & $\functype{x}{(\subst{t_x}{\al}{t_\al})}{\subst{t}{\al}{t_\al}}$ \\ 
  $\subst{(\existype{x}{t_x}{t})}{\al}{t_\al}$     & $\defeq$ & $\existype{x}{(\subst{t_x}{\al}{t_\al})}{\subst{t}{\al}{t_\al}}$ \\ 
  $\subst{(\polytype{\tvarb}{k}{t})}{\al}{t_\al}$  & $\defeq$ & $\polytype{\tvarb}{k}{\subst{t}{\al}{t_\al}}$ \\
  $\subst{\breft{\al}{x}{p}}{\al}{t_\al}$          & $\defeq$ & $\strengthen{t_\al}{\subst{p}{\al}{t_\al}}{x}$ \\
  \end{tabular}
  \begin{tabular}{r@{\hskip 0.05in}c@{\hskip 0.05in}l}
  $\strengthen{\breft{\al}{z}{q}}{p}{x}$           & $\defeq$ & $\breft{\al}{z}{\subst{p}{x}{z} \wedge q}$ \\
  $\strengthen{\existype{z}{t_z}{t}}{p}{x}$        & $\defeq$ & $\existype{z}{t_z}{\strengthen{t}{p}{x}}$ \\
  $\strengthen{\functype{x}{t_x}{t}}{\_}{\_}$      & $\defeq$ & $\functype{x}{t_x}{t}$ \\
  $\strengthen{\polytype{\al}{k}{t}}{\_}{\_}$      & $\defeq$ & $\polytype{\al}{k}{t}$ \\
  && \\
  %\end{array}
  \end{tabular}
  %}
  % $$\begin{array}{rcl}
  % \subst{\breft{\tvarb}{x}{p}}{\al}{t_\al}       & \defeq & \breft{\tvarb}{x}{\subst{p}{\al}{t_\al}} \\ 
  % \subst{(\functype{x}{t_x}{t})}{\al}{t_\al}     & \defeq & \functype{x}{(\subst{t_x}{\al}{t_\al})}{\subst{t}{\al}{t_\al}} \\
  % \subst{(\existype{x}{t_x}{t})}{\al}{t_\al}     & \defeq & \existype{x}{(\subst{t_x}{\al}{t_\al})}{\subst{t}{\al}{t_\al}} \\
  % \subst{(\polytype{\tvarb}{k}{t})}{\al}{t_\al}  & \defeq & \polytype{\tvarb}{k}{\subst{t}{\al}{t_\al}} \\
  % \subst{\breft{\al}{\vv}{p}}{\al}{\stype}       & \defeq & \strengthen{\stype}{\subst{p}{\al}{\stype}} \\ [0.05in] 
  % \strengthen{\breft{\al}{z}{q}}{p}              & \defeq & \breft{\al}{z}{p \wedge q} \\
  % \strengthen{\existype{z}{t_z}{t}}{p}           & \defeq & \existype{z}{t_z}{\strengthen{t}{p}} \\
  % \strengthen{\functype{x}{t_x}{t}}{p}           & \defeq & \functype{x}{t_x}{t} \\
  % \strengthen{\polytype{\al}{k}{t}}{p}           & \defeq & \polytype{\al}{k}{t} \\
  % \end{array}$$
  \vspace{-0.00cm}
  \caption{Type substitution and refinement strengthening.} 
  \label{fig:type-subst}
  %\label{fig:e}
  %\label{fig:opsem}
  %\vspace{-0.00cm}
  \end{figure}
  

\mypara{Primitives}
%
The function $\delta(\sconst, \sval)$ 
evaluates the application $\app{\sconst}{\sval}$ 
of built-in monomorphic primitives.
%
The reductions are defined in a curried 
manner, \ie 
$\app{\app{\leq}{m}}{n}$ evaluates to $\delta(\delta(\leq,m),n)$. 
%we have that $\app{\app{\leq}{m}}{n} \steps \delta(\delta(\leq,m),n)$. 
%
Currying gives us unary relations like $m\!\!\leq$ 
which is a partially evaluated version of the $\leq$ relation.
%
The function $\delta_T(\sconst, \forgetreft{\stype})$
specifies the reduction rules for type 
application on the polymorphic 
built-in primitives. % $=$ and $\leq$.
%
$$\begin{array}{rclrclrcl}
\delta(\wedge,{\tt true}) & \defeq & \lambda x.\, x &
\delta(\leq,m) & \defeq & m\!\!\leq  & 
    \delta_T(=, \tbool) & \defeq & =  \\
\delta(\wedge,{\tt false}) & \defeq & \lambda x.\, {\tt false}\quad\quad &
\delta(m\!\!\leq, n) & \defeq & {\tt}(m \leq n) \quad\quad&
\delta_T(=, \tint) & \defeq & = \\
\delta(\neg,{\tt true}) & \defeq & {\tt false} & 
\delta(=,m) & \defeq & m\!\!= &
\delta_T(\leq, \tbool) & \defeq & \leq  \\
  \delta(\neg,{\tt false}) & \defeq &  {\tt true} &
  \delta(m\!\!=, n) & \defeq &  {\tt}(m = n) &
  \delta_T(\leq, \tint) & \defeq & \leq  \\
%
%  \delta(\vee,{\tt true}) & \defeq & \lambda x.\, {\tt true} &
%  \delta(\leftrightarrow,{\tt true}) & \defeq & \lambda x.\, x & 
%    &  & \\
%    \delta(\vee,{\tt false}) & \defeq & \lambda x.\, x &
%    \delta(\leftrightarrow,{\tt false}) & \defeq & \lambda x.\, \neg x &
%    & & \\
\end{array}$$
         
% SAFETY --> Progress --> + Prim
%                         + Canonical

%        --> Preserve --> + Determinism
%                         + Substitution


  
\mypara{Determinism}
%
Our soundness proof  uses the determinism 
property of the operational semantics.
%
\begin{lemma}[Determinism]\label{lem:step-determ}
For every expression $\sexpr$, 
%\begin{itemize} 
    %\item 
    1) there exists at most one term $\sexpr'$ \suchthat $\sexpr \step \sexpr'$, 
    %\item 
    2) there exists at most one value $\sval$ \suchthat $\evalsTo{\sexpr}{\sval}$, and
    %\item 
    3) if $\sexpr$ is a value there is no term $\sexpr'$ \suchthat $\sexpr \step \sexpr'$.
%\end{itemize}
\end{lemma}    


% Section 3.3: Static Semantics
\section{Static Semantics}
\label{sec:lang:static}

The static semantics of our calculi comprise
four main judgment forms:
%
(\S~\ref{sec:typing:wf}) {\emph{well-formedness}} judgments that determine when a type
or environment is syntactically well-formed (in $\sysf$ and $\sysrf$);
%
(\S~\ref{sec:typing:typ}) {\emph{typing}} judgments that stipulate that a term has
a particular type in a given context (in $\sysf$ and $\sysrf$);
%
(\S~\ref{sec:typing:sub}) {\emph{subtyping}} judgments that establish when one type can
be viewed as a subtype of another (in $\sysrf$); and
%
(\S~\ref{sec:typing:implication}) {\emph{implication}} judgments that establish when one predicate
implies another (in $\sysrf$).
%
Next, we present the static semantics of \sysrf by describing
the rules that establish 
each of these judgments.
%
We use $\greybox{\mbox{grey}}$ to highlight the antecedents and rules
%needed for refinements in $\sysrf$.
specific to $\sysrf$.
%

\begin{fullversion}

\mypara{Co-finite Quantification}
%
We define our rules using the co-finite quantification technique
of~\citet{AydemirCPPW08}.
%
This technique enforces a small (but critical) restriction
in the way fresh names are introduced in the antecedents of rules.
%
For example, below we present the standard (on the left)
and our (on the right) rules for type abstraction.
$$
\inferrule*[Right=\tAbsEx]
    {\notmem{\al'}{\tcenv} \;\;
      \hastype{\bind{\al'}{\skind},\tcenv}{\subst{\sexpr}{\al}{\al'}}{\subst{\stype}{\al}{\al'}} }
    {\hastype{\tcenv}{\tabs{\al}{\skind}{\sexpr}}{\polytype{\al}{\skind}{\stype}}}
\qquad\qquad
\inferrule*[Right=\tTAbs]
    {\forall\notmem{\al'}{L}. \;\;
      \hastype{\bind{\al'}{\skind},\tcenv}{\subst{\sexpr}{\al}{\al'}}{\subst{\stype}{\al}{\al'}} }
    {\hastype{\tcenv}{\tabs{\al}{\skind}{\sexpr}}{\polytype{\al}{\skind}{\stype}}}
$$
%
%\NV{Make the rule consistent with the implementation?}
The standard rule \tAbsEx requires the \text{existence} of a
fresh type variable name $\al'$.
%
Instead our co-finite quantification rule
states that the rule holds for any name excluding
a finite set of names $L$. %(here the ones that already appear in $\tcenv$).
%
As observed by~\citet{AydemirCPPW08} this rephrasing
simplifies the mechanization of metatheory
by eliminating the need for renaming lemmas.

\end{fullversion}

\subsection{Well-formedness}
\label{sec:typing:wf}

\mypara{Judgments}
%
The judgment $\isWellFormed{\tcenv}{\stype}{\skind}$
says that the type $\stype$ is well-formed in the environment
$\tcenv$ and has kind $\skind$.
%
The judgment $\isWellFormedE{\tcenv}$ says that the
environment $\tcenv$ is well formed, meaning
that it only binds to well-formed
types.
%
Well-formedness is also used in the (unrefined) system $\sysf$, where
$\isWFFT{\tcenv}{\sftype}{\skind}$
means that the (unrefined) $\sysf$ type
$\sftype$ is well-formed in environment
$\tcenv$ and has kind $\skind$
and $\isWFFE{\tcenv}$ means
that the free type variables
of the
environment $\tcenv$ are bound earlier in the environment.
%
\begin{fullversion}
Well-formedness is not strictly
required for \sysf, but it 
simplifies the mechanization \cite{Remy21}.
\end{fullversion}


\mypara{Rules}
%
\Cref{fig:wf} summarizes the rules
that establish the well-formedness of types
and environments. 
%, with the grey highlighting
%the parts relevant for refinements.
%
Rule \wtBase states that the two closed base
types (\tint and \tbool, 
refined with \ttrue in \sysrf)
are well-formed and have base kind.
%
Similarly, rule \wtVar says that a %n unrefined or trivially refined
type variable $\tvar$ is well-formed with kind $\skind$
so long as $\bind{\tvar}{\skind}$ is bound in the environment.
%
The rule \wtRefn stipulates that a refined base type $\breft{\sbase}{x}{\spred}$
is well-formed with base kind in some environment
if the unrefined base type $\sbase$
has base kind in the same environment and if
the refinement predicate $\spred$ has type $\tbool$
in the environment augmented by binding a fresh variable to type $\sbase$.
Note that if $\sbase \equiv \tvar$ then we can only form the antecedent
$\isWellFormed{\tcenv}{\breft{\tvar}{x}{\ttrue}}{\skbase}$
when $\bind{\tvar}{\skbase} \in \tcenv$ (rule \wtVar),
which prevents us from refining star-kinded type variables.
%\NV{Do you mean that otherwise the rule var will be used?}
%
\textit{To break a circularity}\NV{check ref in the link in the overview pic}
in which well-formedness judgments 
appear in the antecedents  of
typing judgments and a typing judgment
appears in the antecedents
of \wtRefn, we use the $\sysf$
judgment to check that $\spred$ has
type $\tbool$.
%
\begin{fullversion}
  Our rule \wtFunc states that a function
  type $\functype{x}{t_x}{t}$ is well-formed
  with star kind in some environment $\tcenv$
  if both type $t_x$ is well-formed (with any kind)
  in the same environment and type $t$
  is well-formed (with any kind) in the
  environment $\tcenv$ augmented by binding
  a fresh variable to $t_x$.
  %
  Rule \wtExis states that an existential
  type $\existype{x}{t_x}{t}$ is well-formed
  with some kind $\skind$ in some environment
  $\tcenv$ if both type $t_x$ is well-formed
  (with any kind) in the same environment
  and type $t$ is well-formed with kind
  $\skind$ in the environment $\tcenv$
  augmented by binding a fresh variable
  to $t_x$.
  %
  Rule \wtPoly establishes that a
  polymorphic type $\polytype{\tvar}{\skind}{\stype}$
  has star kind in environment $\tcenv$ if the
  inner type $\stype$ is well-formed (with any kind)
  in environment $\tcenv$ augmented by binding a fresh
  type variable $\tvar$ to kind $\skind$.
\end{fullversion}
%
Finally, rule \wtKind simply states that if a type $\stype$
is well-formed with base kind in some environment, then
it is also well-formed with star kind. % in that environment.
This rule is required by our metatheory to convert base to star
kinds in type variables.
%Rule \wtKind is needed in our formalism because we chose
%that a star kinded variable may only be instantiated with
%a star kinded type in order to simplify other aspects of our
%metatheory.
%\NV{Why would we need to convert from base to star kind?}
%\NV{This rules does the opposite, check again}

\begin{conference}
As for environments, the empty environment
is well-formed.
A well-formed environment remains well-formed 
after binding a fresh term or type variable to \resp any well-formed type or kind. 
\end{conference}
\begin{fullversion}
As for environments, rule \wfeEmp states that the empty environment
is well-formed. Rule \wfeBind says that a well-formed environment
$\tcenv$ remains well-formed after binding a fresh variable $x$ to any
type $t_x$ that is well-formed in $\tcenv$.
%
Finally rule \wfeTBind states that a well-formed environment remains
well-formed after binding a fresh type variable to any kind.
\end{fullversion}


\begin{figure}%[t!]
%
%{\small
\begin{mathpar}
\judgementHead{Well-formed Type}{\isWellFormed{\tcenv}{\stype}{\skind}}

%%%%%%%%%%%% WELL-FORMEDNESS %%%%%%%%%%%%%
%
\inferrule% *[Right=\wtBase]
    {\sbase \in \{\tbool, \tint\}}
    {\isWellFormed{\tcenv}{\sbase\greybox{\!\breft{}{x}{\ttrue}}\!}{\skbase}}
    {\wtBase}
\quad
\inferrule% *[Right=]
    {\bind{\al}{\skind} \in \tcenv}
    {\isWellFormed{\tcenv}{\al\greybox{\!\breft{}{x}{\ttrue}}\!}{\skind}}
    {\wtVar} 
\quad
\inferrule% *[Right=]
    { \isWellFormed{\tcenv}{t}{\skbase} }
    { \isWellFormed{\tcenv}{t}{\skstar} }
    {\wtKind} 

%    
\greybox{\inferrule% *[Right=\;\;\wtRefn]
    { \isWellFormed{\tcenv}{\breft{b}{x}{\ttrue}}{\skbase} \\\\
      \forall\notmem{y}{\tcenv}.
      \hasftype{\bind{y}{b}, \forgetreft{\tcenv}}{\subst{p}{x}{y}}{\tbool}
    }
    {\isWellFormed{\tcenv}{\breft{b}{x}{p}}{\skbase}}
    {\wtRefn} }

\inferrule% *[Right=\wtFunc]
    { \isWellFormed{\tcenv}{t_x}{\skind_x} \\\\
      \greybox{\forall\notmem{y}{\tcenv}.} 
      \isWellFormed{\greybox{\bind{y}{t_x},}\tcenv}{t\greybox{\!\subst{}{x}{y}}}{\skind}
    }
    { \isWellFormed{\tcenv}{\greybox{x\!}\!\functype{}{t_x}{t}}{\skstar} }
    {\wtFunc} 

\greybox{\inferrule% *[Right=\;\;\wtExis]
    { \isWellFormed{\tcenv}{t_x}{\skind_x} \;\;
      \forall\notmem{y}{\tcenv}. \;
      \isWellFormed{\bind{y}{t_x},\tcenv}{\subst{t}{x}{y}}{\skind}
    }
    { \isWellFormed{\tcenv}{\existype{x}{t_x}{t}}{\skind} }
    {\wtExis} }

\inferrule% *[Right=]
    { \forall\notmem{\al'}{\tcenv}. \;\;
      \isWellFormed{\bind{\al'}{k}, \tcenv}{\subst{t}{\al}{\al'}}{\skind_t}
    }
    { \isWellFormed{\tcenv}{\polytype{\al}{\skind}{t}}{\skstar} }
    {\wtPoly}
%
\end{mathpar}
%}


%{\small
\begin{mathpar}
\judgementHead{Well-formed Environment}{\isWellFormedE{\tcenv}}

%
\inferrule% *[Right=\wfeEmp]
    { }
    { \isWellFormedE{\varnothing} }
    {\wfeEmp}
%
\quad
%
\inferrule% *[Right=\wfeBind]
    {\isWellFormed{\tcenv}{\stype_x}{\skind_x} \quad
     \isWellFormedE{\tcenv} \quad
     \notmem{x}{\tcenv}
    }
    { \isWellFormedE{\bind{x}{\stype_x}, \tcenv} }
    {\wfeBind}
%
\quad
%
\inferrule% *[Right=\wfeTBind]
    {\isWellFormedE{\tcenv} \quad
     \notmem{\al}{\tcenv}
    }
    { \isWellFormedE{\bind{\al}{\skind}, \tcenv} }
    {\wfeTBind}
\end{mathpar}
%}
\vspace{-0.00cm}
\caption{Well-formedness of types and environments. The rules for
  $\sysf$ exclude the grey boxes.}
\label{fig:wf}
\label{fig:wfe}
\vspace{-0.00cm}
\end{figure}


% \NV{Commented out kinded polymorphism as is already explained in the overview}
\begin{comment}
\mypara{Kinded (Refinement) Polymorphism}
%
\RJ{fix with OVERVIEW}\NV{all this is pushed to overview now}
%
Recall that \sysrf features two kinds:
\emph{base} ($\skbase$) and \emph{star} ($\skstar$).
%
Only base types \tbool and \tint (and existential
quantifications of them) can be refined in our syntax;
we do not permit refinements for function types
and polymorphic types.
%
This kind system allows us to keep track of types
eligible for refinement in the presence of type variables.
%
A type variable of base kind is eligible for refinements,
but one with star kind could be instantiated to any type
and so may not be refined.
%
For example, consider the polymorphic function definition in code
\ha{let max = \x y -> if x > y then x else y}.
What postconditions can we state about this function?
If $\vv$ is the return value, then we know that
$x \leq \vv$ and $y \leq \vv$. The type for the function is then
$\polytype{\al}{\skbase}{\functype{x}{\al}{\functype{y}{\al}
  {\breft{\al}{\vv}{x \leq \vv \wedge y \leq \vv}}}}$.
This type is well-formed due \wtRefn: we can form the judgment
$\isWellFormed{\bind{y}{\al},\bind{x}{\al}, \bind{\al}{\skbase}}
  {\breft{\al}{\vv}{x\leq\vv \wedge y\leq\vv}}{\skbase}$
only because $\bind{\al}{\skbase}$ appears in the environment.
%
\wtPoly then requires that we quantify over
$\bind{\al}{\skbase}$.
%
The type system will then reject usage like \ha{max f1 f2}
on functions because \ha{max} is only polymorphic over base types.
\end{comment}

% For example, by the typing rule \tVar below
% we can establish the judgment
% $\hastype{\bind{x}{(\polytype{\al}{\skstar}{\al})} }
%          {x}{(\polytype{\al}{\skstar}{\al})}$
% because by rule \wtPoly we know that
% $\isWellFormed{\varnothing}{(\polytype{\al}{\skstar}{\al})}{\skstar}$
% %
% Then by rule \tTApp below,
% for \emph{any} well-formed type
% $\isWellFormed{\bind{x}{(\polytype{\al}{\skstar}{\al})}}{\stype}{\skstar}$
% we have the judgment
% $\hastype{\bind{x}{(\polytype{\al}{\skstar}{\al})}}
%          {\tyapp{x}{\stype}}{\subst{\al}{\al}{\stype}}$.
%
% In particular, $\stype$ could a function type or polymorphic type,
% and so any refinement allowed on $\al$ would have to apply to such
% a type $\stype$.
% %
% Note, however, note that we cannot establish
% $\isWellFormed{\varnothing}{(\polytype{\al}{\skstar}{\breft{\al}{\vv}{\spred}})}{\skstar}$
% for any non-trivial $\spred$,
% and so we cannot establish
% $\hastype{\bind{x}{(\polytype{\al}{\skstar}{\breft{\al}{\vv}{\spred}})}}
%          {\tyapp{x}{t}}{\subst{\breft{\al}{\vv}{\spred}}{\al}{\stype}}$
% for types $\stype$ with arbitrary kind.
% This is prevented by our keeping track of kinds for type variables
% in the binding environment.
% %\RJ{explain how rules establish the above}

\subsection{Typing}
\label{sec:typing:typ}

The judgment $\hastype{\tcenv}{\sexpr}{\stype}$ states
that the term $\sexpr$ has type $\stype$ in the context of
environment $\tcenv$.
%
We write $\hasftype{\tcenv}{\sexpr}{\sftype}$
to indicate that term $\sexpr$ has the (unrefined)
$\sysf$ type $\sftype$ in the (unrefined) context
$\tcenv$.
%
\Cref{fig:typing} summarizes
the rules that establish typing for both $\sysf$ and
$\sysrf$, with grey %boxes
%indicating extensions needed
for the $\sysrf$ extensions.

\begin{figure}
%  {\small
  \begin{mathpar}             %%%%%%%%%%%%% TYPING %%%%%%%%%%%%%%%%%%
  \judgementHead{Typing}{\hastype{\tcenv}{\sexpr}{\stype}} \\
        \inferrule% *[Right=\tPrim]
        { % \ty{\sconst} = \stype
        }{\hastype{\tcenv}{\sconst}{\ty{\sconst}}}
        {\tPrim}
    \quad
        \inferrule% *[Right=\tVar]
        {\bind{x}{\stype} \in \tcenv  \\\\
         \greybox{\isWellFormed{\tcenv}{\stype}{\skind}}}
        {\hastype{\tcenv}{x}{\greybox{\self{\whitebox{\stype}}{x}{\skind}}}}
        {\tVar}
    \quad
    \inferrule%*[Right=\tAnn]
    {\hastype{\tcenv}{\sexpr}{\stype} \\\\
     \greybox{\isWellFormed{\tcenv}{\stype}{\skind}}}
    {\hastype{\tcenv}{\tyann{\sexpr}{\stype}}{\stype}}
    {\tAnn}
        \quad
        \greybox{
          \inferrule% *[Right=\;\;\tSub]
          {
            \isWellFormed{\tcenv}{t}{\skind}\\\\
            \hastype{\tcenv}{\sexpr}{s} \\
           \isSubType{\tcenv}{s}{t}
           }
          {\hastype{\tcenv}{\sexpr}{t}}
          {\tSub}
        }
        \\
    %
    \inferrule% *[Right=\tApp]
    {
      \hastype{\tcenv}{\sexpr_x}{\stype_x}\\\\
      \hastype{\tcenv}{\sexpr}{\greybox{x\!}\!\functype{}{\stype_x}{\stype}}
    }
    {\hastype{\tcenv}{\app{\sexpr}{\sexpr_x}}{\greybox{\existype{x}{\stype_x}{\whitebox{\stype}}}}}
    {\tApp} 
    \quad
    \inferrule%*[Right=\tAbs]
        { 
          \isWellFormed{\tcenv}{\stype_x}{\skind_x}\\\\
          \forall\notmem{y}{\tcenv}.
          \hastype{\bind{y}{\stype_x},\tcenv}{\subst{\sexpr}{x}{y}}{\greybox{\subst{\whitebox{\stype}}{x}{y}}} 
         }
        {\hastype{\tcenv}{\vabs{x}{\sexpr}}{\greybox{x\!}\!\functype{}{\stype_x}{\stype}}}
        {\tAbs}
    %
    \quad
    %
        \inferrule% *[Right=\tTAbs]
        {\forall\notmem{\al'}{\tcenv}.\\\\
        \hastype{\bind{\al'}{\skind},\tcenv}{\subst{\sexpr}{\al}{\al'}}{\subst{\stype}{\al}{\al'}}
        }
        {\hastype{\tcenv}{\tabs{\al}{\skind}{\sexpr}}{\polytype{\al}{\skind}{\stype}}}
        {\tTAbs}
        \and
    %
        \inferrule% *[Right=\tTApp]
        {
          \isWellFormed{\tcenv}{t}{\skind}\\\\
          \hastype{\tcenv}{\sexpr}{\polytype{\al}{\skind}{s}}
        }
        {\hastype{\tcenv}{\tyapp{\sexpr}{t}}{\subst{s}{\al}{t}}}
        {\tTApp}
    \quad    
    %
        \inferrule% *[Right=\tLet]
        {\hastype{\tcenv}{\sexpr_x}{\stype_x} \\
        \greybox{\isWellFormed{\tcenv}{\stype}{\skind}}\\\\
        \forall\notmem{y}{\tcenv}.
        \hastype{\bind{y}{\stype_x},\tcenv}{\subst{\sexpr}{x}{y}}
             {\stype\greybox{\!\subst{}{x}{y}}}
         }
        {\hastype{\tcenv}{\eletin{x}{\sexpr_x}{\sexpr}}{\stype}}
        {\tLet}
    \quad    
    %
        \inferrule% *[Right=\tIf]
        {\hastype{\tcenv}{\sexpr}{\greybox{\breft{\whitebox{\tbool}}{x}{p}}} \\
        \greybox{\isWellFormed{\tcenv}{\stype}{\skind}}\\\\
        \greybox{\forall\notmem{y}{\tcenv}.}
        \hastype{\greybox{\bind{y}{\breft{\tbool}{x}{p \wedge x}},}\tcenv}
             {\sexpr_1}{\stype}\\\\
        \greybox{\forall\notmem{y}{\tcenv}.}
        \hastype{\greybox{\bind{y}{\breft{\tbool}{x}{p \wedge \neg x}},}\tcenv}
             {\sexpr_2}{\stype}
         }
        {\hastype{\tcenv}{\eif{\sexpr}{\sexpr_1}{\sexpr_2}}{\stype}}
        {\tIf}        
    \end{mathpar}
%  }
\vspace{-0.00cm}
\caption{Typing rules.
The judgment $\hasftype{\tcenv}{\sexpr}{\sftype}$ is defined by excluding the grey boxes.}\label{fig:t}\label{fig:typing}
\vspace{-0.00cm}
\end{figure}


\mypara{Typing Primitives}
%
The type of a built-in primitive $\sconst$ is given by
the function $\ty{\sconst}$, which is defined for every
constant of our system. Below we present essential
examples of the $\ty{\sconst}$ definition.
%
{\small
$$\begin{array}{rclrcl}
\ty{\ttrue} & \defeq & \breft{\tbool}{x}{x = \ttrue} &
%&  \ty{\tfalse} & \defeq& \breft{\tbool}{x}{x = \tfalse} \\
%\ty{3} & \defeq& \breft{\tint}{x}{x = 3} \\
%&  \ty{n} & \defeq& \breft{\tint}{x}{x = n} \\
 \ty{\wedge} & \defeq & \functype{x}{\tbool}{\functype{y}{\tbool}{\breft{\tbool}{v}{v = x \wedge y}}} \\
%  \ty{\neg} & \defeq & \functype{x}{\tbool}{\breft{\tbool}{y}{y = \neg x}} \\
\ty{3} & \defeq& \breft{\tint}{x}{x = 3} &
\ty{\leq} & \defeq & \polytype{\al}{\skbase}{\functype{x}{\al}{\functype{y}{\al}{\breft{\tbool}{v}{v = (x \leq y)}}}} \\
\ty{m\!\!\leq} & \defeq & \functype{y}{\tint}{\breft{\tbool}{v}{v = (m \leq y)}} & 
% \ty{m\!\!\leq} \defeq&\; \functype{n}{\tint}{\breft{\tbool}{v}{v = (m \leq n)}} \\
\ty{=} & \defeq & \polytype{\tvar}{\skbase}{\functype{x}{\al}{\functype{y}{\al}{\breft{\tbool}{v}{v = (x = y)}}}}
\end{array}$$
}
We note that the $=$ used in the refinements is the polymorphic
equals with type applications elided.
%
Further, we use $m\!\!\leq$ to represent
an arbitrary member of the infinite family
of primitives $0\!\!\leq,\, 1\!\!\leq,\, 2\!\!\leq,\ldots$.
%
For \sysf we erase the refinements
using $\forgetreft{\ty{\sconst}}$.
%
The rest of the definition is similar.

Our choice to make the typing and reduction
of constants external to our language,
\ie given by the functions
$\ty{\sconst}$ and $\tc{\sconst}$,
makes our system easily extensible with further constants, 
including a terminating \texttt{fix} constant to encode induction.
%
The requirement, for soundness, is that
these two functions % on constants
together satisfy the following four conditions.
%
\begin{requirement}(Primitives) \label{lem:prim-typing}
For every primitive $c$,
\begin{enumerate}
\item If $\ty{\sconst} = \breft{\sbase}{x}{\spred}$, then
  $\isWellFormed{\varnothing}{\ty{\sconst}}{\skbase}$ and
  $\imply{\varnothing}{\ttrue}{\subst{\spred}{x}{\sconst}}$.
\item If $\ty{\sconst} = \functype{x}{\stype_x}{\stype}$ or
         $\ty{\sconst} = \polytype{\al}{\skind}{\stype}$, then
         $\isWellFormed{\varnothing}{\ty{\sconst}}{\skstar}$.
\item If ${\ty{\sconst}} = \functype{x}{\stype_x}{\stype}$,
      then for all $v_x$ such that
     $\hastype{\varnothing}{v_x}{\stype_x}$,
     % $\delta(\sconst,v)$ is defined and
    $\hastype{\varnothing}{\delta(\sconst,v_x)}{\subst{\stype}{x}{v_x}}$.
\item If ${\ty{\sconst}} = \polytype{\al}{\skind}{\stype}$,
      then for all $\stype_\al$ such that
     $\isWellFormed{\varnothing}{\stype_\al}{\skind}$,
     % $\delta_T(\sconst,\stype)$ is defined and we have
    $\hastype{\varnothing}{\delta_T(\sconst,\stype_\al)}{\subst{\stype}{\al}{\stype_\al}}$.
\end{enumerate}
\end{requirement}

Theorem 3 of~\cite{Vazou14} proves that 
a terminating \texttt{fix} constant satisfies
requirement~\ref{lem:prim-typing}. 

To type constants, rule \tPrim gives the type
$ty(\sconst)$ to any built-in
primitive $\sconst$, in any context.
%
\begin{fullversion}
The typing rules for boolean and integer
constants are included in \tPrim.
\end{fullversion}

\mypara{Typing Variables with Selfification}
%
Rule \tVar establishes that any variable $x$ that
appears as $\bind{x}{\stype}$ in environment $\tcenv$
can be given the \emph{selfified} type \cite{Ou2004}
$\self{\stype}{x}{\skind}$ provided that
$\isWellFormed{\tcenv}{\stype}{\skind}$.
%
This rule is crucial in practice,
to enable path-sensitive ``occurrence'' typing \cite{Tob08},
where the types of variables are refined by control-flow guards.
%
\NV{Can we use an example from the overview?}
\NV{go prime require self for if condition}
For example, suppose we want to establish
$\hastype{\bind{\al}{\skbase}}{(\vabs{x}{x})}{\functype{x}{\al}{\breft{\al}{y}{x=y}}}$,
and not just $\hastype{\bind{\al}{\skbase}}{(\vabs{x}{x})}{\funcftype{\al}{\al}}$.
%
The latter would result
if \tVar merely stated that
$\hastype{\tcenv}{x}{\stype}$
whenever $\bind{x}{\stype} \in \tcenv$.
%
Instead, we strengthen the \tVar rule to
be \emph{selfified}.
%:
%
Informally, to get information about $x$
into the refinement level, we need to say
that $x$ is constrained to elements of
type $\al$ that are equal to $x$ itself.
%
In order to express the exact type of
variables, below we define the ``selfification''
function that strengthens a refinement
with the condition that a value is equal
to itself.
%
Since abstractions do not admit equality,
we only selfify the base types and the existential
quantifications of them.
%
$$\begin{array}{r@{\hskip 0.03in}c@{\hskip 0.03in}l@{\hskip 0.2in}r@{\hskip 0.03in}c@{\hskip 0.03in}l}
  \self{\existype{z}{t_z}{t}}{x}{\skind} & \defeq & \existype{z}{t_z}{\self{t}{x}{\skind}} & 
  \self{\breft{\sbase}{z}{p}}{x}{\skbase} & \defeq & \breft{\sbase}{z}{p \wedge z = x} \\
  \self{\functype{x}{t_x}{t}}{\_}{\_} & \defeq & \functype{x}{t_x}{t} &
  \self{\breft{\sbase}{z}{p}}{x}{\skstar} & \defeq & \breft{\sbase}{z}{p} \\ 
  \self{\polytype{\al}{k}{t}}{\_}{\_} & \defeq & \polytype{\al}{k}{t}
% \self{\stype}{x}{\skstar}               & \defeq & \stype &
%  & &
\end{array}$$

\mypara{Typing Applications with Existentials}
%
Our rule \tApp states the conditions for typing
a term application $\app{\sexpr}{\sexpr_x}$.
%
Under the same environment,
we must be able to type $\sexpr$ at
some function type $\functype{x}{\stype_x}{\stype}$ and
$\sexpr_x$ at $\stype_x$. Then we can give $\app{\sexpr}{\sexpr_x}$
the existential type $\existype{x}{\stype_x}{\stype}$.
%
The use of existential types in rule \tApp
is one of the distinctive features of our language
and was introduced by~\citet{Knowles09}.
%
As overviewed in \S~\ref{overview:exists},
we chose this form of \tApp over the conventional
form of $\hastype{\tcenv}{\app{\sexpr}{\sexpr_x}}{\subst{\stype}{x}{\sexpr_x}}$
because our version prevents
the substitution of arbitrary expressions (\eg functions and type abstractions)
into refinements.
%
As an alternative, we could have used ANF (A-Normal Form~\cite{Flanagan93}),
but our metatheory would be more complex
since ANF is not preserved under the small step operational semantics.

\mypara{Other Typing Rules}
%
\begin{fullversion}
    Rule \tAbs says that we can type a lambda abstraction
    $\vabs{x}{\sexpr}$ at a function type $\functype{x}{\stype_x}{\stype}$
    whenever $\stype_x$ is well-formed and the body $\sexpr$
    can be typed at $\stype$ in the environment augmented by binding
    a fresh variable to $\stype_x$.  
\end{fullversion}
%
Our rule \tTApp states that whenever a term $\sexpr$ has polymorphic
type $\polytype{\al}{\skind}{s}$, then for any well-formed type $\stype$
with kind $\skind$, %in the same environment, 
we can give the type
$\subst{s}{\al}{\stype}$ to the type application $\tyapp{\sexpr}{\stype}$.
%
For the \sysf variant of \tTApp, we erase the refinements (via $\forgetreft{\stype}$)
before checking well-formedness and performing the substitution.
%
\begin{fullversion}
    The rule \tTAbs establishes that a type-abstraction $\tabs{\al}{\skind}{\sexpr}$
    can be given a polymorphic type $\polytype{\al}{\skind}{\stype}$ in some % environment
    $\tcenv$ whenever $\sexpr$ can be given the well-formed type $\stype$
    in %the environment 
    $\tcenv$ augmented by binding a fresh type variable
    to kind $\skind$.
    %
    Next, rule $\tLet$ states that an expression $\eletin{x}{\sexpr_x}{e}$ has type
    $\stype$ in some environment whenever $\stype$ is well-formed, $\sexpr_x$ can
    be given some type $\stype_x$, and the body $\sexpr$ can be given type $\stype$
    in the environment augmented by binding a fresh variable to $\stype_x$.
\end{fullversion}
%
Rule $\tAnn$ establishes that an explicit annotation $\tyann{\sexpr}{\stype}$
indeed has type $\stype$ when the underlying %expression 
$\sexpr$ has type $\stype$ and $\stype$ is well-formed.
The \sysf version of the rule erases the refinements and uses
$\forgetreft{\stype}$.
%
Rule $\tIf$ states that a conditional expression $\eif{\sexpr}{\sexpr_1}{\sexpr_2}$
has the type $\stype$ when the guard $\sexpr$ can be 
given type $\tbool$ refined by $p$ and $\sexpr_1$ (\resp $\sexpr_2$)
can be given type $\stype$ in the environment $\tcenv$ 
augmented by the knowledge we have about the type 
and semantics of the guard $\sexpr$.
The extension of the environment $\tcenv$
with a fresh variable that captures the semantics of the guard
when checking the two paths is critical to 
permit path-sensitive reasoning.
%
Finally, rule \tSub tells us that we can exchange a subtype $s$
for a supertype $t$ in a judgment $\hastype{\tcenv}{\sexpr}{t}$
provided 
$t$ is well-formed and
$\isSubType{\tcenv}{s}{t}$, which we present next.
%
\subsection{Subtyping}
\label{sec:typing:sub}

The \emph{subtyping} judgment ${\isSubType{\tcenv}{s}{t}}$, 
defined in \Cref{fig:s}, 
stipulates that the type $s$ is a subtype of
the type $t$ in the environment $\tcenv$
and is used in the subsumption typing rule \tSub (of~\Cref{fig:typing}).

\begin{figure}
\judgementHead{Subtyping}{\isSubType{\tcenv}{s}{t}}

\begin{mathpar}   %%%%%%%%%%%%%%%%%% SUBTYPING %%%%%%%%%%%%%%%%%%
  \inferrule*[Right=\sFunc]
  { \isSubType{\tcenv}{t_{x2}}{t_{x1}} \quad
    \forall\notmem{y}{\tcenv}. \quad
    \isSubType{\bind{y}{t_{x2}},\tcenv}{\subst{t_1}{x}{y}}{\subst{t_2}{x}{y}} }
  {\isSubType{\tcenv}{\functype{x}{t_{x1}}{t_1}}{\functype{x}{t_{x2}}{t_2}}} \\
%
\and
%
  \inferrule*[Right=\sWitn]
  { \hastype{\tcenv}{\sval_x}{\stype_x} \quad
    \isSubType{\tcenv}{\stype}{\subst{\stype'}{x}{\sval_x}}}
  {\isSubType{\tcenv}{\stype}{\existype{x}{\stype_x}{\stype'}}}
%
\and
%
  \inferrule*[Right=\sBind]
  { \forall\notmem{y}{\free{\stype}\cup \tcenv}. \quad
    \isSubType{\bind{y}{\stype_x},\tcenv}{\subst{\stype}{x}{y}}{\stype'}
%    \quad
%    \lc{\stype'}
  }
  {\isSubType{\tcenv}{\existype{x}{\stype_x}{\stype}}{\stype'}} \\
%
\and
%
  \inferrule*[Right=\sPoly]
  { \forall\notmem{\al'}{\tcenv}. \quad
    \isSubType{\bind{\al'}{\skind},\tcenv}{\subst{\stype_1}{\al}{\al'}}{\subst{\stype_2}{\al}{\al'}} }
  {\isSubType{\tcenv}{\polytype{\al}{\skind}{\stype_1}}{\polytype{\al}{\skind}{\stype_2}}}
%
\and
%
  \inferrule*[Right=\sBase]
  {\forall\notmem{y}{\tcenv}. \quad
    \imply{\bind{y}{{\sbase}},\tcenv}{\subst{p_1}{x}{y}}{\subst{p_2}{x}{y}} }
  {\isSubType{\tcenv}{\breft{\sbase}{x}{\spred_1}}{\breft{\sbase}{x}{\spred_2}}}
\end{mathpar}
\vspace{-0.00cm}
\caption{Subtyping Rules.}
\label{fig:s}
\label{fig:subtyping}
\label{fig:ent}
\vspace{-0.00cm}
\end{figure}
\mypara{Subtyping Rules}
\begin{fullversion}  
    The rule \sFunc states that one function type $\functype{x_1}{t_{x1}}{t_1}$
    is a subtype of another function type $\functype{x_2}{t_{x2}}{t_2}$ in a
    given environment $\tcenv$ when both $t_{x2}$ is a subtype of $t_{x1}$
    and $t_1$ is a subtype of $t_2$ when we augment $\tcenv$ by
    binding a fresh variable to type $t_{x2}$.
    As usual, function subtyping is contravariant
    in the input type and covariant in the outputs.
\end{fullversion}
%
Rules \sBind and \sWitn establish subtyping for existential
types \cite{Knowles09}, \resp when the existential
appears on the left or right.
%
Rule \sBind allows us to exchange a universal quantifier
(a variable bound to some type $\stype_x$ in the environment)
for an existential quantifier.
%
If we have a judgment of the form
$\isSubType{\bind{y}{\stype_x},\tcenv}{\subst{\stype}{x}{y}}{\stype'}$
where $y$ does \emph{not} appear free in either $\stype'$ or in the context $\tcenv$,
%$\stype'$ is locally closed, which means that all its deBruijn
%indices point to binders within $\stype'$ (denoted $\lc{t'}$),
% \NV{We should not talk about locally closed since we do not model deBruin.
%     We could instead add a well formedness requirement, but I assume you have a proof
%     that everything in the right hand side of a subtyping is well formed?}
then we can conclude that $\existype{x}{\stype_x}{\stype}$
is a subtype of $\stype'$.
%
Rule \sWitn states that if type $\stype$ is a subtype of
$\subst{\stype'}{x}{\sval_x}$ for some value $\sval_x$
of type $\stype_x$, then we can discard the specific
\emph{witness} for $x$ and quantify existentially to
obtain that $\stype$ is a subtype of $\existype{x}{\stype_x}{\stype'}$.
%
\begin{fullversion} 
    Rule \sPoly states that one polymorphic type
    $\polytype{\al}{\skind}{\stype_1}$
    is a subtype of another polymorphic type
    $\polytype{\al}{\skind}{\stype_2}$
    in some environment $\tcenv$, when
    $\stype_1$ is a subtype of $\stype_2$ in the
    environment $\tcenv$ augmented  by binding
    a fresh type variable to kind $\skind$.
\end{fullversion}

Refinements enter the scene in the rule \sBase which
specifies
%uses implication to specify
that a refined base type $\breft{\sbase}{x}{p_1}$
is a subtype of another $\breft{\sbase}{x}{p_2}$
in context $\tcenv$ when $p_1$ \emph{implies} $\spred_2$
in the environment $\tcenv$ augmented by binding
a fresh variable to the unrefined type $\sbase$.
%
%Next, we describe how we formalized implication.

\subsection{Implication}
\label{sec:typing:implication}

The \emph{implication} judgment $\imply{\tcenv}{\spred_1}{\spred_2}$
states that the implication $\spred_1 \Rightarrow \spred_2$
holds under the assumptions captured by the context $\tcenv$.
%
In refinement type implementations~\cite{newfstar,Seidel14}, this relation
is implemented as an external automated (usually SMT) solver.
%
Since external solvers are not easy to encode in mechanized proofs,
we follow an approach that decouples the mechanization from the implementation. 
Concretely, 
first we define the interface of the implication (\S~\ref{sec:typing:implication:interface})
that precisely captures all the requirements that the implication 
judgement should satisfy to establish the soundness of $\sysrf$. 
Then, we define two alternative implementations of the interface: 
a logical implementation (\S~\ref{sec:typing:implication:logical})
that is used in refinement type implementations 
and a denotational implementation (\S~\ref{sec:typing:implication:denotational})
that we used to complete our mechanized proof. 


\subsubsection{Implication's Interface}
\label{sec:typing:implication:interface}
In our mechanization, 
following~\citet{LehmannTanter}, 
we encode implication
as an axiomatized judgment that satisfies the
requirements below.


\begin{requirement}[Implication Interface]\label{lem:implication}
  The implication relation satisfies the below statements:
  \begin{enumerate}
      \item (Reflexivity) $\imply{\tcenv}{\spred}{\spred}$.
      \item (Transitivity) If $\imply{\tcenv}{\spred_1}{\spred_2}$
        and $\imply{\tcenv}{\spred_2}{\spred_3}$, then
        $\imply{\tcenv}{\spred_1}{\spred_3}$.
      \item (Faithfulness) $\imply{\tcenv}{\spred}{\ttrue}$.
      \item (Introduction) If $\imply{\tcenv}{\spred_1}{\spred_2}$
        and $\imply{\tcenv}{\spred_1}{\spred_3}$, then
        $\imply{\tcenv}{\spred_1}{\csand{\spred_2}{\spred_3}}$.
      \item (Conjunction)
      $\imply{\tcenv}{\csand{\spred_1}{\spred_2}}{\spred_1}$ and
      $\imply{\tcenv}{\csand{\spred_1}{\spred_2}}{\spred_2}$.
      \item (Repetition)
      $\imply{\tcenv}{\csand{\spred_1}{\spred_2}}
                     {\csand{\spred_1}{\csand{\spred_1}{\spred_2}}}$.
      \item (Evaluation) If $\evalsTo{\spred_1}{\spred_2}$, 
        then $\imply{\tcenv}{\spred_1}{\spred_2}$
        and  $\imply{\tcenv}{\spred_2}{\spred_1}$.
      \item (Narrowing) If
      $\imply{\tcenv_1,\bind{x}{t_x},\tcenv_2}{\spred_1}{\spred_2}$
      and $\isSubType{\tcenv_2}{s_x}{t_x}$, then
      $\imply{\tcenv_1,\bind{x}{s_x},\tcenv_2}{\spred_1}{\spred_2}$.
      \item (Weaken) If
      $\imply{\tcenv_1,\tcenv_2}{\spred_1}{\spred_2}$,
       $a,x\not\in\tcenv$, then
      $\imply{\tcenv_1,\bind{x}{t_x},\tcenv_2}{\spred_1}{\spred_2}$ and
      $\imply{\tcenv_1,\bind{a}{k},\tcenv_2}{\spred_1}{\spred_2}$.
      \item (Subst I) If
      $\imply{\tcenv_1,\bind{x}{t_x},\tcenv_2}{\spred_1}{\spred_2}$
      and $\hastype{\tcenv_2}{v_x}{t_x}$, then
      $\imply{\subst{\tcenv_1}{x}{v_x},\tcenv_2}{\subst{\spred_1}{x}{v_x}}
                                             {\subst{\spred_2}{x}{v_x}}$.
      \item (Subst II) If $\imply{\tcenv_1,\bind{a}{k},\tcenv_2}{\spred_1}{\spred_2}$
      and $\isWellFormed{\tcenv_2}{t}{k}$, then
      $\imply{\subst{\tcenv_1}{a}{t},\tcenv_2}{\subst{\spred_1}{a}{t}}
      {\subst{\spred_2}{a}{t}}$.
      \item (Strengthening) If
      $\imply{\bind{y}{\breft{\sbase}{x}{q},\tcenv}}{\spred_1}{\spred_2}$,
      then $\imply{\bind{y}{\sbase},\tcenv}{\csand{\subst{q}{x}{y}}{\spred_1}}
                  {\csand{\subst{q}{x}{y}}{\spred_2}}$.
  \end{enumerate}
\end{requirement}

\noindent
This interface  precisely explicates
the requirements of
the implication checker 
to establish the soundness of the
entire refinement type system.
%
The first six statements are standard properties of implication. 
Evaluation is used to prove that built-in constants satisfy the~\Cref{lem:prim-typing} 
and the rest, as captured by their name, are required to prove  
the narrowing (\ref{lem:narrowing}), weakening (\ref{lem:weakening}), 
substitution (\ref{lem:subst}) lemmas hold in \sysrf. 

Our requirements are very similar to Assumption 1 of~\cite{Knowles09}. 
Our Strengthening and Subst II cases are required for polymorphism, thus 
they do not appear in~\citet{Knowles09}'s assumption.  
Instead they require Consistency and Exact Quantification. 
We do not require Exact Quantification since our relation captures 
the minimum requirements to prove soundness. 
Instead of explicitly requiring Consistency, 
in \S~\ref{sec:typing:implication:denotational} we define (and mechanize) an implementation, \ie inhabitant, 
of the interface thus show our assumptions are consistent. 


\subsubsection{Logical Implementation (non mechanized)}
\label{sec:typing:implication:logical}
The logical implementation of \imply{\tcenv}{\spred_1}{\spred_2}
checks that the logical implication $\spred_1 \Rightarrow \spred_2$
is valid assuming the refinements of the base types in $\tcenv$:
%
$$
  \inferrule
  {\models_{\texttt{LOGIC}} \wedge\{ \subst{\spred}{\vv}{x} \mid \bind{x}{\breft{\sbase}{\vv}{\spred}} \in \tcenv \} \Rightarrow \spred_1 \Rightarrow \spred_2 }
  {\imply{\tcenv}{\spred_1}{\spred_2}}
  {\iLog}
$$
This encoding is imprecise, since some information is ignored from the environment $\tcenv$, 
but when the language of refinements is decidable, implication checking 
is also decidable and can be efficiently checked by an SMT solver.
\lh, for example, uses this encoding to reduce type checking to decidable 
implications checked by Z3~\cite{z3}, while the soundness 
of this implementation (concretely statement 7 of~\Cref{lem:implication}) 
is hinted by Theorem 2 of~\cite{Vazou14}.
\citet{10.1145/3546196.3550162} defines a mechanization of a 
refinement type system in Agda that uses a similar encoding of implication
where logical implications are checked using Agda's logic. 


\subsubsection{Denotational Implementation (mechanized)}
\label{sec:typing:implication:denotational}
The denotational implementation of \imply{\tcenv}{\spred_1}{\spred_2}
checks that if $\spred_1$ evaluates to \ttrue, so does $\spred_2$. 
%
$$
  \inferrule
  { \forall \clsub\!\in\!\denote{\tcenv}. \
      \evalsTo{\applysubst{\clsub}{\spred_1}}{\ttrue}
      \Rightarrow 
      \evalsTo{\applysubst{\clsub}{\spred_2}}{\ttrue}
  }
  {\imply{\tcenv}{\spred_1}{\spred_2}}
  {\iDen}
$$
The refinements $\spred_1$ and $\spred_2$ are boolean expressions, 
so evaluation uses the operational semantics of~\Cref{fig:opsem}. 
But, they are open expressions with variables bound in $\tcenv$,
so before evaluation we apply the closing substitution $\clsub$
that belongs to the denotation of $\tcenv$, as defined next. 

\begin{figure}
$$\begin{array}{r@{\hskip 0.03in}c@{\hskip 0.03in}l}
\denote{\breft{\sbase}{x}{p}} & \defeq &
  \setcomp{\sval}{\hasftype{\varnothing}{\sval}{\sbase} \,\wedge\, \evalsTo{\subst{p}{x}{\sval}}{\ttrue}} \\
\denote{\functype{x}{\stype_x}{\stype}} & \defeq &
  \setcomp{\sval}{\hasftype{\varnothing}{\sval}{\funcftype{\forgetreft{\stype_x}}{\forgetreft{\stype}}} \,\wedge\, (\forall\, \sval_x \in \denote{\stype_x}.\, \evalsTo{\app{\sval}{\sval_x}}{\sval'} \,{\rm s.t.}\, \sval' \in \denote{\subst{\stype}{x}{\sval_x}} }  \\
\denote{\existype{x}{\stype_x}{\stype}} & \defeq &
  \setcomp{\sval}{(\hasftype{\varnothing}{\sval}{\forgetreft{\stype}}) \,\wedge\,
    ( \exists\, \sval_x \in \denote{\stype_x}.\, \sval\in \denote{\subst{\stype}{x}{\sval_x}} }   \\
\denote{\polytype{\al}{\skind}{\stype}} & \defeq &
  \setcomp{\sval}{(\hasftype{\varnothing}{\sval}{\polytype{\al}{\skind}{\forgetreft{\stype}}}) \,\wedge\,
    (\forall\, \stype_\al.\, (\isWellFormed{\varnothing}{\stype_\al}{\skind}) \Rightarrow
    \evalsTo{\tyapp{\sval}{\stype_\al}}{\sval'}
    \,{\rm s.t.}\, \sval' \in \denote{\subst{\stype}{\al}{\stype_\al}}} \\ \\ 
\denote{\tcenv} & \defeq &
\setcomp{\clsub} %= (x_1 \mapsto v_1,\ldots, x_n \mapsto v_n,  \al_1 \mapsto t_1, \,\ldots,\, \al_m\mapsto t_m)}{
{\forall\, (x:t) \in \tcenv.\, \clsub(x) \in \denote{\applysubst{\clsub}{t}} \;\wedge\;
\forall\, (\al:k) \in \tcenv.\, \isWellFormed{\varnothing}{\clsub(\al)}{k}}.
\end{array}$$
\vspace{-0.0cm}
\caption{Denotations of Types and Environments.}
\label{fig:den}
\vspace{-0.0cm}
\end{figure}

\paragraph{Closing Substitutions}
A \emph{closing substitution} is a sequence
of value bindings to variables:
$
    \clsub = (x_1\mapsto v_1,\,\ldots,\, x_n\mapsto v_n, \al_1 \mapsto t_1, \,\ldots,\, \al_m\mapsto t_m)
    \;\; {\rm with\;\; all}\; x_i,\; \al_j\; {\rm distinct.}
$
%
We write $\clsub(x)$ to refer to $v_i$ if $x = x_i$
and we use $\clsub(\al)$ to refer to $t_j$ if $\al = \al_j$.
%
We define $\applysubst{\clsub}{t}$ to be the type derived from $t$
by substituting for all variables in $\clsub$:
%
$
  \applysubst{\clsub}{t} \defeq \subst{t}{x_1}{v_1}\cdots\subst{}{x_n}{v_n}\subst{}{\al_1}{t_1}\cdots\subst{}{\al_m}{t_m}.
$
%


\paragraph{Denotational Semantics}
\Cref{fig:den} defines the denotations of types and environments.
Following~\citet{flanagan06}, 
each closed type has a denotation $\denote{\stype}$
containing the set of closed values
of the appropriate base type
which satisfy the type's refinement
predicate.
(The denotation of a type variable $\al$
is not defined as we only require
denotations for closed types.)
%
We lift the notion of denotations to environments $\denote{\tcenv}$
as the set of closing substitutions, \ie value
and type bindings for the variables in $\tcenv$, such that the
values respect the denotations of the respective $\tcenv$-bound
types and the types are well formed with respect
to the corresponding kinds.


\paragraph{Revisiting rule~\iDen}
The premise of the rule~\iDen quantifies 
over all closing substitutions in the denotations of the 
typing environment 
(\ie $\forall \clsub\!\in\!\denote{\tcenv}$). 
This quantification has two consequences. 

First, the environment denotation appears in a negative position 
on the premise of the rule. Inspecting~\Cref{fig:den}, 
the environment denotation uses the type denotation, which in turn uses 
type checking, thus rendering a \textit{potential circularity
between type and implication checking} (arrow 6 of~\Cref{fig:dependencies}). 
Because of the negative occurrence, this mutual dependency 
would lead to a non-monotonic and thus non-well defined system. 
To break this circularity, we use $\sysf$'s type checking in the definition 
of type denotations. 

Second, the quantification is over all closing substitutions
which are infinite. For example, 
a typing environment that binds $x$ to an integer 
(\ie $\bind{x}{\tint}\in \tcenv$) 
has infinitely many closing substitutions mapping $x$ to a different 
integer. Thus, the denotational implementation 
cannot be used to implement a decidable type checker. 
%
On the positive side, the denotational implementation
connects implication checking to the operational semantics
thus it is amicable to mechanization.
Concretely, we proved (\S~\ref{sec:coq})
that the denotational implementations satisfies 
the statements of~\Cref{lem:implication}. 
 

% new ch 4 : System F Metatheory
\chapter{$\sysf$ Soundness}
\label{ch:systemF}

Next, we present the metatheory  
of the underlying (unrefined) 
$\sysf$. Even though it follows the textbook techniques of~\citet{TAPL},
it is a convenient 
stepping stone \emph{towards} 
the metatheory for (refined) $\sysrf$.
%
In addition, the soundness results 
for $\sysf$ are used \emph{for}                       %% TODO: point to specific
our full metatheory, as our well-formedness           %% places where sysf Soundness
judgments require the refinement                      %% used in Denot Sound metath.
predicate to have the $\sysf$ type 
$\tbool$ thereby avoiding the circularity 
of using a regular typing judgment in the 
antecedents of the well-formedness rules. 
%
% \mypara{Outline}
%
The \colboth boxes in Figure~\ref{fig:graph}
show the high level outline of the metatheory 
for $\sysf$ which provides a miniaturized model
for $\sysrf$ but without the challenges 
of subtyping
and existentials. Next, we describe the top-level 
type safety result, how it is decomposed into 
progress (Lemma~\ref{lem:progressFF}) and 
preservation (Lemma~\ref{lem:preservationFF}) 
lemmas, and the various technical results that 
support the lemmas.

Because the syntax of \sysf is identical to the
syntax of \sysrf, we have the slight complication
that refinements still appear in our terms.
%
Therefore in this chapter we continue to use $t, t', \ldots$ 
in our definitions and theorems to denote the 
refined types of \sysrf that will be used
in type annotations $\tyann{e}{t}$
and type applications $\tyapp{e}{t}$. 
%
To avoid confusion in the development of the
unrefined metatheory, we will
use $\sftype, \sftype',\ldots$ for 
\sysf types. 

\section{Static Semantics}
\label{sec:staticF}

The small-step semantics for \sysf are identical to
those of \sysrf (because the syntax is unchanged),
but the well-formedness and typing rules consist 
of those in Figures \ref{fig:wf} and \ref{fig:t}
with the parts in grey erased.
%
For clarity, 
and to make this chapter self-contained,
we present the $\sysf$ rules for
well-formedness in Figure \ref{fig:wff} and the 
rules for typing in Figure \ref{fig:ft}.

\begin{figure}%[t!]
  %
  %{\small
  \begin{mathpar}
  \judgementHead{Well-formed (Unrefined) Types}{\isWFFT{\tcenv}{\sftype}{\skind}}
  
  %%%%%%%%%%%% WELL-FORMEDNESS %%%%%%%%%%%%%
  %
  \inferrule% *[Right=\wtBase]
      {\sbase \in \{\tbool, \tint\}}
      {\isWFFT{\tcenv}{\sbase}{\skbase}}
      {\wfftBasic}
  \quad
  \inferrule% *[Right=]
      {\bind{\al}{\skind} \in \tcenv}
      {\isWFFT{\tcenv}{\al}{\skind}}
      {\wfftVar} 
  \quad
  \inferrule% *[Right=]
      { \isWFFT{\tcenv}{\sftype}{\skbase} }
      { \isWFFT{\tcenv}{\sftype}{\skstar} }
      {\wfftKind} 
  
  %    
  \inferrule% *[Right=\wtFunc]
      { \isWFFT{\tcenv}{\sftype_x}{\skind_x} \\\\
      \isWFFT{\tcenv}{\sftype}{\skind}
      }
      { \isWFFT{\tcenv}{\funcftype{\sftype_x}{\sftype}}{\skstar} }
      {\wfftFunc} 
  \quad
  \inferrule% *[Right=]
      { \forall\notmem{\al'}{\tcenv}. \;\;
        \isWFFT{\bind{\al'}{k}, \tcenv}{\subst{\sftype}{\al}{\al'}}{\skind_\sftype}
      }
      { \isWFFT{\tcenv}{\polytype{\al}{\skind}{\sftype}}{\skstar} }
      {\wfftPoly}
  %
  \end{mathpar}
  %}
  
  \vspace{-0.00cm}
  \caption{Well-formedness of $\sysf$ types.}
  \label{fig:wff}
  \label{fig:wfft}
  \vspace{-0.00cm}
\end{figure}



\begin{figure}
  %  {\small
    \begin{mathpar}             %%%%%%%%%%%%% TYPING %%%%%%%%%%%%%%%%%%
    \judgementHead{Typing}{\hasftype{\tcenv}{\sexpr}{\sftype}} \\
          \inferrule% *[Right=\tPrim]
          { % \ty{\sconst} = \stype
          }{\hasftype{\tcenv}{\sconst}{\forgetreft{\ty{\sconst}}}}
          {\fPrim}
      \quad
          \inferrule% *[Right=\tVar]
          {\bind{x}{\sftype} \in \tcenv  \\\\
            \isWFFT{\tcenv}{\sftype}{\skind}}
          {\hasftype{\tcenv}{x}{\sftype}}
          {\fVar}
      \quad
      \inferrule%*[Right=\tAnn]
      {\hasftype{\tcenv}{\sexpr}{\forgetreft{\stype}} \\\\
        \isWFFT{\tcenv}{{\forgetreft{\stype}}}{\skind}}
      {\hasftype{\tcenv}{\tyann{\sexpr}{\stype}}{{\forgetreft{\stype}}}}
      {\fAnn}
          \\
      %
      \inferrule% *[Right=\tApp]
      {
        \hasftype{\tcenv}{\sexpr_x}{\sftype_x}\\\\
        \hasftype{\tcenv}{\sexpr}{\funcftype{\sftype_x}{\sftype}}
      }
      {\hasftype{\tcenv}{\app{\sexpr}{\sexpr_x}}{\sftype}}
      {\fApp} 
      \quad
      \inferrule%*[Right=\tAbs]
          { \isWFFT{\tcenv}{\sftype_x}{\skind_x}\\\\
            \forall\notmem{y}{\tcenv}.
            \hastype{\bind{y}{\sftype_x},\tcenv}{\subst{\sexpr}{x}{y}}{\sftype} 
            }
          {\hastype{\tcenv}{\vabs{x}{\sexpr}}{\funcftype{\sftype_x}{\sftype}}}
          {\fAbs}
      %
      \quad
      %
          \inferrule% *[Right=\tTAbs]
          {\forall\notmem{\al'}{\tcenv}.\\\\
          \hastype{\bind{\al'}{\skind},\tcenv}{\subst{\sexpr}{\al}{\al'}}{\subst{\sftype}{\al}{\al'}}
          }
          {\hastype{\tcenv}{\tabs{\al}{\skind}{\sexpr}}{\polytype{\al}{\skind}{\sftype}}}
          {\fTAbs}
          \and
      %
          \inferrule% *[Right=\tTApp]
          {
            \isWFFT{\tcenv}{\forgetreft{\stype}}{\skind}\\\\
            \hastype{\tcenv}{\sexpr}{\polytype{\al}{\skind}{\sftype'}}
          }
          {\hastype{\tcenv}{\tyapp{\sexpr}{\stype}}
              {\subst{\sftype'}{\al}{\forgetreft{\stype}}}}
          {\fTApp}
      \quad    
      %
          \inferrule% *[Right=\tLet]
          {\hastype{\tcenv}{\sexpr_x}{\sftype_x} \\\\
          \forall\notmem{y}{\tcenv}.
          \hastype{\bind{y}{\sftype_x},\tcenv}{\subst{\sexpr}{x}{y}}
                {\sftype}
            }
          {\hastype{\tcenv}{\eletin{x}{\sexpr_x}{\sexpr}}{\sftype}}
          {\fLet}
      \quad    
      %
          \inferrule% *[Right=\tIf]
          {\hastype{\tcenv}{\sexpr}{\tbool} \\
          \hastype{\tcenv}{\sexpr_1}{\sftype}\\
          \hastype{\tcenv}{\sexpr_2}{\sftype}
            }
          {\hastype{\tcenv}{\eif{\sexpr}{\sexpr_1}{\sexpr_2}}{\sftype}}
          {\fIf}        
      \end{mathpar}
  %  }
  \vspace{-0.00cm}
  \caption{Unrefined typing rules.}
  \label{fig:ft}
  \label{fig:ftyping}
  \vspace{-0.00cm}
\end{figure}
    
\section{Metatheory for \sysf}
\label{sec:soundnessF}

% The proof of both of these theorems 
% is given by structural induction on 
% the typing judgment for the closed term. 
% %
% % The typing rules are syntax directed 
% % for $\sysf$, so we can uniquely determine 
% % the top level syntactic structure of our 
% % closed term.  
% % %
% We refer to this notion as Inversion of Typing.
% All of the theorems and lemmas mentioned here 
% appear in \colboth in Figure \ref{fig:graph}.
%
% \mypara{Type Safety}
%
The main type safety theorem for 
\sysf states that a well-typed 
term does not get stuck: \ie either 
evaluates to a value or can step 
to another term (progress) 
of the same type (preservation).
%
The judgment \hasftype{\tcenv}{\sexpr}{\sftype} 
is defined in~\cref{fig:ftyping}, and for clarity we use $\sftype$ 
for \sysf types and $\stype$ for the $\sysrf$ types that appear
in user annotations and in type applications. 

\begin{theorem} (Type Safety of \sysf) 
  \label{lem:soundnessF} %(\cref{app:lem:soundnessF} in App. \ref{ch:proofsF})
  \begin{enumerate}
      \item (Type Safety)
      If $\hastype{\varnothing}{\sexpr}{\sftype}$ and $\evalsTo{\sexpr}{\sexpr'}$,
      then $\sexpr'$ is a value or $\sexpr' \step \sexpr''$
      %and $\hastype{\varnothing}{\sexpr''}{\stype}$
      for some $\sexpr''$.
      \item (No Error)
      If $\hastype{\varnothing}{\sexpr}{\sftype}$ and $\evalsTo{\sexpr}{\sexpr'}$,
      then $\sexpr' \not = \eerr$.
  \end{enumerate}
  \end{theorem}
  %%
\begin{proof}
  (1) 
  We proceed by induction on the number of steps in 
  $\evalsTo{\sexpr}{\sexpr'}$.
  %
  There are two cases for $\evalsTo{\sexpr}{\sexpr'}$: 
  either $\sexpr=\sexpr'$ 
  or there exists a term $\sexpr_1$ 
  such that $\sexpr \step \evalsTo{\sexpr_1}{\sexpr'}$.
  %
  In the former case we conclude immediately 
  by the Progress Lemma (\ref{lem:progressFF} below).
  %
  In the latter case, $\hasftype{\varnothing}{\sexpr_1}{\sftype}$
  by the Preservation Lemma (\ref{lem:preservationFF}). 
  %
  Then by the inductive hypothesis
  applied to the reduction sequence $\evalsTo{\sexpr_1}{\sexpr'}$, 
  we conclude that either $\sexpr'$ is a value 
  or $\sexpr' \step \sexpr''$ for some $\sexpr''$ as desired. 

  (2) The second statement follows immediately from the first:
  we know that either $\sexpr'$ is a value and cannot be $\eerr$
  or $\sexpr'$ can take a step. But no rule in Figure \ref{fig:e}
  can be applied to reduce $\eerr$.
\end{proof}
%%
% %
As mentioned in the proof above,
we prove type safety by induction on the 
length of the sequence of steps comprising 
$\evalsTo{\sexpr}{\sexpr'}$, using the 
preservation and progress lemmas.

\subsection{Progress} \label{sec:sysf:progressF}
%
The progress lemma says a well-typed term is a value 
or steps to some other term.
%
\begin{lemma} (Progress) \label{lem:progressFF} 
If $\hasftype{\varnothing}{\sexpr}{\sftype}$, 
then $\sexpr$ is a value or $\sexpr \step \sexpr'$ for some $\sexpr'$.
\end{lemma}

\begin{proof} 
  We proceed by induction of the structure of 
  $\hasftype{\varnothing}{\sexpr}{\sftype}$. In the cases of rule
  \fPrim, \fVar, \fAbs, or \fTAbs, $\sexpr$ is a value.
  \begin{itemize}
  %%
  \pfcase{\fApp}: We have 
  $\hasftype{\varnothing}{\sexpr}{\sftype}$ where
  $\sexpr \equiv \app{\sexpr_1}{\sexpr_2}$. 
  Inverting, we have that there exists some type $\sftype_2$
  such that $\hasftype{\varnothing}{\sexpr_1}{\funcftype{\sftype_2}{\sftype}}$
  and $\hasftype{\varnothing}{\sexpr_2}{\sftype_2}$.
  We split on five possible cases for 
  the structure of $\sexpr_1$ and $\sexpr_2$. 
  %
  First, suppose $\sexpr_1 \equiv \vabs{x}{\sexpr_0}$ and $\sexpr_2$ is 
  a value. Then by rule \eAppAbs, 
  $e \equiv \app{\vabs{x}{\sexpr_0}}{\sexpr_2} \step \subst{\sexpr_0}{x}{\sexpr_2}$.
  %
  Second, suppose $\sexpr_1 \equiv \vabs{x}{\sexpr_0}$ and $\sexpr_2$
  is not a value. Then by the inductive hypothesis, there exists a term
  $\sexpr'_2$ such that $\sexpr_2 \step \sexpr'_2$. Then by rule \eAppV
  $e \equiv \app{\vabs{x}{\sexpr_0}}{\sexpr_2} \step \app{\vabs{x}{\sexpr_0}}{\sexpr'_2}$.
  %
  Third, suppose $\sexpr_1 \equiv \sconst$, a built in primitive 
  and $\sexpr_2$ is a value. Then by rule \ePrim, 
  $e \equiv \app{\sconst}{\sexpr_2} \step \delta(\sconst,\sexpr_2)$,
  which is well-defined by the Primitives Lemma (\ref{lem:primitivesF}).
  %
  Fourth, suppose $\sexpr_1 \equiv \sconst$ and $\sexpr_2$
  is not a value. Then by the inductive hypothesis, there exists a term
  $\sexpr'_2$ such that $\sexpr_2 \step \sexpr'_2$. Then by rule \eAppV
  $e \equiv \app{\sconst}{\sexpr_2} \step \app{\sconst}{\sexpr'_2}$.
  %
  Finally, by the Canonical Forms Lemma (\ref{lem:canonicalF}),
  $\sexpr_1$ cannot be any other
  value, so it must not be a value. Then by the inductive hypothesis,
  there is a term $\sexpr'_1$ such that $\sexpr_1 \step \sexpr'_1$. 
  Then by rule \eApp,
  $e \equiv \app{\sexpr_1}{\sexpr_2} \step \app{\sexpr'_1}{\sexpr_2}$.
  %%
  \pfcase{\fTApp}: We have 
  $\hasftype{\varnothing}{\sexpr}{\sftype}$ where
  $\sexpr \equiv \tyapp{\sexpr_1}{\stype}$ and
  $\sftype \equiv \subst{\sigma}{\tvar}{\forgetreft{\stype}}$. 
  Inverting, we have that 
  $\hasftype{\varnothing}{\sexpr_1}{\polytype{\tvar}{\skind}{\sigma}}$.
  We split on three cases for the structure of $\sexpr_1$.
  %
  First, suppose $\sexpr_1 \equiv \tabs{\tvar'}{\skind'}{\sexpr_0}$.
  Then by rule \eTAppAbs, 
  $\sexpr \equiv \tyapp{\tabs{\tvar'}{\skind'}{\sexpr_0}}{\stype} 
  \step \subst{\sexpr_0}{\tvar'}{\stype}$.
  %
  Second, suppose $\sexpr_1 \equiv \sconst$, a built in primitive.
  Then by rule \eTPrim, 
  $\sexpr \equiv  \tyapp{\sconst}{\stype} \step \delta_T(\sconst,\forgetreft{\stype})$,
  which is well-defined by the Primitives Lemma (\ref{lem:primitivesF}).
  %
  Finally, by the Canonical Forms Lemma (\ref{canonicalF}), 
  $\sexpr_1$ cannot be any other
  form of value, so it must not be a value. 
  Then by the inductive hypothesis,
  there is a term $\sexpr'_1$ such that $\sexpr_1 \step \sexpr'_1$. 
  Then by rule \eTApp
  $\sexpr \equiv \tyapp{\sexpr_1}{\stype} \step \tyapp{\sexpr'_1}{\stype}$.
  %%
  \pfcase{\fLet}: We have 
  $\hasftype{\varnothing}{\sexpr}{\sftype}$ where
  $\sexpr \equiv \eletin{x}{\sexpr_1}{\sexpr_2}$. Inverting,
  we have that $\hasftype{\varnothing}{\sexpr_1}{\sftype_1}$
  for some type $\sftype_1$. 
  By the inductive hypothesis, either $\sexpr_1$ is a value
  or there is a term $\sexpr'_1$ such that $\sexpr_1 \step \sexpr'_1$.
  In the former case, rule \eLetV gives us
  $e \equiv \eletin{x}{\sexpr_1}{\sexpr_2} \step \subst{\sexpr_2}{x}{\sexpr_1}$.
  In the latter case, by rule \eLet, 
  $e \equiv \eletin{x}{\sexpr_1}{\sexpr_2} \step \eletin{x}{\sexpr'_1}{\sexpr_2}$.
  %%
  \pfcase{\fAnn}: We have 
  $\hasftype{\varnothing}{\sexpr}{\sftype}$ where
  $\sexpr \equiv \tyann{\sexpr_1}{\stype}$. Inverting,
  we have the $\hasftype{\varnothing}{\sexpr_1}{\sftype}$ and
  $\sftype = \forgetreft{\stype}$. 
  By the inductive hypothesis, either $\sexpr_1$ is a value
  or there is a term $\sexpr'_1$ such that $\sexpr_1 \step \sexpr'_1$.
  In the former case, by rule \eAnnV, 
  $e \equiv \tyann{\sexpr_1}{\stype} \step \sexpr_1$.
  In the latter case, rule \eAnn gives us 
  $e \equiv \tyann{\sexpr_1}{\stype} step \tyann{\sexpr'_1}{\stype}$.
   %%
   \pfcase{\fIf}: We have 
   $\hasftype{\varnothing}{\sexpr}{\sftype}$ where
   $\sexpr \equiv \eif{\sexpr}{\sexpr_1}{\sexpr_2}$. Inverting,
   we have that $\hasftype{\varnothing}{\sexpr}{\tbool}$.
   By the inductive hypothesis, either $\sexpr$ is a value
   or there is a term $\sexpr'$ such that $\sexpr \step \sexpr'$.
   In the former case, by the Canonical Forms Lemma (\ref{lem:canonicalF}),
   we have that $\sexpr = \ttrue$ or $\sexpr = \tfalse$, and so
   either 
   $e \equiv \eif{\ttrue}{\sexpr_1}{\sexpr_2} \step \sexpr_1$
   by rule \eIfT or (respectively)
   $e \equiv \eif{\tfalse}{\sexpr_1}{\sexpr_2} \step \sexpr_2$
   by rule \eIfF.
   %
   In the latter case, by rule \eIf, 
   $e \equiv \eif{\sexpr}{\sexpr_1}{\sexpr_2} \step \eif{\sexpr'}{\sexpr_1}{\sexpr_2}$.
  %%
  \end{itemize}
  \end{proof}

\begin{fullversion}
%
The proof of progress given above requires a \emph{Canonical Forms} 
lemma (\cref{lem:canonicalF}) which describes the 
shape of well-typed values and some key properties 
about the built-in \emph{Primitives} (\cref{lem:primitivesF}).
%
We also implicitly use an \emph{Inversion of Typing} 
lemma (\cref{lem:inversionF}) which describes the shape of the 
type of well-typed terms and its subterms. For $\sysf$, 
unlike $\sysrf$, this lemma
is trivial because the typing relation is syntax-directed.


\begin{lemma}\label{lem:canonicalF} (Canonical Forms) 
\begin{enumerate}
    \item If $\hasftype{\varnothing}{v}{\tbool}$, 
        then $v = \ttrue$ or $v = \tfalse$.
    \item If $\hasftype{\varnothing}{v}{\tint}$, then $v$ is an integer constant.
    \item If $\hasftype{\varnothing}{v}{\funcftype{\sftype}{\sftype'}}$, 
        then either $v = \vabs{x}{\sexpr}$ or $v = \sconst$, 
        a constant function where 
        $\sconst \in \{\wedge, \vee, \neg, \leftrightarrow \}$.
    \item If $\hasftype{\varnothing}{v}{\polytype{\al}{\skind}{\sftype}}$, 
        then either $v = \tabs{\al}{\skind}{\sexpr}$ 
        or $v = \sconst$, a polymorphic constant $\sconst \in \{\leq, =\}$.
    \item If $\isWFFT{\varnothing}{\sftype}{\skbase}$,
        then $\sftype = \tbool$ or $\sftype = \tint$.
\end{enumerate}
\end{lemma}
%%
\begin{proof}
    Parts (1) - (4) are easily deduced from the \sysf typing rules 
    in Figure \ref{fig:t} and the definition of $ty(c)$. 
    Part (5) is clear from the well-formedness rules in Figure \ref{fig:wf}.
\end{proof}

We note that 
Lemma \ref{lem:canonicalF-a} is sufficient for our $\sysrf$ metatheory.
Our syntactic typing judgments in $\sysrf$ respect those of $\sysf$.
Specifically, if $\hastype{\tcenv}{\sexpr}{\stype}$ and 
$\isWellFormedE{\tcenv}$, then
$\hasftype{\forgetreft{\tcenv}}{\sexpr}{\forgetreft{\stype}}$.
Therefore, we do not have to state and 
prove a separate Canonical Forms Lemma for $\sysrf$.
%%

\begin{lemma}\label{lem:inversionF} (Inversion of Typing) 
    \begin{enumerate}% [leftmargin=*]
        \item If $\hasftype{\tcenv}{c}{\sftype}$, 
            then $\sftype = \forgetreft{\ty{c}}$.
        \item If $\hasftype{\tcenv}{x}{\sftype}$, 
            then $\bind{x}{\sftype} \in \tcenv$.
        \item If $\hasftype{\tcenv}{\app{e}{e_x}}{\sftype}$,
            then there exists type $\sftype_x$ such that  
            $\hasftype{\tcenv}{e}{\funcftype{\sftype_x}{\sftype}}$ and
            $\hasftype{\tcenv}{e_x}{\sftype_x}$.
        \item If $\hasftype{\tcenv}{\vabs{x}{e}}{\sftype}$,\! 
            then $\sftype = \funcftype{\sftype_x}{\sftype'}$ and
            $\hasftype{\bind{y}{\sftype_x},\tcenv}{\subst{e}{x}{y}}{\sftype'}$
            for any $\notmem{y}{\tcenv}$ and well-formed $\sftype_x$.
        \item If $\hasftype{\tcenv}{\tyapp{e}{t}}{\sftype}$,\! then there exists 
            type $\sigma$ and kind $\skind$ such that 
            $\hasftype{\tcenv}{e}{\polytype{\al}{\skind}{\sigma}}$
            and $\sftype = \subst{\sigma}{\al}{\forgetreft{t}}$.
        \item If $\hasftype{\tcenv}{\tabs{\al}{\skind}{e}}{\sftype}$, then
            there exists type $\sftype'$ and kind $\skind$ such that
            $\sftype = {\polytype{\al}{\skind}{\sftype'}}$ and
            $\hasftype{\bind{\al'}{\skind},\tcenv}{\subst{e}{\al}{\al'}}
            {\subst{\sftype'}{\al}{\al'}}$ for some $\notmem{\al'}{\tcenv}$.
        \item If $\hasftype{\tcenv}{\eletin{x}{e_x}{e}}{\sftype}$, then
            there exists type $\sftype_x$ and $\notmem{y}{\tcenv}$ such that
            $\hasftype{\tcenv}{e_x}{\sftype_x}$ and 
            $\hasftype{\bind{y}{\sftype_x},\tcenv}{\subst{e}{x}{y}}{\sftype}$.
        \item If $\hasftype{\tcenv}{\tyann{e}{t}}{\sftype}$, then 
            $\sftype = \forgetreft{t}$ and $\hasftype{\tcenv}{e}{\sftype}$.
        
        \item If $\hasftype{\tcenv}{\eif{e}{e_1}{e_2}}{\sftype}$, then
            $\hasftype{\tcenv}{e}{\tbool}$,
            $\hasftype{\tcenv}{e_1}{\sftype}$, and 
            $\hasftype{\tcenv}{e_2}{\sftype}$.
    \end{enumerate}
\end{lemma}
%%
\begin{proof}
This is clear from the definition of the typing rules for $\sysf$. Each premise
can match only one rule because the $\sysf$ rules are syntax directed.
\end{proof}
%%
The Inversion of Typing Lemma does not hold in $\sysrf$ due to the subtyping
relation. For instance 
$\hastype{\bind{x}{\breft{\tint}{\vv}{\vv = 5}}}{x}{\tint}$ but
$\notmem{\bind{x}{\tint}}{\bind{x}{\breft{\tint}{\vv}{\vv = 5}}}$.
In Lemma \ref{lem:inversion} we state and prove an analogous result 
for $\sysrf$ in the two cases needed to prove progress and preservation.
%% % TODO ^^^ recheck that this lemma is still there!

%\mypara{Primitives}
%
For each primitive constant or function, we need 
to know that the type $\forgetreft{\ty{\sconst}}$ 
relates to the $\sysf$ type of $\delta(\sconst,v)$ 
in the same manner as \fApp.
%
% While we could derive this from (\ref{prim-typing}), 
% it is straightforward enough to prove from scratch 
% in the mechanization.

\begin{lemma}\label{lem:primitivesF}(Primitives) 
For each built-in primitive $c$, 
%
\begin{enumerate} 
\item If $\forgetreft{\ty{\sconst}} = \funcftype{\sftype_x}{\sftype}$
    and $\hasftype{\varnothing}{v_x}{\sftype_x}$, 
    then % $\delta(\sconst,v)$ is defined and 
    $\hasftype{\varnothing}{\delta(\sconst,v_x)}{\sftype}$.
\item If $\forgetreft{\ty{\sconst}} = \polytype{\al}{\skind}{\sftype}$ 
    and $\isWellFormed{\varnothing}{\sftype_\al}{\skind}$, 
    then % $\delta_T(\sconst,\stype)$ is defined and we have 
    $\hasftype{\varnothing}{\delta_T(\sconst,\sftype_\al)}{\subst{\sftype}{\al}{\sftype_\al}}$.
    \end{enumerate}
\end{lemma}
%%
\begin{proof}
  \begin{enumerate}
      \item First consider $\sconst \in \{\wedge, \vee, \neg\}$. 
          Then $\forgetreft{\ty{\sconst}} = \funcftype{\tbool}{\funcftype{\tbool}{\tbool}}$.
          Then by Lemma \ref{lem:canonicalF}, $\hasftype{\varnothing}{\sval}{\tbool}$
          gives us that $\sval = \ttrue$ or $\sval = \tfalse$.
          For each possibility for $\sconst$ and $\sval$, we can build a judgment 
          that $\hasftype{\varnothing}{\delta(\sconst,\sval)}{\funcftype{\tbool}{\tbool}}$.
          Similarly, if $\sconst = \neg$ 
          then $\forgetreft{\ty{\sconst}} = \funcftype{\tbool}{\tbool}$ and 
          $\delta(\neg,\sval) \in \{\ttrue,\tfalse\}$ can be typed at $\tbool$.
          The analysis for the other monomorphic primitives is entirely similar.
      \item Here $\sconst$ is the polymorphic $=$ and 
          $\forgetreft{\ty{\sconst}} = \polytype{\tvar}{\skbase}{\funcftype{\al}{\funcftype{\al}{\tbool}}}$. By the Canonical Forms Lemma, 
          $\sftype = \tbool$ or $\sftype = \tint$. In the former case,
          $\delta_T(\sconst,\tbool) = \leftrightarrow$, which we can type at 
          $\funcftype{\tbool}{\funcftype{\tbool}{\tbool}} =\subst{\forgetreft{\ty{\sconst}}}{\tvar}{\tbool}$. The case of $\tint$ is entirely similar
          because $\delta_T(\sconst,\tint)$ is the monomorphic integer equality.
  \end{enumerate}
  \end{proof}

%\NV{Michael check the below}
%Lemmas~\ref{lem:canonicalF} and~\ref{lem:inversionF} are proved without
%induction by inspection of the derivation tree, 
%while lemma~\ref{lem:primitivesF} relies on 
%the Primitives Requirement~\ref{lem:prim-typing}. 

\end{fullversion}

\subsection{Preservation} \label{sec:sysf:preservationF}
%
The preservation lemma states that $\sysf$ typing is preserved
by evaluation.
%
\begin{lemma} (Preservation) \label{lem:preservationFF} 
If $\hasftype{\varnothing}{\sexpr}{\sftype}$ and $\sexpr \step \sexpr'$, 
then $\hasftype{\varnothing}{\sexpr'}{\sftype}$.
\end{lemma}    
%%
\begin{proof} 
  We proceed by induction of the structure of 
  $\hasftype{\varnothing}{\sexpr}{\sftype}$. The cases of rules
  \fPrim, \fVar, \fAbs, or \fTAbs cannot occur because $\sexpr$ is a value
  and no value can take a step in our semantics.
  %%
  The interesting cases are for \fApp and \fTApp.
  %
  For applications of primitives, preservation 
  requires the Primitives~\Cref{lem:primitivesF},
  while the general case needs a Substitution~\Cref{lem:substitutionF}.
  %
  We now give the full details of the other five cases:

  \begin{itemize}
  %%
  \pfcase{\fApp}: We have 
  $\hasftype{\varnothing}{\sexpr}{\sftype}$ where
  $\sexpr \equiv \app{\sexpr_1}{\sexpr_2}$. 
  Inverting, we have that there exists some type $\sftype_2$
  such that $\hasftype{\varnothing}{\sexpr_1}{\funcftype{\sftype_2}{\sftype}}$
  and $\hasftype{\varnothing}{\sexpr_2}{\sftype_2}$.
  We split on five possible cases for 
  the structure of $\sexpr_1$ and $\sexpr_2$. 
  %
  First, suppose $\sexpr_1 \equiv \vabs{x}{\sexpr_0}$ and $\sexpr_2$ is 
  a value. Then by rule \eAppAbs and the determinism of our semantics, 
  $e' \equiv \subst{\sexpr_0}{x}{\sexpr_2}$.
  By the Inversion of Typing (\ref{lem:inversionF}), for some $y$ we have
  $\hasftype{\bind{y}{\sftype_2}}{\subst{\sexpr_0}{x}{y}}{\sftype}$.
  By the Substitution Lemma (\ref{lem:substitutionFF}), 
  substituting $\sexpr_2$ through for $y$
  gives us $\hasftype{\varnothing}{\subst{\sexpr_0}{x}{\sexpr_2}}{\sftype}$
  as desired because $\subst{\subst{\sexpr_0}{x}{y}}{y}{\sexpr_2} = \subst{\sexpr_0}{x}{\sexpr_2}$.
  %
  Second, suppose $\sexpr_1 \equiv \vabs{x}{\sexpr_0}$ and $\sexpr_2$
  is not a value. Then by the progress lemma (\ref{lem:progressFF}), 
  there exists a term
  $\sexpr'_2$ such that $\sexpr_2 \step \sexpr'_2$. Then by rule \eAppV
  and the determinism of our semantics,
  $e' \equiv \app{\vabs{x}{\sexpr_0}}{\sexpr'_2}$. 
  Now, by the inductive hypothesis, 
  $\hasftype{\varnothing}{\sexpr'_2}{\sftype_2}$.
  Applying rule \fApp, 
  $\hasftype{\varnothing}{\app{\sexpr_1}{\sexpr'_2}}{\sftype}$
  as desired.
  %
  Third, suppose $\sexpr_1 \equiv \sconst$, a built in primitive, 
  and $\sexpr_2$ is a value. Then by rule \ePrim
  and the determinism of the semantics, 
  $e' \equiv \delta(\sconst,\sexpr_2)$.
  By the primitives lemma, 
  $\hasftype{\varnothing}{\delta(\sconst,\sexpr_2)}{\sftype}$ as desired.
  %
  Fourth, suppose $\sexpr_1 \equiv \sconst$ and $\sexpr_2$
  is not a value. Then we argue in the same manner as the second case.
  %
  Finally, by the canonical forms lemma, $\sexpr_1$ cannot be any other
  value, so it must not be a value. Then by the progress lemma,
  there is a term $\sexpr'_1$ such that $\sexpr_1 \step \sexpr'_1$. 
  Then by rule \eApp and the determinism of the semantics,
  $e' \equiv \app{\sexpr'_1}{\sexpr_2}$. By the inductive hypothesis,
  $\hasftype{\varnothing}{\sexpr'_1}{\funcftype{\sftype_2}{\sftype}}$.
  Applying rule \fApp, $\hasftype{\varnothing}{\app{\sexpr'_1}{\sexpr_2}}{\sftype}$
  as desired.
  %%
  \pfcase{\fTApp}: We have 
  $\hasftype{\varnothing}{\sexpr}{\sftype}$ where
  $\sexpr \equiv \tyapp{\sexpr_1}{\stype}$ and
  $\sftype \equiv \subst{\sigma}{\tvar}{\forgetreft{\stype}}$. 
  Inverting (\ref{lem:inversionF}), we have that 
  $\hasftype{\varnothing}{\sexpr_1}{\polytype{\tvar}{\skind}{\sigma}}$
  and $\isWFFT{\varnothing}{\forgetreft{\stype}}{\skind}$.
  We split on three cases for the structure of $\sexpr_1$.
  %
  First, suppose $\sexpr_1 \equiv \tabs{\tvar}{\skind}{\sexpr_0}$.
  Then by rule \eTAppAbs and the determinism of the semantics, 
  $\sexpr' \equiv \subst{\sexpr_0}{\tvar}{\stype}$.
  By the inversion of typing, for some $\tvar'$, we have
  $\hasftype{\bind{\tvar'}{\skind}}{\subst{\sexpr_0}{\tvar}{\tvar'}}
  {\subst{\sigma}{\tvar}{\tvar'}}$.
  By the Substitution Lemma (\ref{lem:substitutionFF}), 
  substituting $\forgetreft{\stype}$ 
  through for $\tvar$  gives us 
  $\hasftype{\varnothing}{\subst{\sexpr_0}{\tvar}{\stype}}{\subst{\sigma}{\tvar}{\forgetreft{\stype}}}$ as desired.
  %
  Second, suppose $\sexpr_1 \equiv \sconst$, a built in primitive.
  Then by rule \eTPrim and the determinism of the semantics, 
  $\sexpr' \delta_T(\sconst,\forgetreft{\stype})$. By the primitives lemma,
  $\hasftype{\varnothing}{\delta_T(\sconst,\forgetreft{\stype})}{\subst{\sigma}{\tvar}{\forgetreft{\stype}}}$.
  %
  Finally, by the canonical forms lemma, $\sexpr_1$ cannot be any other
  form of value, so it must not be a value. Then by the progress lemma,
  there is a term $\sexpr'_1$ such that $\sexpr_1 \step \sexpr'_1$. 
  Then by rule \eTApp and the deterministic semantics
  $\sexpr' \equiv \tyapp{\sexpr'_1}{\stype}$.  By the inductive hypothesis,
  $\hasftype{\varnothing}{\sexpr'_1}{\polytype{\tvar}{\skind}{\sigma}}$.
  Applying rule \fTApp, 
  $\hasftype{\varnothing}{\tyapp{\sexpr'_1}{\stype}}{\subst{\sigma}{\tvar}{\forgetreft{\stype}}}$
  as desired.
  %%
  \pfcase{\fLet}: We have 
  $\hasftype{\varnothing}{\sexpr}{\sftype}$ where
  $\sexpr \equiv \eletin{x}{\sexpr_1}{\sexpr_2}$. Inverting,
  we have that 
  $\hasftype{\bind{y}{\sftype_1}}{\subst{\sexpr_2}{x}{y}}{\sftype}$
  and $\hasftype{\varnothing}{\sexpr_1}{\sftype_1}$
  for some type $\sftype_1$. 
  By the progress lemma either $\sexpr_1$ is a value
  or there is a term $\sexpr'_1$ such that $\sexpr_1 \step \sexpr'_1$.
  %
  In the former case, rule \eLetV and determinism give us
  $\sexpr' \equiv \subst{\sexpr_2}{x}{\sexpr_1}$.
  By the Substitution Lemma (substituting $\sexpr_1$ for $x$), 
  we have $\hasftype{\varnothing}{\subst{\sexpr_2}{x}{\sexpr_1}}{\sftype}$
  as desired because 
  $\subst{\sexpr_2}{x}{\sexpr_1} = \subst{\subst{\sexpr_2}{x}{y}}{y}{\sexpr_1}$.
  %
  In the latter case, by rule \eLet and determinism give us, 
  $\sexpr' \equiv \eletin{x}{\sexpr'_1}{\sexpr_2}$.
  By the inductive hypothesis we have that 
  $\hasftype{\varnothing}{\sexpr'_1}{\sftype_1}$ and by rule \fLet 
  we have $\hasftype{\varnothing}{\eletin{x}{\sexpr'_1}{\sexpr_2}}{\sftype}$. 
  %%
  \pfcase{\fAnn}: We have 
  $\hasftype{\varnothing}{\sexpr}{\sftype}$ where
  $\sexpr \equiv \tyann{\sexpr_1}{\stype}$. Inverting,
  we have the $\hasftype{\varnothing}{\sexpr_1}{\sftype}$ and
  $\sftype = \forgetreft{\stype}$. 
  By the progress lemma, either $\sexpr_1$ is a value
  or there is a term $\sexpr'_1$ such that $\sexpr_1 \step \sexpr'_1$.
  %
  In the former case, by rule \eAnnV and the determinism of the semantics, 
  $\sexpr' \equiv \sexpr_1$. Then we already have that
  $\hasftype{\varnothing}{\sexpr'}{\sftype}$
  %
  In the latter case, rule \eAnn and determinism give us 
  $\sexpr' \equiv \tyann{\sexpr'_1}{\stype}$. By the inductive hypothesis
  we have that $\hasftype{\varnothing}{\sexpr'_1}{\sftype}$. By rule
  \fAnn we conclude $\hasftype{\varnothing}{\tyann{\sexpr'_1}{\stype}}{\sftype}$.
  %%
  \pfcase{\fIf}: We have 
  $\hasftype{\varnothing}{\sexpr}{\sftype}$ where
  $\sexpr \equiv \eif{\sexpr_0}{\sexpr_1}{\sexpr_2}$. Inverting,
  we have that 
  $\hasftype{\varnothing}{\sexpr_0}{\tbool}$
  and
  $\hasftype{\varnothing}{\sexpr_1}{\sftype}$,
  and
  $\hasftype{\varnothing}{\sexpr_2}{\sftype}$.
  %
  By the progress lemma either $\sexpr_0$ is a value
  or there is a term $\sexpr'_0$ such that $\sexpr_0 \step \sexpr'_0$.
  %
  In the former case, the Canonical Forms Lemma (\ref{lem:canonicalF})
  tells us that $\sexpr_0 = \ttrue$ or $\tfalse$.
  By determinism of the semantics we have that 
  $\sexpr' \equiv {\sexpr_1}$
  or $\sexpr' \equiv {\sexpr_2}$ respectively.
  %
  In either case we can immediately conclude that $\sexpr'$
  has the desired type.
  %
  In the latter case above,rule \eIf and determinism give us, 
  $\sexpr' \equiv \eif{\sexpr'_0}{\sexpr'_1}{\sexpr_2}$.
  By the inductive hypothesis we have that 
  $\hasftype{\varnothing}{\sexpr'_0}{\tbool}$ and by rule \fIf 
  we have 
  $\hasftype{\varnothing}{\eif{\sexpr'_0}{\sexpr_1}{\sexpr_2}}{\sftype}$. 
  %%
  \end{itemize}
\end{proof}
%
The proof of preservation for $\sysrf$ differs in two cases above. 
In \tApp and \tTApp, we must use the 
Inversion of Typing lemma (\ref{lem:inversion})
from $\sysrf$ because the presence of rule \tSub prevents us from 
inferring the last rule used to type a term or type abstraction.
%
Furthermore, in case \tApp the substitution lemma would give us that 
$\hastype{\varnothing}{\sexpr'}{\subst{\stype}{x}{\sval_x}}$ for 
some value $\sval_x$. However we need to show preservation of the
existential type $\existype{x}{\stype_x}{\stype}$. This is done by
using rule \sWitn to show that, in fact, 
$\isSubType{\varnothing}{\subst{\stype}{x}{\sval_x}}{\existype{x}{\stype_x}{\stype}}$.
%%

\mypara{Substitution Lemma}
%
To prove type preservation when
a lambda or type abstraction is applied, 
we proved that the substituted result 
has the same type, as established by 
the Substitution Lemma:
%
\begin{lemma}(Substitution)\label{lem:substitutionFF}
If $\hasftype{\tcenv}{\sval_x}{\sftype_x}$ 
and $\isWFFT{\tcenv}{\forgetreft{\stype_{\al}}}{k_{\al}}$, then 
\begin{enumerate}
\item if\; $\hasftype{\tcenv', \bind{x}{\sftype_x}, \tcenv}{\sexpr}{\sftype}$
    and\; $\isWFFE{\tcenv}$, then
    ${\hasftype{\tcenv', \tcenv}{\subst{\sexpr}{x}{\sval_x}}{\sftype}}$ and 
\item if\; $\hasftype{\tcenv', \bind{\al}{k_{\al}}, \tcenv}{\sexpr}{\sftype}$
    and\; $\isWFFE{\tcenv}$, then
    ${\hasftype{\subst{\tcenv'}{\al}{\forgetreft{\stype_{\al}}}, \tcenv}
               {\subst{\sexpr}{\al}{\stype_{\al}}}
               {\subst{\sftype}{\al}{\forgetreft{\stype_{\al}}}}}$.
\end{enumerate}
\end{lemma}
%%
\begin{proof}
  We give the proofs for part (2); part (1) is similar but slightly
  simpler because term variables do not appear in types in $\sysf$
  We proceed by induction on the derivation tree of the typing judgment
  $\hasftype{\tcenv', \bind{\al}{k_{\al}}, \tcenv}{\sexpr}{\sftype}$.
  %%
  \pfcase{\fPrim}: We have $\sexpr \equiv \sconst$ and
  $\hasftype{\tcenv', \bind{\al}{k_{\al}}, \tcenv}{\sconst}{\forgetreft{\ty{\sconst}}}$.
  Neither $\sconst$ nor $\ty{\sconst}$ has any free variables, so each is 
  unchanged under substitution. Then by rule \tPrim we conclude
  $\hasftype{\subst{\tcenv'}{\al}{\forgetreft{\stype_{\al}}}, \tcenv}{\sconst}{\forgetreft{\ty{\sconst}}}$ because the environment may be chosen arbitrarily.
  %%
  \pfcase{\fVar}: We have $\sexpr \equiv x$; by inversion, we get that 
  $\bind{x}{\sftype} \in \tcenv', \bind{\al}{k_{\al}}, \tcenv$. We must have
  $\tvar \neq x$ so there are two cases to consider for where $x$ can 
  appear in the environment. If $\bind{x}{\sftype} \in \tcenv$,  then $\sftype$
  cannot contain $\tvar$ as a free variable because $x$ is bound first in the environment (which grows from right to left). Then
  $\hasftype{\subst{\tcenv'}{\al}{\forgetreft{\stype_{\al}}}, \tcenv}{x}{\sftype}$
  as desired because $\subst{\sftype}{\tvar}{\forgetreft{\stype_{\al}}} = \sftype$.
  Otherwise $\bind{x}{\sftype} \in \tcenv'$ and so 
  $\bind{x}{\subst{\sftype}{\tvar}{\forgetreft{\stype_{\al}}}} \in \subst{\tcenv'}{\al}{\forgetreft{\stype_{\al}}}, \tcenv$. Thus
  $\hasftype{\subst{\tcenv'}{\al}{\forgetreft{\stype_{\al}}}, \tcenv}{x}{\subst{\sftype}{\tvar}{\forgetreft{\stype_{\al}}}}$.
  %
  (In part (1), we have an additional case where 
  $\hasftype{\tcenv', \bind{x}{\sftype}, \tcenv}{x}{\sftype}$. We have
  $\subst{x}{x}{\sval_x} = \sval_x$ and so we can apply 
  the Weakening Lemma (\ref{lem:weakeningF})
  to $\hasftype{\tcenv}{\sval_x}{\sftype}$ to obtain
  $\hasftype{\tcenv', \tcenv}{\sval_x}{\sftype}$.)
  %%
  \pfcase{\fApp}: We have $\sexpr \equiv \app{\sexpr_1}{\sexpr_2}$. By inversion
  we have that
  $\hasftype{\tcenv', \bind{\al}{k_{\al}}, \tcenv}{\sexpr_1}{\funcftype{\sftype_x}{\sftype}}$
  and $\hasftype{\tcenv', \bind{\al}{k_{\al}}, \tcenv}{\sexpr_2}{\sftype_x}$.
  Applying the inductive hypothesis to both of these, we get
  $\hasftype{\subst{\tcenv'}{\al}{\forgetreft{\stype_{\al}}}, \tcenv}{\subst{\sexpr_1}{\tvar}{\stype_{\tvar}}}{\subst{\funcftype{\sftype_x}{\sftype}}{\tvar}{\forgetreft{\stype_{\tvar}}}}$
  and
  $\hasftype{\subst{\tcenv'}{\al}{\forgetreft{\stype_{\al}}}, \tcenv}{\subst{\sexpr_2}{\tvar}{\stype_{\tvar}}}{\subst{\sftype_x}{\tvar}{\forgetreft{\stype_{\tvar}}}}$. 
  Combining these by rule \fApp, we conclude
  $\hasftype{\subst{\tcenv'}{\al}{\forgetreft{\stype_{\al}}}, \tcenv}{\subst{\app{\sexpr_1}{\sexpr_2}}{\tvar}{\stype_{\tvar}}}{\subst{\sftype}{\tvar}{\forgetreft{\stype_{\tvar}}}}$.
  %%
  \pfcase{\fAbs}: We have $\sexpr \equiv \vabs{x}{\sexpr_1}$ 
  and $\sftype \equiv \funcftype{\sftype_x}{\sftype_1}$. 
  By inversion we have that for any fresh $y$, both
  $\hasftype{\bind{y}{\sftype_x},\tcenv', \bind{\al}{k_{\al}}, \tcenv}
  {\subst{\sexpr_1}{x}{y}}{\sftype_1}$
  and
  $\isWFFT{\tcenv', \bind{\al}{k_{\al}}, \tcenv}{\sftype_x}{\skind_x}$.
  By the inductive hypothesis, and the Substitution Lemma for 
  well-formedness judgments, we have
  $\hasftype{\bind{y}{\subst{\sftype_x}{\al}{\forgetreft{\stype_{\al}}}},\subst{\tcenv'}{\al}{\forgetreft{\stype_{\al}}}, \tcenv}
  {\subst{\subst{\sexpr_1}{\al}{\stype_{\al}}}{x}{y}}{\subst{\sftype_1}{\al}{\forgetreft{\stype_{\al}}}}$
  and
  $\isWFFT{\subst{\tcenv'}{\al}{\forgetreft{\stype_{\al}}}, \tcenv}
  {\subst{\sftype_x}{\al}{\forgetreft{\stype_{\al}}}}{\skind_x}$,
  where we can switch the order of substitutions because $y$ does not 
  appear free in the well-formed type $\stype_{\tvar}$.
  Then we can conclude by applying rule \fAbs that 
  $\hasftype{\subst{\tcenv'}{\al}{\forgetreft{\stype_{\al}}}, \tcenv}
  {\subst{\vabs{x}{\sexpr_1}}{\tvar}{\stype_\tvar}}
  {\subst{\funcftype{\sftype_x}{\sftype_1}}{\tvar}{\forgetreft{\stype_\tvar}}}$.
  %%
  \pfcase{\fTApp}: We have $e \equiv e'\; [\stype']$ 
  and $\sftype \equiv \subst{\sftype'}{\al'}{\forgetreft{\stype'}}$. 
  By inversion, 
  $\hasftype{\tcenv', \bind{\al}{k_{\al}}, \tcenv}{e'}{\polytype{\al'}{k'}{\sftype'}}$
  and $\isWFFT{\tcenv', \bind{\al}{k_{\al}}, \tcenv}{\forgetreft{\stype'}}{k'}$.
  By the inductive hypothesis,\\
  $\hasftype{\subst{\tcenv'}{\al}{\forgetreft{\stype_{\al}}}, \tcenv}
    {\subst{e'}{\al}{\stype_{\al}}}
    {\polytype{\al'}{k'}{\subst{\sftype'}{\al}{\forgetreft{\stype_{\al}}}}}$
  %\]
  and
  %\begin{equation}\label{362T}
  $\isWFFT{\subst{\tcenv'}{\al}{\forgetreft{\stype_{\al}}}, \tcenv}
    {\subst{\forgetreft{\stype'}}{\al}{\forgetreft{\stype_{\al}}}}{k'}$.
  %\end{equation}
  By the definition of refinement erasure,
  $ \subst{\forgetreft{\stype}}{\al'}{\forgetreft{\stype_{\al}}} 
      = \forgetreft{\subst{\stype'}{\al}{\stype_{\al}}}$.
  By applying rule \fTApp, we get
  %\begin{equation}
  $\hasftype{\subst{\tcenv'}{\al}{\forgetreft{\stype_{\al}}}, \tcenv}
    {\tyapp{\subst{e'}{\al}{\stype_{\al}}}{\subst{\stype'}{\al}{\stype_{\al}}}}
    {\subst{\subst{\sftype'}{\al}{\forgetreft{\stype_{\al}}}}
      {\al'}{\forgetreft{\subst{\stype'}{\al}{\stype_{\al}}}}}$.
  %
  By the definition of substitution we have 
  $\tyapp{\subst{e'}{\al}{\stype_{\al}}}{\subst{\stype'}{\al}{\stype_{\al}}} 
      = \subst{\tyapp{e'}{t'}}{\al}{\stype_{\al}}$
  and by the commutativity rules for substitution,\\
  $ \subst{\subst{\sftype'}{\al}{\forgetreft{\stype_{\al}}}}
      {\al'}{\forgetreft{\subst{\stype'}{\al}{\stype_{\al}}}}
    = \subst{\subst{\sftype'}{\al'}{\forgetreft{t'}}}{\al}{\forgetreft{\stype_{\al}}}$.
  %
  Therefore, we conclude that 
  $\hasftype{\subst{\tcenv'}{\al}{\forgetreft{\stype_{\al}}}, \tcenv}
    {\subst{\sexpr}{\al}{\stype_{\al}}}
    {\subst{\sftype}{\al}{\forgetreft{\stype_{\al}}}}$.
  %%
  \pfcase{\fTAbs}: TODO typesetting
  %%
%{\bf Case} {\sc T-AbsT}:We have $\Gamma', x:t_x,\Gamma \vdash e : t$ where $e \equiv \Lambda \al:k. e'$ and $t \equiv \polytype{\al}{k}{t'}$. By inversion, $\al'\bind k, \Gamma', x\bind t_x,\Gamma \vdash e'[\al'/\al] : t'[\al'/\al]$ and $\al'\bind k, \Gamma', x\bind t_x,\Gamma \vdash_w t'[\al'/\al] : k'$ for some $\al'\not\in\dom{\Gamma}$. By the inductive hypothesis
%\begin{equation}
%\al'\bind k, \Gamma'[v_x/x],\Gamma \vdash e'[\al'/\al][v_x/x] : t'[\al'/\al][v_x/x] \;\;{\rm and}\;\; \al'\bind k, \Gamma'[v_x/x],\Gamma \vdash_w t'[\al'/\al][v_x/x] : k'.
%\end{equation}
%We must have $x\neq \al'$ and we have $x \neq \al$ because bound and free variables are taken to be distinct. Moreover, $v_x$ contains only free variables from $\Gamma$, so $e'[\al'/\al][v_x/x] = e'[v_x/x][\al'/\al]$ and $t'[\al'/\al][v_x/x] = t'[v_x/x][\al'/\al]$. Then by rule {\sc T-AbsT}
%\begin{equation}
%\Gamma'[v_x/x], \Gamma' \vdash \Lambda \al:k.(e'[v_x/x]) : \polytype{\al}{k}{t'[v_x/x]}.
%\end{equation}
%By definition of substitution, we can rewrite the above as
%\[
%\Gamma'[v_x/x],\Gamma \vdash (\Lambda \al:k.e')[v_x/x] : \polytype{\al}{k}{t'}[v_x/x].
%\]
%In the type variable substitution lemma, where we have an environment $\Gamma',\beta\bind k_\beta,\Gamma$ and $\Gamma \vdash_w t_\beta : k_\beta$ this proof is similar except that we argue that $e'[\al'/\al][t_\beta/\beta] = e'[t_\beta/\beta][\al'/\al]$ and $t'[\al'/\al][t_\beta/\beta] = t'[t_\beta/\beta][\al'/\al]$ because $\al'\neq\beta$ and only free variables from $\Gamma$ may appear in $t_\beta$.

%% 
\pfcase{\fLet}: We have $\sexpr \equiv \eletin{x}{\sexpr_1}{\sexpr_2}$
and by inversion we have that for some type $\sftype_1$,
$\hasftype{\tcenv', \bind{\al}{k_{\al}}, \tcenv}{\sexpr_1}{\sftype_1}$
and for some $\notmem{y}{\tcenv', \bind{\al}{k_{\al}}, \tcenv}$,
$\hasftype{\bind{y}{\sftype_1},\tcenv', \bind{\al}{k_{\al}}, \tcenv}
{\subst{\sexpr_2}{x}{y}}{\sftype}$.
By the inductive hypothesis, we have that
$\hasftype{\subst{\tcenv'}{\al}{\forgetreft{\stype_{\al}}}, \tcenv}
{\subst{\sexpr_1}{\tvar}{\stype_\tvar}}{\subst{\sftype_1}{\tvar}{\forgetreft{\stype_\tvar}}}$
and
$\hasftype{\bind{y}{\subst{\sftype_1}{\tvar}{\stype_\tvar}},\subst{\tcenv'}{\al}{\forgetreft{\stype_{\al}}}, \tcenv}
{\subst{\sexpr_2}{x}{\stype_\tvar}}{\subst{\sftype}{\tvar}{\forgetreft{\stype_\tvar}}}$.
Then by rule \fLet we conclude
$\hasftype{\subst{\tcenv'}{\al}{\forgetreft{\stype_{\al}}}, \tcenv}
{\eletin{x}{\subst{\sexpr_1}{\tvar}{\stype_\tvar}}{\subst{\sexpr_2}{\tvar}{\stype_\tvar}}}{\subst{\sftype}{\tvar}{\forgetreft{\stype_\tvar}}}$.
%%
\pfcase{\fAnn}: We have $\sexpr \equiv \tyann{\sexpr'}{\stype}$ 
and by inversion we have that $\forgetreft{\stype} = \sftype$ and 
$\hasftype{\tcenv', \bind{\al}{k_{\al}}, \tcenv}
{\sexpr'}{\sftype}$.
By the inductive hypothesis, we have
$\hasftype{\subst{\tcenv'}{\al}{\forgetreft{\stype_{\al}}}, \tcenv}
{\subst{\sexpr'}{\tvar}{\stype_\tvar}}{\subst{\sftype}{\tvar}{\forgetreft{\stype_\tvar}}}$.
By our definition of refinement erasure, we have
$\forgetreft{\subst{\stype}{\tvar}{\stype_\tvar}}
 = \subst{\forgetreft{\stype}}{\tvar}{\forgetreft{\stype_\tvar}}$
and we have 
$\subst{\tyann{\sexpr'}{\stype}}{\tvar}{\stype_\tvar} 
 = \tyann{\subst{\sexpr'}{\tvar}{\stype_\tvar}}{\subst{\stype}{\tvar}{\stype_\tvar}}$. Thus by rule \fAnn,
$\hasftype{\subst{\tcenv'}{\al}{\forgetreft{\stype_{\al}}}, \tcenv}
{\subst{\tyann{\sexpr'}{\stype}}{\tvar}{\stype_\tvar}}{\subst{\forgetreft{\stype}}{\tvar}{\forgetreft{\stype_\tvar}}}$.
%%
\pfcase{\fIf}: TODO Typesetting
%%
\end{proof}  

%\NV{Edited, Michael check}
%The proof goes by induction on the derivation tree. 
Because we encoded our typing rules using 
cofinite quantification % (\S~\ref{sec:lang:static})
the proof above does not require a renaming lemma, but 
the rules that lookup environments 
(rules \tVar and \wtVar) do need a \emph{Weakening Lemma}:
%\Cref{lem:weakeningF}. 
% and \emph{rename} free variables 
%in typing and well-formedness judgments (\cref{lem:freevarsF}).

\begin{lemma}(Weakening) \label{lem:weakeningF}
If $\hasftype{\tcenv_1,\!\tcenv_2\!}{\!\sexpr\!}{\!\sftype}$ and 
   $x,\!\al\! \not\in\! {\tcenv_1, \!\tcenv_2}$,
then 
\begin{enumerate} 
    \item 
    \hasftype{\tcenv_1,\!\bind{x}{\sftype_x},\!\tcenv_2}{\!\sexpr\!}{\!\sftype}, and
    \item 
    \hasftype{\tcenv_1,\!\bind{\al}{\skind},\!\tcenv_2}{\!\sexpr\!}{\!\sftype}.
\end{enumerate}
\end{lemma}
%
The proof is fairly similar to the proof of the Substitution Lemma above,
and we omit it here.


% \begin{proof}
% The proof is by straightforward structural induction on the derivation trees of $\sysf$ typing judgments, weakening the typing and well-formedness judgments in the antecedents. Where needed, we apply the Change of Variables Lemma below to ensure that anything we add to the environment is distinct from the larger, weakened environment.
% \end{proof}

% We also use the lemma that types in typing judgments are well-formed. 

% The proof of the substitution lemma 
% depends on the Weakening Lemma and 
% the Change 
% of Variables Lemma which allow us to 
% perform certain technical manipulations 
% on judgments. 
% %
% Each of these three lemmas has versions 
% that apply to both typing judgments and 
% well-formedness judgments.

% Lemma \ref{canonicalF} further determines all of the possibilities for well-typed closed values; any other well-typed closed term can be further evaluated. The preservation theorem
% also relies on Definition \ref{primitivesF} that types are preserved under the defined semantics for the built-in primitives.
% The main technical challenge in the preservation theorem arises when 
% our closed term has the form $\app{(\vabs{x}{\sexpr})}{\sval}$
% or $\tyapp{(\tabs{\al}{\skind}{\sexpr})}{\stype}$;
% the next step of the evaluation requires substituting 
% for a term variable or type variable 
% in the abstraction body $e$, and in the latter case of a type abstraction, 
% substituting in the type itself. 
% The relationship between a valid substitution and a typing
% judgment is given in the Lemma \ref{substitutionF}, the Substitution Lemma. 
% Two technical lemmas are required to prove the substitution lemma; 
% these are stated in Lemmas \ref{freevarsF} and \ref{weakeningF}.  
% The appearance of well-formedness judgments in our typing rules 
% require us to prove versions of these lemmas 
% for well-formedness judgments in Lemma \ref{WFFTlemmas} 
% and to generate well-formedness judgments from typing judgments in Lemma \ref{ftypes-wfft}.



% \begin{proof}
%     The proof is by straightforward mutual induction 
%     on the derivation trees of each typing judgment, 
%     using the well-formedness version %and (\ref{ftypes-wfft}) 
%     where needed.
% \end{proof}




%%%%%%%%%%%%%%%%%%%%%%%%%%%%%%%%%%%%%%%%%%%%%%%%%%%%%%%%%%%%%%%%%%%%%%%%%%
%% CUT %%%%%%%%%%%%%%%%%%%%%%%%%%%%%%%%%%%%%%%%%%%%%%%%%%%%%%%%%%%%%%%%%%%
%%%%%%%%%%%%%%%%%%%%%%%%%%%%%%%%%%%%%%%%%%%%%%%%%%%%%%%%%%%%%%%%%%%%%%%%%%

% Use this??
%We begin by proving the main sequence of lemmas for judgments on the well-formedness of $\sysf$ types. The Free Variables lemmas are proven by mutual structural induction for the term and type variables versions.
% 
%     \begin{lemma}\label{WFFTlemmas}
%         For any $\sysf$ environments $\tcenv$, $\tcenv'$ and $x,y,\al,\al' \not\in \dom{\tcenv', \tcenv}$:
%         \item (Free Variables) 
%             If $\isWFFT{\tcenv',\bind{x}{\sftype_x},\tcenv}{\sftype}{\skind}$ 
%             then $\isWFFT{\tcenv', \bind{y}{\sftype_x} \tcenv}{\sftype}{\skind}$.
%             If $\isWFFT{\tcenv', \bind{\al}{k_{\al}}, \tcenv}{\sftype}{\skind}$, 
%             then $\isWFFT{\tcenv', \bind{\al'}{k_{\al}}, \tcenv}{\subst{\sftype}{\al}{\al'}}{\skind}$. 
%         \item (Weakening) If $\isWFFT{\tcenv', \tcenv}{\sftype}{\skind}$
%             then $\isWFFT{\tcenv', \bind{x}{\sftype_x}, \tcenv}{\sftype}{\skind}$ 
%             and $\isWFFT{\tcenv', \bind{\al}{k_{\al}}, \tcenv}{\sftype}{\skind}$.
%         \item (Substitution Lemma) 
%             If $\isWFFT{\tcenv', \bind{\al}{k_{\al}}, \tcenv}{\sftype}{\skind}$ and
%             $\isWFFT{\tcenv}{\sftype_{\al}}{k_{\al}}$  
%             then $\isWFFT{\subst{\tcenv'}{\al}{\sftype_{\al}},\tcenv}{\subst{\sftype}{\al}{\sftype_{\al}}}{\skind}$.
%     \end{lemma}

% (don't use these probably)   
% We can also prove analogous statements for the well-formedness judgments for $\sysf$ environments. These judgments never appear in the antecedent positions in any of our $\sysf$ type well-formedness of $\sysf$ typing rules, so they are not part of the chain of logical dependencies. However, they are used in the full metatheory in order to prove facts about closing substitutions.
% 
% Because some $\sysf$ typing judgments have a well-formedness judgment in an antecedent position, we often need to generate a well-formedness judgment for a $\sysf$ type that appears in a typing judgment:
%     
% \begin{lemma}\label{ftypes-wfft}
%     (Well-formedness of $\sysf$ types in judgments) 
%     If $\hasftype{\tcenv}{\sexpr}{\sftype}$ and $\isWFFE{\tcenv}$ 
%     then $\isWFFT{\tcenv}{\sftype}{\skstar}$.
% \end{lemma}
%
% \begin{proof}
% The proof is by structural induction on the typing judgment. In some of the inductive cases we need to use the Weakening lemma above, as well as a strengthening lemma that says we can drop any term variable from the environment because $\sysf$ types cannot contain term variables (no refinements or dependent arrows.)
% \end{proof}
%
% We start the main sequence of lemmas with a lemma that allows us to rename free variables at will. Although this is intuitively obvious, it is required in the proof of the weakening lemma below. Unlike other traditional approaches to System F metatheory \cite{TAPL}, we do not prove a permutation lemma that would allow us to re-order the environment under restricted circumstances.,


% REMOVED FIGURES
% \begin{figure}
%     \begin{tabular}{rrcll}
%     \emphbf{Kinds} 
%       & \skind & $\bnfdef$ & \skbase & \emph{base kind} \\
%       &        & $\spmid$  & \skstar & \emph{star kind} \\[0.05in]
%     
%     \emphbf{Base Types} 
%       & \sbase & $\bnfdef$ & $\tbool$ & \emph{booleans} \\
%       &        & $\spmid$  & $\tint$  & \emph{integers} \\
%       &        & $\spmid$  & $\tvar$  & \emph{type variables} \\[0.05in]
%     
%     \emphbf{Types}
%       & \sftype & $\bnfdef$ & $\sbase$                        & \emph{base type} \\ 
%       &         & $\spmid$  & $\funcftype{\tau}{\tau'}$       & \emph{function type}  \\        
%       &        & $\spmid$  & $\polytype{\tvar}{\skind}{\tau}$ & \emph{polymorphic type}  \\ [0.05in]        
%     
%     \emphbf{Environments}
%       & $\tcenv$ & $\bnfdef$ & $\varnothing$                  & \emph{empty environment} \\
%       &          & $\spmid$  & $\tcenv, \bind{x}{\sftype}$       & \emph{variable binding} \\
%       &          & $\spmid$  & $\tcenv, \bind{\tvar}{\skind}$ & \emph{type binding} \\
%     \end{tabular}
%     \caption{Syntax of $\sysf$ Types and Environments}
%     \label{fig:syn:ftypes}
%     \label{fig:syn:fenv}
%     \end{figure}

%
% First, rule \wfftBasic states that the two closed base types (\tint and \tbool)
% are well-formed and have base kind in any context.
%
% Similarly, rule \wfftVar says that
% type variable $\tvar$ is well-formed having kind $\skind$
% so long as $\bind{\tvar}{\skind}$ is bound in the environment.
%
% Next, our rule \wfftFunc states that 
% a function type $\funcftype{\sftype_x}{\sftype}$
% is well-formed with star kind in some environment $\tcenv$ if 
% both types $\sftype_x$ and $\sftype$ are
% well-formed (with any kind) in the same environment.
%
% Next, rule \wfftPoly establishes that a 
% polymorphic type $\polytype{\tvar}{\skind}{\sftype}$
% has star kind in environment $\tcenv$ if the 
% inner type $\stype$ is well-formed (with any kind)
% in environment $\tcenv$ augmented by binding a fresh
% type variable $\tvar$ to kind $\skind$.
%
% Finally, rule \wfftKind simply states that if a type $\sftype$
% is well-formed with base kind in some environment, then 
% it is also well-formed with star kind in that environment.
%
% As for environments, rule \wffeEmp states that the empty environment 
% is well-formed. Rule \wffeBind says that a well-formed environment
% $\tcenv$ remains well-formed after binding a fresh variable $x$ to any
% type $\sftype_x$ that is well-formed in $\tcenv$.
%
% Finally rule \wffeTBind states that a well-formed environment remains
% well-formed after binding a fresh type variable to any kind.
% 
%     \begin{figure}
%     \judgementHead{Well-formed Type}{\isWFFT{\tcenv}{\sftype}{\skind}}
%     \begin{mathpar}      %%%%%%%%%%%% SYSTEM F WELL-FORMEDNESS %%%%%%%%%%%%%
%         \inferrule*[Right=\wfftBasic]{b\in\{\Bool,\Int\}}
%                     {\isWFFT{\tcenv}{b}{\skbase}}\and
%         \inferrule*[Right=\wfftVar]{\bind{\tvar}{\skind} \in \tcenv}
%                     {\isWFFT{\tcenv}{\tvar}{\skind}}\\
%         \inferrule*[Right=\wfftFunc]
%         {\isWFFT{\tcenv}{\sftype_x}{k_x} \quad\; \isWFFT{\tcenv}{\sftype}{k}}
%         {\isWFFT{\tcenv}{\funcftype{\sftype_x}{\sftype}}{\skstar}} \\
%         \inferrule*[Right=\wfftPoly]
%         {\isWFFT{\bind{\al'}{k}, \tcenv}{\subst{\al}{\al'}{\sftype}}{k_\sftype} 
%             \quad \notmem{\al'}{\dom{\tcenv}} }
%         {\isWFFT{\tcenv}{\polytype{\al}{k}{\sftype}}{\skstar}} \and
%         \inferrule*[Right=\wfftKind]{\isWFFT{\tcenv}{\sftype}{\skbase}}
%             {\isWFFT{\tcenv}{\sftype}{\skstar}} \\
%     \end{mathpar}
%    
%     \judgementHead{Well-formed Environment}{\isWFFE{\tcenv}}
%     \begin{mathpar}
%         \inferrule*[Right=\wffeEmp]{ }{\isWFFE{\varnothing}} \and
%     %
%         \inferrule*[Right=\wffeBind]{\isWFFT{\tcenv}{\sftype_x}{k_x}  \quad 
%           \isWFFE{\tcenv} \quad \notmem{x}{\dom{\tcenv}}}
%           {\isWFFE{\bind{x}{\sftype_x},\tcenv}} \\
%     %
%         \inferrule*[Right=\wffeTBind]
%         {\isWFFE{\tcenv} \quad \notmem{\al}{\dom{\tcenv}}}
%         {\isWFFE{\bind{\al}{\skind}, \tcenv}}
%     \end{mathpar}
%     \caption{Well-formedness of types and environments in $\sysf$}
%     \label{fig:wfft}\label{fig:wffe}
%    \end{figure}

%
% Rule \fPrim states that for any built-in primitive $\sconst$, it can be given base
% type $\forgetreft{\ty{\sconst}}$ in any context.
%
% Rule \fVar establishes that any variable $x$ bound to $\sftype$ 
% in some environment can be given the same type $\sftype$
% in that environment.
%
% Our rule \fApp states the conditions for typing a term application
% $\app{\sexpr}{\sexpr'}$. Under the same environment,
% we must be able to type $\sexpr$ at 
% some function type $\funcftype{\sftype_x}{\sftype}$ and
% $\sexpr'$ at $\sftype_x$. Then we can give $\app{\sexpr}{\sexpr'}$
% type $\sftype$.
% 
% Next, rule \fAbs says that we can type a lambda abstraction
% $\vabs{x}{\sexpr}$ at a function type $\funcftype{\sftype_x}{\sftype}$
% whenever $\sftype_x$ is well-formed and when the body $\sexpr$
% can be typed at $\sftype$ in the environment augmented by binding 
% a fresh variable to $\sftype_x$.
% 
% Our rule \fTApp states that whenever a term $\sexpr$ has polymorphic
% type $\polytype{\al}{\skind}{\sftype'}$, then for any well-formed type 
% of the form $\forgetreft{\stype}$
% with kind $\skind$ in the same environment, we can give the type
% $\subst{\sftype'}{\al}{\forgetreft{\stype}}$ 
% to the type application $\tyapp{\sexpr}{\stype}$.
% 
% The rule \fTAbs establishes that a 
% type-abstraction $\tabs{\al}{\skind}{\sexpr}$ can be given 
% a polymorphic type $\polytype{\al}{\skind}{\sftype}$ in some
% environment $\tcenv$ whenever $\sexpr$ can be given type $\sftype$ 
% in the environment $\tcenv$ augmented by binding a fresh type variable 
% to kind $\skind$.
% 
% Next, rule $\fLet$ states that an expression 
% $\eletin{x}{\sexpr_x}{\sexpr}$ has type
% $\sftype$ in some environment provided that $\sexpr_x$ can
% be given some type $\sftype_x$ and the body $\sexpr$ can be given type $\sftype$
% in the augmented environment formed by binding a fresh variable to $\sftype_x$.
% 
% Rule $\fAnn$ establishes that an explicit annotation $\tyann{\sexpr}{\stype}$
% can be given type $\forgetreft{\stype}$ when the underlying expression has 
% type $\forgetreft{\stype}$.


% \begin{figure}
%     \begin{mathpar}             %%%%%%%%%%%%% SYSTEM F TYPING %%%%%%%%%%%%%%%%%%
%         \inferrule*[Right=\fPrim]
%             {\forgetreft{\ty{\sconst}} = \sftype}
%             {\hasftype{\tcenv}{\sconst}{\sftype}}
%     %
%         \and
%     %
%         \inferrule*[Right=\fVar]
%             {\bind{x}{\sftype} \in \tcenv}{\hasftype{\tcenv}{x}{\sftype}} 
%     %
%         \and
%     %
%         \inferrule*[Right=\fApp]
%         {\hasftype{\tcenv}{\sexpr}{\funcftype{\sftype_x}{\sftype}} \qquad 
%             \hasftype{\tcenv}{\sexpr'}{\sftype_x}}
%         {\hasftype{\tcenv}{\app{\sexpr}{\sexpr'}}{\sftype} } \\
%     %
%         \inferrule*[Right=\fAbs]
%         {\hasftype{\bind{y}{\sftype_x}, \tcenv}{\subst{\sexpr}{x}{y}}{\sftype} \qquad
%          \isWFFT{\tcenv}{\sftype_x}{\skind_x}  \qquad
%          \notmem{y}{\dom{\tcenv}}}
%         {\hasftype{\tcenv}{\vabs{x}{\sexpr}}{\funcftype{\sftype_x}{\sftype}}} \\
%     %
%         \inferrule*[Right=\fTApp]
%         {\hasftype{\tcenv}{\sexpr}{\polytype{\al}{\skind}{\sftype'}} \qquad 
%          \isWFFT{\tcenv}{\forgetreft{\stype}}{\skind}}
%         {\hasftype{\tcenv}{\tyapp{\sexpr}{\stype}}{\subst{\sftype'}{\al}{\forgetreft{\stype}}}} \\
%     %
%         \inferrule*[Right=\fTAbs]
%         {\hasftype{\bind{\al'}{\skind}, \tcenv}{\subst{\sexpr}{\al}{\al'}}{\subst{\sftype}{\al}{\al'}} \qquad \notmem{\al'}{\dom{\tcenv}}}
%         {\hasftype{\tcenv}{\tabs{\al}{\skind}{\sexpr}}{\polytype{\al}{\skind}{\sftype}}} \\
%     %
%         \inferrule*[Right=\fLet]
%         {\hasftype{\tcenv}{\sexpr_x}{\sftype_x} \qquad
%          \hasftype{\bind{y}{\sftype_x},\tcenv}{\subst{\sexpr}{x}{y}}{\sftype} \qquad
%          \notmem{y}{\dom{\tcenv}}}
%         {\hasftype{\tcenv}{\eletin{x}{\sexpr_x}{\sexpr}}{\sftype}} \\
%     %
%         \inferrule*[Right=\fAnn]
%         {\hasftype{\tcenv}{\sexpr}{\sftype} \qquad \forgetreft{\stype} = \sftype}{\hasftype{\tcenv}{\tyann{\sexpr}{\stype}}{\sftype}}
%     \end{mathpar}
%     \caption{$\sysf$ typing rules}\label{fig:ft}
%     \end{figure}

% Ch 5: System RF Soundness
\chapter{Soundness of \sysrf} 
\label{ch:soundness}


% Node fill colors
\definecolor{palegrey}{rgb}{0.72, 0.72, 0.72}%{0.90, 0.90, 0.90}
\definecolor{lightgreen}{rgb}{0.92, 0.92, 0.92}%{0.56, 0.93, 0.56}
% Arrow colors
\definecolor{medgreen}{rgb}{0.02, 0.02, 0.02}%{0.24, 0.39, 0.24}
\definecolor{darkgreen}{rgb}{0.02, 0.02, 0.02}%{0.19, 0.31, 0.19}
\definecolor{sysfcolor}{rgb}{0.02, 0.02, 0.02}%{0.52, 0.52, 0.52}
\begin{figure}[t]
\begin{center}
\scalebox{0.77}{
    \begin{tikzpicture}[
        > = stealth, %arrowhead=1.25cm,  % arrow head style
        %shorten > = 1pt,               % don't touch arrow head to node
        auto,
        node distance = 1.0cm,         % distance between nodes
        semithick                      % line style
    ]

    \tikzstyle{every state}=[
        rectangle,
        minimum width=3.70cm,          % make most nodes the same size
        draw=black, rounded corners
    ]
    %\tikzstyle{every box}=[
    %    rectangle,
    %    opacity=0.00
    %    minimum width=3.85cm,         % make most nodes the same size
    %    draw=thin grey, rounded corners
    %]

    %% ORANGE REGION: Substitution Lemmas and friends
    \node[state, fill=palegrey]   (we1)                    {Weaken: tv in sub};
    \node[state, fill=lightgreen] (we2) [above of=we1]     {Weaken: tv in typ};
    \node[state, fill=palegrey]   (we3) [above of=we2]     {Weaken: var in sub};
    \node[state, fill=lightgreen] (we4) [above of=we3]     {Weaken: var in typ};

    \node[state, fill=lightgreen] (we5) [right=5.40cm of we1] {Weaken: tv in wf};
    \node[state, fill=lightgreen] (we6) [right=5.40cm of we2] {Weaken: var in wf};

    \node[state, fill=palegrey]   (su1) [above right=-0.25cm and 0.75cm of we2] {Substitute: tv in sub};% [250]};
    \node[state, fill=lightgreen] (su2) [above of=su1]     {Substitute: tv in typ};% [300]};
    \node[state, fill=palegrey]   (su3) [above of=su2]     {Substitute: var in sub};% [220]};
    \node[state, fill=lightgreen] (su4) [above of=su3]     {Substitute: var in typ};% [270]};

    \node[state, fill=lightgreen] (su5) [above=1.00cm of we6] {Substitute: tv in wf};
    \node[state, fill=lightgreen] (su6) [above of=su5] {Substitute: var in wf};

%%%    \begin{pgfonlayer}{background}
%%%        \path (su4.west |- su4.north)+(-0.25,0.60) node (oa) {};
%%%        \path (su1.south -| su1.east)+(0.25,-0.25) node (oc) {};
        %\path[fill=orange!50,rounded corners, draw=black!50, dashed]
        %    (oa) rectangle (oc);
%%%        \path[rounded corners, draw=black!50, dashed]
%%%            (oa) rectangle (oc) node (orangefill) {};
%%%        \fill[pattern=custom north west lines,hatchspread=6pt,hatchthickness=1pt,hatchcolor=gray!40]
%%%            (oa) rectangle (oc) ;
%%%    \end{pgfonlayer}
%%%    \begin{pgfonlayer}{background}
%%%        \path (we4.west |- we4.north)+(-0.25,0.60) node (oa1) {};
%%%        \path (we1.south -| we1.east)+(0.25,-0.25) node (oc1) {};
        %\path[fill=orange!50,rounded corners, draw=black!50, dashed]
        %    (oa) rectangle (oc);
%%%        \path[rounded corners, draw=black!50, dashed]
%%%            (oa1) rectangle (oc1) node (orangefill) {};
%%%        \fill[pattern=custom north west lines,hatchspread=6pt,hatchthickness=1pt,hatchcolor=gray!40]
%%%            (oa1) rectangle (oc1) ;
%%%    \end{pgfonlayer}

    \begin{pgfonlayer}{middleground}
        \path (we4.west |- we4.north)+(-0.10,0.45) node (b3a) {};
        \path (we1.south -| we1.east)+(0.10,-0.10) node (b3b) {};
        \path (we4.north) +(0,0.20) node (b3cap) {Weakening Lemma (\ref{lem:weakening})};
        \path[draw=black!70]
            (b3a) rectangle (b3b) node (b3) {};
    \end{pgfonlayer}
    \begin{pgfonlayer}{middleground}
        \path (we6.west |- we6.north)+(-0.10,0.45) node (b4a) {};
        \path (we5.south -| we5.east)+(0.10,-0.10) node (b4b) {};
        \path (we6.north) +(0,0.20) node (b4cap) {Weakening Lemma};
        \path[draw=black!70]
            (b4a) rectangle (b4b) node (b4) {};
    \end{pgfonlayer}
    \begin{pgfonlayer}{middleground}
        \path (su4.west |- su4.north)+(-0.10,0.45) node (b5a) {};
        \path (su1.south -| su1.east)+(0.10,-0.10) node (b5b) {};
        \path (su4.north) +(0,0.20) node (b5cap) {Substitution Lemma (\ref{lem:subst})};
        \path[draw=black!70]
            (b5a) rectangle (b5b) node (b5) {};
    \end{pgfonlayer}
    \begin{pgfonlayer}{middleground}
        \path (su6.west |- su6.north)+(-0.10,0.45) node (b6a) {};
        \path (su5.south -| su5.east)+(0.10,-0.10) node (b6b) {};
        \path (su6.north) +(0,0.20) node (b6cap) {Substitution Lemma};
        \path[draw=black!70]
            (b6a) rectangle (b6b) node (b6) {};
    \end{pgfonlayer}

    \begin{pgfonlayer}{foreground}
        \path[every edge/.style={draw, ->, >={Stealth[width = 5pt, length = 7pt, inset=1pt,sep]}}]
            (we5.north east) edge[sysfcolor, dashed, bend right] node {} (su6.south east)
            (we2.north east) edge[sysfcolor, dashed, bend left] node {} (su3.south west)
            (we5.north west) edge[sysfcolor, dashed, bend left] node {} (we3.south east)
            (su5.north west) edge[sysfcolor, dashed, bend right] node {} (su3.south east);
    \end{pgfonlayer}
    \begin{pgfonlayer}{foreground}
        \path[every edge/.style={draw, >-<, >={Stealth[width = 6pt, length = -5pt, inset=-8pt]}}]
            (we1) edge[darkgreen, bend right=15] node {} (we2)
            (we3) edge[darkgreen, bend right=15] node {} (we4)
            (su1) edge[darkgreen, bend right=15] node {} (su2)
            (su3) edge[darkgreen, bend right=15] node {} (su4);
    \end{pgfonlayer}

    % NEW REGION: Implication Interface 
    \node[state, fill=palegrey]   (imp) [left=0.75cm of we1]   {Implication Interf. (Rq.~\ref{lem:implication})};% [55]};

%%%    \begin{pgfonlayer}{background}
%%%        \path (imp.west |- imp.north)+(-0.25,0.25) node (ia) {};
%%%        \path (imp.south -| imp.east)+(0.25,-0.25) node (ic) {};
        %\path[fill=red!50,rounded corners, draw=black!50, dashed]
        %    (ra) rectangle (rc);
%%%        \path[rounded corners, draw=black!50, dashed]
%%%            (ia) rectangle (ic) node (redfill) {};
%%%        \fill[pattern=dots, pattern color=gray!25]
%%%            (ia) rectangle (ic);
%%%    \end{pgfonlayer}

    \begin{pgfonlayer}{foreground}
        \path[every edge/.style={draw, ->, >={Stealth[width = 5pt, length = 7pt, inset=1pt,sep]}}]
            (imp.east) edge[medgreen] node {} (we2.north west)
            (imp.south east) edge[medgreen, bend right=60] node {} (su1.south west);
    \end{pgfonlayer}

    % PURPLE REGION: Denotational Soundness
    \node[state, fill=palegrey]   (de3) [below=1.1cm of imp] {Den. Sound: subtyping};% [320]};
    \node[state, fill=palegrey]   (de2) [right=0.75cm of de3] {Denot. Sound: typing};% [375]};
    \node[state, fill=palegrey]   (de1) [right=0.75cm of de2] {Selfified Den. (\ref{denote-selfification})};% [115]};

%%%    \begin{pgfonlayer}{background}
%%%        \path (de2.west |- de2.south)+(-0.25,-0.25) node (pa) {};
%%%        \path (de1.north -| de1.east)+(0.25,0.60) node (pc) {};
        %\path[fill=violet!50,rounded corners, draw=black!50, dashed]
        %   (pa) rectangle (pc);
%%%        \path[rounded corners, draw=black!50, dashed]
%%%            (pa) rectangle (pc) node (purplefill) {};
%%%        \fill[pattern=custom horizontal lines, hatchspread=5pt, hatchthickness=0.75pt, hatchcolor=gray!35]
%%%            (pa) rectangle (pc);
%%%    \end{pgfonlayer}
    \begin{pgfonlayer}{middleground}
        \path (de3.west |- de3.north)+(-0.10,0.45) node (b7a) {};
        \path (de2.south -| de2.east)+(0.10,-0.10) node (b7b) {};
        \path (de3.north east) +(0,0.20) node (b7cap) {Denotational Soundness (\ref{denote-sound})};
        \path[draw=black!70]
            (b7a) rectangle (b7b) node (b7) {};
    \end{pgfonlayer}

    \begin{pgfonlayer}{foreground}
        \path[every edge/.style={draw, ->, >={Stealth[width = 5pt, length = 7pt, inset=1pt,sep]}}]
            (de1.west) edge[medgreen] node {} (de2.east);
    \end{pgfonlayer}
    \begin{pgfonlayer}{foreground}
        \path[every edge/.style={draw, >-<, >={Stealth[width = 6pt, length = -5pt, inset=-8pt]}}]
            (de2.west) edge[darkgreen, bend left=15] node {} (de3.east);
    \end{pgfonlayer}
    \begin{pgfonlayer}{foreground}
        \path[every edge/.style={draw, ->, >={Stealth[width = 6pt, length = 5pt, inset=1pt,sep]}}]
            (de2.north west) edge[darkgreen, bend right=15] node {} (imp);
    \end{pgfonlayer}    

    % RED REGION: Narrowing Lemma (and Exactness)
    \node[state, fill=palegrey]   (na1) [left=5.25cm of su2]   {Exact Subtypes (\ref{lem:exact})};% [55]};
    \node[state, fill=palegrey]   (na2) [above=0.70cm of na1]   {Exact Types (\ref{lem:exact})};% [100]};
    \node[state, fill=palegrey]   (na3) [above=0.70cm of na2]   {Narrowing Lemma (\ref{lem:narrowing})};% [390]};

%%%    \begin{pgfonlayer}{background}
%%%        \path (na3.west |- na3.north)+(-0.25,0.25) node (ra) {};
%%%        \path (na1.south -| na1.east)+(0.25,-0.25) node (rc) {};
        %\path[fill=red!50,rounded corners, draw=black!50, dashed]
        %    (ra) rectangle (rc);
%%%        \path[rounded corners, draw=black!50, dashed]
%%%            (ra) rectangle (rc) node (redfill) {};
%%%        \fill[pattern=custom checkerboard, hatchcolor=gray!25]
%%%            (ra) rectangle (rc);
%%%    \end{pgfonlayer}

    \begin{pgfonlayer}{foreground}
        \path[every edge/.style={draw, ->, >={Stealth[width = 5pt, length = 7pt, inset=1pt,sep]}}]
            (na1.north) edge[medgreen] node {} (na2.south)
            (na2.north) edge[medgreen] node {} (na3.south)
            (imp.north west) edge[medgreen, bend left=30] node {} (na3.west)
            (na2.east) edge[medgreen, bend left=25] node {} (su2.north west);
    \end{pgfonlayer}

    % YELLOW REGION: Inversion of Typing
    \node[state, fill=palegrey]   (in1) [above left=-0.25cm and 0.80cm of su4]   {Transitivity (\ref{lem:transitivity})};% [295]};
    \node[state, fill=lightgreen] (in2) [above=0.60cm of in1]   {Inversion of Typing (\ref{lem:inversion})};% [80]};

%%%    \begin{pgfonlayer}{background}
%%%        \path (in2.west |- in2.north)+(-0.25,0.25) node (ya) {};
%%%        \path (in1.south -| in1.east)+(0.25,-0.25) node (yc) {};
        %\path[fill=yellow!50,rounded corners, draw=black!50, dashed]
        %    (ya) rectangle (yc);
%%%        \path[rounded corners, draw=black!50, dashed]
%%%            (ya) rectangle (yc) node (yellowfill) {};
%%%        \fill[pattern=custom vertical lines, hatchspread=5pt, hatchthickness=1pt, hatchcolor=gray!35]
%%%            (ya) rectangle (yc);
%%%    \end{pgfonlayer}
    \begin{pgfonlayer}{middleground}
        \path (in2.west |- in2.north)+(-0.10,0.10) node (in6a) {};
        \path (in1.south -| in1.east)+(0.10,-0.45) node (in6b) {};
        \path (in1.south) +(0,-0.20) node (in6cap) {Inversion};
        \path[draw=black!70]
            (in6a) rectangle (in6b) node (in6) {};
    \end{pgfonlayer}
    \begin{pgfonlayer}{foreground}
        \path[every edge/.style={draw, ->, >={Stealth[width = 5pt, length = 7pt, inset=1pt,sep]}}]
            (in1.north) edge[medgreen] node {} (in2.south)
            (na3.east) edge[medgreen] node {} (in2.west)
            (na3.east) edge[medgreen] node {} (in1.west)
            (su4.west) edge[medgreen] node {} (in1.east);
    \end{pgfonlayer}

    % TOP REGION: Buildup to Progress and Preservation
    %%\node[state, fill=lightgreen] (pr1) [left=0.75cm of in2]   {Canonical Forms};
    \node[state, fill=lightgreen] (pr2) [right=0.75cm of in2]   {Primitives (Req. \ref{lem:prim-typing})};% [30]};
    \node[state, fill=lightgreen] (pr3) [right=0.75cm of pr2]   {Polym. Prim. (Req. \ref{lem:prim-typing})};% [30]};
    \node[state, fill=lightgreen] (pr4) [above left=0.70cm and 0.75cm of in2]   {Progress (\ref{lem:progressF})};% [55]};
    \node[state, fill=lightgreen] (pr5) [right=0.75cm of pr4]   {Preservation (\ref{lem:preservationF})};% [125]};
    \node[state, fill=lightgreen] (pr6) [right=0.75cm of pr5]   {Values Stuck};% [60]};
    \node[state, fill=lightgreen] (pr7) [right=0.75cm of pr6]   {Det. Semantics};% [85]};

    \begin{pgfonlayer}{foreground}
        \path[every edge/.style={draw, ->, >={Stealth[width = 5pt, length = 7pt, inset=1pt,sep]}}]
            (su4.north) edge[medgreen] node {} (pr2.south)          %% TO SECOND ROW
            (su4.east) edge[medgreen] node {} (pr3.south)
            (in2.north west) edge[sysfcolor, dashed] node {} (pr4.south)    %% TO FIRST ROW
            (in2.north) edge[sysfcolor, dashed] node {} (pr5.south)
            (pr2.north west) edge[sysfcolor, dashed] node {} (pr4.south)
            (pr2.north west) edge[sysfcolor, dashed] node {} (pr5.south)
            (pr3.north west) edge[sysfcolor, dashed] node {} (pr4.south east)
            (pr3.north west) edge[sysfcolor, dashed] node {} (pr5.south east)
            (su4.north west) edge[sysfcolor, dashed, bend right=0] node {} (pr5.south east)
            (pr6.west) edge[sysfcolor, dashed] node {} (pr5.east)
            (pr7.north west) edge[sysfcolor, dashed, bend right=20] node {} (pr5.north east);
    %%        (pr1.north) edge[sysfcolor, bend left] node {} (pr4.south west)
    %%        (pr1.north east) edge[sysfcolor] node {} (pr5.south west)
    %%        (pr4.east) edge[sysfcolor] node {} (pr5.west)  %% REMOVED DEPENDENCY FROM PROOF
    \end{pgfonlayer}

    \end{tikzpicture}
}
\end{center}
\vspace{-0.00cm}
\caption{Dependencies in the metatheory. We write
``var'' and ``tv'' to resp. abbreviate term and type variables.}
%The number in brackets indicates lines of code.}
\label{fig:graph}
\vspace{-0.00cm}
\end{figure}

    % BLUE REGION: Substitution Lemma for Entailment Judgments
    %\node[state, fill=palegrey]   (se1) [above=1.5cm of de3]   {Substitute: var in entail};% [165]};
    %\node[state, fill=palegrey]   (se2) [above=0.4cm of se1] {Substitute: tv in entail};% [175]};

    %\begin{pgfonlayer}{background}
    %    \path (se2.west |- se2.north)+(-0.25,0.60) node (ba) {};
    %    \path (se1.south -| se1.east)+(0.25,-0.25) node (bc) {};
    %    %\path[fill=blue!50,rounded corners, draw=black!50, dashed]
    %    %    (ba) rectangle (bc);
    %    \path[rounded corners, draw=black!50, dashed]
    %        (ba) rectangle (bc) node (bluefill) {};
    %    \fill[pattern=custom crosshatch dots, hatchspread=2.5pt, hatchthickness=0.25pt, hatchcolor=gray!35]
    %        (ba) rectangle (bc);
    %\end{pgfonlayer}
    %\begin{pgfonlayer}{middleground}
    %    \path (se2.west |- se2.north)+(-0.10,0.45) node (b8a) {};
    %    \path (se1.south -| se1.east)+(0.10,-0.10) node (b8b) {};
    %    \path (se2.north) +(0,0.20) node (b8cap) {Substitution Lemma};
    %    \path[draw=black!70]
    %        (b8a) rectangle (b8b) node (b8) {};
    %\end{pgfonlayer}

    %\begin{pgfonlayer}{foreground}
    %    \path[every edge/.style={draw, ->, >={Stealth[width = 5pt, length = 7pt, inset=1pt,sep]}}]
    %        (de3.north) edge[medgreen, bend right=15] node {} (se1.south)
    %        (se1.east) edge[medgreen] node {} (su1.west)
    %        (se2.east) edge[medgreen] node {} (su3.west);
    %\end{pgfonlayer}
    %\begin{pgfonlayer}{foreground}
    %    \path[every edge/.style={draw, >-<, >={Stealth[width = 6pt, length = -5pt, inset=-8pt]}}]
    %        (se1) edge[darkgreen, bend right=15] node {} (se2);
    %\end{pgfonlayer}


Our development of the metatheory 
for \sysf (\S~\ref{sec:soundnessF})
followed the standard presentation of System F's
metatheory by \citet{TAPL}.
%
The main difference 
is that ours includes well-formedness of types and
environments, which help with mechanization \cite{Remy21}
and are crucial in \sysrf when formalizing refinements.
%

%
%We proceed to the metatheory
%of the two systems $\sysf$ and $\sysrf$
%and concretely, 
%type safety for both systems (\Cref{lem:soundness})
%and denotational soundness for $\sysrf$ (\Cref{lem:denote-sound-first}).
%
\Cref{fig:graph} charts the overall
landscape of our formal development as a
dependency graph of the main lemmas which
establish meta-theoretic properties of the
different judgments for \sysf and \sysrf.
%
\NV{I think RJ says to remove colors?}
Nodes shaded \colboth represent lemmas
in the metatheories for both \sysf and \sysrf.
%
The \colref nodes denote lemmas
that only appear in %the metatheory for 
\sysrf.
%
An arrow shows a dependency: the lemma at the
\emph{tail} is used in the proof of the lemma at
the \emph{head}. %A double-headed arrow indicates
%a mutual dependency, \ie mutually recursive proofs.
Solid arrows are dependencies in \sysrf only.
The chart already shows that the metatheory of 
the refined calculus \sysrf is much more complex 
that the one of the unrefined system \sysf, 
as also shown by the summary 
of our mechanization (\Cref{fig:empirical}).

In the rest of this chapter
we establish 
denotational soundness (\S~\ref{sec:soundness:denotational}) and 
type safety (\S~\ref{sec:soundness:safety})
and we flesh out the skeleton of~\Cref{fig:graph}, \ie  
the inversion (\S~\ref{sec:soundness:inversion}),
substitution (\S~\ref{sec:soundness:substitution}),
and narrowing (\S~\ref{sec:soundness:narrowing}) lemmas.





\section{Denotational Soundness}
\label{sec:soundness:denotational}
\label{sec:denot:soundness}

Denotational soundness connects syntactic typing and subtyping 
with the type denotations (of~\Cref{fig:den}).
For typing, it states that if
$\hastype{\tcenv}{e}{t}$,
then when $e$ is closed by any closing substitution of $\tcenv$
it evaluates to a value that belongs in the denotation of the closed $t$.
For subtyping,
if $\isSubType{\tcenv}{s}{t}$, then
under all closing substitutions, the denotation of the
former type is contained in the latter:

\begin{theorem}\label{lem:denote-sound-first}
\label{denote-sound} (Denotational Soundness)
\begin{enumerate}
\item If $\hastype{\tcenv}{e}{t}$
      and $\isWellFormedE{\tcenv}$
      and $\theta \in \denote{\tcenv}$
      then $\evalsTo{\theta(e)}{v} \in \denote{\theta(t)}$ for some value $v$.
\item If $\isSubType{\tcenv}{t_1}{t_2}$
      and $\isWellFormedE{\tcenv}$
      and $\isWellFormed{\tcenv}{t_1}{k_1}$
      and $\isWellFormed{\tcenv}{t_2}{k_2}$ and $\theta \in \denote{\tcenv}$ then $\denote{\theta(t_1)} \subseteq\denote{\theta(t_2)}$.
\end{enumerate}
\end{theorem}

The proof is by mutual induction
on the structure of the judgments
$\hastype{\tcenv}{e}{t}$ and
$\isSubType{\tcenv}{t_1}{t_2}$ respectively.
%
Our rule \tVar mentions selfification,
so we use \Cref{denote-selfification}
for that case.

\begin{lemma}\label{denote-selfification} (Selfified Denotations)
    If $\isWellFormed{\varnothing}{\stype}{\skind}$,
       $\hastype{\varnothing}{\sexpr}{\stype}$,
       $\evalsTo{\sexpr}{\sval}$ for some $\sval \in \denote{\stype}$
    then $\sval \in \denote{\self{t}{e}{k}}$.
    %(DenotationsSelfify.hs) Depends on: ${\tt lem\_typing\_wf}$
\end{lemma}

This lemma captures the intuition
that if $v \in \denote{\breft{\sbase}{x}{p}}$
(\ie if $v$ has base type $\sbase$
and $\evalsTo{\subst{p}{x}{v}}{\ttrue}$),
then we have $v \in \denote{\breft{\sbase}{x}{ p \wedge x = v}}$
as $\subst{(p \wedge x=v)}{x}{v}$ certainly evaluates to $\ttrue$.
%
The full proof also handles the case that $t$
is an existential type as well as selfification
by an expression $e$ that evaluates to $v$.



\section{Type Safety}
\label{sec:soundness:safety}

The type safety theorem  states that a well-typed 
term does not get stuck: \ie either 
evaluates to a value or can step 
to another term (progress) 
of the same type (preservation).
%

\begin{theorem} (Type Safety of \sysrf) 
\label{lem:soundness} 
\begin{enumerate}
    \item (Type Safety)
    If $\hastype{\varnothing}{\sexpr}{\stype}$ and $\evalsTo{\sexpr}{\sexpr'}$,
    then $\sexpr'$ is a value or $\sexpr' \step \sexpr''$
    %and $\hastype{\varnothing}{\sexpr''}{\stype}$
    for some $\sexpr''$.
    \item (No Error)
    If $\hastype{\varnothing}{\sexpr}{\stype}$ and $\evalsTo{\sexpr}{\sexpr'}$,
    then $\sexpr' \not = \eerr$.
\end{enumerate}
\end{theorem}
The No Error property explicitly states that well-typed terms 
cannot evaluate to the term $\eerr$ (that encodes stuck terms)
and is a direct implication of type safety.
We prove type safety by induction on the 
length of the sequence of steps  
$\evalsTo{\sexpr}{\sexpr'}$, using  
preservation and progress.

\mypara{Progress} \label{sec:sysf:progress}
%
The progress lemma says a well-typed term is a value 
or steps to some other term.
%
\begin{lemma} (Progress) \label{lem:progressF} 
If $\hastype{\varnothing}{\sexpr}{\stype}$, 
then $\sexpr$ is a value or $\sexpr \step \sexpr'$ for some $\sexpr'$.
\end{lemma}

The proof is by induction on the typing derivation 
using the primitives~\Cref{lem:prim-typing}, that we proved 
for our built-in primitives, and the inversion of typing lemma.



\mypara{Preservation} \label{sec:sysf:preservation}
%
The preservation lemma states that typing is preserved
by evaluation.
%
\begin{lemma} (Preservation) \label{lem:preservationF} 
If $\hastype{\varnothing}{\sexpr}{\stype}$ and $\sexpr \step \sexpr'$, 
then $\hastype{\varnothing}{\sexpr'}{\stype}$.
\end{lemma}    

The proof is by structural induction on the 
derivation of the typing judgment and implicitly uses the 
inversion lemma. 
We use the determinism of the operational 
semantics (lemma~\ref{lem:step-determ}) and 
the canonical forms lemma to case split 
on $\sexpr$ to determine $\sexpr'$.
%
The interesting cases are for \fApp and \fTApp that require 
a substitution~\Cref{lem:subst}.
%
Next, let's see the three main lemmas used in 
the preservation and progress proofs.

\section{Inversion of Typing Judgments}
\label{sec:soundness:inversion}

The region of~\Cref{fig:graph} labelled ``Inversion''
accounts for the fact that, due to subtyping
chains, the typing judgment in \sysrf is not
syntax-directed.
%
% \mypara{Transitivity of Subtyping}
%
First, we establish that subtyping is transitive:

\begin{lemma} (Transitivity) \label{lem:transitivity} 
    If\; $\isWellFormed{\tcenv}{t_1\!}{\! k_1}$,
       %$\isWellFormed{\tcenv}{t_2}{k_2}$,
       $\isWellFormed{\tcenv}{t_3\!}{\! k_3}$,
       $\isWellFormedE{\tcenv}$,
       $\isSubType{\tcenv}{t_1}{t_2}$, 
       $\isSubType{\tcenv}{t_2}{t_3}$, then
       $\isSubType{\tcenv}{t_1}{t_3}$.
    % (LemmasTransitive.hs) Our ${\tt lem\_sub\_trans}$ depends on ${\tt lem\_subst\_sub}$.
\end{lemma}
%
The proof consists of a case-split
on the possible rules for $\isSubType{\tcenv}{t_1}{t_2}$
and $\isSubType{\tcenv}{t_2}{t_3}$.
%
When the last rule used
in the former is \sWitn and the
latter is \sBind, we require the
substitution~\Cref{lem:subst}.
%
As \citet{Aydemir05},
we use the narrowing~\Cref{lem:narrowing}
for the transitivity for function types.

\mypara{Inverting Typing Judgments}
%
We use the transitivity of subtyping to prove
some non-trivial lemmas that let us ``invert''
the typing judgments to recover information
about the underlying terms and types.
%
We describe the non-trivial case which
pertains to type and value abstractions:

\begin{lemma} (Inversion of $\tAbs$, $\tTAbs$) \label{lem:inversion} 
    \begin{enumerate}
    \item If $\hastype{\tcenv}{(\vabs{w}{\sexpr})}{\functype{x}{\stype_x}{\stype}}$
    and $\isWellFormedE{\tcenv}$,
    then for all $\notmem{y}{{\tcenv}}$,
    $\hastype{\bind{y}{\stype_x},\tcenv}{\subst{\sexpr}{w}{y}}{\subst{\stype}{x}{y}}$.
    \item If $\hastype{\tcenv}{(\tabs{\al_1}{\skind_1}{\sexpr})}{\polytype{\al}{\skind}{\stype}}$
    and $\isWellFormedE{\tcenv}$,
    then for all $\notmem{\al'}{{\tcenv}}$, 
    $\hastype{\bind{\al'}{\skind},\tcenv}{\subst{\sexpr}{\al_1}{\al'}}{\subst{\stype}{\al}{\al'}}$. %(LemmasInvertLambda.hs) Both of these depend on ${\tt lem_sub_trans}$. Our ${\tt lem\_invert\_tabs}$ depends on ${\tt lem\_narrow\_typ}$, which in turn depends on ${\tt lem\_exact\_type}$.
    \end{enumerate}
\end{lemma}
%
If $\hastype{\tcenv}{(\vabs{w}{e})}{\functype{x}{t_x}{t}}$,
then we cannot directly invert the typing judgment
to get a judgment for the body $e$ of $\vabs{w}{e}$.
%
Perhaps the last rule used was \tSub,
and inversion only tells us that there
exists a type $t_1$ such that
$\hastype{\tcenv}{(\vabs{w}{e})}{t_1}$
and $\isSubType{\tcenv}{t_1}{\functype{x}{t_x}{t}}$.
%
Inverting again, we may in fact find a chain
of types $t_{i+1} \subt t_i \subt \cdots \subt t_2 \subt t_1$
which can be arbitrarily long.
%
But the proof tree must be finite so eventually
we find a type $\functype{w}{s_w}{s}$ such that
$\hastype{\tcenv}{(\vabs{w}{e})}{\functype{w}{s_w}{s}}$
and $\isSubType{\tcenv}{\functype{w}{s_w}{s}}{\functype{x}{t_x}{t}}$
(by transitivity)
and the last rule was $\tAbs$.
%
Then inversion gives us that for any $\notmem{y}{\tcenv}$
we have
$\hastype{\bind{y}{s_w}, \tcenv}{e}{\subst{s}{w}{y}}$.
To get the desired typing judgment, we must
use the narrowing~\Cref{lem:narrowing} to obtain
$\hastype{\bind{y}{t_x}, \tcenv}{e}{\subst{s}{w}{y}}$.
Finally, we use $\tSub$ to derive
$\hastype{\bind{y}{t_x}, \tcenv}{e}{\subst{t}{w}{y}}$.

\section{Substitution Lemma}
\label{sec:soundness:substitution}

%The main result in the \colsubtyping
%region of~\Cref{fig:graph} is
%the substitution lemma.
%
In $\sysrf$, unlike unrefined calculi such as $\sysf$, 
typing and subtyping are mutual dependent. 
%
Due to this dependency, both the substitution
the weakening lemmas
must now be proven
in a mutually recursive form: %  for both judgments: 

\begin{lemma}\label{substitution} \label{lem:subst} (Substitution)
    \begin{enumerate}
        %If $\entails{\tcenv',\bind{\al}{\skind}\tcenv}{p}$ and $\vdash_w \tcenv',\bind{\al}{\skind}\tcenv$ and $\hastype{\tcenv}{t_a}{k}$ then $\entails{\subst{\tcenv'}{\al}{t_\al},\tcenv}{p[t_a/a]}$. (SubstitutionLemmaEntTV.hs) \\
        \item If $\isSubType{\tcenv_1,\bind{x}{t_x},\tcenv_2}{s}{t}$,
                 $\isWellFormedE{\tcenv_2}$,
            and $\hastype{\tcenv_2}{v_x}{t_x}$,
            then $\isSubType{\subst{\tcenv_1}{x}{v_x},\tcenv_2}{\subst{s}{x}{v_x}}{\subst{t}{x}{v_x}}$.
            %(SubstitutionLemmaTyp.hs) Our ${\tt lem\_subst\_sub}$ depends on ${\tt lem\_subst\_ent}$. \\
        \item If $\hastype{\tcenv_1,\bind{x}{t_x},\tcenv_2}{e}{t}$,
                 $\isWellFormedE{\tcenv_2}$,
            and $\hastype{\tcenv_2}{v_x}{t_x}$,
            then $\hastype{\subst{\tcenv_1}{x}{v_x},\tcenv_2}{\subst{e}{x}{v_x}}{\subst{t}{x}{v_x}}$.
            % (SubstitutionLemmaTyp.hs) Mutually recursive with ${\tt lem\_subst\_sub}$. \\
        \item If $\isSubType{\tcenv_1,\bind{\al}{\skind},\tcenv_2}{s}{t}$,
                $\isWellFormedE{\tcenv_2}$,
            and $\isWellFormed{\tcenv_2}{t_\al}{k}$,
            then $\isSubType{\subst{\tcenv_1}{\al}{t_\al},\tcenv_2}{\subst{s}{\al}{t_\al}}{\subst{t}{\al}{t_\al}}$.
            %(SubstitutionLemmaTypTV.hs) Our ${\tt lem\_subst\_tv\_sub}$ depends on ${\tt lem\_subst\_tv\_ent}$.\\
        \item If $\hastype{\tcenv_1,\bind{\al}{\skind},\tcenv_2}{e}{t}$,
                 $\isWellFormedE{\tcenv_2}$,
            and $\isWellFormed{\tcenv_2}{t_\al}{\skind}$,
            then $\hastype{\subst{\tcenv_1}{\al}{t_\al},\tcenv_2}{\subst{e}{\al}{t_\al}}{\subst{t}{\al}{t_\al}}$.
        %(SubstitutionLemmaTypTV.hs) Mutually recursive with ${\tt lem\_subst\_tv\_sub}$.
    \end{enumerate}
\end{lemma}

The proof goes by induction on the derivation trees. 
The main difficulty arises
in substituting some type
$t_\al$ for variable $\al$
in $\isSubType{\tcenv_1,\bind{\al}{\skind},\tcenv_2}{\breft{\al}{x_1}{p}}{\breft{\al}{x_2}{q}}$
because $t_\al$  must be
strengthened  by
the refinements $p$ and
$q$ respectively.
%
Because we encoded our typing rules using 
cofinite quantification~\cite{AydemirCPPW08}
the proof does not require a renaming lemma, but 
the rules that lookup environments 
(rules \tVar and \wtVar) do need a \emph{weakening Lemma}:
%\Cref{lem:weakeningF}. 
% and \emph{rename} free variables 
%in typing and well-formedness judgments (\Cref{lem:freevarsF}).

%As with the $\sysf$ version, the proof 
%requires the \emph{Weakening}~\Cref{lem:weakening}
%but now both for typing and subtyping.

%\RJ{where is "weakening" used? the label is unused in the text. CUT or REFER}
\begin{lemma}\label{lem:weakening} (Weakening)
If $\notmem{x,\al}{{\tcenv_1, \tcenv_2}}$, then
    \begin{enumerate}
        \item if $\hastype{\tcenv_1, \tcenv_2}{\sexpr}{\stype}$
        then $\hastype{\tcenv_1, \bind{x}{\stype_x}, \tcenv_2}{\sexpr}{\stype}$ and $\hastype{\tcenv_1, \bind{\al}{\skind}, \tcenv_2}{\sexpr}{\stype}$.
        \item if $\isSubType{\tcenv_1, \tcenv_2}{s}{t}$
        then $\isSubType{\tcenv_1, \bind{x}{\stype_x}, \tcenv_2}{s}{t}$
        and $\isSubType{\tcenv_1, \bind{\al}{\skind}, \tcenv_2}{s}{t}$.
            %\item If $\tcenv', \tcenv \vdash_e p$ then $\tcenv', \bind{x}{t_x}, \tcenv \vdash_e p$ and $\tcenv', \al\bindt k, \tcenv \vdash_e p$.
            %\item If $\tcenv', \tcenv \vdash_w t:k$ then $\tcenv', \bind{x}{t_x}, \tcenv     \vdash_w t:k$ and $\tcenv', \al\bindt k, \tcenv \vdash_w t:k$.
    \end{enumerate}
\end{lemma}

The proof is by mutual induction
on the derivation of the typing and subtyping judgments.

\section{Narrowing} 
\label{sec:soundness:narrowing}

The narrowing lemma says that whenever
we have a judgment where a binding
$x\bindt t_x$ appears in the binding
environment, we can replace $t_x$
by any subtype $s_x$.
%
The intuition here is that the judgment
holds under the replacement because we
are making the context more specific.

\begin{lemma} \label{subtype-env} \label{lem:narrowing} (Narrowing)
    If $\tcenv_2 \vdash s_x <: t_x$,  $\isWellFormed{\tcenv_2}{s_x}{k_x}$,
    and $\isWellFormedE{\tcenv_2}$ then
    \begin{enumerate}
        \item if $\tcenv_1, \bind{x}{t_x}, \tcenv_2 \vdash_w t : k$, then
                 $\tcenv_1, x\bindt s_x, \tcenv_2 \vdash_w t : k $.
        \item if $\tcenv_1, \bind{x}{t_x}, \tcenv_2 \vdash t_1 <: t_2$, then
                 $\tcenv_1, x\bindt s_x, \tcenv_2 \vdash t_1 <: t_2$.
        \item if $\tcenv_1, \bind{x}{t_x}, \tcenv_2 \vdash e : t$, then
                 $\tcenv_1, x\bindt s_x, \tcenv_2 \vdash e : t$.
    \end{enumerate}
\end{lemma}

\begin{fullversion}
The narrowing proof
requires an exact typing~\Cref{lem:exact}
which says that a subtyping
judgment $\isSubType{\tcenv}{s}{t}$
is preserved after
selfification on both types.
%
Similarly, whenever we can type
a value $v$ at type $t$ then we
also type $v$ at the type $t$
selfified by $v$.
\end{fullversion}
\begin{conference}
    The narrowing proof
    requires an exact typing~\Cref{lem:exact}
    which says that both subtyping and typing is preserved 
    after selfification.
\end{conference}
\begin{lemma} (Exact Typing) \label{lem:exact}
\begin{enumerate}
  \item If $\hastype{\tcenv}{e}{t}$, $\isWellFormedE{\tcenv}$, $\isWellFormed{\tcenv}{t}{k}$, and $\isSubType{\tcenv}{s}{t}$, then $\isSubType{\tcenv}{{\rm self}(s, v, k)}{{\rm self}(t, v, k)}$.
  \item If $\hastype{\tcenv}{v}{t}$, $\isWellFormedE{\tcenv}$, and $\isWellFormed{\tcenv}{t}{k}$, then $\hastype{\tcenv}{v}{{\rm self}(t, v, k)}$. %(LemmasExactness.hs) Our ${\tt lem\_exact\_type}$ depends on ${\tt lem\_exact\_subtype}$.
\end{enumerate}
\end{lemma}
%\NV{Put these two in one line if we need space}

\section*{Acknowledgements for Chapter 5}
%
This chapter is adapted from 
``Mechanizing Refinement Types'' in the proceedings of the 
$51^{st}$ ACM SIGPLAN Symposium on Principles of Programming
Languages (POPL 2024), by Michael Borkowski, Niki Vazou, and
Ranjit Jhala.
%
The dissertation author was the primary investigator 
and author of this material.


% REMOVED LEMMAS

%We can also prove analogous statements for the well-formedness judgments for environments. Next, we have the weakening lemma for well-formedness judgments. As with the free variables lemma, it even suffices for the erased environments to be well-formed.

%\begin{lemma}\label{WF-weaken}
%    For any environments $\tcenv$, $\tcenv'$ and $x,\al, \not\in dom(\tcenv', \tcenv)$:
%    If $\tcenv', \tcenv \vdash_w t:k$ and $\vdash_{w} \tcenv', \tcenv$, then $\tcenv', \bind{x}{t_x}, \tcenv \vdash_w t:k$ and $\tcenv', \al\bindt k_{\al}, \tcenv \vdash_w t:k$.
%\end{lemma}

%Next, we have the substitution lemma for well-formedness judgments.

%\begin{lemma}\label{WF-substitution}
%    For any environments $\tcenv$, $\tcenv'$ and $x,\al \not\in dom(\tcenv', \tcenv)$:
%    \item If $\tcenv', \bind{x}{t_x}, \tcenv \vdash_w t:k$ and
%    $\hastype{\tcenv}{v_x}{t_x}$, then
%    then $\tcenv'[v_x/x], \tcenv \vdash_w \subst{t}{x}{v_x}:k$.
%    \item If $\tcenv', \al\bindt k_{\al}, \tcenv \vdash_w t:k$ and
%    $\isWFFT{\tcenv}{t_{\al}}{k_{\al}}$,
%    then $\tcenv'[t_{\al}/\al], \tcenv \vdash_w t[t_{\al}/\al]:k$.
%\end{lemma}

%\begin{lemma}
%    {\em Typing implies Well-Formedness}: If $\hastype{\tcenv}{e}{t}$ and $\vdash \tcenv$ then $\isWellFormed{\tcenv}{t}{{}^{*}}$. (LemmasTyping.hs) Used in many, many places.
%\end{lemma}

%    \begin{lemma}
%        {\em ``Assumption 1'' / Typing of $\delta$}. In the mechanization we  assume something more conservative and then show:  If $\hastype{\varnothing}{c}{ x\bindt t_x->t'}$ and $\hastype{\varnothing}{v}{t_x}$ then $\hastype{\varnothing}{\delta(c,v)}{t'[v/x]}$ (PrimitivesDeltaTyping.hs) (and similarly for the polymorphic Eql). The latter one, our ${\tt lem\_deltaT\_typ}$ depends on ${\tt lem\_subst\_tv\_sub}$.
%    \end{lemma}

%    \begin{theorem}
%        Progress Theorem: If $\varnothing \vdash e : t$ then either $e$ is a value or there exists a term $e'$ such that $e \hookrightarrow$ e'. (MainTheorems.hs) No significant dependencies.
%    \end{theorem}

%    \begin{theorem}
%        Preservation Theorem: If $\varnothing \vdash e : t$ and $e \hookrightarrow e'$, then $\varnothing \vdash e' : t$. (MainTheorems.hs)  Our ${\tt thm\_preservation}$ depends on ${\tt lem\_sem\_det}$, ${\tt lem\_value\_stuck}$, ${\tt lem\_subst\_typ}$, ${\tt lem\_invert\_tabs}$, ${\tt lem\_invert\_tabst}$, ${\tt lem\_delta\_typ}$, ${\tt lem\_detaT\_typ}$, and ${\tt thm\_progress}$.
%    \end{theorem}

\begin{comment}
\section{Subtyping and Entailments}

In order to prove the substitution lemma (\Cref{lem:subst})
for the subtyping judgments, the case of \sBase requires
us to prove a corresponding lemma for entailment judgments.

\begin{lemma} (Entailment Substitution) \label{lem:entail-base}
    \begin{enumerate}
        \item If $\entails{\tcenv',\bind{x}{t_x},\tcenv}{p}$,
                 $\isWellFormedE{\tcenv',\bind{x}{t_x},\tcenv}$,
            and $\hastype{\tcenv}{v_x}{t_x}$,
            then $\entails{\subst{\tcenv'}{x}{v_x},\tcenv}{\subst{p}{x}{v_x}}$.
            %(SubstitutionLemmaEnt.hs) This one uses ${\tt lem\_denote\_sound\_typ}$ \\
        \item If $\entails{\tcenv',\bind{\al}{\skind},\tcenv}{p}$,
                 $\isWellFormedE{\tcenv',\bind{\al}{\skind},\tcenv}$,
            and $\hastype{\tcenv}{t_\al}{\skind}$,
            then $\entails{\subst{\tcenv'}{\al}{t_\al},\tcenv}{\subst{p}{\al}{t_\al}]}$.
            %(SubstitutionLemmaEntTV.hs) \\
    \end{enumerate}
\end{lemma}

The proof of the above lemma depends
on Denotational Soundness (\Cref{lem:denote-sound-first})
to prove the particular statement about denotations
in the antecedent of rule \entPred holds under substitution.
%
Specifically, we are given the statement
$\foralltheta{\theta}{\tcenv',\bind{x}{t_x},\tcenv}
    \Rightarrow \evalsTo{\theta(p)}{\ttrue}$
and we need to show the statement
$\foralltheta{\theta'}{\subst{\tcenv'}{x}{v_x},\tcenv}
    \Rightarrow \evalsTo{\theta'(\subst{p}{x}{v_x})}{\ttrue}$ holds.
The second statement is a special case of the first;
for any appropriate $\theta'$ and $v_x$, the closing substitution
$\theta \defeq (\theta', x\mapsto v_x)$ is appropriate to the first statement.
%
The proof hinges on Denotational Soundness,
described below, which gives us $v_x \in \denote{\theta'(t_x)}$
and on the commutativity of substitution
($v_x$ is a closed value and so has no variables appearing in it).

\end{comment}

% Ch 6: Data Props
\chapter{\lh \& Refined Data Propositions}
\label{ch:data-props}

In Chapter~\ref{ch:implementation} we will present how 
we proved 
\sysrf soundness in \lh. 
%
To do so, we developed \textit{refined data propositions}, 
a novel feature of \lh that made such 
a meta-theoretic proof possible. 
%% NEW
Although \lh had been previously used to prove theorems
inductively about provably terminating user defined 
functions, these extensions were needed to 
reason about potentially non-terminating properties, 
such as the typing judgment of \sysrf.
%%


\section{\lh}
\lh's core proof system is \sysrf, 
that is, it is using the typing judgement 
presented in Figure~\ref{fig:t} to check whether a Haskell program 
satisfies its refinement type annotations.
%
The expression language checked by \lh is GHC's intermediate language 
(\coresyn~\cite{coresyn}) which is a superset of \sysrf that also includes 
literals, datatypes, and coercions. Thus, \lh's typing judgement is extended to 
include these constructs. 
%
To guess the unknown types of Figure~\ref{fig:t} (\ie in the rules \tSub and \tLet)
and make the typing judgment algorithmic,
\lh  implements the refinement type inference algorithm of 
liquid types~\cite{LiquidPLDI08}.
%
To check the implications \lh uses the \iLog 
rule of~\S~\ref{sec:typing:implication:logical}
which are automatically discharged by an SMT solver.
% }

% \newtext{
\paragraph{\lh as a Theorem Prover}
Equipped with the SMT solver, \lh can be used to prove theorems 
over theories known to the SMT solver.
For example, that addition over integers is associative 
and that for every integer there exists a larger one, 
as encoded by the below functions: 
% }
\begin{mcode}
  assoc :: x:Int -> y:Int -> {v:() | x + y == y + x } 
  assoc _ _ = () 
  
  exLg :: x:Int -> (y::Int,{v:() | y > x })
  exLg x = (x+1, ())   
\end{mcode}
%
% \newtext{
These definitions use lambda abstraction and dependent pairs 
to respectively encode the universal and existential quantifiers.
To encode logical terms, such as \texttt{y > x}, they refine the unit type 
with such terms. 
Building upon this idea, \lh has been extensively used 
to prove theorems, using recursive Haskell definitions to 
encode inductive proofs and refinement reflection~\cite{Vazou18}
to allow user-defined terminating functions into the refinement logic. 
Yet, the proving power of \lh was limited because 
only provably terminating functions can be used in the refinement logic
and the proofs were implicitly performed by the SMT solver.  
Thus, the programmer could not inspect the proof terms. 
% }   

\section{Refined Data Propositions}
\label{sec:data-props}
%
Refined data propositions encode \coq-style
inductive predicates in \lh by refining Haskell's 
data types, allowing the programmer to write plain
Haskell functions to provide constructive proofs 
for user-defined propositions.
Here, for exposition, we present the four steps we followed  
in the mechanization of \sysrf to 
define the ``has-type'' proposition
and then use it to type the primitive one. 

\mypara{Step 1: Reifying Propositions as Data}
%
Our first step is to represent the propositions
of interest as plain Haskell data.
%
For example, we can define the following types 
(suffixed \ha{Pr} for ``proposition''):
%
\begin{mcode}
    data HasTyPr   = HasTyPr    Env  Expr Type
    data IsSubTyPr = IsSubTyPr  Env  Type Type
\end{mcode}
%
Thus, \ha{HasTyPr $\Gamma$ e t} and \ha{IsSubTyPr $\Gamma$ s t}
respectively represent the \emph{propositions} 
$\hastype{\Gamma}{e}{t}$ and 
$\isSubType{\Gamma}{s}{t}$,
which say
that \ha{e} can be typed as \ha{t} under environment $\Gamma$ and
that \ha{t} is a subtype of \ha{t'} under $\Gamma$.

\mypara{Step 2: Reifying Evidence as Data}
%
Next, we reify evidence, \ie \emph{derivation trees}
as data by defining Haskell data types with a
\emph{single constructor per derivation rule}.
%
For example, we define the data type \ha{HasTyEv}
to encode the typing rules of~\Cref{fig:t}, 
with constructors that match the names of each rule.
%
\begin{mcode}
  data HasTyEv where
    TPrim :: Env ->Prim ->HasTyEv
    TSub  :: Env ->Expr ->Type ->Type ->HasTyEv ->IsSubTyEv ->HasTyEv
    ...
\end{mcode}
Using these data one can construct derivation trees. 
For instance,  
\ha{TPrim Empty (PInt 1) :: HasTyEv} is the tree that types 
the primitive one under the empty environment. 

\mypara{Step 3: Relating Evidence to its Propositions}
%
Next, we specify the relationship between the evidence and the proposition
that it establishes, via a refinement-level \emph{uninterpreted function}:
%that maps evidence to its corresponding proposition:
%
\begin{mcode}
  measure hasTyEvPr   :: HasTyEv ->HasTyPr
  measure isSubTyEvPr :: IsSubTyEv ->IsSubTyPr
\end{mcode}
%
The above signatures declare that \ha{hasTyEvPr} (resp. \ha{isSubTyEvPr})
is a refinement-level function that maps has-type (resp. is-subtype)
evidence to its corresponding proposition.
%
We can now use these uninterpreted functions to define \emph{type aliases}
that denote well-formed evidence that establishes a proposition.
%
For example, consider the (refined) type aliases
%
\begin{mcode}
  type HasTy   $\Gamma$ e t = {ev:HasTyEv   | hasTyEvPr ev == HasTyPr $\Gamma$ e t }
  type IsSubTy $\Gamma$ s t = {ev:IsSubTyEv | isSubTyEvPr ev == IsSubTyPr $\Gamma$ s t }
\end{mcode}
%
The definition stipulates that the type \ha{HasTy $\Gamma$ e t}
is inhabited by evidence (of type \ha{HasTyEv}) that
establishes the typing proposition \ha{HasTyPr $\Gamma$ e t}.
%
Similarly \ha{IsSubTy $\Gamma$ s t} is inhabited by evidence
(of type \ha{IsSubTyEv}) that establishes the subtyping
proposition \ha{IsSubTyPr $\Gamma$ s t}.
%
Note that the first three steps have only defined separate data types
for propositions and evidence, and \emph{specified} the relationship
between them via uninterpreted functions in the refinement logic.

\mypara{Step 4: Refining Evidence to Establish Propositions}
%
Finally, we \emph{implement} the relationship between
evidence and propositions \emph{refining} the types of the evidence
data constructors (rules) with pre-conditions that require
the rules' premises and post-conditions that ensure
the rules' conclusions.
%
For example, we connect the evidence and proposition for the
typing relation by refining the data constructors for \ha{HasTyEv}
using their respecting typing rule
from ~\Cref{fig:t}.
%
\begin{mcode}
  data HasTyEv where
    TPrim :: $\Gamma$:Env ->c:Prim ->HasTy $\Gamma$ (Prim c) (ty c)
    TSub  :: $\Gamma$:Env ->e:Expr ->s:Type ->t:Type 
          ->HasTy $\Gamma$ e s ->IsSubTy $\Gamma$ s t ->HasTy $\Gamma$ e t
    ...
\end{mcode}
%
The constructors \ha{TPrim} and \ha{TSub} respectively
encode the rules \tPrim and \tSub (with 
well-formedness elided for simplicity).
%
The refinements on the input types, 
which encode the premises of the rules,
are checked whenever these constructors are used. 
The refinement on the 
output type (being evidence of a specific proposition)
is axiomatized to encode the conclusion of the rules.
%
For example, the type for \ha{TSub}
says that ``for all $\Gamma, e, s, t$,
given evidence that $\hastype{\Gamma}{e}{s}$ and $\isSubType{\Gamma}{s}{t}$'',
the constructor returns ``evidence that $\hastype{\Gamma}{e}{t}$''.

\begin{fullversion}

\mypara{Programs as Constructive Proofs}
%
Thus, the constructor refinements crucially ensure that only well-formed pieces of evidence
can be constructed, and simultaneously, precisely track the proposition established
by the evidence.
%
This lets the programmer write plain Haskell terms as constructive proofs, and \lh ensures
that those terms indeed establish the proposition stipulated by their type.
%
For example, the below Haskell term is proof that the literal \ha{1} has the type
$\breft{\Int}{\vv}{\vv = 1}$
%
\begin{mcode}
  oneTy :: HasTy Empty (EPrim (PInt 1)) {v:Int | v == 1}
  oneTy = TPrim Empty (PInt 1)
\end{mcode}
%
If instead, the programmer wrote
\ha{oneTy = TPrim Empty (PInt 2)}, \lh would reject  
this
%the program 
as the modified evidence does not establish
the proposition described in the type.
%
\end{fullversion}

% As an example, below we use the primitive and subtyping constructors
% to construct the witness (equivalently derivation tree) that \ha{1}
% is a positive integer.
% %
% \begin{mcode}
%     onePos :: HasTy Empty (EPrim (PInt 1)) {v:Int | 0 < v}
%     onePos = TSub Empty (EPrim (PInt 1))
%                   {v:Int | 1 = v} (TPrim Empty (PInt 1)) -- $\cmt{\hastype{\varnothing}{1}{\breft{\tint}{\vv}{\vv = 1}}}$
%                   {v:Int | 0 < v} (SBase ...) -- $\cmt{\isSubType{\varnothing}{\breft{\tint}{\vv}{\vv = 1}}{\breft{\tint}{\vv}{0 < \vv }}}$
% \end{mcode}

\mypara{Implementation of Data Propositions}
Data propositions are a novel feature required to encode 
inductive propositions in
the mechanization of \sysrf. 
(\citet{lweb} developed a \lh metatheoretic proof  
but before data propositions and thus had to axiomatize a terminating 
evaluation relation; see~\S~\ref{ch:related}.)
%
% \newtext{
Refined data propositions are implemented as part of \lh's existing 
refined data types that already supported subtyping on constructor arguments 
using variant and contravariant rules, as described but not formalized in~\cite{sprite}. 
The essential extension to support data propositions is that by refining the output types 
of inductive data types, \lh can support constructive derivation-tree-style proofs.
% }                                              
%
To use this feature in practice, we had to extend the refinement logic 
of \lh to use existing SMT support to make data constructors \emph{injective}, 
\ie if $C$ is a constructor then $\forall x, y.\, C(x) = C(y) => x = y$.
Thus, refined data types and injectivity are the two required 
components to implement data propositions.



% Ch 7: 7.1, Implementation
\chapter{Implementation and Mechanization}
\label{ch:implementation}

\section{\lh Mechanization}

We mechanized type safety (\Cref{lem:soundness}) 
of \sysrf in both \coq 8.15.1 and \lh 8.10.7.1
(available online as supplementary
material).
%
In \lh we use refined data propositions 
(\S~\ref{sec:data-props}) to specify the
static (\eg typing, subtyping, well-formedness)
and dynamic (\ie small-step transitions and their closure)
semantics of \sysrf.
%
\begin{fullversion}
  The \lh mechanization is simplified by 
  SMT-automation (\S~\ref{impl:settheory}),
  uses a co-finite encoding for reasoning about
  variables (\S~\ref{subsec:implementation:co-finite}),
\end{fullversion}
\begin{conference}
  The \lh mechanization is simplified by 
  SMT-automation (\S~\ref{impl:settheory})
\end{conference}
and consists of proofs implemented as recursive functions that
construct  evidence to establish propositions by
induction (\S~\ref{impl:proofs}).
%


  Other that the development of data propositions, we extended \lh 
  with two more features during the development of this proof. 
  First, we implemented an interpreter that 
  critically dropped the verification time from 10 hours 
  to only 29 minutes (\S\ref{subsec:quantitiative}). 
  Second, we implemented a (\coq-style) strictly-positive-occurrence checker
  to ensure data propositions are well defined, 
  since early versions of our proof used negative occurrences. 

%
Note that while Haskell types are inhabited
by diverging $\bot$ values, \lh's totality,
termination, and type checks ensure that all
cases are handled, the induction (recursion)
is well-founded, and that the proofs (programs)
indeed inhabit the propositions (types).



\subsection{Quantitative Results}
\label{subsec:quantitiative}

We provide a mechanically checked
proof of the type safety in~\S~\ref{sec:soundness}, 
that only assumes the requirements
\ref{lem:prim-typing} and \ref{lem:implication}. 
%
%The only facts that are assumed are 
%Req. \ref{lem:implication}, \ie the implication interface
%which we encoded as a data proposition, 
%and 
%Req. \ref{lem:prim-typing}, \ie assumptions about built-in primitives. 
Concretely, we assumed the primitives \Cref{lem:prim-typing} for some 
constants of \sysrf because 
it was too strenuous to mechanically prove 
without interactive aid. 
%
In \lh type denotations (of \Cref{fig:den}) 
cannot be currently encoded: 
since they include $\forall$-quantification
they could only be encoded as data propositions, 
but the strictly-positive-occurrence checker 
rejects the definition of the function denotation. 
Due to this limitation, 
we can neither define the denotational implementation
of the implication (\S~\ref{sec:typing:implication:denotational})
nor prove 
the denotational soundness (\Cref{lem:denote-sound-first}). \NV{CHECK}


% \newtext{
  \mypara{Representing Binders}
One main challenge in the mechanized
metatheory is the syntactic representation
of variables and binders~\cite{Aydemir05}.
%
The \emph{named} representation
has severe difficulties because
of variable capturing substitutions
and the \emph{nameless} (\aka de Bruijn)
requires heavy index shifting.
%
The variable representation
of $\sysrf$ is
\emph{locally nameless representation}~\cite{Pollack93,AydemirCPPW08},
where free variables are named, but
bound variables are represented by
%syntactically distinct 
deBruijn indices.
%
Our mechanization still resembles
the paper and pencil proofs (performed
before  mechanization),
yet it clearly addresses
the following two problems with named bound variables.
%
First, when different refinements are strengthened
(as in \Cref{fig:type-subst}) the variable
capturing problem reappears
because we are substituting underneath a binder.
%
Second, subtyping usually permits
alpha-renaming of binders,
which breaks a required invariant
that each \sysrf derivation tree
is a valid \sysf tree after erasure. \\
% }

\begin{fullversion}
\mypara{Representing Binders}
%
In our mechanization, we use the
\emph{locally-nameless representation} \cite{AydemirCPPW08,Chargueraud12}.
%
Free variables and bound variables
are taken to be separate syntactic
objects, so we do not need to worry
about alpha renaming of free variables
to avoid capture in substitutions.
%
We also use de Bruijn indices only
for bound variables. This enables us to avoid
taking binder names into account in the
$\mathsf{strengthen}$ function used to define
substitution (\Cref{fig:type-subst}).
%
% While the uniqueness of names in the environment
% is convenient for the mechanized metatheory,
% it comes at the cost of renaming lemmas that
% let us change the name of bound variables.
% %
% In \S~\ref{sec:related} we discuss another
% approach to the mechanization \cite{AydemirCPPW08}
% that avoids the frequent use of renaming.
%
\end{fullversion}

\Cref{fig:empirical} summarizes the development 
of our metatheory,
which was checked using \lh 8.10.7.1
and a Lenovo ThinkPad T15p laptop with
an Intel Core i7-11800H processor.
%with 8 physical cores and 32 GB of RAM.
%
Our mechanized proofs are substantial. The entire
\lh development comprises over 12,800 lines \NV{CHECK these numbers are outdated? cannot find them in the table}
across about 35 files. 
%
Currently, the whole \lh proof can be checked
in 29 minutes, which makes interactive
development difficult, especially compared to 
the \coq proof (\S~\ref{sec:coq}) that 
is checked in about 60 seconds.
%
While incremental modular checking provides
a modicum of interactivity, improving the
ergonomics of \lh, 
\ie verification time and 
actionable error messages, remains an important
direction for future work.

\begin{table}
% \begin{center}
\setlength{\tabcolsep}{10pt}
\scalebox{0.90}{
\begin{tabular}{ lrrrr|rrr  }
  \multicolumn{1}{l}{ } & 
  \multicolumn{4}{c|}{\lh Mechanization} & 
  \multicolumn{3}{l}{\coq Mechanization} \\
  \toprule
  \textbf{Subject} & \textbf{Files} & \textbf{Time (m)} &
  \textbf{Spec} & \textbf{Proof} &  
  \textbf{Files} & \textbf{Spec} & \textbf{Proof} \\
  \hline
  Definitions             & 6      &  1     & 1805  &  374  & 7      &  941  &  190 \\
  Basic Properties        & 8      &  4     &  646  & 2117  & 8      & 1201  & 2360 \\
  $\sysf$ Soundness       & 4      &  3     &  138  &  685  & 4      &  173  &  773 \\
  Weakening               & 4      &  1     &  379  &  467  & 4      &  110  &  568 \\
  Substitution            & 4      &  7     &  458  &  846  & 4      &  158  &  859 \\
  Exact Typing            & 2      &  4     &   70  &  230  & 2      &   33  &  182 \\
  Narrowing               & 1      &  1     &   88  &  166  & 1      &   54  &  262 \\
  Inversion               & 1      &  1     &  124  &  206  & 1      &   57  &  258 \\
  Primitives              & 3      &  4     &  120  &  277  & 3      &   89  &  508 \\
  $\sysrf$ Soundness      & 1      &  1     &   14  &  181  & 1      &   12  &  233 \\
  Denotational Soundness  & -      &  -     &    -  &    -  & 13     &  815  & 3010 \\
  \midrule
  \textbf{Total}          & 35     & 29     & 3842  & 5549  & 49     & 3643  & 9203 \\
  \bottomrule
\end{tabular}
}
% \end{center}
\caption{Quantitative mechanization details. We split each development into sets of
         modules pertaining to regions of \Cref{fig:graph} and for each we
         count lines of specification (definitions, lemma statements)
         and of proof.}
\label{fig:empirical}
\vspace{-0.00cm}
\end{table}


% Section 7.2, Coq
\section{\coq Mechanization}
\label{sec:coq}

Our \coq mechanization 
proves both type safety and
denotation soundness, \ie all the statements of~\S~\ref{sec:soundness}
and serves as a comparison for the metatheoretical 
development abilities of the two theorem provers. 
%is a translation from \lh
%and was built to compare the two developments. 
%
%All theorems from~\S~\ref{sec:soundness} are proven in \coq. %(\ie the proof has no \ha{Admitted}.)% and zero trusted code base. 
%
In \coq, 
Req. \ref{lem:prim-typing} 
is proved (using \coq's interactive development)
and type denotations (of \Cref{fig:den})
are defined as recursive functions using 
Equations~\cite{10.1145/3341690}, 
which make both the 
definition the denotational implementation
of the implication (\S~\ref{sec:typing:implication:denotational})
and the proof  
the denotational soundness (\Cref{lem:denote-sound-first})
possible. 
\begin{fullversion}
    The implication judgment
    is  axiomatized per Requirement \ref{lem:implication}.
\end{fullversion}
%
To fairly compare the two developments
%In order to better understand the relative strengths 
%and tradeoffs of \lh vs. \coq 
in terms of effort and ergonomics,
we did not use external \coq libraries 
%and implemented our own infrastructure 
because no such libraries exist yet for \lh.
%
\citet{Vazou17} previously compared \lh and \coq 
as theorem provers, but their mechanizations were an order of magnitude
smaller than ours and did not use data propositions (\S~\ref{sec:data-props}),
which permit constructive \lh proofs. 

\mypara{\coq \vs \lh}
\coq has a tiny trusted code base (TCB) 
and strong foundational mechanized soundness 
guarantees \cite{coqcoqcorrect}.
%
In contrast, \lh trusts the Haskell compiler (GHC), 
the SMT solver (Z3), and its constraint generation rules 
which have not been formalized. 
%
This work, \sysrf, serves precisely that purpose: by
formalizing and mechanizing a significant subset of \lh, 
leaving out literals, casts, and data types. 
%
As far as the user experience is concerned, 
\coq metatheoretical developments 
are much faster to check, which was 
expected since \lh comes with expensive 
inference, and can be aided by relevant libraries. 
The two tools come with different kinds of automation: 
tactics \vs SMT, which we found to be useful in \emph{complementary} 
parts of the proofs, pointing the way to possible improvements
for both verification styles.
Finally, \lh facilitates reasoning over 
mutually defined and partial functions. 
%
%Next, we expand upon the last two points
%with snippets from our mechanizations. 
%
\begin{fullversion}
We begin by looking at aspects of mechanized metatheory in \coq
that are easier or more feasible than in \lh, and then we
turn to aspects that are easier in \lh.
\end{fullversion}

\mypara{Negative Occurrences and \coq's Equations}\NV{NEW CHECK}
Our original \lh mechanization defined  
denotations as refined data propositions
and proved denotational soundness. 
Though, we realized that the definition 
of the function type denotation
has a negative occurrence and permitting negative occurrences
can, in general, lead to unsoundness~\cite{CP90}. 
%
Our mechanization is the first big-scale user of 
\lh's data propositions thus it was not surprising that it revealed 
this potential unsoundness.
%
To remove this source of unsoundness in \lh, 
we implemented a \coq-style positivity checker
that unsurprisingly rejected the type denotation definitions. 
%
A similar challenge appears in the proof of strong normalization 
of the simply-type lambda calculus that because of negative occurrences
cannot use inductive propositions \cite{Pierce:SF2}.
There, the solution is to use a recursive function \ha{expr -> type -> Prop}
because a definition doesn't need to be computable.
%
In our \coq mechanization, we followed a similar solution, 
but since our definition was not structurally recursive and 
was needed for the proofs, we used 
the full power of \coq's Equations~\cite{10.1145/3341690} to define 
the type denotations. 
%
Unfortunately, a similar approach cannot currently carry over to \lh
because all Haskell functions must be computable and all
\lh annotations must be decidable. Therefore, quantifiers
are neither allowed on the right-hand side of Haskell
definitions nor in the refinements. % of Liquid types.


\mypara{Tactics and Automation}
\coq's tactics and automation often permit shorter
proofs as lemmas and constructors can be used with the
\ha{apply} tactic without writing out all arguments. 
%
For example, in \lh 
safety (thm.~\ref{lem:soundness})
is encoded using 
Haskell's \ha{Either}
for disjunction
and dependent pairs for existentials.
%
(\ha{Steps} is defined, using data propositions, as the 
% reflexive and transitive 
closure of \ha{Step}.)
%
\begin{mcode}
  safety :: e$_0$:Expr -> t:Type -> e:Expr -> HasTy Empty e$_0$ t 
         -> Steps e$_0$ e -> Either {isVal e}  (e$_i$::Expr, Step e e$_i$)
  safety _e$_0$ t _e e$_0$_has_t e$_0$_evals_e = case e$_0$_evals_e of
     Refl e$_0$ -> progress e$_0$ t e$_0$_has_t       -- $\cmt{e_0 = e}$
     AddStep e$_0$ e$_1$ e$_0$_step_e$_1$ e e$_1$_eval_e ->  -- $\cmt{e_0 \step \evalsTo{e_1}{e}}$
       safety e$_1$ t e (preservation e$_0$ t e$_0$_has_t e$_1$ e$_0$_step_e$_1$) e$_1$_eval_e
\end{mcode}
%
%\begin{fullversion}
    The reflexive case is proved by \ha{progress}.
    In the inductive case the evaluation
    sequence is $e_0 \step \evalsTo{e_1}{e}$
    and the proof goes by induction,
    using preservation to ensure that
    $e_1$ is typed.  
%\end{fullversion}
%
In \coq safety is proved 
without any of the three fully applied calls above:
%
\begin{mcode}
  Theorem safety : forall (e$_0$ e:expr) (t:type),
     Steps e$_0$ e -> HasTy Empty e$_0$ t -> isVal e \/ exists e$_i$, Steps e e$_i$.
  Proof. intros; induction H.
    - (* Refl *) apply progress with t; assumption.
    - (* Add  *) apply IHSteps; apply preservation with e; assumption. Qed.
\end{mcode}
%
Automation tactics could make this proof even shorter,
but we retain the essential proof structure.

\mypara{Mutual Recursion}
\lh makes it easy to define and work with mutually
recursive data types, 
such as our typing and subtyping judgments,
and to prove mutually inductive lemmas.
%
\begin{fullversion}
    Similarly, our 
    expressions, types, and predicates are
    three mutually recursive data types.
\end{fullversion}
%
Mutually recursive types are not a natural fit for \coq: 
the automatically generated induction principles do not work,
so we need to use the \ha{Scheme} keyword to generate  
suitable principles.
%
Theorems involving these types cannot be
broken up into separate lemmas for each type 
involved. 
%Rather, a unified approach must
%be taken which leads to theorems that are difficult to use
%directly in the \ha{rewrite} tactic.
%
Rather, one combined statement must be given,
which is difficult to use %directly 
in the \ha{rewrite} tactic.


Another weakness of \coq is that all 
information about the hypothesis is lost during the induction
tactic, so the normal structural \ha{induction} tactic only works when
a judgment contains no information, \ie the data constructor
is instantiated solely with universally quantified
variables. 
%
For instance, in the proof of the weakening~\Cref{lem:weakening},
to do structural induction on %a judgment like 
\ha{HasTy (concat g g')  e t}
we must introduce a universally quantified variable \ha{g0}
and strengthen the theorem with the hypothesis
\ha{g0 = concat g g'}.
%
While the standard library contains an ``experimental'' tactic 
\ha{dependent induction}, we also need to work with 
the special mutual induction principles that we generate for
our types, so we have to directly instantiate the principle
with a strengthened, complex hypothesis%. 
%\begin{comment}
and state the lemma as:
\begin{mcode}
  Lemma lem_weaken_typ' : ( forall (g0 : env) (e : expr) (t : type),
    HasTy g0 e t ->( forall (g g' : env) (x : vname) (t_x : type),
        g0 = concatE g g' ->unique g ->unique g' ->
        (binds g) $\cap $ (binds g') = empty ->~ (in_env x g) ->~ (in_env x g')
        ->HasTy (concatE (Cons x t_x g) g') e t ) ) /\ (
  forall (g0 : env) (t : type) (t' : type),
    Subtype g0 t t' ->( forall (g g' : env) (x : vname) (t_x : type),
        g0 = concatE g g' ->unique g ->unique g' -> 
        (binds g) $\cap $ (binds g') = empty ->~ (in_env x g) ->~ (in_env x g')
        ->Subtype (concatE (Cons x t_x g) g') t t') ).
\end{mcode}
%Proof. apply ( judgments_mutind 
%  (fun (g0 : env) (e : expr) (t : type) (p_e_t : Hastype g0 e t) => 
%    forall (g g':env) (x:vname) (t_x:type),
%      g0 = concatE g g' -> unique g -> unique g'
%                         -> intersect (binds g) (binds g') = empty
%                         -> ~ (in_env x g) -> ~ (in_env x g') 
%                         -> Hastype (concatE (Cons x t_x g) g') e t )
%  (fun (g0 : env) (t : type) (t' : type) (p_t_t' : Subtype g0 t t') => 
%    forall (g g':env) (x:vname) (t_x:type),
%      g0 = concatE g g' -> unique g -> unique g'
%                         -> intersect (binds g) (binds g') = empty
%                         -> ~ (in_env x g) -> ~ (in_env x g') 
%                         -> Subtype (concatE (Cons x t_x g) g') t t')  
%  ); ...
%
%\end{comment}
By contrast, in \lh we can state two separate mutually recursive
lemma functions for weakening: one for typing
and one for subtyping.
%
Then we may call either lemma in their own proofs on any
smaller instance of the typing (resp. subtyping) judgment.
%
%\begin{fullversion}
In practice, developments in \coq sidestep some of these issues
by collapsing the language of terms, types, \etc 
into a single inductive data type. 
%
This approach has the advantage of reducing the number of substitution
operations, but allows highly ungrammatical
combinations like \ha{App Bool False} into our syntax. 
%
We could still use this approach 
combined with a pre-term encoding common in \coq developments,  
but we preferred to keep a closer comparison to the \lh mechanization.
%The solution to this is to define these expressions to be pre-terms and then
%to define terms and types to be an inductive proposition consisting of a 
%pre-term and a proof of being a well-formed term or type. 
% 
%We avoided this common approach in order to keep a closer comparison
%to the \lh mechanization.
%\end{fullversion}

%Sometimes, a lemma cannot be proven 
%For example, the Transitivity of Subtyping cannot be proven 
%by structural induction on the two subtyping judgments because
%in one case we need to use the Substitution Lemma before
%invoking the inductive hypothesis.
%Instead, it must proceed by induction on the combined size of
%the three types involved. In \coq, this requires defining and 
%arguing about an ad-hoc measure and encoding strong induction
%over \ha{Nat} into the statement of the Transitivity Lemma itself.

\mypara{Partial Functions} 
\lh facilitates the definition of partial Haskell functions 
and proves totality with respect to the refined types,
usually automatically, without 
having to reason about impossible cases in mechanized proofs.
%
For instance, our syntax does not contain an explicit 
\ha{error} value, so we only want the function
$\delta(c,\sval)$ to be defined where $\app{c}{\sval}$ can 
step in our semantics.
%
This is straightforward in \lh: we define a predicate 
\ha{isCompat :: Prim ->Value ->Bool} and refine the input
types of $\delta$ to satisfy \ha{isCompat}.
%
%\begin{fullversion}
% NV: isCompat' is not even used in the text jeje
%\begin{mcode}
%  Definition isCompat' (c : prim) (e : expr) : bool :=
%    match c, e with | And , (Bc _)      => true
%                    | Or  , (Bc _)      => true
%                    ...
%                    | _        , _      => false end.
%\end{mcode}
%
%\end{fullversion}
In \coq a more roundabout approach is needed: we have to define 
\ha{isCompat} as an inductive type and include this object as
an explicit argument to our $\delta$ function:
%\begin{fullversion}
%
\begin{mcode}
  Inductive isCompat : prim -> expr -> Set : =   
    | isCpt_And  : forall b, isCompat And (Bc b)
    | isCpt_Or   : forall b, isCompat Or  (Bc b) 
    ...
\end{mcode}
%
%\end{fullversion}
However, this makes it harder to prove the determinism of our
semantics due to the dependence on the proof object. 
%
One solution would be to define a partial version of 
$\delta$ with type
\ha{Prim ->Expr ->option Expr} and prove the two functions
always agree regardless of proof object, \eg using \emph{subset types};
but since  each value
comes wrapped with a term-level proof object,
agreement proofs would require a \emph{Proof Irrelevance} axiom.
%, and 
%as reasoning about equality between objects of a 
%subset type is not possible in general sans 
%a \emph{Proof Irrelevance} axiom \cite{Vazou17}. 
%


% Section 7.3, Conclusions and Future Work
\chapter{Conclusions \& Future Work}
\label{ch:future}
  
We presented and formalized, for the first time, the soundness of
\sysrf and \sysrfd : the former is 
a refinement calculus with semantic subtyping, existential 
types, and parametric polymorphism, which are critical for practical 
refinement typing. 
%
The latter adds basic concrete data types and measures: lists and
a built-in length function.
%
Our metatheory is mechanized in both \coq and (for the core $\sysrf$) 
\lh, the latter
using the novel feature of refined data propositions to reify 
derivations as (refined) Haskell datatypes, using SMT
to automate invariants about variables.

While our proof can be mechanized in other proof assistants like
\agda~\cite{agda},
\isabelle~\cite{NPW2002},
\beluga~\cite{beluga},
\dafny~\cite{Dafny}, or
\fstar~\cite{metafstar},
% or those equipped with SMT-based
% automation like \dafny \cite{Dafny}
% or \fstar \cite{metafstar},
our goal here is not to compare \lh against every system.
%
Instead, our primary contribution is to,
for the first time, \emph{establish the soundness}
of the combination of features critical for practical
refinement typing and show that such a proof can be 
\emph{mechanized as a plain program} with refinement types.
%
% To achieve this mechanization we had to develop data propositions 
% that allow constructive verification in \lh, which naturally led to a 
% comparison with a \coq mechanization.
%
% \newtext{
\mypara{Metatheory} Looking ahead, 
we envision two lines
of work on mechanizing metatheory \emph{of}
and \emph{with} refinement types.
% }

% \newtext{
\mypara{1. Mechanization of Refinements}
%
\sysrf covers a crucial but small
fragment of the features of modern refinement
type checkers.
%
Building on this, \sysrfd takes the first step 
towards data types.
% 
The next step is to extend the language 
to include literals, casts, and arbitrary user-specified data types, 
thus covering \textit{all} GHC's core calculus. 
Next, \sysrf can be extended to more sophisticated 
features of refinement types, such as 
abstract and bounded refinements and refinement reflection. 
%
Similarly, our current work axiomatizes the
requirements of the semantic implication
checker (\ie SMT solver).
It would be interesting to implement a solver
and verify that it satisfies that contract, or
alternatively, show how proof certificates \cite{pcc}
could be used in place of such axioms.
% }

% \newtext{
\mypara{2. Mechanization with Refinements}
%
While this work shows that non-trivial
meta-theoretic proofs are \emph{possible}
with SMT-based refinement types, our experience
is that much remains to make such developments
\emph{pleasant}.
%
For example, programming would be far more convenient
with support for automatically \emph{splitting cases}
or filling in \emph{holes} as done in Agda \cite{agda}
and envisioned by \citet{hole_driven_liquid}.
%
Similarly, when a proof fails, the user has little
choice but to think really hard about the internal
proof state and what extra lemmas are needed to prove
their goal.
%
Finally, the stately pace of verification --- 9400 lines
across 35 files take about 30 minutes --- hinders
interactive development.
%
Thus, rapid incremental checking, lightweight synthesis,
and actionable error messages would go a long way towards
improving the ergonomics of verification, and hence remain
important directions for future work.
% }
%% Verified SMT Axioms or Certificates
%% Denotations ~ Implication via negative types
%% Datatypes + Polymorphism
%% Ergonomics

\mypara{Algorithmic type checking} The key motivation behind
the use of liquid types in practical systems is the ability to 
typecheck programs decidably.
%
When refinements are restricted to a decidable logic, then 
verification can be done by an SMT solver
without reliance on brittle heuristics.
%
As a next step, we aim to show that our typing system can be
equivalently cast in the form of a bi-modal or ``bidirectional''
algorithm that combines checking typing obligations (from 
explicit annotations left by the programmer or at function
application sites, for instance) 
with synthesizing types from program subterms where possible
\cite{bidir-survey}. 
%
Combined with restrictions on the syntax of refinements, this would 
make typechecking for a significant subset of \sysrfd programs decidable.


\section*{Acknowledgements for Chapter 10}
%
This chapter is adapted from 
``Mechanizing Refinement Types'' in the proceedings of the 
$51^{st}$ ACM SIGPLAN Symposium on Principles of Programming
Languages (POPL 2024), by Michael Borkowski, Niki Vazou, and
Ranjit Jhala.
%
The dissertation author was the primary investigator 
and author of this material.

% Chapter 8: Comparison of Coq and LH
\chapter{Comparison of Proof Assistants}

\section{Proving Theorems in \lh}

\subsection{SMT Solvers, Arithmetic, and Set Theory} \label{impl:settheory}

The most tedious part in the mechanization of metatheories
is the establishment of invariants about variables,
for example uniqueness and freshness.
%
\lh offers a built-in, SMT automated support for the
theory of sets, which simplifies establishing such
invariants.

\begin{fullversion}
  \mypara{Set of Free Variables}
  Our proof mechanization defines the Haskell function
  \ha{fv} that returns the \ha{Set} of free variable names
  that appear in its argument.
  \begin{mcode}
  measure fv
  fv :: Expr -> S.Set VName
  fv (EVar x)    = S.singleton x
  fv (ELam e)    = fv e
  fv (EApp e e') = S.union (fv e) (fv e')
  ... -- other cases
  \end{mcode}
  In the above (incomplete) definition, \ha{S}
  is used to qualify the standard \verb+Data.Set+ Haskell library.
  \lh embeds the functions of \verb+Data.Set+ to SMT
  set operators (encoded as a map to booleans).
  For example, \ha{S.union} is treated as the
  logical set union operator $\cup$.
  %
  Further, we lift \ha{fv} into the refinement
  logic using the \ha{measure fv} annotation.
  %
  % (and in general we use the special comments
  % \ha{$\{$-@ ... @-$\}$}) to provide \lh specific
  % annotations.
  %
  The measure definition  defines the logical
  function \ha{fv} in the logic in a way that
  lets the SMT solver reason about the semantics
  of \ha{fv} in a \textit{decidable} fashion,
  as an uninterpreted function refining the
  type of each \ha{Expr} data constructor
  \cite{sprite}.
  %
  This embedding, combined with the SMT solver's
  support for the theory of sets, lets \lh prove
  properties about expressions' free variables
  ``for free''.
\end{fullversion}

\mypara{Intrinsic Verification}
%
\begin{conference}
  \lh embeds the functions of the standard
  \verb+Data.Set+ Haskell library as SMT
  set operators. Given a Haskell function,
  \eg the set of free variables in an expression,
  this embedding, combined with SMT's
  support for set theory, 
  lets \lh prove
  properties about  free variables
  ``for free''.
\end{conference}
%
For an example of properties for free, consider the function
\ha{subFV x vx e} which substitutes
the variable \ha{x} with
\ha{vx} in \ha{e}.
%
The refinement type of \ha{subFV} 
describes the free variables of the result.
%
\begin{mcode}
  subFV :: x:VName ->vx:{Expr | isVal vx } ->e:Expr
        ->{e':Expr | fv e' $\subseteq$ (fv vx $\cup$ (fv e \ x)) && (isVal e => isVal e')}
  subFV x vx (EVar y)    = if x == y then vx else EVar y
  subFV x vx (ELam   e)  = ELam (subFV x vx e)
  subFV x vx (EApp e e') = EApp (subFV x vx e) (subFV x vx e')
  ... -- other cases
\end{mcode}
%
The refinement type % post condition 
specifies
that the free variables after substitution is a subset
of the free variables in the two argument expressions,
excluding \ha{x}, \ie
$\mathtt{fv}(\subst{e}{x}{v_x}) \subseteq
\mathtt{fv}(v_x) \cup (\mathtt{fv}(e) \setminus \{x\})$.
This specification is proved \emph{intrinsically},
\ie the definition of \ha{subFV} is the proof
(no user aid is required) and, importantly,
the specification is automatically established each time
the function \ha{subFV} is called 
without any need for explicit hints.
%So, the user does not have to provide explicit hints to reason
%about free variables of substituted expressions.
%
The specification of \ha{subFV} above shows another example
of SMT-based proof simplification. 
It intrinsically proves that the value property is preserved
by substitution, 
using 
the Haskell boolean
function \ha{isVal} that defines
when an expression is a \emph{value}. 

%as stated by the second (implication) conjunct
%in the output of \ha{subFV}.


\begin{fullversion}
\mypara{Freshness}
%
\lh's support for sets simplifies
defining a \ha{fresh} function,
which is often challenging\footnote{For example, \coq cannot fold over a set,
and so a more complex combination of tactics is
required to generate a fresh name.}.
%
\ha{fresh xs} returns a variable that provably
does not belong to its input \ha{xs}.
%
\begin{mcode}
  fresh :: xs:S.Set VName -> { x:VName | x $\not\in$ xs }
  fresh xs = n ? above_max n xs'
    where n    = 1 + maxs xs'
          xs'  = S.fromList xs

  maxs :: [VName] -> VName
  maxs []     = 0
  maxs (x:xs) = if maxs xs < x then x else maxs xs

  above_max :: x:VName -> xs:{[VName]|maxs xs < x} -> {x $\not\in$ elems xs}
  above_max _ []     = ()
  above_max x (_:xs) = above_max x xs
\end{mcode}
%
The \ha{fresh} function returns \ha{n}: the maximum element of the set
increased by one.
%
To compute the maximum element we convert the set to a list
and use the inductively defined \ha{maxs} functions.
To prove \ha{fresh}'s intrinsic specification we use
an extrinsic, \ie explicit, lemma \ha{above_max n xs'}
that, via the \ha{(?)} combinator of type \ha{a ->  b ->  a},
tells \lh that \ha{n} is not in the set \ha{xs}.
%
This extrinsic lemma is itself trivially proved
by induction on \ha{xs} and SMT automation.
\end{fullversion}

\begin{fullversion}
  \subsection{Co-finite Quantification}
  \label{subsec:implementation:co-finite}
  \begin{figure}[t!]
  \begin{mcode}
-- Standard Existential Rule
TAbsEx  :: $\gamma$:Env -> t$_x$:Type -> e:Expr -> t:Type
        -> y:{VName | y $\not\in$ dom $\gamma$ }
        -> HasTy ((y,t$_x$):$\gamma$) (unbind y e) (unbindT y t)
        -> HasTy $\gamma$ (ELam e) (TFunc t$_x$ t)

-- Co-finitely Quantified Rule
TAbs    :: $\gamma$:Env -> t$_x$:Type -> e:Expr -> t:Type -> l:S.Set VName
        -> (y:{VName|y $\not\in$ l} -> HasTy ((y,t$_x$):$\gamma$) (unbind y e) (unbindT y t))
        -> HasTy $\gamma$ (ELam e) (TFunc t$_x$ t)

-- Note: All rules also include k$_{\color{gray_ulisses}x}$:Kind -> WfType ${\color{gray_ulisses}\gamma}$ t$_{\color{gray_ulisses}x}$ k$_{\color{gray_ulisses}x}$ elided for clarity.
  \end{mcode}
  \caption{Encoding of Co-finitely Quantified Rules.}
  \label{fig:impl:co-finite}
  \end{figure}

  %-- Final Rule: Co-finitely Quantified and Explicit Size
  %TAbs    :: n:Nat -> $\gamma$:Env -> t$_x$:Type -> e:Expr -> t:Type
  %        -> l:[VName]
  %        -> ( y:{VName | y $\not\in$ S.fromList l } ->
  %             { pf:HasTy ((y,t$_x$):$\gamma$) (unbind y e) (unbindT y t)
  %                 | typSize pf < typSize this } )
  %        -> HasTy $\gamma$ (ELam e) (TFunc t$_x$ t)
  %

  To encode the rules that need a fresh free variable name
  we use the co-finite quantification of~\citet{AydemirCPPW08}, as discussed
  in~\S~\ref{sec:lang:static}.
  %
  \Cref{fig:impl:co-finite} presents
  this encoding using the \tAbs rule
  as an example.
  %
  The standard abstraction rule
  (rule \tAbsEx in~\S~\ref{sec:lang:static})
  requires to provide a concrete fresh name,
  which is encoded in the second line of
  \ha{TAbsEx} as the \ha{y:$\{$VName | y $\not\in$ dom $\gamma\}$} argument.
  %
  The co-finitely quantified encoding of the rule \ha{TAbs}, instead,
  states that there exists a specified finite set of excluded names, namely \ha{l},
  and requires that the sub-derivation holds for any name \ha{y}
  that does not belong in \ha{l}.
  %\NV{I really do not understand how l connects with gamma... I think we need to say something about it}
  %\NV{Also, why encoded as list and then }
  That is, the premise is turned into a function
  that, given the name \ha{y}, returns the sub-derivation.
  This encoding greatly simplifies our mechanization,
  since the premises are no more tied to concrete names,
  eliminating the need for renaming lemmas.
  %
  We will often take \ha{l} to be the domain
  of the environment, but the ability to choose \ha{l}
  gives us the flexibility when constructing derivations
  to exclude additional names that clash
  with another part of a proof.
\end{fullversion}
%However, this encoding introduces an interesting challenge
%in the construction of proofs by induction on the derivation
%tree (\S~\ref{impl:proofs}).
%
%\lh cannot conclude that the size of the derivation
%subtree is independent of $y$, which makes it impossible
%accept inductive hypotheses on quantified subtrees.
%This challenge does not appear in the \coq formalization
%of~\citet{AydemirCPPW08} as \coq generates induction
%principles that works over co-finitely quantified
%inductive hypotheses.
%
%Our final \ha{TAbs} rule takes the extra ghost
%size argument \ha{n} and ensures that the size
%of the conclusion is bounded by \ha{n+1}, if
%the size of the premise is bounded by \ha{n}.
%
%Now, induction on derivation trees is permitted,
%at the cost of providing an explicit size argument
%to quantified judgments.
%


\subsection{Inductive Proofs as Recursive Functions}
\label{impl:proofs}

The majority of our proofs are by induction on derivations.
These proofs are recursive Haskell functions that
operate over refined data propositions. 
%types reifying the respective derivations. 
\lh ensures the proofs are valid by checking that
they are inductive (\ie the recursion is well-founded),
handle all cases (\ie the function is total), and
establish the desired properties 
(\ie witnesses the appropriate proposition).

\mypara{Preservation (\Cref{lem:preservationF})}
relates the \ha{HasTy} data proposition of~\S~\ref{sec:data-props}
with a \ha{Step} data proposition that encodes~\Cref{fig:opsem}
and is proved by induction on the type derivation tree.
Below we present a snippet of the proof, where 
the subtyping case is by induction
while the primitive case is impossible: 
% (\Cref{lem:step-determ}): this does not seem the correct lemma
% also does not fit :)
%
\begin{mcode}
  preservation :: e:Expr ->t:Type ->e':Expr ->HasTy Empty e t 
               ->Step e e' ->HasTy Empty e' t
  preservation _e _t e' (TSub Empty e t' t e_has_t' t'_sub_t) e_step_e'
    = TSub Empty e' t' t (preservation e t' e' e_has_t' e_step_e') t'_sub_t
  preservation  e _t e' (TPrim _ _) step
    = impossible "value" ? lemValStep e e' step -- $e \step e' \Rightarrow \neg (\text{isVal}\ e)$
  ...
  impossible :: {v:String | false} ->a
  lemValStep :: e:Expr ->e':Expr ->Step e e' ->{$\neg$(isVal e)}
\end{mcode}
%where e'_has_t' = preservation e t' e' e_has_t' e_step_e'
%  preservation e _ e' (TAbs {}) step
%    = impossible "value" ? lemValStep e e' step -- $e \step e' \Rightarrow \neg (\text{isVal}\ e)$

In the \ha{TSub} case we note that \lh knows that
the argument \ha{_e} is equal to
the subtyping parameter \ha{e}.
%
The termination checker ensures
the inductive call happens on a smaller
derivation subtree.
%
The \ha{TPrim} case is by contradiction since
primitives cannot step:
%
we proved values cannot step
in the \ha{lemValStep} lemma, which
is combined 
via the \ha{(?)} combinator of type \ha{a ->  b ->  a}
with the fact that \ha{e} is a value
to allow the call of the false-precondition \ha{impossible}.

\lh's totality checker ensures
all cases of \ha{HasTyEv} are covered
and the termination checker ensures
the proof is well-founded. %(There are no
%size bounds on \ha{ProofOf (HasType Empty e' t)}
%as the preservation lemma does not use any
%co-finitely quantified rules.)

\begin{fullversion}
  \mypara{Progress (Theorem~\ref{lem:progressF})}
  ensures that a well-typed expression is a value
  \textit{or} there exists an expression to which
  it steps.
  %
  To express this claim we used Haskell's \ha{Either}
  to encode disjunction that contain pairs (refined to be dependent)
  to encode existentials.
  %
  % progress :: Expr -> Type -> HasType -> Either () (Expr,Step)
  %
  \begin{mcode}
  progress :: e:Expr -> t:Type -> HasTy Empty e t 
                     -> Either {isVal e}  (e'::Expr, Step e e')
  progress _ _ (TSub Empty e t' t e_has_t' _) = progress e t' e_has_t'
  progress _ _ (TPrim _ _)                    = Left ()
  progress _ _ (TAbs {})                      = Left ()
  ...
  \end{mcode}
  %
  The proofs of the \ha{TSub} and \ha{TPrim}
  cases are easily done by, respectively,
  an inductive call and establishing is-Value.
  %
  The more interesting cases require us to case-split
  on the inductive call in order to get access
  to the existential witness.
\end{fullversion}

% Note: condensed into the Coq Section
%\mypara{Soundness (Theorem~\ref{lem:soundness})}
%ensures that a well-typed expression will not get stuck,
%that is, it will either reach a value or keep evaluating.
%
%\begin{conference}
%  To express this claim we used Haskell's \ha{Either}
%  to encode disjunction that contain pairs (refined to be dependent)
%  to encode existentials.
%\end{conference}
%
%We reify evaluation sequences with a refined data
%proposition \ha{Steps e$_0$ e} with a reflexive
%and a transitive (recursive) constructor.
%
%Our soundness proof goes by induction on the
%evaluation sequence.
%
%\begin{mcode}
%  soundness :: e$_0$:Expr -> t:Type -> e:Expr -> HasTy Empty e$_0$ t -> Steps e$_0$ e
%                          -> Either {isVal e}  (e$_i$::Expr, Step e e$_i$)
%  soundness _e$_0$ t _e e$_0$_has_t e$_0$_evals_e = case e$_0$_evals_e of
%     Refl e$_0$ -> progress e$_0$ t e$_0$_has_t       -- $\cmt{e_0 = e}$
%     AddStep e$_0$ e$_1$ e$_0$_step_e$_1$ e e$_1$_eval_e ->  -- $\cmt{e_0 \step \evalsTo{e_1}{e}}$
%       soundness e$_1$ t e (preservation e$_0$ t e$_0$_has_t e$_1$ e$_0$_step_e$_1$) e$_1$_eval_e
%\end{mcode}
%
%The reflexive case is proved by \ha{progress}.
%In the inductive case the evaluation
%sequence is $e_0 \step \evalsTo{e_1}{e}$
%and the proof goes by induction,
%using preservation to ensure that
%$e_1$ is typed.

}

\section{\coq \vs \lh}
\coq has a tiny trusted code base (TCB) 
and strong foundational mechanized soundness 
guarantees \cite{coqcoqcorrect}.
%
In contrast, \lh trusts the Haskell compiler (GHC), 
the SMT solver (Z3), and its constraint generation rules 
which have not been formalized. 
%
This work, \sysrf, serves precisely that purpose: by
formalizing and mechanizing a significant subset of \lh, 
leaving out literals, casts, and data types. 
%
As far as the user experience is concerned, 
\coq metatheoretical developments 
are much faster to check, which was 
expected since \lh comes with expensive 
inference, and can be aided by relevant libraries. 
The two tools come with different kinds of automation: 
tactics \vs SMT, which we found to be useful in \emph{complementary} 
parts of the proofs, pointing the way to possible improvements
for both verification styles.
Finally, \lh facilitates reasoning over 
mutually defined and partial functions. 
%
%Next, we expand upon the last two points
%with snippets from our mechanizations. 
%
\begin{fullversion}
We begin by looking at aspects of mechanized metatheory in \coq
that are easier or more feasible than in \lh, and then we
turn to aspects that are easier in \lh.
\end{fullversion}

\mypara{Negative Occurrences and \coq's Equations}\NV{NEW CHECK}
Our original \lh mechanization defined  
denotations as refined data propositions
and proved denotational soundness. 
Though, we realized that the definition 
of the function type denotation
has a negative occurrence and permitting negative occurrences
can, in general, lead to unsoundness~\cite{CP90}. 
%
Our mechanization is the first big-scale user of 
\lh's data propositions thus it was not surprising that it revealed 
this potential unsoundness.
%
To remove this source of unsoundness in \lh, 
we implemented a \coq-style positivity checker
that unsurprisingly rejected the type denotation definitions. 
%
A similar challenge appears in the proof of strong normalization 
of the simply-type lambda calculus that because of negative occurrences
cannot use inductive propositions \cite{Pierce:SF2}.
There, the solution is to use a recursive function \ha{expr -> type -> Prop}
because a definition doesn't need to be computable.
%
In our \coq mechanization, we followed a similar solution, 
but since our definition was not structurally recursive and 
was needed for the proofs, we used 
the full power of \coq's Equations~\cite{10.1145/3341690} to define 
the type denotations. 
%
Unfortunately, a similar approach cannot currently carry over to \lh
because all Haskell functions must be computable and all
\lh annotations must be decidable. Therefore, quantifiers
are neither allowed on the right-hand side of Haskell
definitions nor in the refinements. % of Liquid types.


\mypara{Tactics and Automation}
\coq's tactics and automation often permit shorter
proofs as lemmas and constructors can be used with the
\ha{apply} tactic without writing out all arguments. 
%
For example, in \lh 
safety (thm.~\ref{lem:soundness})
is encoded using 
Haskell's \ha{Either}
for disjunction
and dependent pairs for existentials.
%
(\ha{Steps} is defined, using data propositions, as the 
% reflexive and transitive 
closure of \ha{Step}.)
%
\begin{mcode}
  safety :: e$_0$:Expr -> t:Type -> e:Expr -> HasTy Empty e$_0$ t 
         -> Steps e$_0$ e -> Either {isVal e}  (e$_i$::Expr, Step e e$_i$)
  safety _e$_0$ t _e e$_0$_has_t e$_0$_evals_e = case e$_0$_evals_e of
     Refl e$_0$ -> progress e$_0$ t e$_0$_has_t       -- $\cmt{e_0 = e}$
     AddStep e$_0$ e$_1$ e$_0$_step_e$_1$ e e$_1$_eval_e ->  -- $\cmt{e_0 \step \evalsTo{e_1}{e}}$
       safety e$_1$ t e (preservation e$_0$ t e$_0$_has_t e$_1$ e$_0$_step_e$_1$) e$_1$_eval_e
\end{mcode}
%
%\begin{fullversion}
    The reflexive case is proved by \ha{progress}.
    In the inductive case the evaluation
    sequence is $e_0 \step \evalsTo{e_1}{e}$
    and the proof goes by induction,
    using preservation to ensure that
    $e_1$ is typed.  
%\end{fullversion}
%
In \coq safety is proved 
without any of the three fully applied calls above:
%
\begin{mcode}
  Theorem safety : forall (e$_0$ e:expr) (t:type),
     Steps e$_0$ e -> HasTy Empty e$_0$ t -> isVal e \/ exists e$_i$, Steps e e$_i$.
  Proof. intros; induction H.
    - (* Refl *) apply progress with t; assumption.
    - (* Add  *) apply IHSteps; apply preservation with e; assumption. Qed.
\end{mcode}
%
Automation tactics could make this proof even shorter,
but we retain the essential proof structure.

\mypara{Mutual Recursion}
\lh makes it easy to define and work with mutually
recursive data types, 
such as our typing and subtyping judgments,
and to prove mutually inductive lemmas.
%
\begin{fullversion}
    Similarly, our 
    expressions, types, and predicates are
    three mutually recursive data types.
\end{fullversion}
%
Mutually recursive types are not a natural fit for \coq: 
the automatically generated induction principles do not work,
so we need to use the \ha{Scheme} keyword to generate  
suitable principles.
%
Theorems involving these types cannot be
broken up into separate lemmas for each type 
involved. 
%Rather, a unified approach must
%be taken which leads to theorems that are difficult to use
%directly in the \ha{rewrite} tactic.
%
Rather, one combined statement must be given,
which is difficult to use %directly 
in the \ha{rewrite} tactic.


Another weakness of \coq is that all 
information about the hypothesis is lost during the induction
tactic, so the normal structural \ha{induction} tactic only works when
a judgment contains no information, \ie the data constructor
is instantiated solely with universally quantified
variables. 
%
For instance, in the proof of the weakening~\Cref{lem:weakening},
to do structural induction on %a judgment like 
\ha{HasTy (concat g g')  e t}
we must introduce a universally quantified variable \ha{g0}
and strengthen the theorem with the hypothesis
\ha{g0 = concat g g'}.
%
While the standard library contains an ``experimental'' tactic 
\ha{dependent induction}, we also need to work with 
the special mutual induction principles that we generate for
our types, so we have to directly instantiate the principle
with a strengthened, complex hypothesis%. 
%\begin{comment}
and state the lemma as:
\begin{mcode}
  Lemma lem_weaken_typ' : ( forall (g0 : env) (e : expr) (t : type),
    HasTy g0 e t ->( forall (g g' : env) (x : vname) (t_x : type),
        g0 = concatE g g' ->unique g ->unique g' ->
        (binds g) $\cap $ (binds g') = empty ->~ (in_env x g) ->~ (in_env x g')
        ->HasTy (concatE (Cons x t_x g) g') e t ) ) /\ (
  forall (g0 : env) (t : type) (t' : type),
    Subtype g0 t t' ->( forall (g g' : env) (x : vname) (t_x : type),
        g0 = concatE g g' ->unique g ->unique g' -> 
        (binds g) $\cap $ (binds g') = empty ->~ (in_env x g) ->~ (in_env x g')
        ->Subtype (concatE (Cons x t_x g) g') t t') ).
\end{mcode}
%Proof. apply ( judgments_mutind 
%  (fun (g0 : env) (e : expr) (t : type) (p_e_t : Hastype g0 e t) => 
%    forall (g g':env) (x:vname) (t_x:type),
%      g0 = concatE g g' -> unique g -> unique g'
%                         -> intersect (binds g) (binds g') = empty
%                         -> ~ (in_env x g) -> ~ (in_env x g') 
%                         -> Hastype (concatE (Cons x t_x g) g') e t )
%  (fun (g0 : env) (t : type) (t' : type) (p_t_t' : Subtype g0 t t') => 
%    forall (g g':env) (x:vname) (t_x:type),
%      g0 = concatE g g' -> unique g -> unique g'
%                         -> intersect (binds g) (binds g') = empty
%                         -> ~ (in_env x g) -> ~ (in_env x g') 
%                         -> Subtype (concatE (Cons x t_x g) g') t t')  
%  ); ...
%
%\end{comment}
By contrast, in \lh we can state two separate mutually recursive
lemma functions for weakening: one for typing
and one for subtyping.
%
Then we may call either lemma in their own proofs on any
smaller instance of the typing (resp. subtyping) judgment.
%
%\begin{fullversion}
In practice, developments in \coq sidestep some of these issues
by collapsing the language of terms, types, \etc 
into a single inductive data type. 
%
This approach has the advantage of reducing the number of substitution
operations, but allows highly ungrammatical
combinations like \ha{App Bool False} into our syntax. 
%
We could still use this approach 
combined with a pre-term encoding common in \coq developments,  
but we preferred to keep a closer comparison to the \lh mechanization.
%The solution to this is to define these expressions to be pre-terms and then
%to define terms and types to be an inductive proposition consisting of a 
%pre-term and a proof of being a well-formed term or type. 
% 
%We avoided this common approach in order to keep a closer comparison
%to the \lh mechanization.
%\end{fullversion}

%Sometimes, a lemma cannot be proven 
%For example, the Transitivity of Subtyping cannot be proven 
%by structural induction on the two subtyping judgments because
%in one case we need to use the Substitution Lemma before
%invoking the inductive hypothesis.
%Instead, it must proceed by induction on the combined size of
%the three types involved. In \coq, this requires defining and 
%arguing about an ad-hoc measure and encoding strong induction
%over \ha{Nat} into the statement of the Transitivity Lemma itself.

\mypara{Partial Functions} 
\lh facilitates the definition of partial Haskell functions 
and proves totality with respect to the refined types,
usually automatically, without 
having to reason about impossible cases in mechanized proofs.
%
For instance, our syntax does not contain an explicit 
\ha{error} value, so we only want the function
$\delta(c,\sval)$ to be defined where $\app{c}{\sval}$ can 
step in our semantics.
%
This is straightforward in \lh: we define a predicate 
\ha{isCompat :: Prim ->Value ->Bool} and refine the input
types of $\delta$ to satisfy \ha{isCompat}.
%
%\begin{fullversion}
% NV: isCompat' is not even used in the text jeje
%\begin{mcode}
%  Definition isCompat' (c : prim) (e : expr) : bool :=
%    match c, e with | And , (Bc _)      => true
%                    | Or  , (Bc _)      => true
%                    ...
%                    | _        , _      => false end.
%\end{mcode}
%
%\end{fullversion}
In \coq a more roundabout approach is needed: we have to define 
\ha{isCompat} as an inductive type and include this object as
an explicit argument to our $\delta$ function:
%\begin{fullversion}
%
\begin{mcode}
  Inductive isCompat : prim -> expr -> Set : =   
    | isCpt_And  : forall b, isCompat And (Bc b)
    | isCpt_Or   : forall b, isCompat Or  (Bc b) 
    ...
\end{mcode}
%
%\end{fullversion}
However, this makes it harder to prove the determinism of our
semantics due to the dependence on the proof object. 
%
One solution would be to define a partial version of 
$\delta$ with type
\ha{Prim ->Expr ->option Expr} and prove the two functions
always agree regardless of proof object, \eg using \emph{subset types};
but since  each value
comes wrapped with a term-level proof object,
agreement proofs would require a \emph{Proof Irrelevance} axiom.
%, and 
%as reasoning about equality between objects of a 
%subset type is not possible in general sans 
%a \emph{Proof Irrelevance} axiom \cite{Vazou17}. 
%


% Chapter 9: Lists
\chapter{Lists: The Language \sysrfd}
\label{ch:lists}

One of the key features of GHC's core calculus
missing from $\sysrf$ is data types. The addition of
data types would also enable us to replace the ad hoc
kind system of $\sysrf$ with a more versatile system
of type classes.
%
As a first step in this direction, we augment our
calculus with polymorphic refined list types.
We also add a measure @length@ that describes the 
length of our lists.

\section{Syntax and Semantics} \label{sec:lang:syntaxD}

We present the syntax and semantics of \sysrfd in terms 
of the additions to \sysrf. The reader may refer to 
the figures in Chapter \ref{ch:language} for the syntactic
forms and rules that are inherited from \sysrf.
%
As before, we use the $\greybox{\mbox{gray}}$ to highlight
the extensions to $\sysf$ needed to support refinements 
in $\sysrfd$.

\begin{figure}%[t!]
%  \scalebox{0.80}{
    \begin{tabular}{rrcll}
\emphbf{Primitives} 
  & \sconst & $\bnfdef$ & $\cdots$    & \\ %\emph{booleans and integers} \\
  &         & $\spmid$  & $\suc$     & \emph{integer ops.} \\
  &         & $\spmid$  & $\len$      & \emph{polymorphic list ops} \\ [0.05in] 

\emphbf{List Values}
  & \slval  & $\bnfdef$ & $\nil{\stype}$  & \emph{empty list} \\
  &         & $\spmid$  & $\cons{\stype}{\sval}{\slval}$  
                                      & \emph{list constructor} \\ [0.05in]
\emphbf{Values}
  & \sval   & $\bnfdef$ & $\cdots$               & \\ % \emph{primitives} \\ 
  &         & $\spmid$  & \slval                & \emph{list values} \\[0.05in]

\emphbf{Terms}
  & \sexpr  & $\bnfdef$ & $\cdots$              & \\ %\emph{values} \\ 
  &         & $\spmid$  & \cons{\stype}{e_1}{e_2} & \emph{list constructor} \\
  &         & $\spmid$  & \eswitch{e}{e_n}{e_c} & \emph{list destructor} \\
\end{tabular}
%  }
  \vspace{-0.0cm}
  \caption{Syntax of Primitives, Values, and Expressions.}
\label{fig:syn:termsD}
\vspace{-0.0cm}
\end{figure}


\begin{figure}%[b!]
%{\small
  \begin{tabular}{rrcll}

  \emphbf{Types}
   & \stype & $\bnfdef$ & $\cdots$              & \\ %\emph{values} \\ 
   &        & $\spmid$  & $\listtype{\stype}\greybox{\!\breft{}{\vv}{\spred}}$  & \emph{\greytextbox{refined} list type} \\ 
  \end{tabular}
%}
\vspace{-0.0cm}
  \caption{Syntax of Types.
           The gray boxes are the extensions 
           to $\sysf$ needed by $\sysrfd$.}
           %We use $\sftype$ for $\sysf$-only types.}
  \label{fig:syn:typesD}
  \vspace{-0.00cm}
\end{figure}

\mypara{Constants, Values, and Terms}
%
\Cref{fig:syn:termsD} summarizes the syntax of terms in both 
\sysrfd and in the unrefined calculus.
%
The \emph{primitives} $\sconst$ now
include a $\suc$ function on integers that adds one
and a (polymorphic) $\len$ measure that computes the length of any list.
%
Although a user could easily define $\len$ using
\texttt{switch} below, we add $\len$ as a built-in primitive
in order to use it in our typing judgments.
%
Our \emph{terms} $\sexpr$ now additionally contain two list constructors:
the empty list $\nil{\stype}$ and the non-empty $\cons{\stype}{\sexpr_1}{\sexpr_2}$,
which builds a list from a head element and a tail.
%
Both of these constructors take a refined type annotation. For instance, we should
be able to type check
\[
\cons{\breft{\tint}{\vv}{\vv\geq 0}}{1}({\cons{\breft{\tint}{\vv}{\vv > 0}}{2}{3}})
\]
but not
\[
\cons{\breft{\tint}{\vv}{\vv > 0}}{1}({\cons{\breft{\tint}{\vv}{\vv\geq 0}}{2}{3}})
\]

because the tail of the latter list is only known to consist of non-negative integers.
%
Although we mechanized the metatheory in the same manner without 
type annotations as well, these annotations are needed to enable future work on
a bidirectional type checking algorithm (Chapter \ref{ch:future}).
%
The terms also now contain a list destructor 
$\eswitch{\sexpr}{\sexpr_n}{\sexpr_c}$, which
case splits on the shape of the match scrutinee $\sexpr$.
%
Finally, \emph{values} $\sval$ are augmented by lists that contain only
values as elements; these list values are defined inductively 
in Figure \ref{fig:syn:termsD}.

\mypara{Kinds \& Types}       
%
\Cref{fig:syn:typesD} shows the syntax of the types,
with the gray boxes indicating the extensions to $\sysf$ 
required by $\sysrfd$.
%
In contrast \sysrf, our list types are not base types,
but they can be refined.
%
Both of these are key to data types: our lists may contain
incomparable data such as lambda abstractions, and so they
cannot support the polymorphic comparison operators within
our simple kind system.
%
However, refinements on lists are key to any model of
data types. We must be able to express the type of a program 
such as the following safe tail function:
\begin{code}
  tail :: forall a. {v:[a] | length v > 0 } -> [a]
  tail xs = switch (xs) error (\y ys -> ys)
\end{code}
%
\sysrfd keeps the simple kind system from \sysrf and 
enforces list types as $\skstar$-kinded to prevent the substitution
of a list type for a refined type variable. 
%
This prevents a refinement of a list type from attempting
to compare a list using one of the polymorphic list operators.


\mypara{Dynamic Semantics} %\label{sec:lang:dynamic}
\Cref{fig:opsemD} summarizes the small-step semantics 
for both calculi.

\begin{figure}
%  {\small
  \begin{mathpar} %%%%%%%%% SMALL-STEP SEMANTICS %%%%%%%%%%
    \judgementHead{Operational Semantics (ext. Figure \ref{fig:e})}{$\sexpr \step \sexpr'$}\\
        \inferrule%*[Right=\eCons]
        {\sexpr \step \sexpr'}
        {\cons{\stype}{\sexpr}{\sexpr_1} \step \cons{\stype}{\sexpr'}{\sexpr_1}}
        {\eCons} 
        \quad
        \inferrule%*[Right=\eConsV]
        {\sexpr \step \sexpr'}
        {\cons{\stype}{\sval}{\sexpr} \step \cons{\stype}{\sval}{\sexpr'}}
        {\eConsV} 
        \\
        \inferrule%*[Right=\eSwitch]
          {\sexpr \step \sexpr'}
          {\eswitch{\sexpr}{\sexpr_n}{\sexpr_c} \step \eswitch{\sexpr'}{\sexpr_n}{\sexpr_c}}
          {\eSwitch} 
        \quad
        \inferrule% *[Right=\eSwitchN]
          {   }
          {\eswitch{\nil{\stype}}{\sexpr_n}{\sexpr_c} \step \sexpr_n}
          {\eSwitchN} 
          \\
        \inferrule% *[Right=\eSwitchC]
          { }
          {\eswitch{\cons{\stype}{\sval_1}{\sval_2}}{\sexpr_n}{\sexpr_c} 
              \step \app{(\app{\sexpr_c}{\sval_1})}{\sval_2}}
          {\eSwitchC} 
        \end{mathpar}        
%    }
\caption{The small-step semantics for \sysrfd.} 
\label{fig:eD}
\label{fig:opsemD}
\end{figure}


%\mypara{Primitives}
%%
%The function $\delta(\sconst, \sval)$ 
%evaluates the application $\app{\sconst}{\sval}$ 
%of built-in monomorphic primitives.
%%
%The reductions are defined in a curried 
%manner, \ie 
%$\app{\app{\leq}{m}}{n}$ evaluates to $\delta(\delta(\leq,m),n)$. 
%%we have that $\app{\app{\leq}{m}}{n} \steps \delta(\delta(\leq,m),n)$. 
%%
%Currying gives us unary relations like $m\!\!\leq$ 
%which is a partially evaluated version of the $\leq$ relation.
%%
%The function $\delta_T(\sconst, \forgetreft{\stype})$
%specifies the reduction rules for type 
%application on the polymorphic 
%built-in primitives. % $=$ and $\leq$.
%%
%$$\begin{array}{rclrclrcl}
%\delta(\wedge,{\tt true}) & \defeq & \lambda x.\, x &
%\delta(\leq,m) & \defeq & m\!\!\leq  & 
%    \delta_T(=, \tbool) & \defeq & =  \\
%\delta(\wedge,{\tt false}) & \defeq & \lambda x.\, {\tt false}\quad\quad &
%\delta(m\!\!\leq, n) & \defeq & {\tt}(m \leq n) \quad\quad&
%\delta_T(=, \tint) & \defeq & = \\
%\delta(\neg,{\tt true}) & \defeq & {\tt false} & 
%\delta(=,m) & \defeq & m\!\!= &
%\delta_T(\leq, \tbool) & \defeq & \leq  \\
%  \delta(\neg,{\tt false}) & \defeq &  {\tt true} &
%  \delta(m\!\!=, n) & \defeq &  {\tt}(m = n) &
%  \delta_T(\leq, \tint) & \defeq & \leq  \\
%%
%%  \delta(\vee,{\tt true}) & \defeq & \lambda x.\, {\tt true} &
%%  \delta(\leftrightarrow,{\tt true}) & \defeq & \lambda x.\, x & 
%%    &  & \\
%%    \delta(\vee,{\tt false}) & \defeq & \lambda x.\, x &
%%    \delta(\leftrightarrow,{\tt false}) & \defeq & \lambda x.\, \neg x &
%%    & & \\
%\end{array}$$
         
\mypara{Typing and Well-formedness}
%
Next, we present the static semantics of \sysrfd by describing
the additional rules used to establish our
well-formedness, typing, and subtyping judgments.
%%
%We use $\greybox{\mbox{grey}}$ to highlight the antecedents and rules
%specific to $\sysrf$.
%%
%
\Cref{fig:wfD} summarizes the new rules
that establish the well-formedness of types.
%
Rule \wtList states that a list type 
$\breft{\tlist{\stype}}{x}{\ttrue}$
with empty refinement
is well-formed with star kind provided that 
$\stype$ is well-formed with some kind $\skind$.
%
Similar to rule \wtRefn, our rule \wtListR 
stipulates that a refined list type $\breft{\tlist{\stype}}{x}{\spred}$
is well-formed with star kind in some environment
if the trivially refined type $\breft{\tlist{\stype}}{x}{\ttrue}$
has star kind in the same environment and if
the refinement predicate $\spred$ has type $\tbool$
in the environment augmented by binding a fresh variable to type $\tlist{\stype}$.

\begin{figure}%[t!]
%
%{\small
\begin{mathpar}
\judgementHead{Well-formed Type (ext. Figure \ref{fig:wf})}{\isWellFormed{\tcenv}{\stype}{\skind}}

%%%%%%%%%%%% WELL-FORMEDNESS %%%%%%%%%%%%%
%
\inferrule% *[Right=\wtList]
    {\isWellFormed{\tcenv}{\stype}{\skind}}
    {\isWellFormed{\tcenv}{\tlist{\stype}\greybox{\!\breft{}{x}{\ttrue}}\!}{\skstar}}
    {\wtList}
\quad
%    
\greybox{\inferrule% *[Right=\;\;\wtListR]
    { \isWellFormed{\tcenv}{\breft{\tlist{\stype}}{x}{\ttrue}}{\skstar} \\\\
      \forall\notmem{y}{\tcenv}.
      \hasftype{\bind{y}{\tlist{\stype}}, \forgetreft{\tcenv}}{\subst{p}{x}{y}}{\tbool}
    }
    {\isWellFormed{\tcenv}{\breft{\tlist{\stype}}{x}{p}}{\skstar}}
    {\wtListR} }
%
\end{mathpar}
%}
\vspace{-0.00cm}
\caption{Well-formedness of $\sysrfd$ types. The rules for
  $\sysf$ exclude the gray boxes.}
\label{fig:wfD}
\vspace{-0.00cm}
\end{figure}

%The judgment $\hastype{\tcenv}{\sexpr}{\stype}$ states
%that the term $\sexpr$ has type $\stype$ in the context of
%environment $\tcenv$.
\Cref{fig:typingD} summarizes
the rules that establish typing for both $\sysf$ and
$\sysrf$, with gray %boxes
%indicating extensions needed
for the $\sysrf$ extensions.
%
Rule $\tNil$ states that whenever $\stype$ is a well-formed
type in some environment, then  $\nil{\stype}$ has the type
$\breft{\tlist{\stype}}{x}{\len x = 0}$ of lists of elements
of type $\stype$ of length zero.
%
The rule $\tCons$ is slightly more complex: whenever $\sexpr_h$
can be given type $\stype$ in some environment, and whenever
$\sexpr_t$ can be given type
$\breft{\tlist{\stype}}{x}{\spred}$ in the same environment,
then the list $\cons{\stype}{\sexpr_h}{\sexpr_t}$ can be given
the type $\tlist{\stype}$ with the refinement that says that
this list has length one more than some list of type
$\breft{\tlist{\stype}}{x}{\spred}$.
%
The purpose of this refinement is to embed the information about
the specific length of a list at the refinement level.
%
Our typing rule for the list destructor 
$\eswitch{\sexpr}{\sexpr_n}{\sexpr_c}$ is 
best thought of as analogous to our rule $\tIf$
because it enables path-sensitive reasoning about lists.
%
This rule $\tSwitch$ says that whenever the match scrutinee
$\sexpr$ can be given a list type 
$\breft{\tlist{\stype}}{x}{\spred}$ in some environment,
whenever $\stype'$ is well-typed in the same environment,
and whenever each branch can be given this type
$\stype'$ in this environment augmented by the knowledge 
that the about the scrutinee and its length, then the full
term $\eswitch{\sexpr}{\sexpr_n}{\sexpr_c}$ can be given type
$\stype'$. 
%
Note that, per the semantics in \ref{fig:eD}, the cons-branch
$\sexpr_c$ is a function expecting two arguments: the head
of the scrutinee and the tail (which we know has length
one less than the scrutinee).
%
In our formalism, it is necessary for us to augment the 
environment with \emph{two} dummy variables here. We need
to preserve $\spred$,
the knowledge obtained from the match scrutinee,
and the fact that the scrutinee is one element longer than
the tail to which $\sexpr_c$ is being applied.
%
We could express the antecedent judgment as
\begin{equation}
  \label{badswitch}
\forall\notmem{z}{\tcenv}. 
  \hastype{\bind{z}{\breft{\tlist{\stype}}{x}{p},\tcenv}}
     {\sexpr_c}
     {\funcftype{\stype}{\funcftype{\tlist{\stype}\greybox{\breft{}{\vv}{\suc \len \vv = \len z}}}{\stype'}}},
\end{equation}
but this would pose a problem for proving type soundness 
for \sysrfd (specifically preservation)
from a minimal interface of axioms (\ref{lem:implicationD}) 
for implication.
%
These axioms are purely syntactic, and so are not sufficient
to prove implications (and hence subtyping obligations) 
where there are slight variations in refinement syntax. 
%
Rule \tCons gives us a refinement that says the length of this list
is one \emph{longer} than some other list;
judgment \ref{badswitch} would require us to show that the second
argument to $\sexpr_c$ has length one \emph{shorter} than some other 
list. There is no way to derive this knowledge without either a 
semantic notion of entailment, or a large, unwieldy set of axioms,
or by changing the form of rule \tSwitch as we did.


\begin{figure}
%  {\small
  \begin{mathpar}             %%%%%%%%%%%%% TYPING %%%%%%%%%%%%%%%%%%
  \judgementHead{Typing (ext. Figure \ref{fig:t})}{\hastype{\tcenv}{\sexpr}{\stype}} \\

    %
        \inferrule
        {\isWellFormed{\tcenv}{\stype}{\skind}}
        {\hastype{\tcenv}{\nil{\stype}}{\tlist{\stype}\greybox{\breft{}{x}{\len x = 0}}}}
        {\tNil}
    %
    \\  
    %
        \inferrule
        {\hastype{\tcenv}{\sexpr_h}{\stype} \\
         \hastype{\tcenv}{\sexpr_t}{\tlist{\stype}\greybox{\breft{}{x}{\spred}}}
        }
        {\hastype{\tcenv}{\cons{\stype}{\sexpr_h}{\sexpr_t}}
                 {\greybox{\existype{y}{\tlist{\stype}\greybox{\breft{}{x}{\spred}}}}
                  \tlist{\stype}\greybox{\breft{}{\vv}{\len \vv = \suc \len y}}}}
        {\tCons}
    \\    
    %
    {\small
        \inferrule
        {\hastype{\tcenv}{\sexpr}{\tlist{\stype}\greybox{\breft{}{x}{\spred}}} \\
         \isWellFormed{\tcenv}{\stype'}{\skind} \\\\
          %
         \greybox{\forall\notmem{y}{\tcenv}.}
         \hastype{\greybox{\bind{y}{\breft{\tlist{\stype}}{x}{p \wedge \len x = 0}},}\tcenv}
             {\sexpr_n}{\stype'}\\\\
          %
         \greybox{\forall\notmem{y, z}{\tcenv}.}
         \hastype{\greybox{\bind{z}{\breft{\tlist{\stype}}{x}{p \wedge \suc \len y = \len x}},
                          \bind{y}{\tlist{\stype}},}\tcenv}
            {\sexpr_c}{\funcftype{\stype}{\funcftype{\tlist{\stype}\greybox{\breft{}{\vv}{\len y = \len \vv}}}{\stype'}}}
        }
        {\hastype{\tcenv}{\eswitch{\sexpr}{\sexpr_n}{\sexpr_c}}{\stype'}}
        {\tSwitch} 
    }       
    \end{mathpar}
%  }
\vspace{-0.00cm}
\caption{Typing rules.
The judgment $\hasftype{\tcenv}{\sexpr}{\sftype}$ is extended by excluding the gray boxes.}
\label{fig:tD}\label{fig:typingD}
\vspace{-0.00cm}
\end{figure}

\mypara{New Primitives}
%
The function $\ty{\sconst}$, which gives the type of every
built-in primitives, is extended for the new primitives
$\suc$ and $\len$. Below we present essential
examples of the $\ty{\sconst}$ definition.
%
{\small
$$\begin{array}{rcl}
\ty{\suc} & \defeq & \functype{y}{\tint}{\breft{\tint}{v}{v = (\suc y)}} \\
\ty{\len} & \defeq & \polytype{\tvar}{\skstar}{\functype{y}{\al}{\breft{\tint}{v}{v = (\len y)}}}
\end{array}$$
}
For ease of reading, the $\len$ used in the refinements 
is the polymorphic $\len$,
but with the type applications elided.


\subsection{Subtyping}
\label{sec:typing:subD}

\Cref{fig:s} presents the new rule to establish the 
subtyping judgment ${\isSubType{\tcenv}{s}{t}}$.  
%
Rule \sList states that one list type
$\breft{\listtype{\stype_1}}{\vv}{\spred_1}$
is a subtype of another list type
$\breft{\listtype{\stype_2}}{\vv}{\spred_2}$
in some environment $\tcenv$, when
$\stype_1$ is a subtype of $\stype_2$ and 
$\spred_1$ implies $\spred_2$
in the
environment $\tcenv$ augmented  by binding
a fresh type variable to kind $\skind$.

\begin{figure}
\judgementHead{Subtyping}{\isSubType{\tcenv}{s}{t}}

\begin{mathpar}   %%%%%%%%%%%%%%%%%% SUBTYPING %%%%%%%%%%%%%%%%%%
%
  \inferrule*[Right=\sList]
  {\isSubType{\tcenv}{\stype_1}{\stype_2} \quad
    \forall\notmem{y}{\tcenv}. \;\;
    \imply{\bind{y}{\listtype{\stype_1}\{\ttrue\}},\tcenv}{\subst{p_1}{x}{y}}{\subst{p_2}{x}{y}} }
  {\isSubType{\tcenv}{\breft{\listtype{\stype_1}}{x}{\spred_1}}
                     {\breft{\listtype{\stype_2}}{x}{\spred_2}}}
\end{mathpar}
\vspace{-0.00cm}
\caption{Subtyping Rules.}
\label{fig:sD}
\label{fig:subtypingD}
%\label{fig:ent}
\vspace{-0.00cm}
\end{figure}

\mypara{Implication}
In \S~\ref{sec:typing:implication} we discussed our approach to formalizing
implication by giving both an axiomatized interface and a 
denotational implementation. 
%
First, we give the additional axioms for implication in \sysrfd.


\begin{requirement}[Implication Interface]\label{lem:implicationD}
  The implication relation satisfies the statements
  in Requirement~\ref{lem:implication} and the statements below:
  \begin{enumerate}
      \item (Exact Quantification) If
      $\hastype{\tcenv}{v_x}{t_x}$ and
      $\isWellFormed{\tcenv}{t_x}{\skbase}$ and 
      $\bind{x}{\self{t_x}{v_x}{\skbase}} \in \tcenv$ 
      and $x\not\in \free{v_x}$ then
      $\imply{\tcenv}{\spred}{\subst{\spred}{x}{v_x}}$.
      %
      \item (Equal Length Quantification) If
      $\hastype{\tcenv}{\sval}{\breft{\listtype{\stype}}{x}{\spred}}$ and
      $\isWellFormed{\tcenv}{\breft{\listtype{\stype}}{x}{\spred}}{\skstar}$ and \\
      $\bind{x}{\breft{\listtype{\stype}}{x}{\len x = \len \sval \wedge \spred}} \in \tcenv$ 
      and $x\not\in \free{\sval}$ and \safeListVar{x}{q} then
      $\imply{\tcenv}{q}{\subst{q}{x}{\sval}}$\\
      and
      $\imply{\tcenv}{\subst{q}{x}{\sval}}{q}$.
  \end{enumerate}
\end{requirement}

\noindent
The first statement says that whenever 
$\bind{x}{\self{t_x}{v_x}{\skbase}}$ appears 
bound in $\tcenv$, then $x$ is effectively being
universally quantified over a type with just one
inhabitant (up to equality). Then we assume 
that a refinement $\spred$ implies $\spred$ with
all occurrences of $x$ replaced by $v_x$.
%
This is semantically valid because any encoding of 
our refinements in an external logic (such as SMT)
would map $=$ in our refinement syntax to equality.
%
And we also prove that under the denotational definition of
implication (\S~\ref{sec:typing:implication:denotational})
our syntactic primitive $=$ actually corresponds to semantic
equality, and we are thus able to show that this axiom
follows from the denotational definition.

The second statement, which we call Equal Length Quantification,
expresses the best possible analogue for lists. We cannot
appeal to Exact Quantification for any variable $x$ bound to
a list type in the environment; these types cannot be
selfified because lists cannot be compared, even for equality,
in our system.
%
However, the refinements that do occur in our system (by virtue
of appearing in our various rules) do not use the 
$\texttt{switch}$ statement to destruct lists to inspect
their contents. Rather, these refinements are entirely agnostic
about the data contained in lists and are only concerned with
the length of lists.
%
We capture this notion in the recursive function 
$\safeListVar{x}{q}$ which states that $x$ only appears 
in $q$ as the argument to the $\len$ function.
% 
If this is the case, then the only information that $q$ uses 
about $x$ is the length. The Exact Length Quantification
axiom then says that if $x$ is bound to a type that constrains
the length of $x$ to be equal to the length of a list $v$, 
then the refinement $q$ is equivalent (under implication)
to $\subst{q}{x}{v}$.

Previously, we noted in \S~\ref{sec:typing:implication:interface}
that we did not require the Exact Quantification axiom in
Assumption 1 of~\cite{Knowles09} to formalize implication in \sysrf.
% 
However, we do require it for \sysrfd in our mechanization 
to verify Requirement~\ref{lem:prim-typing} as it applies to the
length of lists.

\subsection{Denotational Semantics}
\label{sec:lists:denotational}

We extend the definition of denotations of types given
in Figure~\ref{fig:den} to define the denotation of a list type.
%
In order to define the denotation 
$\denote{\breft{\listtype{\stype}}{x}{p}}$
as the set of closed values
of the appropriate base type $\listtype{\forgetreft{t}}$
which satisfy the type's refinement predicate,
we have to take into account that there are at least two
refinements here. The entire list $\sval$ must satisfy
the refinement $\spred$, and additionally the type
$\stype$ itself may contain refinements which must
be satisfied by each of the elements of $\sval$.

\begin{figure}
$$\begin{array}{r@{\hskip 0.03in}c@{\hskip 0.03in}l}
\denote{\breft{\listtype{\stype}}{x}{p}} & \defeq &
  \setcomp{\sval}{\hasftype{\varnothing}{\sval}{\listtype{\forgetreft{\stype}}} 
  \,\wedge\, (\forall\, v_i\in v.\, v_i \in \denote{t})
  \,\wedge\, \evalsTo{\subst{p}{x}{\sval}}{\ttrue}}.
\end{array}$$
\vspace{-0.0cm}
\caption{Denotations of Types and Environments.}
\label{fig:denD}
\vspace{-0.0cm}
\end{figure}

\section{Metatheory of Lists}

Our proof of type soundness of $\sysrfd$ generally follows
the same structure as the soundness proof for $\sysrf$,
illustrated in Figure~\ref{fig:graph}.
%
However, some lemmas have additional cases that are highly 
non-trivial and add additional difficulty to the proof. 
%
For instance, we need to be able to invert typing judgments
such as 
$\hastype{\tcenv}{\cons{\stype}{\sexpr_h}{\sexpr_t}}{\stype'}$
both to obtain typing judgments for $\sexpr_h$ and $\sexpr_t$
and also to obtain a typing judgment that contains the 
knowledge about the length of $\cons{\stype}{\sexpr_h}{\sexpr_t}$,
which may have been discarded through subsumption.

We give the highlights for the metatheory of \sysrfd.


\mypara{Inversion of Typing Judgments}
\label{sec:soundnessD:inversion}
%
In \S~\ref{sec:soundness:inversion} we discussed Inversion Lemma
\ref{lem:inversion}, which allowed us to invert the typing judgment
for a term- or type-abstraction. We need to extend this 
lemma to include inverting typing judgments for lists.
%
We can then recover typing information about the head and tail 
of this list as well as information about the length of the list that
may have been lost through use of the subsumption \tSub rule.

\begin{lemma} (Inversion of $\tNil$, $\tCons$) \label{lem:inversionD} 
    (extends Lemma~\ref{lem:inversion})
    \begin{enumerate}
    \item If $\hastype{\tcenv}{\nil{\stype_0}}{\breft{\listtype{\stype}}{x}{\spred}}$
    and $\isWellFormedE{\tcenv}$,
    then $\isSubType{\tcenv}{\stype_0}{\stype}$, and
    $\isWellFormed{\tcenv}{\stype_0}{\skstar}$,\\
    and $\isSubType{\tcenv}{\breft{\listtype{\stype_0}}{x}{\len x = 0}}
          {\breft{\listtype{\stype}}{x}{\spred}}$.

    \item If $\hastype{\tcenv}{\cons{\stype_0}{v_1}{v_2}}
                  {\breft{\listtype{\stype}}{x}{\spred}}$
    and $\isWellFormedE{\tcenv}$,
    then 
    $\isSubType{\tcenv}{\stype_0}{\stype}$,
    $\isWellFormed{\tcenv}{\stype_0}{\skstar}$,
    $\hastype{\tcenv}{v_1}{\stype_0}$
    and for some refinement $q$,
    $\hastype{\tcenv}{v_2}{\breft{\listtype{\stype_0}}{x}{q}}$
    and $\isSubType{\tcenv}
          {\existype{y}{\breft{\listtype{\stype}}{x}{q}}
                    {\breft{\listtype{\stype_0}}{x}{\len x = \suc \len y}} }
          {\breft{\listtype{\stype}}{x}{\spred}}$.

  \end{enumerate}                    

\end{lemma}
%
The second statement above is important because if
$\hastype{\tcenv}{\cons{\stype_0}{v_1}{v_2}}
                  {\breft{\listtype{\stype}}{x}{\spred}}$
then we cannot get directly to a typing judgment for $v_1$
or for $v_2$. Indeed, rule \tCons gives us an existentially
quantified list of type $\listtype{\stype_0}$ with a refinement
relating the length of the list to its tail; but here $\stype$
may not even be the same as $\stype_0$, so the derivation tree
must have used one or more applications of rule \tSub.
%
The proof goes by induction on the size of the derivation tree,
 which must be finite.   

\mypara{Exact Typing}
Although we cannot apply the exact typing lemma to lists (only 
base types can be selfified), we can derive an analogous
statement in terms of equality of length.
%
whenever we can type
a list value $v$ at type $\breft{\listtype{t}}{x}{p}$ then we
also type $v$ at the type $\breft{\listtype{t}}{x}{p}$
with the refinement strengthened by $\len x = \len v$.
%
\begin{lemma} (Equal Length in Typing, compare to Lemma~\ref{lem:exact}) 
\label{lem:equalLength}
%\begin{enumerate}
  %\item If $\hastype{\tcenv}{e}{t}$, $\isWellFormedE{\tcenv}$, $\isWellFormed{\tcenv}{t}{k}$, and $\isSubType{\tcenv}{s}{t}$, then $\isSubType{\tcenv}{{\rm self}(s, v, k)}{{\rm self}(t, v, k)}$.
  %\item 
  If $\hastype{\tcenv}{v}{\breft{\listtype{t}}{x}{p}}$ and $\isWellFormedE{\tcenv}$
  then $\hastype{\tcenv}{v}{\breft{\listtype{t}}{x}{\len x = \len v \wedge p}}$. 
%\end{enumerate}
\end{lemma}


\begin{comment}

\mypara{Denotational Soundness}
\label{sec:soundness:denotationalD}
\label{sec:denot:soundnessD}

Our statement of denotational soundness remains the same as
Theorem~\ref{lem:denote-sound-first}.
%
Here we emphasize that 



\begin{lemma}\label{denote-selfification} (Selfified Denotations)
  If $\isWellFormed{\varnothing}{\stype}{\skind}$,
     $\hastype{\varnothing}{\sexpr}{\stype}$,
     $\evalsTo{\sexpr}{\sval}$ for some $\sval \in \denote{\stype}$
  then $\sval \in \denote{\self{t}{e}{k}}$.
  %(DenotationsSelfify.hs) Depends on: ${\tt lem\_typing\_wf}$
\end{lemma}

This lemma captures the intuition
that if $v \in \denote{\breft{\sbase}{x}{p}}$
(\ie if $v$ has base type $\sbase$
and $\evalsTo{\subst{p}{x}{v}}{\ttrue}$),
then we have $v \in \denote{\breft{\sbase}{x}{ p \wedge x = v}}$
as $\subst{(p \wedge x=v)}{x}{v}$ certainly evaluates to $\ttrue$.
%
The full proof also handles the case that $t$
is an existential type as well as selfification
by an expression $e$ that evaluates to $v$.

\end{comment}

\section{Implementation}

We implemented the metatheory of \sysrfd in \coq.
This mechanization 
proves both type safety and
denotation soundness. 
%
Compared to mechanization of \sysrf, this proof
was about $35\%$ longer in lines of code and takes
about twice as long (two minutes) to check.
%
This mechanization is also included as supplementary
material.
%
Table \ref{fig:comparisonD} summarizes the development 
of our metatheory, and points to which areas of the 
proof increased the most in length in \sysrfd as compared
with \sysrf. 
%
Finally, in the last two columns, the table provides a breakdown for each
broad area of the soundness proof of \sysrfd into
lines of specification (definitions and theorem statements
in Gallina) and lines of proof (proofs written in Ltac,
\coq's tactic language).


The source code for our mechanization of in \coq 
of the soundness proof for $\sysrfd$ is available on Zenodo 
\cite{michael_h_borkowski_2024_13352164}.

\begin{table}
  \caption{Comparative mechanization details for \sysrf versus \sysrfd. 
  }
  \label{fig:comparisonD}
  \vspace{0.00cm}
  \begin{tabular}{l}
  \end{tabular}
  \end{table}

\begin{table}
  % \end{center}
  \vspace{-0.00cm}
  % \begin{center}
  \setlength{\tabcolsep}{6pt}
  \scalebox{0.97}{
  \begin{tabular}{ lrrrr|rr  }
    \multicolumn{2}{l}{ } & 
    \multicolumn{2}{c}{\coq Mechanizations} &
    \multicolumn{3}{r}{ } \\
    \toprule
    \textbf{Subject} &  
    \textbf{Files} & \textbf{\sysrf lines} &
    \textbf{\sysrfd Lines} & \textbf{\% Increase} &
    \textbf{\sysrfd Spec} & \textbf{\sysrfd Proof} \\
    \hline
    Definitions             & 7      & 1155 &  1476 &  27.8\% & 1193 &  283 \\ % &  941  &  190 \\
    Basic Properties        & 8      & 3626 &  3980 &   9.8\% & 1236 & 2744 \\ % & 1201  & 2360 \\
    $\sysf$ Soundness       & 4      &  983 &  1156 &  17.5\% &  178 &  978 \\ % &  173  &  773 \\
    Weakening               & 4      &  719 &   868 &  20.7\% &  110 &  758 \\ % &  110  &  568 \\
    Substitution            & 4      & 1079 &  1248 &  15.7\% &  165 & 1083 \\ % &  158  &  859 \\
    Exact Typing            & 2      &  246 &   318 &  29.3\% &   73 &  245 \\ % &   33  &  182 \\
    Narrowing               & 1      &  333 &   381 &  14.4\% &   54 &  327 \\ % &   54  &  262 \\
    Inversion               & 1      &  339 &   813 & 139.8\% &  145 &  668 \\ % &   57  &  258 \\
    Primitives              & 2      &  635 &  1192 &  87.7\% &  137 & 1055 \\ % &   89  &  508 \\
    $\sysrf$ Soundness      & 1      &  263 &   657 & 149.8\% &   12 &  645 \\ % &   12  &  233 \\
    Denot. Soundness        & 13     & 4033 &  6061 &  50.3\% & 1147 & 4914 \\ % &  815  & 3010 \\
    \midrule
    \textbf{Total}          & 50    & 13411 & 18150 &  35.3\% & 4450 & 13700 \\ % & 3643  & 9203 \\
    \bottomrule
  \end{tabular}
  }
  \end{table}
  
\section*{Acknowledgements for Chapter 9}
%
This chapter consists of unpublished work done in collaboration
with Ranjit Jhala. This material is 
being prepared for future submission for publication.
%
The dissertation author was the primary investigator 
and author of this material.



% Chapter 10: Birdirectional Algorithm
%\input{10-bidirectional}

%% APPENDIX
\appendix

% App A: System F Soundness Proofs
\chapter{Proofs of \sysf Soundness}
\label{ch:proofsF}
\label{sec:proofsF}


%%%%%%%%%%%%%%%%%%%%%%%%%%%%%%%
%%%%%%%%%%%%%%%%%%%%%%%%%%%%%%%
%%%%%%%%%%%%%%%%%%%%%%%%%%%%%%%
 





%\begin{lemma}\label{WFFTlemmas}
%    For any System F environments $\tcenv$, $\tcenv'$ and $x,y,\al,\al' \not\in dom(\tcenv', \tcenv)$:
%    \item (Change of Free Variables) If $\tcenv', x\bindt \tau_x, \tcenv \vdash_F^w \tau:k$ then $\tcenv', y\bindt t_x, \tcenv \vdash_F^w \tau:k$.  
%    If $\tcenv', \al\bindt k_{\al}, \tcenv \vdash_F^w \tau:k$, then $\tcenv', \al'\bindt k_{\al}, \tcenv \vdash_F^w \tau[\al'/\al]:k$. 
%    \item (Weakening) If $\tcenv', \tcenv \vdash_F^w \tau:k$ then $\tcenv', x\bindt \tau_x, \tcenv \vdash_F^w \tau:k$ and $\tcenv', \al\bindt k_{\al}, \tcenv \vdash_F^w \tau:k$.
%    \item (Substitution Lemma) If $\tcenv', \al\bindt k_{\al}, \tcenv \vdash_F^w \tau:k$ and 
%    $\isWFFT{\tcenv}{\tau_{\al}}{k_{\al}}$ 
%    then $\tcenv'[\tau_{\al}/\al], \tcenv \vdash_F^w \tau[\tau_{\al}/\al]:k$. 
%\end{lemma}
%
%\begin{proof}
%\begin{enumerate}
%\item Change of Free Variables is proven by 
%a straightforward mutual induction on the structure of 
%$\isWFFT{\tcenv', x\bindt \tau_x, \tcenv}{\tau}{k}$, 
%and on the structure of 
%$\isWFFT{\tcenv', \al\bindt k_{\al}, \tcenv}{\tau}{k}$. 
%The one non-trivial detail, which will arise frequently below, is that in the case \wfftPoly, we invert 
%$\isWFFT{\tcenv', x\bindt \tau_x, \tcenv}{\forall\,\al\bindt k.\tau}{*}$
%to obtain
%$\isWFFT{\al'\bindt k, \tcenv', x\bindt \tau_x, \tcenv}{\tau[\al'/\al]}{k_\tau}$
%for some $\al' \not\in {\rm dom}(\tcenv', x\bindt \tau_x, \tcenv)$.
%Before we can apply the inductive hypothesis, we must choose some 
%new fresh 
%$\al'' \not\in {\rm dom}(\tcenv', y\bindt \tau_x, \tcenv) \cup \{x\}$
%because $y$ was not chosen as fresh with respect to an environment
%containing $\al'$. Then we apply the inductive hypothesis twice,
%first obtaining
%$\isWFFT{\al''\bindt k, \tcenv', x\bindt \tau_x, \tcenv}{\tau[\al''/\al]}{k_\tau}$
%and then
%$\isWFFT{\al''\bindt k, \tcenv', y\bindt \tau_x, \tcenv}{\tau[\al''/\al]}{k_\tau}$
%before finally concluding 
%$\isWFFT{\tcenv', y\bindt \tau_x, \tcenv}{\forall\,\al\bindt k.\tau}{*}$
%by rule \wfftPoly.
%\item The proof is a straightforward induction on the structure of
%$\isWFFT{\tcenv',\tcenv}{\tau}{k}$, similar to the Change of Free Variables lemma above. In case \wfftPoly, we invert 
%$\isWFFT{\tcenv', \tcenv}{\forall\,\al\bindt k.\tau}{*}$
%to obtain
%$\isWFFT{\al'\bindt k, \tcenv', \tcenv}{\tau[\al'/\al]}{k_\tau}$
%and use the Change of Free Variables lemma to pick a new $\al''$ 
%that is fresh with respect to $x$ and then apply the inductive hypothesis.
%\item The Substitution Lemma here is proven by induction on the
%structure of
%$\isWFFT{\tcenv', \al\bindt k_{\al}, \tcenv}{\tau}{k}$.
%The proof is straightforward except in the case of rule \wfftVar.
%Here $\tau \equiv \al'$ and we invert 
%$\isWFFT{\tcenv', \al\bindt k_{\al}, \tcenv}{\al'}{k}$
%to get $\al'\bindt k \in \tcenv', \al\bindt k_{\al}, \tcenv$.
%We have three cases depending on where $\al'$ is bound in 
%$\tcenv', \al\bindt k_{\al}, \tcenv$.
%First, if $\al' \bindt k \in \tcenv$, then $\al' \neq \al$ 
%and so $\al'[\tau_{\al}/{\al}] = \al'$.
%...
% 
%Next, if $\al' \bindt k = \al\bindt k_{\al}$, then ...
%
%Finally, if $\al' \bindt k \in \tcenv'$, then $\al' \neq \al$ 
%and so $\al'[\tau_{\al}/{\al}] = \al'$. In particular,
%$\al' \bindt k \in \tcenv'[\tau_{\al}/{\al}]$. ....
%
%$\isWFFT{\tcenv}{\tau_{\al}}{k_{\al}}$ 
%\end{enumerate}
%\end{proof}


% App B: System RF Soundness Proofs
\chapter{Proofs of System RF Soundness}
\label{ch:proofs}

We present the proofs in this appendix in the same order 

\mypara{Inversion of Typing Judgments.} 
In the yellow region of Figure \ref{fig:graph}, we discuss how the fact that the typing judgment is no longer syntax-directed leads to an involved proof for the inversion lemma below. First, we need a lemma about subtyping:

\begin{lemma}
    {\em Transitivity of Subtyping}: If $\isWellFormed{\tcenv}{t}{k}$, $\isWellFormed{\tcenv}{t'}{k'}$, $\isWellFormed{\tcenv}{t''}{k''}$, and $\vdash_w \tcenv$ and $\isSubType{\tcenv}{t}{t'}$ and $\isSubType{\tcenv}{t'}{t''}$ then $\isSubType{\tcenv}{t}{t''}$. (LemmasTransitive.hs) Our ${\tt lem\_sub\_trans}$ depends on ${\tt lem\_subst\_sub}$.
\end{lemma}

\begin{proof}
    We proceed by induction on the combined size of the derivation trees of $\tcenv \vdash t' <: t'$ and $\tcenv \vdash t' <: t''$.
    We first consider the cases where none of $t, t',$ or $t''$ are existential types. By examination of the subtyping rules, we see that $t$, $t'$, and $t''$ must have the same form, and so there are three such cases:
    
    \pfcase{Refinement types}: 
    In this case $t \equiv b\{x_1\col p_1\}$, $t' \equiv b\{x_2\col p_2\}$, and $t'' \equiv b\{x_3\col p_3\}$. The last rule used in each of $\tcenv \vdash t' <: t'$ and $\tcenv \vdash t' <: t''$ must have been {\sc S-Base}. Inverting each of these we have
    \begin{equation}
    \entails{y\bindt b\{x_1\col p_1\}, \tcenv}{p_2[y/x_2]} \;\;\;\;{\rm and}\;\;\;\; \entails{z\bindt b\{x_2\col p_2\}, \tcenv}{p_3[z/x_3]}
    \end{equation}
    for some $y,z \not\in\dom{\tcenv}$. We can invert {\sc Ent-Pred}  to get
    \begin{equation}
    \forall \theta.\, \theta \in\lb y\bindt b\{x_1\col p_1\}, \tcenv\rb \Rightarrow \theta(p_2[y/x_2]) \steps \ttrue
    \end{equation}and\begin{equation}
    \forall \theta.\, \theta \in\lb y\bindt b\{x_2\col p_2\}, \tcenv\rb \Rightarrow \theta(p_3[y/x_3]) \steps \ttrue.
    \end{equation}
    where we also use the change of free variables lemma in the last equation.
    Let $\theta \in\lb y\bindt b\{x_1\col p_1\}, \tcenv\rb$. 
    Let $v = \theta(y)$ Then $v \in \lb \theta(b\{x_1\col p_1\})\rb,$ and so $v \in \lb \theta(b\{x_2\col p_2\})\rb$ by the Denotational Soundness Lemma. Then $\theta \in\lb y\bindt b\{x_2\col p_2\}, \tcenv\rb$ also, and so 
    $\theta(p_3)[y/x_3] \steps \ttrue.$
    Therefore we can conclude that
    \begin{equation}
        \entails{y\bindt b\{x_1\col p_1\}, \tcenv}{p_3[y/x_3]}
    \end{equation} 
    and thus that $\isSubType{\tcenv}{b\{x_1\col p_1\}}{b\{x_3\col p_3\}}$.
    
    \mypara{Case function types}: 
    In this case $t \equiv \functype{x_1}{s_1}{t_1}$, $t' \equiv \functype{x_2}{s_2}{t_2}$, and $t'' \equiv \functype{x_3}{s_3}{t_3}$. The last rule used in each of $\tcenv \vdash t' <: t'$ and $\tcenv \vdash t' <: t''$ must have been {\sc S-Func}. Inverting each of these we have
    \begin{equation*}
    \isSubType{\tcenv}{s_2}{s_1}, \;\;\; \isSubType{\tcenv}{s_3}{s_2}, \;\;\;
    \isSubType{y\bindt s_2, \tcenv}{t_1[y/x_1]}{t_2[y/x_2]}, \;\;{\rm and} \;\;
    \isSubType{z\bindt s_3, \tcenv}{t_2[z/x_2]}{t_3[z/x_3]}
    \end{equation*}
    for some $y,z \not\in\dom{\tcenv}$. By the inductive hypothesis we can combine the first two judgments to get $\isSubType{\tcenv}{s_3}{s1}$. By the narrowing lemma (see below) and the change of variables lemma we have $\isSubType{z\bindt s_3, \tcenv}{t_1[z/x_1]}{t_2[z/x_2]}$; then we conclude by the inductive hypothesis that $\isSubType{z\bindt s_3, \tcenv}{t_1[z/x_1]}{t_3[z/x_3]}$. 
    Then by rule {\sc S-Func} we conclude that $\isSubType{\tcenv}{\functype{x_1}{s_1}{t_1}}{\functype{x_3}{s_3}{t_3}}$.
    The well-formedness judgments required to apply the inductive hypothesis can be constructed by inverting $\isWellFormed{\tcenv}{t}{*}$, $\isWellFormed{\tcenv}{t'}{*}$, and $\isWellFormed{\tcenv}{t''}{*}$ and using the change of free variables and narrowing lemmas (for well-formedness).
    
    \mypara{Case polymorphic types}: In this case $t \equiv \polytype{\al_1}{k_1}{t_1}$, $t' \equiv \polytype{\al_2}{k_1}{t_2}$, and $t'' \equiv \polytype{\al_3}{k_1}{t_3}$. The last rule used in each of $\tcenv \vdash t' <: t'$ and $\tcenv \vdash t' <: t''$ must have been {\sc S-Poly}. Inverting each of these we have
    \begin{equation*}
    \isSubType{\al\bindt k_1, \tcenv}{t_1[\al/\al_1]}{t_2[\al/\al_2]}, \;\;\;{\rm and} \;\;\;
    \isSubType{\al'\bindt k_1, \tcenv}{t_2[\al'/\al_2]}{t_3[\al'/\al_3]}
    \end{equation*}
    for some $\al,\al' \not\in\dom{\tcenv}$. By the change of variables lemma, 
    we have $\isSubType{\al\bindt k_1, \tcenv}{t_2[\al/\al_2]}{t_3[\al/\al_3]}$. By the inductive hypothesis we can conclude that
    $\isSubType{\al\bindt k_1, \tcenv}{t_1[\al/\al_1]}{t_3[\al/\al_3]}$
    and thus
    $\isSubType{\tcenv}{\polytype{\al_1}{k_1}{t_1}}{\polytype{\al_3}{k_1}{t_3}}$.
    The well-formedness judgments required to apply the inductive hypothesis can be constructed by inverting $\isWellFormed{\tcenv}{t}{*}$, $\isWellFormed{\tcenv}{t'}{*}$, and $\isWellFormed{\tcenv}{t''}{*}$ and using the change of free variables lemma.

    \mypara{Case $t''$ }

    \mypara{}

    \mypara{}
    
    \end{proof}

\begin{lemma}
    {\em Inversion of TAbs/TAbsT}: If $\hastype{\tcenv}{(\lambda w. e)}{x\bindt t_x->t}$ and $\vdash_w \tcenv$ then there exists $y \not\in \dom(\tcenv)$ such that $\hastype{y\bindt tx,\tcenv}{e[y/w]}{t[y/x]}$.  If $\hastype{\tcenv}{(\Lambda a_1\bindt k_1. e)}{\forall a\bindt k. t}$ and $\vdash_w \tcenv$ then exists $a' \not\in \dom(\tcenv)$ such that $\hastype{a'\bindt k,\tcenv}{e[a'/a_1]}{t[a'/a]}$. %(LemmasInvertLambda.hs) Both of these depend on ${\tt lem_sub_trans}$. Our ${\tt lem\_invert\_tabs}$ depends on ${\tt lem\_narrow\_typ}$, which in turn depends on ${\tt lem\_exact\_type}$.
\end{lemma}

% Another appendix for additions?
%

%% END MATTER
% \printindex %% Uncomment to display the index
% \nocite{}  %% Put any references that you want to include in the bib 
%               but haven't cited in the braces.
%\bibliographystyle{alpha}  %% This is just my personal favorite style. 
%                              There are many others.
%\setlength{\bibleftmargin}{0.25in}  % indent each item
%\setlength{\bibindent}{-\bibleftmargin}  % unindent the first line
%\def\baselinestretch{1.0}  % force single spacing
%\setlength{\bibitemsep}{0.16in}  % add extra space between items

\newcommand{\newblock}{}
\bibliographystyle{plainnat}
\singlespace  %to force bibilography environment to use single spacing for each entry 
              %double spacing between entries remains
\setcitestyle{numbers, open={[},close={]}}
% The various parameters that can be passed to this command are the following:
%Citation mode: authoryear, numbers or super.
%Brackets: round or square. You can manually set any other opening 
%   and closing characters with open={char} adn close={char}.
%Citation separator: semicolon, comma.
%Separator between author and year: aysep{char}.
%Separator between years with common author: yysep={char}.
%Text before post-note: notesep={text}.


\bibliography{dissertation}  %% This looks for the bibliography in template.bib 
%                          which should be formatted as a bibtex file.
%                          and needs to be separately compiled into a bbl file.
\end{document}

