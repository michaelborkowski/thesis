\chapter{Proofs of \sysf Soundness}
\label{ch:proofsF}
\label{sec:proofsF}
%
In this appendix, we present the proofs for 
each lemma of our \sysf metatheory presented in \S~\ref{sec:soundnessF}.
%
\begin{theorem} (Type Safety of \sysf) 
  \label{app:lem:soundnessF}
  \begin{enumerate}
      \item (Type Safety)
      If $\hastype{\varnothing}{\sexpr}{\stype}$ and $\evalsTo{\sexpr}{\sexpr'}$,
      then $\sexpr'$ is a value or $\sexpr' \step \sexpr''$
      %and $\hastype{\varnothing}{\sexpr''}{\stype}$
      for some $\sexpr''$.
      \item (No Error)
      If $\hastype{\varnothing}{\sexpr}{\stype}$ and $\evalsTo{\sexpr}{\sexpr'}$,
      then $\sexpr' \not = \eerr$.
  \end{enumerate}
  \end{theorem}
  %
\begin{theorem} (Type Safety)
    If $\hasftype{\varnothing}{\sexpr}{\sftype}$ and $\evalsTo{\sexpr}{\sexpr'}$,
    then $\sexpr'$ is a value or $\sexpr' \step \sexpr''$ 
    %and $\hasftype{\varnothing}{\sexpr''}{\sftype}$ 
    for some $\sexpr''$. 
\end{theorem}
%%
\begin{proof}
We proceed by induction on the number of steps in $\evalsTo{\sexpr}{\sexpr'}$.
There are two cases for $\evalsTo{\sexpr}{\sexpr'}$: either $\sexpr=\sexpr'$ 
or there exists a term $\sexpr_1$ 
such that $\sexpr \step \evalsTo{\sexpr_1}{\sexpr'}$.
In the former case we conclude immediately by the Progress Lemma.
In the latter case, $\hasftype{\varnothing}{\sexpr_1}{\sftype}$
by the Preservation Lemma. Then by the inductive hypothesis
applied to $\sexpr_1$, we conclude that either $\sexpr'$ is a value 
or $\sexpr' \step \sexpr''$ for some $\sexpr''$. 
\end{proof}
%%
\section{Progress}
%%
\begin{lemma} (Progress) \label{lem:progressF-a} 
    If $\hasftype{\varnothing}{\sexpr}{\sftype}$ 
    then $\sexpr$ is a value or $\sexpr \step \sexpr'$ for some $\sexpr'$.
\end{lemma}
%%
\begin{proof} 
We proceed by induction of the structure of 
$\hasftype{\varnothing}{\sexpr}{\sftype}$. In the cases of rule
\fPrim, \fVar, \fAbs, or \fTAbs, $\sexpr$ is a value.
\begin{itemize}
%%
\pfcase{\fApp}: We have 
$\hasftype{\varnothing}{\sexpr}{\sftype}$ where
$\sexpr \equiv \app{\sexpr_1}{\sexpr_2}$. 
Inverting, we have that there exists some type $\sftype_2$
such that $\hasftype{\varnothing}{\sexpr_1}{\funcftype{\sftype_2}{\sftype}}$
and $\hasftype{\varnothing}{\sexpr_2}{\sftype_2}$.
We split on five possible cases for 
the structure of $\sexpr_1$ and $\sexpr_2$. 
%
First, suppose $\sexpr_1 \equiv \vabs{x}{\sexpr_0}$ and $\sexpr_2$ is 
a value. Then by rule \eAppAbs, 
$e \equiv \app{\vabs{x}{\sexpr_0}}{\sexpr_2} \step \subst{\sexpr_0}{x}{\sexpr_2}$.
%
Second, suppose $\sexpr_1 \equiv \vabs{x}{\sexpr_0}$ and $\sexpr_2$
is not a value. Then by the inductive hypothesis, there exists a term
$\sexpr'_2$ such that $\sexpr_2 \step \sexpr'_2$. Then by rule \eAppV
$e \equiv \app{\vabs{x}{\sexpr_0}}{\sexpr_2} \step \app{\vabs{x}{\sexpr_0}}{\sexpr'_2}$.
%
Third, suppose $\sexpr_1 \equiv \sconst$, a built in primitive 
and $\sexpr_2$ is a value. Then by rule \ePrim, 
$e \equiv \app{\sconst}{\sexpr_2} \step \delta(\sconst,\sexpr_2)$,
which is well-defined by the primitives lemma.
%
Fourth, suppose $\sexpr_1 \equiv \sconst$ and $\sexpr_2$
is not a value. Then by the inductive hypothesis, there exists a term
$\sexpr'_2$ such that $\sexpr_2 \step \sexpr'_2$. Then by rule \eAppV
$e \equiv \app{\sconst}{\sexpr_2} \step \app{\sconst}{\sexpr'_2}$.
%
Finally, by the canonical forms lemma, $\sexpr_1$ cannot be any other
value, so it must not be a value. Then by the inductive hypothesis,
there is a term $\sexpr'_1$ such that $\sexpr_1 \step \sexpr'_1$. 
Then by rule \eApp,
$e \equiv \app{\sexpr_1}{\sexpr_2} \step \app{\sexpr'_1}{\sexpr_2}$.
%%
\pfcase{\fTApp}: We have 
$\hasftype{\varnothing}{\sexpr}{\sftype}$ where
$\sexpr \equiv \tyapp{\sexpr_1}{\stype}$ and
$\sftype \equiv \subst{\sigma}{\tvar}{\forgetreft{\stype}}$. 
Inverting, we have that 
$\hasftype{\varnothing}{\sexpr_1}{\polytype{\tvar}{\skind}{\sigma}}$.
We split on three cases for the structure of $\sexpr_1$.
%
First, suppose $\sexpr_1 \equiv \tabs{\tvar'}{\skind'}{\sexpr_0}$.
Then by rule \eTAppAbs, 
$\sexpr \equiv \tyapp{\tabs{\tvar'}{\skind'}{\sexpr_0}}{\stype} 
\step \subst{\sexpr_0}{\tvar'}{\stype}$.
%
Second, suppose $\sexpr_1 \equiv \sconst$, a built in primitive.
Then by rule \eTPrim, 
$\sexpr \equiv  \tyapp{\sconst}{\stype} \step \delta_T(\sconst,\forgetreft{\stype})$,
which is well-defined by the primitives lemma.
%
Finally, by the canonical forms lemma, $sexpr_1$ cannot be any other
form of value, so it must not be a value. Then by the inductive hypothesis,
there is a term $\sexpr'_1$ such that $\sexpr_1 \step \sexpr'_1$. 
Then by rule \eTApp
$\sexpr \equiv \tyapp{\sexpr_1}{\stype} \step \tyapp{\sexpr'_1}{\stype}$.
%%
\pfcase{\fLet}: We have 
$\hasftype{\varnothing}{\sexpr}{\sftype}$ where
$\sexpr \equiv \eletin{x}{\sexpr_1}{\sexpr_2}$. Inverting,
we have that $\hasftype{\varnothing}{\sexpr_1}{\sftype_1}$
for some type $\sftype_1$. 
By the inductive hypothesis, either $\sexpr_1$ is a value
or there is a term $\sexpr'_1$ such that $\sexpr_1 \step \sexpr'_1$.
In the former case, rule \eLetV gives us
$e \equiv \eletin{x}{\sexpr_1}{\sexpr_2} \step \subst{\sexpr_2}{x}{\sexpr_1}$.
In the latter case, by rule \eLet, 
$e \equiv \eletin{x}{\sexpr_1}{\sexpr_2} \step \eletin{x}{\sexpr'_1}{\sexpr_2}$.
%%
\pfcase{\fAnn}: We have 
$\hasftype{\varnothing}{\sexpr}{\sftype}$ where
$\sexpr \equiv \tyann{\sexpr_1}{\stype}$. Inverting,
we have the $\hasftype{\varnothing}{\sexpr_1}{\sftype}$ and
$\sftype = \forgetreft{\stype}$. 
By the inductive hypothesis, either $\sexpr_1$ is a value
or there is a term $\sexpr'_1$ such that $\sexpr_1 \step \sexpr'_1$.
In the former case, by rule \eAnnV, 
$e \equiv \tyann{\sexpr_1}{\stype} \step \sexpr_1$.
In the latter case, rule \eAnn gives us 
$e \equiv \tyann{\sexpr_1}{\stype} step \tyann{\sexpr'_1}{\stype}$.
%%
\end{itemize}
\end{proof}
%%
The progress proof is substantially the same for $\sysrf$. 
The only difference is that there is 
another straightforward inductive case for rule \tSub. 
%%
\begin{lemma}\label{lem:canonicalF-a}
    ($\sysf$ Canonical Forms) \begin{enumerate}
    \item If $\hasftype{\varnothing}{v}{\Bool}$ 
        then $v = {\tt true}$ or $v = {\tt false}$.
    \item If $\hasftype{\varnothing}{v}{\Int}$ 
        then $v$ is an integer constant.
    \item If $\hasftype{\varnothing}{v}{\funcftype{\sftype}{\sftype'}}$ 
        then either $v = \vabs{x}{\sexpr}$ or $v = \sconst$, 
        a built in primitive function where 
        $\sconst \in \{\wedge, \vee, \neg, \leftrightarrow,\leq,=\}$.
    \item If $\hasftype{\varnothing}{v}{\polytype{a}{k}{\tau}}$ 
        then either $v = \Lambda a\bindt k.\, e$ 
        or $v$ is the polymorphic equality $=$.
    \item If $\isWFFT{\varnothing}{\tau}{\skbase}$ 
        then $\tau = \tbool$ or $\tau = \tint$.
    \end{enumerate}
\end{lemma}
%%
\begin{proof}
    Parts (1) - (4) are easily deduced from the \sysf typing rules 
    in Figure \ref{fig:t} and the definition of $ty(c)$. 
    Part (5) is clear from the well-formedness rules in Figure \ref{fig:wf}.
\end{proof}
%%
Lemma \ref{lem:canonicalF-a} is sufficient for our $\sysrf$ metatheory.
Our syntactic typing judgments in $\sysrf$ respect those of $\sysf$.
Specifically, if $\hastype{\tcenv}{\sexpr}{\stype}$ and 
$\isWellFormedE{\tcenv}$, then
$\hasftype{\forgetreft{\tcenv}}{\sexpr}{\forgetreft{\stype}}$.
Therefore, we do not have to state and 
prove a separate Canonical Forms Lemma for $\sysrf$.
%%
\begin{lemma}\label{lem:inversionF-a} (Inversion of Typing) 
    \begin{enumerate}
        \item If $\hasftype{\tcenv}{c}{\sftype}$ 
            then $\sftype = \forgetreft{\ty{c}}$.
        \item If $\hasftype{\tcenv}{x}{\sftype}$ 
            then $\bind{x}{\sftype} \in \tcenv$.
        \item If $\hasftype{\tcenv}{\app{e}{e'}}{\sftype}$
            then there is some type $\sftype_x$ such that  
            $\hasftype{\tcenv}{e}{\funcftype{\sftype_x}{\sftype}}$ and
            $\hasftype{\tcenv}{e'}{\sftype_x}$.
        \item If $\hasftype{\tcenv}{\vabs{x}{e}}{\sftype}$ 
            then $\sftype = \funcftype{\sftype_x}{\sftype'}$ and
            $\hasftype{\bind{y}{\sftype_x},\tcenv}{\subst{e}{x}{y}}{\sftype'}$
            for some $\notmem{y}{\tcenv}$ and well-formed $\sftype_x$.
        \item If $\hasftype{\tcenv}{\tyapp{e}{t}}{\sftype}$ then there is some
            type $\sigma$ and kind $\skind$ such that 
            $\hasftype{\tcenv}{e}{\polytype{\al}{\skind}{\sigma}}$
            and $\sftype = \subst{\sigma}{\al}{\forgetreft{t}}$.
        \item If $\hasftype{\tcenv}{\tabs{\al}{\skind}{e}}{\sftype}$ then
            there is some type $\sftype'$ and kind $\skind$ such that
            $\sftype = {\polytype{\al}{\skind}{\sftype'}}$ and
            $\hasftype{\bind{\al'}{\skind},\tcenv}{\subst{e}{\al}{\al'}}
            {\subst{\sftype'}{\al}{\al'}}$ for some $\notmem{\al'}{\tcenv}$.
        \item If $\hasftype{\tcenv}{\eletin{x}{e_x}{e}}{\sftype}$ then
            there is some type $\sftype_x$ and $\notmem{y}{\tcenv}$ such that
            $\hasftype{\tcenv}{e_x}{\sftype_x}$ and 
            $\hasftype{\bind{y}{\sftype_x},\tcenv}{\subst{e}{x}{y}}{\sftype}$.
        \item If $\hasftype{\tcenv}{\tyann{e}{t}}{\sftype}$ then 
            $\sftype = \forgetreft{t}$ and $\hasftype{\tcenv}{e}{\sftype}$.
    \end{enumerate}
\end{lemma}
%%
\begin{proof}
This is clear from the definition of the typing rules for $\sysf$. Each premise
can match only one rule because the $\sysf$ rules are syntax directed.
\end{proof}
%%
The Inversion of Typing Lemma does not hold in $\sysrf$ due to the subtyping
relation. For instance 
$\hastype{\bind{x}{\breft{\tint}{\vv}{\vv = 5}}}{x}{\tint}$ but
$\notmem{\bind{x}{\tint}}{\bind{x}{\breft{\tint}{\vv}{\vv = 5}}}$.
In Lemma \ref{lem:inversion-a} we state and prove an analogous result 
for $\sysrf$ in the two cases needed to prove progress and preservation.
%%
\begin{lemma}\label{lem:primitivesF-a}(Primitives) 
For each built-in primitive $\sconst$, 
%
\begin{enumerate}
\item If $\forgetreft{\ty{\sconst}} = \funcftype{\sftype_x}{\sftype}$, 
        and $\hasftype{\varnothing}{\sval}{\sftype_x}$ 
        then % $\delta(\sconst,v)$ is defined and 
        $\hasftype{\varnothing}{\delta(\sconst,\sval)}{\sftype}$.
\item If $\forgetreft{\ty{\sconst}} = \polytype{\al}{\skbase}{\sftype'}$, 
        and $\isWFFT{\varnothing}{\sftype}{\skbase}$, 
        then % $\delta_T(\sconst,\stype)$ is defined and we have 
        $\hasftype{\varnothing}{\delta_T(\sconst,\sftype)}{\subst{\sftype'}{\al}{\sftype}}$.
\end{enumerate}
\end{lemma}
%%
\begin{proof}
\begin{enumerate}
    \item First consider $\sconst \in \{\wedge, \vee, \leftrightarrow\}$. 
        Then $\forgetreft{\ty{\sconst}} = \funcftype{\tbool}{\funcftype{\tbool}{\tbool}}$.
        Then by Lemma \ref{lem:canonicalF-a}, $\hasftype{\varnothing}{\sval}{\tbool}$
        gives us that $\sval = \ttrue$ or $\sval = \tfalse$.
        For each possibility for $\sconst$ and $\sval$, we can build a judgment 
        that $\hasftype{\varnothing}{\delta(\sconst,\sval)}{\funcftype{\tbool}{\tbool}}$.
        Similarly, if $\sconst = \neg$ 
        then $\forgetreft{\ty{\sconst}} = \funcftype{\tbool}{\tbool}$ and 
        $\delta(\neg,\sval) \in \{\ttrue,\tfalse\}$ can be typed at $\tbool$.
        The analysis for the other monomorphic primitives is entirely similar.
    \item Here $\sconst$ is the polymorphic $=$ and 
        $\forgetreft{\ty{\sconst}} = \polytype{\tvar}{\skbase}{\funcftype{\al}{\funcftype{\al}{\tbool}}}$. By the Canonical Forms Lemma, 
        $\sftype = \tbool$ or $\sftype = \tint$. In the former case,
        $\delta_T(\sconst,\tbool) = \leftrightarrow$, which we can type at 
        $\funcftype{\tbool}{\funcftype{\tbool}{\tbool}} =\subst{\forgetreft{\ty{\sconst}}}{\tvar}{\tbool}$. The case of $\tint$ is entirely similar
        because $\delta_T(\sconst,\tint)$ is the monomorphic integer equality.
\end{enumerate}
\end{proof}
%%
\section{Preservation}
%
\begin{lemma} (Preservation) \label{lem:preservationF} 
    If $\hasftype{\varnothing}{\sexpr}{\sftype}$ and $\sexpr \step \sexpr'$, 
    then $\hasftype{\varnothing}{\sexpr'}{\sftype}$.
\end{lemma}   

\begin{proof} 
    We proceed by induction of the structure of 
    $\hasftype{\varnothing}{\sexpr}{\sftype}$. The cases of rules
    \fPrim, \fVar, \fAbs, or \fTAbs cannot occur because $\sexpr$ is a value
    and no value can take a step in our semantics.
    \begin{itemize}
    %%
    \pfcase{\fApp}: We have 
    $\hasftype{\varnothing}{\sexpr}{\sftype}$ where
    $\sexpr \equiv \app{\sexpr_1}{\sexpr_2}$. 
    Inverting, we have that there exists some type $\sftype_2$
    such that $\hasftype{\varnothing}{\sexpr_1}{\funcftype{\sftype_2}{\sftype}}$
    and $\hasftype{\varnothing}{\sexpr_2}{\sftype_2}$.
    We split on five possible cases for 
    the structure of $\sexpr_1$ and $\sexpr_2$. 
    %
    First, suppose $\sexpr_1 \equiv \vabs{x}{\sexpr_0}$ and $\sexpr_2$ is 
    a value. Then by rule \eAppAbs and the determinism of our semantics, 
    $e' \equiv \subst{\sexpr_0}{x}{\sexpr_2}$.
    By the Inversion of Typing, for some $y$ we have
    $\hasftype{\bind{y}{\sftype_2}}{\subst{\sexpr_0}{x}{y}}{\sftype}$.
    By the Substitution Lemma, substituting $\sexpr_2$ through for $y$
    gives us $\hasftype{\varnothing}{\subst{\sexpr_0}{x}{\sexpr_2}}{\sftype}$
    as desired because $\subst{\subst{\sexpr_0}{x}{y}}{y}{\sexpr_2} = \subst{\sexpr_0}{x}{\sexpr_2}$.
    %
    Second, suppose $\sexpr_1 \equiv \vabs{x}{\sexpr_0}$ and $\sexpr_2$
    is not a value. Then by the progress lemma, there exists a term
    $\sexpr'_2$ such that $\sexpr_2 \step \sexpr'_2$. Then by rule \eAppV
    and the determinism of our semantics,
    $e' \equiv \app{\vabs{x}{\sexpr_0}}{\sexpr'_2}$. 
    Now, by the inductive hypothesis, $\hasftype{\varnothing}{\sexpr'_2}{\sftype_2}$.
    Applying rule \fApp, $\hasftype{\varnothing}{\app{\sexpr_1}{\sexpr'_2}}{\sftype}$
    as desired.
    %
    Third, suppose $\sexpr_1 \equiv \sconst$, a built in primitive, 
    and $\sexpr_2$ is a value. Then by rule \ePrim
    and the determinism of the semantics, 
    $e' \equiv \delta(\sconst,\sexpr_2)$.
    By the primitives lemma, 
    $\hasftype{\varnothing}{\delta(\sconst,\sexpr_2)}{\sftype}$ as desired.
    %
    Fourth, suppose $\sexpr_1 \equiv \sconst$ and $\sexpr_2$
    is not a value. Then we argue in the same manner as the second case.
    %
    Finally, by the canonical forms lemma, $\sexpr_1$ cannot be any other
    value, so it must not be a value. Then by the progress lemma,
    there is a term $\sexpr'_1$ such that $\sexpr_1 \step \sexpr'_1$. 
    Then by rule \eApp and the determinism of the semantics,
    $e' \equiv \app{\sexpr'_1}{\sexpr_2}$. By the inductive hypothesis,
    $\hasftype{\varnothing}{\sexpr'_1}{\funcftype{\sftype_2}{\sftype}}$.
    Applying rule \fApp, $\hasftype{\varnothing}{\app{\sexpr'_1}{\sexpr_2}}{\sftype}$
    as desired.
    %%
    \pfcase{\fTApp}: We have 
    $\hasftype{\varnothing}{\sexpr}{\sftype}$ where
    $\sexpr \equiv \tyapp{\sexpr_1}{\stype}$ and
    $\sftype \equiv \subst{\sigma}{\tvar}{\forgetreft{\stype}}$. 
    Inverting, we have that 
    $\hasftype{\varnothing}{\sexpr_1}{\polytype{\tvar}{\skind}{\sigma}}$
    and $\isWFFT{\varnothing}{\forgetreft{\stype}}{\skind}$.
    We split on three cases for the structure of $\sexpr_1$.
    %
    First, suppose $\sexpr_1 \equiv \tabs{\tvar}{\skind}{\sexpr_0}$.
    Then by rule \eTAppAbs and the determinism of the semantics, 
    $\sexpr' \equiv \subst{\sexpr_0}{\tvar}{\stype}$.
    By the inversion of typing, for some $\tvar'$, we have
    $\hasftype{\bind{\tvar'}{\skind}}{\subst{\sexpr_0}{\tvar}{\tvar'}}
    {\subst{\sigma}{\tvar}{\tvar'}}$.
    By the Substitution Lemma, substituting $\forgetreft{\stype}$ 
    through for $\tvar$  gives us 
    $\hasftype{\varnothing}{\subst{\sexpr_0}{\tvar}{\stype}}{\subst{\sigma}{\tvar}{\forgetreft{\stype}}}$ as desired.
    %
    Second, suppose $\sexpr_1 \equiv \sconst$, a built in primitive.
    Then by rule \eTPrim and the determinism of the semantics, 
    $\sexpr' \delta_T(\sconst,\forgetreft{\stype})$. By the primitives lemma,
    $\hasftype{\varnothing}{\delta_T(\sconst,\forgetreft{\stype})}{\subst{\sigma}{\tvar}{\forgetreft{\stype}}}$.
    %
    Finally, by the canonical forms lemma, $sexpr_1$ cannot be any other
    form of value, so it must not be a value. Then by the progress lemma,
    there is a term $\sexpr'_1$ such that $\sexpr_1 \step \sexpr'_1$. 
    Then by rule \eTApp and the deterministic semantics
    $\sexpr' \equiv \tyapp{\sexpr'_1}{\stype}$.  By the inductive hypothesis,
    $\hasftype{\varnothing}{\sexpr'_1}{\polytype{\tvar}{\skind}{\sigma}}$.
    Applying rule \fTApp, 
    $\hasftype{\varnothing}{\tyapp{\sexpr'_1}{\stype}}{\subst{\sigma}{\tvar}{\forgetreft{\stype}}}$
    as desired.
    %%
    \pfcase{\fLet}: We have 
    $\hasftype{\varnothing}{\sexpr}{\sftype}$ where
    $\sexpr \equiv \eletin{x}{\sexpr_1}{\sexpr_2}$. Inverting,
    we have that 
    $\hasftype{\bind{y}{\sftype_1}}{\subst{\sexpr_2}{x}{y}}{\sftype}$
    and $\hasftype{\varnothing}{\sexpr_1}{\sftype_1}$
    for some type $\sftype_1$. 
    By the progress lemma either $\sexpr_1$ is a value
    or there is a term $\sexpr'_1$ such that $\sexpr_1 \step \sexpr'_1$.
    %
    In the former case, rule \eLetV and determinism give us
    $\sexpr' \equiv \subst{\sexpr_2}{x}{\sexpr_1}$.
    By the Substitution Lemma (substituting $\sexpr_1$ for $x$), 
    we have $\hasftype{\varnothing}{\subst{\sexpr_2}{x}{\sexpr_1}}{\sftype}$
    as desired because 
    $\subst{\sexpr_2}{x}{\sexpr_1} = \subst{\subst{\sexpr_2}{x}{y}}{y}{\sexpr_1}$.
    %
    In the latter case, by rule \eLet and determinism give us, 
    $\sexpr' \equiv \eletin{x}{\sexpr'_1}{\sexpr_2}$.
    By the inductive hypothesis we have that 
    $\hasftype{\varnothing}{\sexpr'_1}{\sftype_1}$ and by rule \fLet 
    we have $\hasftype{\varnothing}{\eletin{x}{\sexpr'_1}{\sexpr_2}}{\sftype}$. 
    %%
    \pfcase{\fAnn}: We have 
    $\hasftype{\varnothing}{\sexpr}{\sftype}$ where
    $\sexpr \equiv \tyann{\sexpr_1}{\stype}$. Inverting,
    we have the $\hasftype{\varnothing}{\sexpr_1}{\sftype}$ and
    $\sftype = \forgetreft{\stype}$. 
    By the progress lemma, either $\sexpr_1$ is a value
    or there is a term $\sexpr'_1$ such that $\sexpr_1 \step \sexpr'_1$.
    %
    In the former case, by rule \eAnnV and the determinism of the semantics, 
    $\sexpr' \equiv \sexpr_1$. Then we already have that
    $\hasftype{\varnothing}{\sexpr'}{\sftype}$
    %
    In the latter case, rule \eAnn and determinism give us 
    $\sexpr' \equiv \tyann{\sexpr'_1}{\stype}$. By the inductive hypothesis
    we have that $\hasftype{\varnothing}{\sexpr'_1}{\sftype}$. By rule
    \fAnn we conclude $\hasftype{\varnothing}{\tyann{\sexpr'_1}{\stype}}{\sftype}$.
    %%
    \end{itemize}
\end{proof}

The proof of preservation for $\sysrf$ differs in two cases above. 
In \tApp and \tTApp, we must use the Inversion of Typing lemma (\ref{lem:inversion-a})
from $\sysrf$ because the presence of rule \tSub prevents us from 
inferring the last rule used to type a term or type abstraction.
%
Furthermore, in case \tApp the substitution lemma would give us that 
$\hastype{\varnothing}{\sexpr'}{\subst{\stype}{x}{\sval_x}}$ for 
some value $\sval_x$. However we need to show preservation of the
existential type $\existype{x}{\stype_x}{\stype}$. This is done by
using rule \sWitn to show that, in fact, 
$\isSubType{\varnothing}{\subst{\stype}{x}{\sval_x}}{\existype{x}{\stype_x}{\stype}}$.
%%
\begin{lemma}(Substitution)\label{lem:substitutionF}
    If $\hasftype{\tcenv}{\sval_x}{\sftype_x}$ 
    and if $\isWFFT{\tcenv}{\forgetreft{\stype_{\al}}}{k_{\al}}$ then 
    \begin{enumerate}
    \item If\; $\hasftype{\tcenv', \bind{x}{\sftype_x}, \tcenv}{\sexpr}{\sftype}$
        and $\isWFFE{\tcenv', \bind{x}{\sftype_x}, \tcenv}$ then
        ${\hasftype{\tcenv', \tcenv}{\subst{\sexpr}{x}{\sval_x}}{\sftype}}$.
    \item If\; $\hasftype{\tcenv', \bind{\al}{k_{\al}}, \tcenv}{\sexpr}{\sftype}$
        and $\isWFFE{\tcenv', \bind{\al}{k_{\al}}, \tcenv}$ then
        ${\hasftype{\subst{\tcenv'}{\al}{\forgetreft{\stype_{\al}}}, \tcenv}
                   {\subst{\sexpr}{\al}{\stype_{\al}}}
                   {\subst{\sftype}{\al}{\forgetreft{\stype_{\al}}}}}$
    \end{enumerate}
\end{lemma} 
%%
\begin{proof}
We give the proofs for part (2); part (1) is similar but slightly
simpler because term variables do not appear in types in $\sysf$
We proceed by induction on the derivation tree of the typing judgment
$\hasftype{\tcenv', \bind{\al}{k_{\al}}, \tcenv}{\sexpr}{\sftype}$.
%%
\pfcase{\fPrim}: We have $\sexpr \equiv \sconst$ and
$\hasftype{\tcenv', \bind{\al}{k_{\al}}, \tcenv}{\sconst}{\forgetreft{\ty{\sconst}}}$.
Neither $\sconst$ nor $\ty{\sconst}$ has any free variables, so each is 
unchanged under substitution. Then by rule \tPrim we conclude
$\hasftype{\subst{\tcenv'}{\al}{\forgetreft{\stype_{\al}}}, \tcenv}{\sconst}{\forgetreft{\ty{\sconst}}}$ because the environment may be chosen arbitrarily.
%%
\pfcase{\fVar}: We have $\sexpr \equiv x$; by inversion, we get that 
$\bind{x}{\sftype} \in \tcenv', \bind{\al}{k_{\al}}, \tcenv$. We must have
$\tvar \neq x$ so there are two cases to consider for where $x$ can 
appear in the environment. If $\bind{x}{\sftype} \in \tcenv$,  then $\sftype$
cannot contain $\tvar$ as a free variable because $x$ is bound first in the environment (which grows from right to left). Then
$\hasftype{\subst{\tcenv'}{\al}{\forgetreft{\stype_{\al}}}, \tcenv}{x}{\sftype}$
as desired because $\subst{\sftype}{\tvar}{\forgetreft{\stype_{\al}}} = \sftype$.
Otherwise $\bind{x}{\sftype} \in \tcenv'$ and so 
$\bind{x}{\subst{\sftype}{\tvar}{\forgetreft{\stype_{\al}}}} \in \subst{\tcenv'}{\al}{\forgetreft{\stype_{\al}}}, \tcenv$. Thus
$\hasftype{\subst{\tcenv'}{\al}{\forgetreft{\stype_{\al}}}, \tcenv}{x}{\subst{\sftype}{\tvar}{\forgetreft{\stype_{\al}}}}$.
%
(In part (1), we have an additional case where 
$\hasftype{\tcenv', \bind{x}{\sftype}, \tcenv}{x}{\sftype}$. We have
$\subst{x}{x}{\sval_x} = \sval_x$ and so we can apply the weakening Lemma
to $\hasftype{\tcenv}{\sval_x}{\sftype}$ to obtain
$\hasftype{\tcenv', \tcenv}{\sval_x}{\sftype}$.)
%%
\pfcase{\fApp}: We have $\sexpr \equiv \app{\sexpr_1}{\sexpr_2}$. By inversion
we have that
$\hasftype{\tcenv', \bind{\al}{k_{\al}}, \tcenv}{\sexpr_1}{\funcftype{\sftype_x}{\sftype}}$
and $\hasftype{\tcenv', \bind{\al}{k_{\al}}, \tcenv}{\sexpr_2}{\sftype_x}$.
Applying the inductive hypothesis to both of these, we get
$\hasftype{\subst{\tcenv'}{\al}{\forgetreft{\stype_{\al}}}, \tcenv}{\subst{\sexpr_1}{\tvar}{\stype_{\tvar}}}{\subst{\funcftype{\sftype_x}{\sftype}}{\tvar}{\forgetreft{\stype_{\tvar}}}}$
and
$\hasftype{\subst{\tcenv'}{\al}{\forgetreft{\stype_{\al}}}, \tcenv}{\subst{\sexpr_2}{\tvar}{\stype_{\tvar}}}{\subst{\sftype_x}{\tvar}{\forgetreft{\stype_{\tvar}}}}$. 
Combining these by rule \fApp, we conclude
$\hasftype{\subst{\tcenv'}{\al}{\forgetreft{\stype_{\al}}}, \tcenv}{\subst{\app{\sexpr_1}{\sexpr_2}}{\tvar}{\stype_{\tvar}}}{\subst{\sftype}{\tvar}{\forgetreft{\stype_{\tvar}}}}$.
%%
\pfcase{\fAbs}: We have $\sexpr \equiv \vabs{x}{\sexpr_1}$ 
and $\sftype \equiv \funcftype{\sftype_x}{\sftype_1}$. 
By inversion we have that for some $y$, both
$\hasftype{\bind{y}{\sftype_x},\tcenv', \bind{\al}{k_{\al}}, \tcenv}
{\subst{\sexpr_1}{x}{y}}{\sftype_1}$
and
$\isWFFT{\tcenv', \bind{\al}{k_{\al}}, \tcenv}{\sftype_x}{\skind_x}$.
By the inductive hypothesis, and the Substitution Lemma for 
well-formedness judgments, we have
$\hasftype{\bind{y}{\subst{\sftype_x}{\al}{\forgetreft{\stype_{\al}}}},\subst{\tcenv'}{\al}{\forgetreft{\stype_{\al}}}, \tcenv}
{\subst{\subst{\sexpr_1}{\al}{\stype_{\al}}}{x}{y}}{\subst{\sftype_1}{\al}{\forgetreft{\stype_{\al}}}}$
and
$\isWFFT{\subst{\tcenv'}{\al}{\forgetreft{\stype_{\al}}}, \tcenv}
{\subst{\sftype_x}{\al}{\forgetreft{\stype_{\al}}}}{\skind_x}$,
where we can switch the order of substitutions because $y$ does not 
appear free in the well-formed type $\stype_{\tvar}$.
Then we can conclude by applying rule \fAbs that 
$\hasftype{\subst{\tcenv'}{\al}{\forgetreft{\stype_{\al}}}, \tcenv}
{\subst{\vabs{x}{\sexpr_1}}{\tvar}{\stype_\tvar}}
{\subst{\funcftype{\sftype_x}{\sftype_1}}{\tvar}{\forgetreft{\stype_\tvar}}}$.
%%


%{\bf Case} {\sc T-AppT}:We have $\Gamma', x\bind t_x,\Gamma \vdash e : t$ where $e \equiv e'\; [t']$ and $t \equiv s[t'/\al']$. 
%By inversion, $\Gamma', x\bind t_x, \Gamma \vdash e' : \polytype{\al'}{k'}{s}$ and $\Gamma', x\bind t_x, \Gamma \vdash_w t' : k'$. By the inductive hypothesis,
%\begin{equation}\label{361T}
%\Gamma'[v_x/x],\Gamma\vdash e'[v_x/x] : \polytype{\al'}{k'}{s[v_x/x]}
%\end{equation}
%and
%\begin{equation}\label{362T}
%\Gamma'[v_x/x],\Gamma\vdash_w t'[v_x/x] : k'.
%\end{equation}
%By applying rule {\sc T-AppT}
%\begin{equation}
%\Gamma'[v_x/x],\Gamma\vdash e'[v_x/x]\;[t'[v_x/x]] : s[v_x/x][t'[v_x/x]/\al'].
%\end{equation}
%Now by the definition of substitution we have $e'[v_x/x]\; [t'[v_x/x]] = (e'\;[t'])[v_x/x] \equiv e[v_x/x]$ and $s[v_x/x][t'[v_x/x]/\al'] = s[t'/\al'][v_x/x] \equiv t[v_x/x]$. Therefore, we conclude that $\Gamma'[v_x/x],\Gamma \vdash e[v_x/x] : t[v_x/x]$.

%{\bf Case} {\sc T-AbsT}:We have $\Gamma', x:t_x,\Gamma \vdash e : t$ where $e \equiv \Lambda \al:k. e'$ and $t \equiv \polytype{\al}{k}{t'}$. By inversion, $\al'\bind k, \Gamma', x\bind t_x,\Gamma \vdash e'[\al'/\al] : t'[\al'/\al]$ and $\al'\bind k, \Gamma', x\bind t_x,\Gamma \vdash_w t'[\al'/\al] : k'$ for some $\al'\not\in\dom{\Gamma}$. By the inductive hypothesis
%\begin{equation}
%\al'\bind k, \Gamma'[v_x/x],\Gamma \vdash e'[\al'/\al][v_x/x] : t'[\al'/\al][v_x/x] \;\;{\rm and}\;\; \al'\bind k, \Gamma'[v_x/x],\Gamma \vdash_w t'[\al'/\al][v_x/x] : k'.
%\end{equation}
%We must have $x\neq \al'$ and we have $x \neq \al$ because bound and free variables are taken to be distinct. Moreover, $v_x$ contains only free variables from $\Gamma$, so $e'[\al'/\al][v_x/x] = e'[v_x/x][\al'/\al]$ and $t'[\al'/\al][v_x/x] = t'[v_x/x][\al'/\al]$. Then by rule {\sc T-AbsT}
%\begin{equation}
%\Gamma'[v_x/x], \Gamma' \vdash \Lambda \al:k.(e'[v_x/x]) : \polytype{\al}{k}{t'[v_x/x]}.
%\end{equation}
%By definition of substitution, we can rewrite the above as
%\[
%\Gamma'[v_x/x],\Gamma \vdash (\Lambda \al:k.e')[v_x/x] : \polytype{\al}{k}{t'}[v_x/x].
%\]
%In the type variable substitution lemma, where we have an environment $\Gamma',\beta\bind k_\beta,\Gamma$ and $\Gamma \vdash_w t_\beta : k_\beta$ this proof is similar except that we argue that $e'[\al'/\al][t_\beta/\beta] = e'[t_\beta/\beta][\al'/\al]$ and $t'[\al'/\al][t_\beta/\beta] = t'[t_\beta/\beta][\al'/\al]$ because $\al'\neq\beta$ and only free variables from $\Gamma$ may appear in $t_\beta$.

%{\bf Case} {\sc T-Let}: We have $\Gamma', x\bind t_x,\Gamma \vdash e : t$ where $e \equiv (\letin{y}{e_1}{e_2})$ and $t \equiv t_2$. By inversion, $\Gamma', x\bind t_x,\Gamma \vdash e_1 : t_1$ and $z\bind t_1, \Gamma,x\bind t_x,\Gamma \vdash e_2[z/y] : t_2[z/y]$ for some type $t_1$ and some $y\not\in\dom{\Gamma}$. By the inductive hypothesis we have
%\begin{equation}
%\Gamma'[v_x/x],\Gamma \vdash e_1[v_x/x] : t_1[v_x/x]
%\end{equation} and
%\begin{equation}
%y\bind t_1[v_x/x], \Gamma'[v_x/x], \Gamma \vdash e_2[z/y][v_x/x] : t_2[z/y][e_x/x]
%\end{equation}
%We must have $x\neq z$ and we have $x \neq y$ because bound and free variables are taken to be distinct. Moreover, $v_x$ contains only free variables from $\Gamma$, so $e_2[z/y][v_x/x] = e_2[v_x/x][z/y]$ and $t_2[z/y][v_x/x] = t_2[v_x/x][z/y]$. Then by rule {\sc T-Let},
%%\begin{equation}
%\Gamma'[v_x/x], \Gamma \vdash \letin{y}{e_1[v_x/x]}{e_2[v_x/x]} : t_2[v_x/x]
%\end{equation} which we can write as \[
%\Gamma'[v_x/x], \Gamma \vdash (\letin{y}{e_1}{e_2})[v_x/x]:t_2[v_x/x].
%% 
\pfcase{\fLet}: We have $\sexpr \equiv \eletin{x}{\sexpr_1}{\sexpr_2}$
and by inversion we have that for some type $\sftype_1$,
$\hasftype{\tcenv', \bind{\al}{k_{\al}}, \tcenv}{\sexpr_1}{\sftype_1}$
and for some $\notmem{y}{\tcenv', \bind{\al}{k_{\al}}, \tcenv}$,
$\hasftype{\bind{y}{\sftype_1},\tcenv', \bind{\al}{k_{\al}}, \tcenv}
{\subst{\sexpr_2}{x}{y}}{\sftype}$.
By the inductive hypothesis, we have that
$\hasftype{\subst{\tcenv'}{\al}{\forgetreft{\stype_{\al}}}, \tcenv}
{\subst{\sexpr_1}{\tvar}{\stype_\tvar}}{\subst{\sftype_1}{\tvar}{\forgetreft{\stype_\tvar}}}$
and
$\hasftype{\bind{y}{\subst{\sftype_1}{\tvar}{\stype_\tvar}},\subst{\tcenv'}{\al}{\forgetreft{\stype_{\al}}}, \tcenv}
{\subst{\sexpr_2}{x}{\stype_\tvar}}{\subst{\sftype}{\tvar}{\forgetreft{\stype_\tvar}}}$.
Then by rule \fLet we conclude
$\hasftype{\subst{\tcenv'}{\al}{\forgetreft{\stype_{\al}}}, \tcenv}
{\eletin{x}{\subst{\sexpr_1}{\tvar}{\stype_\tvar}}{\subst{\sexpr_2}{\tvar}{\stype_\tvar}}}{\subst{\sftype}{\tvar}{\forgetreft{\stype_\tvar}}}$.
%%
\pfcase{\fAnn}: We have $\sexpr \equiv \tyann{\sexpr'}{\stype}$ 
and by inversion we have that $\forgetreft{\stype} = \sftype$ and 
$\hasftype{\tcenv', \bind{\al}{k_{\al}}, \tcenv}
{\sexpr'}{\sftype}$.
By the inductive hypothesis, we have
$\hasftype{\subst{\tcenv'}{\al}{\forgetreft{\stype_{\al}}}, \tcenv}
{\subst{\sexpr'}{\tvar}{\stype_\tvar}}{\subst{\sftype}{\tvar}{\forgetreft{\stype_\tvar}}}$.
By our definition of refinement erasure, we have
$\forgetreft{\subst{\stype}{\tvar}{\stype_\tvar}}
 = \subst{\forgetreft{\stype}}{\tvar}{\forgetreft{\stype_\tvar}}$
and we have 
$\subst{\tyann{\sexpr'}{\stype}}{\tvar}{\stype_\tvar} 
 = \tyann{\subst{\sexpr'}{\tvar}{\stype_\tvar}}{\subst{\stype}{\tvar}{\stype_\tvar}}$. Thus by rule \fAnn,
$\hasftype{\subst{\tcenv'}{\al}{\forgetreft{\stype_{\al}}}, \tcenv}
{\subst{\tyann{\sexpr'}{\stype}}{\tvar}{\stype_\tvar}}{\subst{\forgetreft{\stype}}{\tvar}{\forgetreft{\stype_\tvar}}}$.
%%
\end{proof}
%%%%%%%%%%%%%%%%%%%%%%%%%%%%%%%
%%%%%%%%%%%%%%%%%%%%%%%%%%%%%%%
%%%%%%%%%%%%%%%%%%%%%%%%%%%%%%%
 





%\begin{lemma}\label{WFFTlemmas}
%    For any System F environments $\tcenv$, $\tcenv'$ and $x,y,\al,\al' \not\in dom(\tcenv', \tcenv)$:
%    \item (Change of Free Variables) If $\tcenv', x\bindt \tau_x, \tcenv \vdash_F^w \tau:k$ then $\tcenv', y\bindt t_x, \tcenv \vdash_F^w \tau:k$.  
%    If $\tcenv', \al\bindt k_{\al}, \tcenv \vdash_F^w \tau:k$, then $\tcenv', \al'\bindt k_{\al}, \tcenv \vdash_F^w \tau[\al'/\al]:k$. 
%    \item (Weakening) If $\tcenv', \tcenv \vdash_F^w \tau:k$ then $\tcenv', x\bindt \tau_x, \tcenv \vdash_F^w \tau:k$ and $\tcenv', \al\bindt k_{\al}, \tcenv \vdash_F^w \tau:k$.
%    \item (Substitution Lemma) If $\tcenv', \al\bindt k_{\al}, \tcenv \vdash_F^w \tau:k$ and 
%    $\isWFFT{\tcenv}{\tau_{\al}}{k_{\al}}$ 
%    then $\tcenv'[\tau_{\al}/\al], \tcenv \vdash_F^w \tau[\tau_{\al}/\al]:k$. 
%\end{lemma}
%
%\begin{proof}
%\begin{enumerate}
%\item Change of Free Variables is proven by 
%a straightforward mutual induction on the structure of 
%$\isWFFT{\tcenv', x\bindt \tau_x, \tcenv}{\tau}{k}$, 
%and on the structure of 
%$\isWFFT{\tcenv', \al\bindt k_{\al}, \tcenv}{\tau}{k}$. 
%The one non-trivial detail, which will arise frequently below, is that in the case \wfftPoly, we invert 
%$\isWFFT{\tcenv', x\bindt \tau_x, \tcenv}{\forall\,\al\bindt k.\tau}{*}$
%to obtain
%$\isWFFT{\al'\bindt k, \tcenv', x\bindt \tau_x, \tcenv}{\tau[\al'/\al]}{k_\tau}$
%for some $\al' \not\in {\rm dom}(\tcenv', x\bindt \tau_x, \tcenv)$.
%Before we can apply the inductive hypothesis, we must choose some 
%new fresh 
%$\al'' \not\in {\rm dom}(\tcenv', y\bindt \tau_x, \tcenv) \cup \{x\}$
%because $y$ was not chosen as fresh with respect to an environment
%containing $\al'$. Then we apply the inductive hypothesis twice,
%first obtaining
%$\isWFFT{\al''\bindt k, \tcenv', x\bindt \tau_x, \tcenv}{\tau[\al''/\al]}{k_\tau}$
%and then
%$\isWFFT{\al''\bindt k, \tcenv', y\bindt \tau_x, \tcenv}{\tau[\al''/\al]}{k_\tau}$
%before finally concluding 
%$\isWFFT{\tcenv', y\bindt \tau_x, \tcenv}{\forall\,\al\bindt k.\tau}{*}$
%by rule \wfftPoly.
%\item The proof is a straightforward induction on the structure of
%$\isWFFT{\tcenv',\tcenv}{\tau}{k}$, similar to the Change of Free Variables lemma above. In case \wfftPoly, we invert 
%$\isWFFT{\tcenv', \tcenv}{\forall\,\al\bindt k.\tau}{*}$
%to obtain
%$\isWFFT{\al'\bindt k, \tcenv', \tcenv}{\tau[\al'/\al]}{k_\tau}$
%and use the Change of Free Variables lemma to pick a new $\al''$ 
%that is fresh with respect to $x$ and then apply the inductive hypothesis.
%\item The Substitution Lemma here is proven by induction on the
%structure of
%$\isWFFT{\tcenv', \al\bindt k_{\al}, \tcenv}{\tau}{k}$.
%The proof is straightforward except in the case of rule \wfftVar.
%Here $\tau \equiv \al'$ and we invert 
%$\isWFFT{\tcenv', \al\bindt k_{\al}, \tcenv}{\al'}{k}$
%to get $\al'\bindt k \in \tcenv', \al\bindt k_{\al}, \tcenv$.
%We have three cases depending on where $\al'$ is bound in 
%$\tcenv', \al\bindt k_{\al}, \tcenv$.
%First, if $\al' \bindt k \in \tcenv$, then $\al' \neq \al$ 
%and so $\al'[\tau_{\al}/{\al}] = \al'$.
%...
% 
%Next, if $\al' \bindt k = \al\bindt k_{\al}$, then ...
%
%Finally, if $\al' \bindt k \in \tcenv'$, then $\al' \neq \al$ 
%and so $\al'[\tau_{\al}/{\al}] = \al'$. In particular,
%$\al' \bindt k \in \tcenv'[\tau_{\al}/{\al}]$. ....
%
%$\isWFFT{\tcenv}{\tau_{\al}}{k_{\al}}$ 
%\end{enumerate}
%\end{proof}
