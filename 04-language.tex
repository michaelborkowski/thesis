\chapter{The Languages \sysf and \sysrf}
\label{ch:language}

%For brevity, clarity and also 
To cut the circularities in the metatheory,
%(in rule \wtRefn in \S~\ref{sec:typing:wf}), 
we formalize 
refinements using two calculi. 
% 
The first is the \emph{base} language 
\sysf: a classic System F \cite{TAPL} 
with call-by-value semantics extended 
with primitive \tint and \tbool types 
and operations.
%
The second is the \emph{refined} language
\sysrf which extends \sysf with refinements.
%
By using the first calculus to express the typing judgments
for our refinements,
we avoid making 
the well-formedness (in rule \wtRefn in \S~\ref{sec:typing:wf})
and the implication (in type denotations of~\Cref{fig:den}) judgments
mutually dependent with the typing judgments. 
%
We use the $\greybox{\mbox{grey}}$ highlights 
for the extensions to  
$\sysf$ required % needed to support refinements 
for $\sysrf$.

\begin{conference}
\begin{figure}[t!]
  {\small
    \begin{tabular}{rrcll}
\emphbf{Primitives} 
  & \sconst & $\bnfdef$ & $\ttrue$ $\spmid$ $\tfalse$ $\spmid$  
  $0, 1, 2, \ldots$ 
  $\spmid$  $\wedge, \neg$ 
  $\spmid$  $\leq, \sconst\!\!\leq, =,  \sconst\!\!=$                             
  \\
\emphbf{Values}
  & \sval   & $\bnfdef$ & \sconst  
  $\spmid  x, y, \ldots
  \spmid   \vabs{x}{e}
  \spmid   \tabs{\alpha}{k}{e}
  $
  &\\
\emphbf{Terms}
  & \sexpr  & $\bnfdef$ & 
  $
  \sval \spmid \app{e_1}{e_2} \spmid \tyapp{e}{t} \spmid \eletin{x}{e_1}{e_2} 
  \spmid \tyann{e}{t} \spmid \eif{e}{e_1}{e_2} \spmid \eerr
  $ 
\end{tabular}
}
\caption{Syntax of Primitives, Values, and Expressions.}
\label{fig:syn:termsC}
\vspace{-0.4cm}
\end{figure}
\end{conference}


\begin{fullversion}
\begin{figure}[t!]
%  \scalebox{0.80}{
    \begin{tabular}{rrcll}
\emphbf{Primitives} 
  & \sconst & $\bnfdef$ & $\ttrue$ $\spmid$ $\tfalse$ $\spmid$  $0, 1, 2, \ldots$    & \emph{booleans and integers} \\
%  &         & $\spmid$  & $0, 1, 2, \ldots$                     & \emph{integers} \\ 
  &         & $\spmid$  & $\wedge, \vee, \neg, \leftrightarrow$ & \emph{boolean ops.} \\
  &         & $\spmid$  & $\leq, =$                             & \emph{polymorphic comparisons} \\ [0.05in] 

\emphbf{Values}
  & \sval   & $\bnfdef$ & \sconst               & \emph{primitives} \\ 
  &         & $\spmid$  & x, y, \ldots          & \emph{variables} \\
  &         & $\spmid$  & \vabs{x}{e}           & \emph{abstractions} \\
  &         & $\spmid$  & \tabs{\alpha}{k}{e}   & \emph{type abstractions} \\[0.05in]

\emphbf{Terms}
  & \sexpr  & $\bnfdef$ & \sval                 & \emph{values} \\ 
  &         & $\spmid$  & \app{e_1}{e_2}        & \emph{applications} \\
  &         & $\spmid$  & \tyapp{e}{t}          & \emph{type applications} \\
  &         & $\spmid$  & \eletin{x}{e_1}{e_2}  & \emph{let-binders} \\
  &         & $\spmid$  & \tyann{e}{t}          & \emph{annotations} \\
  &         & $\spmid$  & \eif{e_0}{e_1}{e_2}   & \emph{conditionals} \\
  &         & $\spmid$  & \eerr                 & \emph{runtime errors} \\
\end{tabular}
%  }
  \vspace{-0.0cm}
  \caption{Syntax of Primitives, Values, and Expressions.}
\label{fig:syn:terms}
\vspace{-0.0cm}
\end{figure}
\end{fullversion}



\begin{figure}%[b!]
%{\small
  \begin{tabular}{rrcll}
  \emphbf{Kinds} 
    & \skind & $\bnfdef$ & $\skbase \spmid \skstar$ & \emph{base and star kind} \\[0.05in]
%    &        & $\spmid$  & \skstar & \emph{star kind} \\[0.05in]
  
  \emphbf{Predicates}  
    & \spred & $\bnfdef$ & $\greybox{\{ e \;|\; \exists\, \tcenv.\, \hasftype{\tcenv}{\sexpr}{\tbool} \}}$ & \greytextbox{\emph{boolean-typed terms}} 
    \\[0.05in] 
  
  \emphbf{Base Types} 
    & \sbase & $\bnfdef$ & $\tbool \spmid \tint \spmid \tvar$ 
    & \emph{bool, ints, and type variables} 
    \\[0.05in] 

  \emphbf{Types}
%  & \stype & $\bnfdef$ & $\sbase\greybox{\!\breft{}{\vv}{\spred}}$  
%             $\spmid$   $\greybox{x\!}\!\functype{}{t_x}{t}$ 
%             $\spmid$   \greybox{\existype{x}{t_x}{t}} 
%            $\spmid$   $\polytype{\tvar}{\skind}{t}$  & \\ [0.05in]    
% RESTORE IF THERE IS SPACE     
   & \stype & $\bnfdef$ & $\sbase\greybox{\!\breft{}{\vv}{\spred}}$  & \emph{\greytextbox{refined} base type} \\ 
   &        & $\spmid$  & $\greybox{x\!}\!\functype{}{t_x}{t}$         & \emph{function type}     \\        
   &        & $\spmid$  & \greybox{\existype{x}{t_x}{t}} & \greytextbox{\emph{existential type}}  \\        
   &        & $\spmid$  & $\polytype{\tvar}{\skind}{t}$  & \emph{polymorphic type}  \\ [0.05in]        

  \emphbf{Environments}
    & $\tcenv$ & $\bnfdef$ & 
    $\varnothing \spmid \tcenv, \bind{x}{t} \spmid \tcenv, \bind{\tvar}{\skind}$                  
    & \emph{variable and type bindings} 
  \end{tabular}
%}
\vspace{-0.0cm}
  \caption{Syntax of Types. % and Environments.
           The gray boxes are the extensions 
           to $\sysf$ needed by $\sysrf$.
           We use $\sftype$ for $\sysf$-only types.}
  \label{fig:syn:types}
  \label{fig:syn:reft}
  \label{fig:syn:env}
  \vspace{-0.00cm}
\end{figure}

\section{Syntax} \label{sec:lang:syntax}

We start by describing the syntax of terms and types 
in the two calculi.  

\mypara{Constants, Values and Terms}
%
\Cref{fig:syn:terms} summarizes the 
syntax of terms in both calculi.
%
%Terms are stratified into primitive \emph{constants} 
%and \emph{values}.
%
The \emph{primitives} $\sconst$
include \tint and \tbool constants, 
boolean operations, 
the polymorphic comparison and equality, 
and their curried versions.
%
%For clarity we will write both the 
%polymorphic equality and the monomorphic 
%equality for $\tint$ as $=$ in our refinements
%and elide the type application.
%\NV{Why do we need both int and polymorphic equality? 
%Isn't inequalities also polymorphic? This is what I assumed in the overview max example}
%
\emph{Values} $\sval$ are 
%those expressions 
%which cannot be evaluated any further, 
%including primitive 
constants, binders 
and $\lambda$- and type- abstractions.
%
Finally, the \emph{terms} $\sexpr$ comprise values,
value- and type- applications, let-binders, 
annotated expressions, conditionals, and runtime errors.
The types in annotations are, potentially wrong, specifications 
written by the user and checked by the type checker.  


\mypara{Kinds \& Types}       
%
\Cref{fig:syn:types} shows the syntax of the types,
with the gray boxes indicating the extensions to $\sysf$ 
required by $\sysrf$.
%
In \sysrf, only base types % \tbool, \tint, and \tvar
can be refined: we do not permit refinements 
for functions and polymorphic types. 
%
\sysrf enforces this restriction using two kinds
which denote types that may ($\skbase$) or may not ($\skstar$)
be refined.
%
The (unrefined) \emph{base} types $\sbase$ comprise 
$\tint$, $\tbool$, and type variables $\tvar$. 
%
The simplest type is of the form
$\breft{\sbase}{\vv}{\spred}$ 
comprising a base type $\sbase$ 
and a \emph{refinement} that restricts 
$\sbase$ to the subset of values 
$\vv$ that satisfy $\spred$ \ie 
for which $\spred$ evaluates to $\ttrue$.
%
We use refined base types to build up
dependent function types (where the input 
parameter $x$ can appear in the output type's 
refinement), existential and polymorphic 
types.
%
%Note that $\sbase$ by itself denotes the trivially 
%refined type $\breft{\sbase}{\vv}{\ttrue}$.
In the sequel, we write $\sbase$ to abbreviate 
$\breft{\sbase}{\vv}{\ttrue}$
%, particularly for type variables, 
and call types refined with only \ttrue 
``trivially refined'' types. 

\mypara{Refinement Erasure}
%
The reduction semantics of our polymorphic 
primitives are defined 
using an \emph{erasure} function that 
returns the unrefined, \sysf version of a refined \sysrf type:
\[
  \forgetreft{ \breft{\sbase}{\vv}{\spred} } \defeq \sbase, \quad 
  \forgetreft{ \functype{x}{t_x}{t} } \defeq \forgetreft{t_x} \rightarrow \forgetreft{t}, \quad
  \forgetreft{ \existype{x}{t_x}{t} } \defeq \forgetreft{t}, \quad{\rm and} \quad
  \forgetreft{ \polytype{\al}{k}{t} } \defeq \polytype{\al}{k}{\forgetreft{t}}
\]


\mypara{Environments}
%
\Cref{fig:syn:env} describes 
the syntax of typing environments 
$\tcenv$ which contain both term 
variables bound to types and type 
variables bound to kinds. 
%
These variables may appear in types 
bound later in the environment.
%
In our formalism, environments grow 
from right to left.
%

\mypara{Note on Variable Representation}
Our metatheory requires that all variables 
bound in the environment are distinct. 
%
Our mechanization enforces this invariant 
via the locally nameless representation~\cite{Aydemir05}: 
free and bound variables are distinct objects 
in the syntax, as are type and term variables.
%
All free variables have unique names which
never conflict with bound variables represented as 
de Bruijn indices. This eliminates
the possibility of capture in substitution
and the need to perform alpha-renaming during
substitution. 
The locally nameless representation avoids %the need for 
technical manipulations such as index shifting by using 
names instead of indices for  free variables 
(we discuss alternatives in~\S~\ref{sec:related}).
%
To simplify the presentation of the syntax and rules, we 
use names for bound variables to make the dependent nature
of the function arrow clear.

\section{Dynamic Semantics} \label{sec:lang:dynamic}
\begin{figure}
%  {\small
  \begin{mathpar} %%%%%%%%% SMALL-STEP SEMANTICS %%%%%%%%%%
    \judgementHead{Operational Semantics}{$\sexpr \step \sexpr'$}\\
        \inferrule%*[Right=\ePrim]
        {  }
        {\app{\sconst}{\sval} \step \delta(\sconst,\sval)}
        {\ePrim} 
        \quad
        \inferrule%*[Right=\eTPrim]
        {  }
        {\tyapp{\sconst}{\stype} \step \delta_T(\sconst,\forgetreft{\stype})}
        {\eTPrim} 
        \quad
        \inferrule%*[Right=\eAnn]
        {\sexpr \step \sexpr'}{\tyann{\sexpr}{\stype} \step \tyann{\sexpr'}{\stype}}
        {\eAnn}
        \quad
        \inferrule%*[Right=\eAnnV]
        { }
        {\tyann{\sval}{\stype} \step \sval}
        {\eAnnV}
        \\
        \inferrule%*[Right=\eApp]
        {\sexpr \step \sexpr'}
        {\app{\sexpr}{\sexpr_1} \step \app{\sexpr'}{\sexpr_1}}
        {\eApp} 
        \quad
        \inferrule%*[Right=\eAppV]
        {\sexpr \step \sexpr'}
        {\app{\sval}{\sexpr} \step \app{\sval}{\sexpr'}}
        {\eAppV} 
        \quad
        \inferrule%*[Right=\eAppAbs]
        { }
          {\app{(\vabs{x}{\sexpr})}{\sval} \step \subst{\sexpr}{x}{\sval}}
          {\eAppAbs} 
          \\
          \inferrule%*[Right=\eTAppAbs]
          { }
          {\tyapp{(\tabs{\tvar}{\skind}{\sexpr})}{\stype} \step \subst{\sexpr}{\tvar}{\stype}}
          {\eTAppAbs} 
        \quad
        \inferrule%*[Right=\eTApp]
        {\sexpr \step \sexpr'}
        {\tyapp{\sexpr}{t} \step \tyapp{\sexpr'}{t}}
        {\eTApp} 
        \\
        \inferrule% *[Right=\eLet]
          { \sexpr_x \step \sexpr'_x}
          {\eletin{x}{\sexpr_x}{\sexpr} \step \eletin{x}{\sexpr'_x}{\sexpr}}
          {\eLet} 
          \quad
        \inferrule% *[Right=\eLetV]
        { }
        {\eletin{x}{\sval}{\sexpr} \step \subst{\sexpr}{x}{\sval}}
        {\eLetV} 
        \\
        \inferrule%*[Right=\eIf]
          {\sexpr \step \sexpr'}
          {\eif{\sexpr}{\sexpr_1}{\sexpr_2} \step \eif{\sexpr'}{\sexpr_1}{\sexpr_2}}
          {\eIf} 
        \\
        \inferrule% *[Right=\eIfT]
          {   }
          {\eif{\ttrue}{\sexpr_1}{\sexpr_2} \step \sexpr_1}
          {\eIfT} 
          \quad
        \inferrule% *[Right=\eIfF]
          { }
          {\eif{\tfalse}{\sexpr_1}{\sexpr_2} \step \sexpr_2}
          {\eIfF} 
        \end{mathpar}        
%    }
\caption{The small-step semantics.} 
\label{fig:e}
\label{fig:opsem}
\end{figure}


\Cref{fig:opsem} summarizes the substitution-based, 
call-by-value, contextual, small-step semantics 
for both calculi.
%
We specify the reduction semantics 
of the primitives using the functions 
$\delta$ and $\delta_T$.

\mypara{Substitution} 
%
% \NV{Put the definitions in Subst Figure one after the other, now that there is space?}
The key difference with standard formulations
is the notion of substitution for type variables 
at (polymorphic) type-application sites as shown 
in rule $\eTAppAbs$.
%
Type 
substitution is defined on the left of~\Cref{fig:type-subst},
and it is standard except 
for the last line which defines the substitution 
of a type variable $\al$ in a refined type variable 
${\breft{\al}{x}{p}}$ with a 
(potentially refined)
type $\stype_\al$.
%
To do this substitution, we combine $p$ with the type $\stype_\al$ 
by using $\strengthen{\stype_\al}{p}{x}$ 
which essentially conjoins the refinement $p$ 
to the top-level refinement of a base-kinded 
$\stype_\al$. 
%
For existential types, $\mathsf{refine}$ 
\emph{pushes} the refinement through the 
existential quantifier. 
%
Function and quantified types are left unchanged 
as they cannot instantiate a \emph{refined} 
type variable (which must be of base kind).

% Given the $\mathsf{strengthen}$ function we can define 
% \begin{equation*}
%   \subst{\breft{\al}{\vv}{p}}{\al}{\stype} \defeq \strengthen{\stype}{\subst{p}{\al}{\stype}}
% \end{equation*}

% In order to define substitution of a type variable
% by an arbitrary type in $\subst{\breft{\al}{\vv}{p}}{\al}{\stype}$, 
% we would simply write $\breft{\stype}{\vv}{\subst{p}{\al}{\stype}}$ 
% if we could stack refinements arbitrarily.
% %
% We have already covered the case of $\stype$ being a refinement type
% above. In the case of an existential type, we can 
% ``push'' the refinement $\breft{\xspace}{\vv}{\subst{p}{\al}{\stype}}$
% through the existential quantifier and define:
% \begin{equation*}
%   \subst{\breft{\al}{\vv}{p}}{\al}{\existype{z}{t_z}{t}} \defeq
%   \existype{z}{t_z}{ \subst{\breft{\al}{\vv}{p}}{\al}{t} }.
% \end{equation*}
% %
% These are the only two cases we need to elaborate: $\stype$ must have
% base kind unless $\al$ is unrefined (\ie, $p = \ttrue$), so
% $\stype$ cannot be a function type or a polymorphic type in a valid
% substitution into a refined type variable. 
% %
% For an unrefined type variable we can define simply
% $\subst{\al}{\al}{t} \defeq t$.

% To facilitate a unified definition for the substitution of a
% type variable by an arbitrary type, we define an auxiliary 
% function $\mathsf{strengthen}$ that describes how we ``push''
% a predicate into a base kinded type:
% \begin{align*}
%   \strengthen{\breft{\al}{z}{q}}{p}    & \defeq \breft{\al}{z}{p \wedge q} \\
%   \strengthen{\functype{x}{t_x}{t}}{p} & \defeq \functype{x}{t_x}{t} \\
%   \strengthen{\existype{z}{t_z}{t}}{p} & \defeq \existype{z}{t_z}{\strengthen{t}{p}} \\
%   \strengthen{\polytype{\al}{k}{t}}{p} & \defeq \polytype{\al}{k}{t}
% \end{align*}
% Given the $\mathsf{strengthen}$ function we can define 
% \begin{equation*}
%   \subst{\breft{\al}{\vv}{p}}{\al}{\stype} \defeq \strengthen{\stype}{\subst{p}{\al}{\stype}}
% \end{equation*}
% % \RJ{define and use a SUBST macro}
% % \RJ{i cannot even understand defs below}



\begin{figure}%[t!]
  %{\small
  %\begin{array}{r@{\hskip 0.05in}c@{\hskip 0.05in}l}
  \begin{tabular}{r@{\hskip 0.05in}c@{\hskip 0.05in}l}
  $\subst{\breft{\tvarb}{x}{p}}{\al}{t_\al}$       & $\defeq$ & $\breft{\tvarb}{x}{\subst{p}{\al}{t_\al}}, \al \not = \tvarb$ \\ 
  $\subst{(\functype{x}{t_x}{t})}{\al}{t_\al}$     & $\defeq$ & $\functype{x}{(\subst{t_x}{\al}{t_\al})}{\subst{t}{\al}{t_\al}}$ \\ 
  $\subst{(\existype{x}{t_x}{t})}{\al}{t_\al}$     & $\defeq$ & $\existype{x}{(\subst{t_x}{\al}{t_\al})}{\subst{t}{\al}{t_\al}}$ \\ 
  $\subst{(\polytype{\tvarb}{k}{t})}{\al}{t_\al}$  & $\defeq$ & $\polytype{\tvarb}{k}{\subst{t}{\al}{t_\al}}$ \\
  $\subst{\breft{\al}{x}{p}}{\al}{t_\al}$          & $\defeq$ & $\strengthen{t_\al}{\subst{p}{\al}{t_\al}}{x}$ \\
  \end{tabular}
  \begin{tabular}{r@{\hskip 0.05in}c@{\hskip 0.05in}l}
  $\strengthen{\breft{\al}{z}{q}}{p}{x}$           & $\defeq$ & $\breft{\al}{z}{\subst{p}{x}{z} \wedge q}$ \\
  $\strengthen{\existype{z}{t_z}{t}}{p}{x}$        & $\defeq$ & $\existype{z}{t_z}{\strengthen{t}{p}{x}}$ \\
  $\strengthen{\functype{x}{t_x}{t}}{\_}{\_}$      & $\defeq$ & $\functype{x}{t_x}{t}$ \\
  $\strengthen{\polytype{\al}{k}{t}}{\_}{\_}$      & $\defeq$ & $\polytype{\al}{k}{t}$ \\
  && \\
  %\end{array}
  \end{tabular}
  %}
  % $$\begin{array}{rcl}
  % \subst{\breft{\tvarb}{x}{p}}{\al}{t_\al}       & \defeq & \breft{\tvarb}{x}{\subst{p}{\al}{t_\al}} \\ 
  % \subst{(\functype{x}{t_x}{t})}{\al}{t_\al}     & \defeq & \functype{x}{(\subst{t_x}{\al}{t_\al})}{\subst{t}{\al}{t_\al}} \\
  % \subst{(\existype{x}{t_x}{t})}{\al}{t_\al}     & \defeq & \existype{x}{(\subst{t_x}{\al}{t_\al})}{\subst{t}{\al}{t_\al}} \\
  % \subst{(\polytype{\tvarb}{k}{t})}{\al}{t_\al}  & \defeq & \polytype{\tvarb}{k}{\subst{t}{\al}{t_\al}} \\
  % \subst{\breft{\al}{\vv}{p}}{\al}{\stype}       & \defeq & \strengthen{\stype}{\subst{p}{\al}{\stype}} \\ [0.05in] 
  % \strengthen{\breft{\al}{z}{q}}{p}              & \defeq & \breft{\al}{z}{p \wedge q} \\
  % \strengthen{\existype{z}{t_z}{t}}{p}           & \defeq & \existype{z}{t_z}{\strengthen{t}{p}} \\
  % \strengthen{\functype{x}{t_x}{t}}{p}           & \defeq & \functype{x}{t_x}{t} \\
  % \strengthen{\polytype{\al}{k}{t}}{p}           & \defeq & \polytype{\al}{k}{t} \\
  % \end{array}$$
  \vspace{-0.00cm}
  \caption{Type substitution and refinement strengthening.} 
  \label{fig:type-subst}
  %\label{fig:e}
  %\label{fig:opsem}
  %\vspace{-0.00cm}
  \end{figure}
  

\mypara{Primitives}
%
The function $\delta(\sconst, \sval)$ 
evaluates the application $\app{\sconst}{\sval}$ 
of built-in monomorphic primitives.
%
The reductions are defined in a curried 
manner, \ie 
$\app{\app{\leq}{m}}{n}$ evaluates to $\delta(\delta(\leq,m),n)$. 
%we have that $\app{\app{\leq}{m}}{n} \steps \delta(\delta(\leq,m),n)$. 
%
Currying gives us unary relations like $m\!\!\leq$ 
which is a partially evaluated version of the $\leq$ relation.
%
The function $\delta_T(\sconst, \forgetreft{\stype})$
specifies the reduction rules for type 
application on the polymorphic 
built-in primitives. % $=$ and $\leq$.
%
$$\begin{array}{rclrclrcl}
\delta(\wedge,{\tt true}) & \defeq & \lambda x.\, x &
\delta(\leq,m) & \defeq & m\!\!\leq  & 
    \delta_T(=, \tbool) & \defeq & =  \\
\delta(\wedge,{\tt false}) & \defeq & \lambda x.\, {\tt false}\quad\quad &
\delta(m\!\!\leq, n) & \defeq & {\tt}(m \leq n) \quad\quad&
\delta_T(=, \tint) & \defeq & = \\
\delta(\neg,{\tt true}) & \defeq & {\tt false} & 
\delta(=,m) & \defeq & m\!\!= &
\delta_T(\leq, \tbool) & \defeq & \leq  \\
  \delta(\neg,{\tt false}) & \defeq &  {\tt true} &
  \delta(m\!\!=, n) & \defeq &  {\tt}(m = n) &
  \delta_T(\leq, \tint) & \defeq & \leq  \\
%
%  \delta(\vee,{\tt true}) & \defeq & \lambda x.\, {\tt true} &
%  \delta(\leftrightarrow,{\tt true}) & \defeq & \lambda x.\, x & 
%    &  & \\
%    \delta(\vee,{\tt false}) & \defeq & \lambda x.\, x &
%    \delta(\leftrightarrow,{\tt false}) & \defeq & \lambda x.\, \neg x &
%    & & \\
\end{array}$$
         
% SAFETY --> Progress --> + Prim
%                         + Canonical

%        --> Preserve --> + Determinism
%                         + Substitution


  
\mypara{Determinism}
%
Our soundness proof  uses the determinism 
property of the operational semantics.
%
\begin{lemma}[Determinism]\label{lem:step-determ}
For every expression $\sexpr$, 
%\begin{itemize} 
    %\item 
    1) there exists at most one term $\sexpr'$ \suchthat $\sexpr \step \sexpr'$, 
    %\item 
    2) there exists at most one value $\sval$ \suchthat $\evalsTo{\sexpr}{\sval}$, and
    %\item 
    3) if $\sexpr$ is a value there is no term $\sexpr'$ \suchthat $\sexpr \step \sexpr'$.
%\end{itemize}
\end{lemma}    
