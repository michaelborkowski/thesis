\chapter{Lists: The Language \sysrfd}
\label{ch:lists}

One of the key features of GHC's core calculus
missing from $\sysrf$ is data types. The addition of
data types would also enable us to replace the ad hoc
kind system of $\sysrf$ with a more versatile system
of type classes.
%
As a first step in this direction, we augment our
calculus with polymorphic refined list types.
We also add a measure @length@ that describes the 
length of our lists.

\section{Syntax and Semantics} \label{sec:lang:syntaxD}

We present the syntax and semantics of \sysrfd in terms 
of the additions to \sysrf. The reader may refer to 
the figures in Chapter \ref{ch:language} for the syntactic
forms and rules that are inherited from \sysrf.
%
As before, we use the $\greybox{\mbox{gray}}$ to highlight
the extensions to $\sysf$ needed to support refinements 
in $\sysrfd$.

\begin{figure}%[t!]
%  \scalebox{0.80}{
    \begin{tabular}{rrcll}
\emphbf{Primitives} 
  & \sconst & $\bnfdef$ & $\cdots$    & \\ %\emph{booleans and integers} \\
  &         & $\spmid$  & $\suc$     & \emph{integer ops.} \\
  &         & $\spmid$  & $\len$      & \emph{polymorphic list ops} \\ [0.05in] 

\emphbf{List Values}
  & \slval  & $\bnfdef$ & $\nil{\stype}$  & \emph{empty list} \\
  &         & $\spmid$  & $\cons{\stype}{\sval}{\slval}$  
                                      & \emph{list constructor} \\ [0.05in]
\emphbf{Values}
  & \sval   & $\bnfdef$ & $\cdots$               & \\ % \emph{primitives} \\ 
  &         & $\spmid$  & \slval                & \emph{list values} \\[0.05in]

\emphbf{Terms}
  & \sexpr  & $\bnfdef$ & $\cdots$              & \\ %\emph{values} \\ 
  &         & $\spmid$  & \cons{\stype}{e_1}{e_2} & \emph{list constructor} \\
  &         & $\spmid$  & \eswitch{e}{e_n}{e_c} & \emph{list destructor} \\
\end{tabular}
%  }
  \vspace{-0.0cm}
  \caption{Syntax of Primitives, Values, and Expressions.}
\label{fig:syn:termsD}
\vspace{-0.0cm}
\end{figure}


\begin{figure}%[b!]
%{\small
  \begin{tabular}{rrcll}

  \emphbf{Types}
   & \stype & $\bnfdef$ & $\cdots$              & \\ %\emph{values} \\ 
   &        & $\spmid$  & $\listtype{\stype}\greybox{\!\breft{}{\vv}{\spred}}$  & \emph{\greytextbox{refined} list type} \\ 
  \end{tabular}
%}
\vspace{-0.0cm}
  \caption{Syntax of Types.
           The gray boxes are the extensions 
           to $\sysf$ needed by $\sysrfd$.}
           %We use $\sftype$ for $\sysf$-only types.}
  \label{fig:syn:typesD}
  \vspace{-0.00cm}
\end{figure}

\mypara{Constants, Values, and Terms}
%
\Cref{fig:syn:termsD} summarizes the syntax of terms in both 
\sysrfd and in the unrefined calculus.
%
The \emph{primitives} $\sconst$ now
include a $\suc$ function on integers that adds one
and a (polymorphic) $\len$ measure that computes the length of any list.
%
Although a user could easily define $\len$ using
\texttt{switch} below, we add $\len$ as a built-in primitive
in order to use it in our typing judgments.
%
Our \emph{terms} $\sexpr$ now additionally contain two list constructors:
the empty list $\nil{\stype}$ and the non-empty $\cons{\stype}{\sexpr_1}{\sexpr_2}$,
which builds a list from a head element and a tail.
%
Both of these constructors take a refined type annotation. For instance, we should
be able to type check
\[
\cons{\breft{\tint}{\vv}{\vv\geq 0}}{1}({\cons{\breft{\tint}{\vv}{\vv > 0}}{2}{3}})
\]
but not
\[
\cons{\breft{\tint}{\vv}{\vv > 0}}{1}({\cons{\breft{\tint}{\vv}{\vv\geq 0}}{2}{3}})
\]

because the tail of the latter list is only known to consist of non-negative integers.
%
Although we mechanized the metatheory in the same manner without 
type annotations as well, these annotations are needed to enable future work on
a bidirectional type checking algorithm (Chapter \ref{ch:future}).
%
The terms also now contain a list destructor 
$\eswitch{\sexpr}{\sexpr_n}{\sexpr_c}$, which
case splits on the shape of the match scrutinee $\sexpr$.
%
Finally, \emph{values} $\sval$ are augmented by lists that contain only
values as elements; these list values are defined inductively 
in Figure \ref{fig:syn:termsD}.

\mypara{Kinds \& Types}       
%
\Cref{fig:syn:typesD} shows the syntax of the types,
with the gray boxes indicating the extensions to $\sysf$ 
required by $\sysrfd$.
%
In contrast \sysrf, our list types are not base types,
but they can be refined.
%
Both of these are key to data types: our lists may contain
incomparable data such as lambda abstractions, and so they
cannot support the polymorphic comparison operators within
our simple kind system.
%
However, refinements on lists are key to any model of
data types. We must be able to express the type of a program 
such as the following safe tail function:
\begin{code}
  tail :: forall a. {v:[a] | length v > 0 } -> [a]
  tail xs = switch (xs) error (\y ys -> ys)
\end{code}
%
\sysrfd keeps the simple kind system from \sysrf and 
enforces list types as $\skstar$-kinded to prevent the substitution
of a list type for a refined type variable. 
%
This prevents a refinement of a list type from attempting
to compare a list using one of the polymorphic list operators.


\mypara{Dynamic Semantics} %\label{sec:lang:dynamic}
\Cref{fig:opsemD} summarizes the small-step semantics 
for both calculi.

\begin{figure}
%  {\small
  \begin{mathpar} %%%%%%%%% SMALL-STEP SEMANTICS %%%%%%%%%%
    \judgementHead{Operational Semantics (ext. Figure \ref{fig:e})}{$\sexpr \step \sexpr'$}\\
        \inferrule%*[Right=\eCons]
        {\sexpr \step \sexpr'}
        {\cons{\stype}{\sexpr}{\sexpr_1} \step \cons{\stype}{\sexpr'}{\sexpr_1}}
        {\eCons} 
        \quad
        \inferrule%*[Right=\eConsV]
        {\sexpr \step \sexpr'}
        {\cons{\stype}{\sval}{\sexpr} \step \cons{\stype}{\sval}{\sexpr'}}
        {\eConsV} 
        \\
        \inferrule%*[Right=\eSwitch]
          {\sexpr \step \sexpr'}
          {\eswitch{\sexpr}{\sexpr_n}{\sexpr_c} \step \eswitch{\sexpr'}{\sexpr_n}{\sexpr_c}}
          {\eSwitch} 
        \quad
        \inferrule% *[Right=\eSwitchN]
          {   }
          {\eswitch{\nil{\stype}}{\sexpr_n}{\sexpr_c} \step \sexpr_n}
          {\eSwitchN} 
          \\
        \inferrule% *[Right=\eSwitchC]
          { }
          {\eswitch{\cons{\stype}{\sval_1}{\sval_2}}{\sexpr_n}{\sexpr_c} 
              \step \app{(\app{\sexpr_c}{\sval_1})}{\sval_2}}
          {\eSwitchC} 
        \end{mathpar}        
%    }
\caption{The small-step semantics for \sysrfd.} 
\label{fig:eD}
\label{fig:opsemD}
\end{figure}


%\mypara{Primitives}
%%
%The function $\delta(\sconst, \sval)$ 
%evaluates the application $\app{\sconst}{\sval}$ 
%of built-in monomorphic primitives.
%%
%The reductions are defined in a curried 
%manner, \ie 
%$\app{\app{\leq}{m}}{n}$ evaluates to $\delta(\delta(\leq,m),n)$. 
%%we have that $\app{\app{\leq}{m}}{n} \steps \delta(\delta(\leq,m),n)$. 
%%
%Currying gives us unary relations like $m\!\!\leq$ 
%which is a partially evaluated version of the $\leq$ relation.
%%
%The function $\delta_T(\sconst, \forgetreft{\stype})$
%specifies the reduction rules for type 
%application on the polymorphic 
%built-in primitives. % $=$ and $\leq$.
%%
%$$\begin{array}{rclrclrcl}
%\delta(\wedge,{\tt true}) & \defeq & \lambda x.\, x &
%\delta(\leq,m) & \defeq & m\!\!\leq  & 
%    \delta_T(=, \tbool) & \defeq & =  \\
%\delta(\wedge,{\tt false}) & \defeq & \lambda x.\, {\tt false}\quad\quad &
%\delta(m\!\!\leq, n) & \defeq & {\tt}(m \leq n) \quad\quad&
%\delta_T(=, \tint) & \defeq & = \\
%\delta(\neg,{\tt true}) & \defeq & {\tt false} & 
%\delta(=,m) & \defeq & m\!\!= &
%\delta_T(\leq, \tbool) & \defeq & \leq  \\
%  \delta(\neg,{\tt false}) & \defeq &  {\tt true} &
%  \delta(m\!\!=, n) & \defeq &  {\tt}(m = n) &
%  \delta_T(\leq, \tint) & \defeq & \leq  \\
%%
%%  \delta(\vee,{\tt true}) & \defeq & \lambda x.\, {\tt true} &
%%  \delta(\leftrightarrow,{\tt true}) & \defeq & \lambda x.\, x & 
%%    &  & \\
%%    \delta(\vee,{\tt false}) & \defeq & \lambda x.\, x &
%%    \delta(\leftrightarrow,{\tt false}) & \defeq & \lambda x.\, \neg x &
%%    & & \\
%\end{array}$$
         
\mypara{Typing and Well-formedness}
%
Next, we present the static semantics of \sysrfd by describing
the additional rules used to establish our
well-formedness, typing, and subtyping judgments.
%%
%We use $\greybox{\mbox{grey}}$ to highlight the antecedents and rules
%specific to $\sysrf$.
%%
%
\Cref{fig:wfD} summarizes the new rules
that establish the well-formedness of types.
%
Rule \wtList states that a list type 
$\breft{\tlist{\stype}}{x}{\ttrue}$
with empty refinement
is well-formed with star kind provided that 
$\stype$ is well-formed with some kind $\skind$.
%
Similar to rule \wtRefn, our rule \wtListR 
stipulates that a refined list type $\breft{\tlist{\stype}}{x}{\spred}$
is well-formed with star kind in some environment
if the trivially refined type $\breft{\tlist{\stype}}{x}{\ttrue}$
has star kind in the same environment and if
the refinement predicate $\spred$ has type $\tbool$
in the environment augmented by binding a fresh variable to type $\tlist{\stype}$.

\begin{figure}%[t!]
%
%{\small
\begin{mathpar}
\judgementHead{Well-formed Type (ext. Figure \ref{fig:wf})}{\isWellFormed{\tcenv}{\stype}{\skind}}

%%%%%%%%%%%% WELL-FORMEDNESS %%%%%%%%%%%%%
%
\inferrule% *[Right=\wtList]
    {\isWellFormed{\tcenv}{\stype}{\skind}}
    {\isWellFormed{\tcenv}{\tlist{\stype}\greybox{\!\breft{}{x}{\ttrue}}\!}{\skstar}}
    {\wtList}
\quad
%    
\greybox{\inferrule% *[Right=\;\;\wtListR]
    { \isWellFormed{\tcenv}{\breft{\tlist{\stype}}{x}{\ttrue}}{\skstar} \\\\
      \forall\notmem{y}{\tcenv}.
      \hasftype{\bind{y}{\tlist{\stype}}, \forgetreft{\tcenv}}{\subst{p}{x}{y}}{\tbool}
    }
    {\isWellFormed{\tcenv}{\breft{\tlist{\stype}}{x}{p}}{\skstar}}
    {\wtListR} }
%
\end{mathpar}
%}
\vspace{-0.00cm}
\caption{Well-formedness of $\sysrfd$ types. The rules for
  $\sysf$ exclude the gray boxes.}
\label{fig:wfD}
\vspace{-0.00cm}
\end{figure}

%The judgment $\hastype{\tcenv}{\sexpr}{\stype}$ states
%that the term $\sexpr$ has type $\stype$ in the context of
%environment $\tcenv$.
\Cref{fig:typingD} summarizes
the rules that establish typing for both $\sysf$ and
$\sysrf$, with gray %boxes
%indicating extensions needed
for the $\sysrf$ extensions.
%
Rule $\tNil$ states that whenever $\stype$ is a well-formed
type in some environment, then  $\nil{\stype}$ has the type
$\breft{\tlist{\stype}}{x}{\len x = 0}$ of lists of elements
of type $\stype$ of length zero.
%
The rule $\tCons$ is slightly more complex: whenever $\sexpr_h$
can be given type $\stype$ in some environment, and whenever
$\sexpr_t$ can be given type
$\breft{\tlist{\stype}}{x}{\spred}$ in the same environment,
then the list $\cons{\stype}{\sexpr_h}{\sexpr_t}$ can be given
the type $\tlist{\stype}$ with the refinement that says that
this list has length one more than some list of type
$\breft{\tlist{\stype}}{x}{\spred}$.
%
The purpose of this refinement is to embed the information about
the specific length of a list at the refinement level.
%
Our typing rule for the list destructor 
$\eswitch{\sexpr}{\sexpr_n}{\sexpr_c}$ is 
best thought of as analogous to our rule $\tIf$
because it enables path-sensitive reasoning about lists.
%
This rule $\tSwitch$ says that whenever the match scrutinee
$\sexpr$ can be given a list type 
$\breft{\tlist{\stype}}{x}{\spred}$ in some environment,
whenever $\stype'$ is well-typed in the same environment,
and whenever each branch can be given this type
$\stype'$ in this environment augmented by the knowledge 
that the about the scrutinee and its length, then the full
term $\eswitch{\sexpr}{\sexpr_n}{\sexpr_c}$ can be given type
$\stype'$. 
%
Note that, per the semantics in \ref{fig:eD}, the cons-branch
$\sexpr_c$ is a function expecting two arguments: the head
of the scrutinee and the tail (which we know has length
one less than the scrutinee).
%
In our formalism, it is necessary for us to augment the 
environment with \emph{two} dummy variables here. We need
to preserve $\spred$,
the knowledge obtained from the match scrutinee,
and the fact that the scrutinee is one element longer than
the tail to which $\sexpr_c$ is being applied.
%
We could express the antecedent judgment as
\begin{equation}
  \label{badswitch}
\forall\notmem{z}{\tcenv}. 
  \hastype{\bind{z}{\breft{\tlist{\stype}}{x}{p},\tcenv}}
     {\sexpr_c}
     {\funcftype{\stype}{\funcftype{\tlist{\stype}\greybox{\breft{}{\vv}{\suc \len \vv = \len z}}}{\stype'}}},
\end{equation}
but this would pose a problem for proving type soundness 
for \sysrfd (specifically preservation)
from a minimal interface of axioms (\ref{lem:implicationD}) 
for implication.
%
These axioms are purely syntactic, and so are not sufficient
to prove implications (and hence subtyping obligations) 
where there are slight variations in refinement syntax. 
%
Rule \tCons gives us a refinement that says the length of this list
is one \emph{longer} than some other list;
judgment \ref{badswitch} would require us to show that the second
argument to $\sexpr_c$ has length one \emph{shorter} than some other 
list. There is no way to derive this knowledge without either a 
semantic notion of entailment, or a large, unwieldy set of axioms,
or by changing the form of rule \tSwitch as we did.


\begin{figure}
%  {\small
  \begin{mathpar}             %%%%%%%%%%%%% TYPING %%%%%%%%%%%%%%%%%%
  \judgementHead{Typing (ext. Figure \ref{fig:t})}{\hastype{\tcenv}{\sexpr}{\stype}} \\

    %
        \inferrule
        {\isWellFormed{\tcenv}{\stype}{\skind}}
        {\hastype{\tcenv}{\nil{\stype}}{\tlist{\stype}\greybox{\breft{}{x}{\len x = 0}}}}
        {\tNil}
    %
    \\  
    %
        \inferrule
        {\hastype{\tcenv}{\sexpr_h}{\stype} \\
         \hastype{\tcenv}{\sexpr_t}{\tlist{\stype}\greybox{\breft{}{x}{\spred}}}
        }
        {\hastype{\tcenv}{\cons{\stype}{\sexpr_h}{\sexpr_t}}
                 {\greybox{\existype{y}{\tlist{\stype}\greybox{\breft{}{x}{\spred}}}}
                  \tlist{\stype}\greybox{\breft{}{\vv}{\len \vv = \suc \len y}}}}
        {\tCons}
    \\    
    %
    {\small
        \inferrule
        {\hastype{\tcenv}{\sexpr}{\tlist{\stype}\greybox{\breft{}{x}{\spred}}} \\
         \isWellFormed{\tcenv}{\stype'}{\skind} \\\\
          %
         \greybox{\forall\notmem{y}{\tcenv}.}
         \hastype{\greybox{\bind{y}{\breft{\tlist{\stype}}{x}{p \wedge \len x = 0}},}\tcenv}
             {\sexpr_n}{\stype'}\\\\
          %
         \greybox{\forall\notmem{y, z}{\tcenv}.}
         \hastype{\greybox{\bind{z}{\breft{\tlist{\stype}}{x}{p \wedge \suc \len y = \len x}},
                          \bind{y}{\tlist{\stype}},}\tcenv}
            {\sexpr_c}{\funcftype{\stype}{\funcftype{\tlist{\stype}\greybox{\breft{}{\vv}{\len y = \len \vv}}}{\stype'}}}
        }
        {\hastype{\tcenv}{\eswitch{\sexpr}{\sexpr_n}{\sexpr_c}}{\stype'}}
        {\tSwitch} 
    }       
    \end{mathpar}
%  }
\vspace{-0.00cm}
\caption{Typing rules.
The judgment $\hasftype{\tcenv}{\sexpr}{\sftype}$ is extended by excluding the gray boxes.}
\label{fig:tD}\label{fig:typingD}
\vspace{-0.00cm}
\end{figure}

\mypara{New Primitives}
%
The function $\ty{\sconst}$, which gives the type of every
built-in primitives, is extended for the new primitives
$\suc$ and $\len$. Below we present essential
examples of the $\ty{\sconst}$ definition.
%
{\small
$$\begin{array}{rcl}
\ty{\suc} & \defeq & \functype{y}{\tint}{\breft{\tint}{v}{v = (\suc y)}} \\
\ty{\len} & \defeq & \polytype{\tvar}{\skstar}{\functype{y}{\al}{\breft{\tint}{v}{v = (\len y)}}}
\end{array}$$
}
For ease of reading, the $\len$ used in the refinements 
is the polymorphic $\len$,
but with the type applications elided.


\subsection{Subtyping}
\label{sec:typing:subD}

\Cref{fig:s} presents the new rule to establish the 
subtyping judgment ${\isSubType{\tcenv}{s}{t}}$.  
%
Rule \sList states that one list type
$\breft{\listtype{\stype_1}}{\vv}{\spred_1}$
is a subtype of another list type
$\breft{\listtype{\stype_2}}{\vv}{\spred_2}$
in some environment $\tcenv$, when
$\stype_1$ is a subtype of $\stype_2$ and 
$\spred_1$ implies $\spred_2$
in the
environment $\tcenv$ augmented  by binding
a fresh type variable to kind $\skind$.

\begin{figure}
\judgementHead{Subtyping}{\isSubType{\tcenv}{s}{t}}

\begin{mathpar}   %%%%%%%%%%%%%%%%%% SUBTYPING %%%%%%%%%%%%%%%%%%
%
  \inferrule*[Right=\sList]
  {\isSubType{\tcenv}{\stype_1}{\stype_2} \quad
    \forall\notmem{y}{\tcenv}. \;\;
    \imply{\bind{y}{\listtype{\stype_1}\{\ttrue\}},\tcenv}{\subst{p_1}{x}{y}}{\subst{p_2}{x}{y}} }
  {\isSubType{\tcenv}{\breft{\listtype{\stype_1}}{x}{\spred_1}}
                     {\breft{\listtype{\stype_2}}{x}{\spred_2}}}
\end{mathpar}
\vspace{-0.00cm}
\caption{Subtyping Rules.}
\label{fig:sD}
\label{fig:subtypingD}
%\label{fig:ent}
\vspace{-0.00cm}
\end{figure}

\mypara{Implication}
In \S~\ref{sec:typing:implication} we discussed our approach to formalizing
implication by giving both an axiomatized interface and a 
denotational implementation. 
%
First, we give the additional axioms for implication in \sysrfd.


\begin{requirement}[Implication Interface]\label{lem:implicationD}
  The implication relation satisfies the statements
  in Requirement~\ref{lem:implication} and the statements below:
  \begin{enumerate}
      \item (Exact Quantification) If
      $\hastype{\tcenv}{v_x}{t_x}$ and
      $\isWellFormed{\tcenv}{t_x}{\skbase}$ and 
      $\bind{x}{\self{t_x}{v_x}{\skbase}} \in \tcenv$ 
      and $x\not\in \free{v_x}$ then
      $\imply{\tcenv}{\spred}{\subst{\spred}{x}{v_x}}$.
      %
      \item (Equal Length Quantification) If
      $\hastype{\tcenv}{\sval}{\breft{\listtype{\stype}}{x}{\spred}}$ and
      $\isWellFormed{\tcenv}{\breft{\listtype{\stype}}{x}{\spred}}{\skstar}$ and \\
      $\bind{x}{\breft{\listtype{\stype}}{x}{\len x = \len \sval \wedge \spred}} \in \tcenv$ 
      and $x\not\in \free{\sval}$ and \safeListVar{x}{q} then
      $\imply{\tcenv}{q}{\subst{q}{x}{\sval}}$\\
      and
      $\imply{\tcenv}{\subst{q}{x}{\sval}}{q}$.
  \end{enumerate}
\end{requirement}

\noindent
The first statement says that whenever 
$\bind{x}{\self{t_x}{v_x}{\skbase}}$ appears 
bound in $\tcenv$, then $x$ is effectively being
universally quantified over a type with just one
inhabitant (up to equality). Then we assume 
that a refinement $\spred$ implies $\spred$ with
all occurrences of $x$ replaced by $v_x$.
%
This is semantically valid because any encoding of 
our refinements in an external logic (such as SMT)
would map $=$ in our refinement syntax to equality.
%
And we also prove that under the denotational definition of
implication (\S~\ref{sec:typing:implication:denotational})
our syntactic primitive $=$ actually corresponds to semantic
equality, and we are thus able to show that this axiom
follows from the denotational definition.

The second statement, which we call Equal Length Quantification,
expresses the best possible analogue for lists. We cannot
appeal to Exact Quantification for any variable $x$ bound to
a list type in the environment; these types cannot be
selfified because lists cannot be compared, even for equality,
in our system.
%
However, the refinements that do occur in our system (by virtue
of appearing in our various rules) do not use the 
$\texttt{switch}$ statement to destruct lists to inspect
their contents. Rather, these refinements are entirely agnostic
about the data contained in lists and are only concerned with
the length of lists.
%
We capture this notion in the recursive function 
$\safeListVar{x}{q}$ which states that $x$ only appears 
in $q$ as the argument to the $\len$ function.
% 
If this is the case, then the only information that $q$ uses 
about $x$ is the length. The Exact Length Quantification
axiom then says that if $x$ is bound to a type that constrains
the length of $x$ to be equal to the length of a list $v$, 
then the refinement $q$ is equivalent (under implication)
to $\subst{q}{x}{v}$.

Previously, we noted in \S~\ref{sec:typing:implication:interface}
that we did not require the Exact Quantification axiom in
Assumption 1 of~\cite{Knowles09} to formalize implication in \sysrf.
% 
However, we do require it for \sysrfd in our mechanization 
to verify Requirement~\ref{lem:prim-typing} as it applies to the
length of lists.

\subsection{Denotational Semantics}
\label{sec:lists:denotational}

We extend the definition of denotations of types given
in Figure~\ref{fig:den} to define the denotation of a list type.
%
In order to define the denotation 
$\denote{\breft{\listtype{\stype}}{x}{p}}$
as the set of closed values
of the appropriate base type $\listtype{\forgetreft{t}}$
which satisfy the type's refinement predicate,
we have to take into account that there are at least two
refinements here. The entire list $\sval$ must satisfy
the refinement $\spred$, and additionally the type
$\stype$ itself may contain refinements which must
be satisfied by each of the elements of $\sval$.

\begin{figure}
$$\begin{array}{r@{\hskip 0.03in}c@{\hskip 0.03in}l}
\denote{\breft{\listtype{\stype}}{x}{p}} & \defeq &
  \setcomp{\sval}{\hasftype{\varnothing}{\sval}{\listtype{\forgetreft{\stype}}} 
  \,\wedge\, (\forall\, v_i\in v.\, v_i \in \denote{t})
  \,\wedge\, \evalsTo{\subst{p}{x}{\sval}}{\ttrue}}.
\end{array}$$
\vspace{-0.0cm}
\caption{Denotations of Types and Environments.}
\label{fig:denD}
\vspace{-0.0cm}
\end{figure}

\section{Metatheory of Lists}

Our proof of type soundness of $\sysrfd$ generally follows
the same structure as the soundness proof for $\sysrf$,
illustrated in Figure~\ref{fig:graph}.
%
However, some lemmas have additional cases that are highly 
non-trivial and add additional difficulty to the proof. 
%
For instance, we need to be able to invert typing judgments
such as 
$\hastype{\tcenv}{\cons{\stype}{\sexpr_h}{\sexpr_t}}{\stype'}$
both to obtain typing judgments for $\sexpr_h$ and $\sexpr_t$
and also to obtain a typing judgment that contains the 
knowledge about the length of $\cons{\stype}{\sexpr_h}{\sexpr_t}$,
which may have been discarded through subsumption.

We give the highlights for the metatheory of \sysrfd.


\mypara{Inversion of Typing Judgments}
\label{sec:soundnessD:inversion}
%
In \S~\ref{sec:soundness:inversion} we discussed Inversion Lemma
\ref{lem:inversion}, which allowed us to invert the typing judgment
for a term- or type-abstraction. We need to extend this 
lemma to include inverting typing judgments for lists.
%
We can then recover typing information about the head and tail 
of this list as well as information about the length of the list that
may have been lost through use of the subsumption \tSub rule.

\begin{lemma} (Inversion of $\tNil$, $\tCons$) \label{lem:inversionD} 
    (extends Lemma~\ref{lem:inversion})
    \begin{enumerate}
    \item If $\hastype{\tcenv}{\nil{\stype_0}}{\breft{\listtype{\stype}}{x}{\spred}}$
    and $\isWellFormedE{\tcenv}$,
    then $\isSubType{\tcenv}{\stype_0}{\stype}$, and
    $\isWellFormed{\tcenv}{\stype_0}{\skstar}$,\\
    and $\isSubType{\tcenv}{\breft{\listtype{\stype_0}}{x}{\len x = 0}}
          {\breft{\listtype{\stype}}{x}{\spred}}$.

    \item If $\hastype{\tcenv}{\cons{\stype_0}{v_1}{v_2}}
                  {\breft{\listtype{\stype}}{x}{\spred}}$
    and $\isWellFormedE{\tcenv}$,
    then 
    $\isSubType{\tcenv}{\stype_0}{\stype}$,
    $\isWellFormed{\tcenv}{\stype_0}{\skstar}$,
    $\hastype{\tcenv}{v_1}{\stype_0}$
    and for some refinement $q$,
    $\hastype{\tcenv}{v_2}{\breft{\listtype{\stype_0}}{x}{q}}$
    and $\isSubType{\tcenv}
          {\existype{y}{\breft{\listtype{\stype}}{x}{q}}
                    {\breft{\listtype{\stype_0}}{x}{\len x = \suc \len y}} }
          {\breft{\listtype{\stype}}{x}{\spred}}$.

  \end{enumerate}                    

\end{lemma}
%
The second statement above is important because if
$\hastype{\tcenv}{\cons{\stype_0}{v_1}{v_2}}
                  {\breft{\listtype{\stype}}{x}{\spred}}$
then we cannot get directly to a typing judgment for $v_1$
or for $v_2$. Indeed, rule \tCons gives us an existentially
quantified list of type $\listtype{\stype_0}$ with a refinement
relating the length of the list to its tail; but here $\stype$
may not even be the same as $\stype_0$, so the derivation tree
must have used one or more applications of rule \tSub.
%
The proof goes by induction on the size of the derivation tree,
 which must be finite.   

\mypara{Exact Typing}
Although we cannot apply the exact typing lemma to lists (only 
base types can be selfified), we can derive an analogous
statement in terms of equality of length.
%
whenever we can type
a list value $v$ at type $\breft{\listtype{t}}{x}{p}$ then we
also type $v$ at the type $\breft{\listtype{t}}{x}{p}$
with the refinement strengthened by $\len x = \len v$.
%
\begin{lemma} (Equal Length in Typing, compare to Lemma~\ref{lem:exact}) 
\label{lem:equalLength}
%\begin{enumerate}
  %\item If $\hastype{\tcenv}{e}{t}$, $\isWellFormedE{\tcenv}$, $\isWellFormed{\tcenv}{t}{k}$, and $\isSubType{\tcenv}{s}{t}$, then $\isSubType{\tcenv}{{\rm self}(s, v, k)}{{\rm self}(t, v, k)}$.
  %\item 
  If $\hastype{\tcenv}{v}{\breft{\listtype{t}}{x}{p}}$ and $\isWellFormedE{\tcenv}$
  then $\hastype{\tcenv}{v}{\breft{\listtype{t}}{x}{\len x = \len v \wedge p}}$. 
%\end{enumerate}
\end{lemma}


\begin{comment}

\mypara{Denotational Soundness}
\label{sec:soundness:denotationalD}
\label{sec:denot:soundnessD}

Our statement of denotational soundness remains the same as
Theorem~\ref{lem:denote-sound-first}.
%
Here we emphasize that 



\begin{lemma}\label{denote-selfification} (Selfified Denotations)
  If $\isWellFormed{\varnothing}{\stype}{\skind}$,
     $\hastype{\varnothing}{\sexpr}{\stype}$,
     $\evalsTo{\sexpr}{\sval}$ for some $\sval \in \denote{\stype}$
  then $\sval \in \denote{\self{t}{e}{k}}$.
  %(DenotationsSelfify.hs) Depends on: ${\tt lem\_typing\_wf}$
\end{lemma}

This lemma captures the intuition
that if $v \in \denote{\breft{\sbase}{x}{p}}$
(\ie if $v$ has base type $\sbase$
and $\evalsTo{\subst{p}{x}{v}}{\ttrue}$),
then we have $v \in \denote{\breft{\sbase}{x}{ p \wedge x = v}}$
as $\subst{(p \wedge x=v)}{x}{v}$ certainly evaluates to $\ttrue$.
%
The full proof also handles the case that $t$
is an existential type as well as selfification
by an expression $e$ that evaluates to $v$.

\end{comment}

\section{Implementation}

We implemented the metatheory of \sysrfd in \coq.
This mechanization 
proves both type safety and
denotation soundness. 
%
Compared to mechanization of \sysrf, this proof
was about $35\%$ longer in lines of code and takes
about twice as long (two minutes) to check.
%
This mechanization is also included as supplementary
material.
%
Table \ref{fig:comparisonD} summarizes the development 
of our metatheory, and points to which areas of the 
proof increased the most in length in \sysrfd as compared
with \sysrf. 
%
Finally, in the last two columns, the table provides a breakdown for each
broad area of the soundness proof of \sysrfd into
lines of specification (definitions and theorem statements
in Gallina) and lines of proof (proofs written in Ltac,
\coq's tactic language).


The source code for our mechanization of in \coq 
of the soundness proof for $\sysrfd$ is available on Zenodo 
\cite{michael_h_borkowski_2024_13352164}.

\begin{table}
  \caption{Comparative mechanization details for \sysrf versus \sysrfd. 
  }
  \label{fig:comparisonD}
  \vspace{0.00cm}
  \begin{tabular}{l}
  \end{tabular}
  \end{table}

\begin{table}
  % \end{center}
  \vspace{-0.00cm}
  % \begin{center}
  \setlength{\tabcolsep}{6pt}
  \scalebox{0.97}{
  \begin{tabular}{ lrrrr|rr  }
    \multicolumn{2}{l}{ } & 
    \multicolumn{2}{c}{\coq Mechanizations} &
    \multicolumn{3}{r}{ } \\
    \toprule
    \textbf{Subject} &  
    \textbf{Files} & \textbf{\sysrf lines} &
    \textbf{\sysrfd Lines} & \textbf{\% Increase} &
    \textbf{\sysrfd Spec} & \textbf{\sysrfd Proof} \\
    \hline
    Definitions             & 7      & 1155 &  1476 &  27.8\% & 1193 &  283 \\ % &  941  &  190 \\
    Basic Properties        & 8      & 3626 &  3980 &   9.8\% & 1236 & 2744 \\ % & 1201  & 2360 \\
    $\sysf$ Soundness       & 4      &  983 &  1156 &  17.5\% &  178 &  978 \\ % &  173  &  773 \\
    Weakening               & 4      &  719 &   868 &  20.7\% &  110 &  758 \\ % &  110  &  568 \\
    Substitution            & 4      & 1079 &  1248 &  15.7\% &  165 & 1083 \\ % &  158  &  859 \\
    Exact Typing            & 2      &  246 &   318 &  29.3\% &   73 &  245 \\ % &   33  &  182 \\
    Narrowing               & 1      &  333 &   381 &  14.4\% &   54 &  327 \\ % &   54  &  262 \\
    Inversion               & 1      &  339 &   813 & 139.8\% &  145 &  668 \\ % &   57  &  258 \\
    Primitives              & 2      &  635 &  1192 &  87.7\% &  137 & 1055 \\ % &   89  &  508 \\
    $\sysrf$ Soundness      & 1      &  263 &   657 & 149.8\% &   12 &  645 \\ % &   12  &  233 \\
    Denot. Soundness        & 13     & 4033 &  6061 &  50.3\% & 1147 & 4914 \\ % &  815  & 3010 \\
    \midrule
    \textbf{Total}          & 50    & 13411 & 18150 &  35.3\% & 4450 & 13700 \\ % & 3643  & 9203 \\
    \bottomrule
  \end{tabular}
  }
  \end{table}
  
\section*{Acknowledgements for Chapter 9}
%
This chapter consists of unpublished work done in collaboration
with Ranjit Jhala. This material is 
being prepared for future submission for publication.
%
The dissertation author was the primary investigator 
and author of this material.

