\chapter{Lists: The Language \sysrfd}
\label{ch:lists}

One of the key features of GHC's core calculus
missing from $\sysrf$ is data types. The addition of
data types would also enable us to replace the ad hoc
kind system of $\sysrf$ with a more versatile system
of type classes.
%
As a first step in this direction, we augment our
calculus with polymorphic refined list types.
We also add a measure @length@ that describes the 
length of our lists.

\section{Syntax and Semantics} \label{sec:lang:syntaxD}

We present the syntax and semantics of \sysrfd in terms 
of the additions to \sysrf. The reader may refer to 
the figures in Chapter \ref{ch:language} for the syntactic
forms and rules that are inherited from \sysrf.
%
As before, we use the $\greybox{\mbox{grey}}$ to highlight
the extensions to $\sysf$ needed to support refinements 
in $\sysrfd$.

\begin{figure}%[t!]
%  \scalebox{0.80}{
    \begin{tabular}{rrcll}
\emphbf{Primitives} 
  & \sconst & $\bnfdef$ & $\cdots$    & \\ %\emph{booleans and integers} \\
  &         & $\spmid$  & $\suc$     & \emph{integer ops.} \\
  &         & $\spmid$  & $\len$      & \emph{polymorphic list ops} \\ [0.05in] 

\emphbf{List Values}
  & \slval  & $\bnfdef$ & $\nil{\stype}$  & \emph{empty list} \\
  &         & $\spmid$  & $\cons{\stype}{\sval}{\slval}$  
                                      & \emph{list constructor} \\ [0.05in]
\emphbf{Values}
  & \sval   & $\bnfdef$ & $\cdots$               & \\ % \emph{primitives} \\ 
  &         & $\spmid$  & \slval                & \emph{list values} \\[0.05in]

\emphbf{Terms}
  & \sexpr  & $\bnfdef$ & $\cdots$              & \\ %\emph{values} \\ 
  &         & $\spmid$  & \cons{\stype}{e_1}{e_2} & \emph{list constructor} \\
  &         & $\spmid$  & \eswitch{e}{e_n}{e_c} & \emph{list destructor} \\
\end{tabular}
%  }
  \vspace{-0.0cm}
  \caption{Syntax of Primitives, Values, and Expressions.}
\label{fig:syn:termsD}
\vspace{-0.0cm}
\end{figure}


\begin{figure}%[b!]
%{\small
  \begin{tabular}{rrcll}

  \emphbf{Types}
   & \stype & $\bnfdef$ & $\cdots$              & \\ %\emph{values} \\ 
   &        & $\spmid$  & $\listtype{\stype}\greybox{\!\breft{}{\vv}{\spred}}$  & \emph{\greytextbox{refined} list type} \\ 
  \end{tabular}
%}
\vspace{-0.0cm}
  \caption{Syntax of Types.
           The grey boxes are the extensions 
           to $\sysf$ needed by $\sysrfd$.}
           %We use $\sftype$ for $\sysf$-only types.}
  \label{fig:syn:typesD}
  \vspace{-0.00cm}
\end{figure}

\mypara{Constants, Values, and Terms}
%
\Cref{fig:syn:termsD} summarizes the syntax of terms in both 
\sysrfd and in the unrefined calculus.
%
The \emph{primitives} $\sconst$ now
include a $\suc$ function on integers that adds one
and a (polymorphic) $\len$ measure that computes the length of any list.
%
Although a user could easily define $\len$ using
\texttt{switch} below, we add $\len$ as a built-in primitive
in order to use it in our typing judgments.
%
Our \emph{terms} $\sexpr$ now additionally contain two list constructors:
the empty list $\nil{\stype}$ and the non-empty $\cons{\stype}{\sexpr_1}{\sexpr_2}$,
which builds a list from a head element and a tail.
%
Both of these constructors take a refined type annotation. For instance, we should
be able to type check
\[
\cons{\breft{\tint}{\vv}{\vv\geq 0}}{1}({\cons{\breft{\tint}{\vv}{\vv > 0}}{2}{3}})
\]
but not
\[
\cons{\breft{\tint}{\vv}{\vv > 0}}{1}({\cons{\breft{\tint}{\vv}{\vv\geq 0}}{2}{3}})
\]

because the tail of the latter list is only known to consist of non-negative integers.
%
Although we mechanized the metatheory in the same manner without 
type annotations as well, these annotations are needed to enable 
the birdirectional type checking algorithm described in Chapter \ref{ch:bidirectional}.
%
The terms also now contain a list destructor 
$\eswitch{\sexpr}{\sexpr_n}{\sexpr_c}$, which
case splits on the shape of the match scrutinee $\sexpr$.
%
Finally \emph{values} $\sval$ are augmented by lists that contain only
values as elements; these list values are defined inductively 
in Figure \ref{fig:syn:termsD}.

\mypara{Kinds \& Types}       
%
\Cref{fig:syn:typesD} shows the syntax of the types,
with the grey boxes indicating the extensions to $\sysf$ 
required by $\sysrfd$.
%
In contrast \sysrf, our list types are not base types,
but they can be refined.
%
Both of these are key to data types: our lists may contain
incomparable data such as lambda abstractions and so they
cannot support the polymorphic comparison operators within
our simple kind system.
%
However, refinements on lists are key to any model of
data types. We must be able to express the type of a program 
such as the following safe tail function:
\begin{code}
  tail :: forall a. {v:[a] | length v > 0 } -> [a]
  tail xs = switch (xs) error (\y ys -> ys)
\end{code}
%
\sysrfd keeps the simple kind system from \sysrf and 
enforces list types as $\skstar$-kinded to prevent the substitution
of a list type for a refined type variable. 
%
This prevents a refinment of a list type from attempting
to compare a list using one of the polymorphic list operators.


\mypara{Dynamic Semantics} %\label{sec:lang:dynamic}
\Cref{fig:opsemD} summarizes the small-step semantics 
for both calculi.

\begin{figure}
%  {\small
  \begin{mathpar} %%%%%%%%% SMALL-STEP SEMANTICS %%%%%%%%%%
    \judgementHead{Operational Semantics (ext. Figure \ref{fig:e})}{$\sexpr \step \sexpr'$}\\
        \inferrule%*[Right=\eCons]
        {\sexpr \step \sexpr'}
        {\cons{\stype}{\sexpr}{\sexpr_1} \step \cons{\stype}{\sexpr'}{\sexpr_1}}
        {\eCons} 
        \quad
        \inferrule%*[Right=\eConsV]
        {\sexpr \step \sexpr'}
        {\cons{\stype}{\sval}{\sexpr} \step \cons{\stype}{\sval}{\sexpr'}}
        {\eConsV} 
        \\
        \inferrule%*[Right=\eSwitch]
          {\sexpr \step \sexpr'}
          {\eswitch{\sexpr}{\sexpr_n}{\sexpr_c} \step \eswitch{\sexpr'}{\sexpr_n}{\sexpr_c}}
          {\eSwitch} 
        \quad
        \inferrule% *[Right=\eSwitchN]
          {   }
          {\eswitch{\nil{\stype}}{\sexpr_n}{\sexpr_c} \step \sexpr_n}
          {\eSwitchN} 
          \\
        \inferrule% *[Right=\eSwitchC]
          { }
          {\eswitch{\cons{\stype}{\sval_1}{\sval_2}}{\sexpr_n}{\sexpr_c} 
              \step \app{(\app{\sexpr_c}{\sval_1})}{\sval_2}}
          {\eSwitchC} 
        \end{mathpar}        
%    }
\caption{The small-step semantics for \sysrfd.} 
\label{fig:eD}
\label{fig:opsemD}
\end{figure}


%\mypara{Primitives}
%%
%The function $\delta(\sconst, \sval)$ 
%evaluates the application $\app{\sconst}{\sval}$ 
%of built-in monomorphic primitives.
%%
%The reductions are defined in a curried 
%manner, \ie 
%$\app{\app{\leq}{m}}{n}$ evaluates to $\delta(\delta(\leq,m),n)$. 
%%we have that $\app{\app{\leq}{m}}{n} \steps \delta(\delta(\leq,m),n)$. 
%%
%Currying gives us unary relations like $m\!\!\leq$ 
%which is a partially evaluated version of the $\leq$ relation.
%%
%The function $\delta_T(\sconst, \forgetreft{\stype})$
%specifies the reduction rules for type 
%application on the polymorphic 
%built-in primitives. % $=$ and $\leq$.
%%
%$$\begin{array}{rclrclrcl}
%\delta(\wedge,{\tt true}) & \defeq & \lambda x.\, x &
%\delta(\leq,m) & \defeq & m\!\!\leq  & 
%    \delta_T(=, \tbool) & \defeq & =  \\
%\delta(\wedge,{\tt false}) & \defeq & \lambda x.\, {\tt false}\quad\quad &
%\delta(m\!\!\leq, n) & \defeq & {\tt}(m \leq n) \quad\quad&
%\delta_T(=, \tint) & \defeq & = \\
%\delta(\neg,{\tt true}) & \defeq & {\tt false} & 
%\delta(=,m) & \defeq & m\!\!= &
%\delta_T(\leq, \tbool) & \defeq & \leq  \\
%  \delta(\neg,{\tt false}) & \defeq &  {\tt true} &
%  \delta(m\!\!=, n) & \defeq &  {\tt}(m = n) &
%  \delta_T(\leq, \tint) & \defeq & \leq  \\
%%
%%  \delta(\vee,{\tt true}) & \defeq & \lambda x.\, {\tt true} &
%%  \delta(\leftrightarrow,{\tt true}) & \defeq & \lambda x.\, x & 
%%    &  & \\
%%    \delta(\vee,{\tt false}) & \defeq & \lambda x.\, x &
%%    \delta(\leftrightarrow,{\tt false}) & \defeq & \lambda x.\, \neg x &
%%    & & \\
%\end{array}$$
         
\mypara{Typing, Subtyping, and Well-formedness}
%
Next, we present the static semantics of \sysrfd by describing
the additional rules used to establish our
well-formedness, typing, and subtyping judgments.
%%
%We use $\greybox{\mbox{grey}}$ to highlight the antecedents and rules
%specific to $\sysrf$.
%%
%
\Cref{fig:wfD} summarizes the new rules
that establish the well-formedness of types.
%
Rule \wtList states that a the list type 
$\breft{\tlist{\stype}}{x}{\ttrue}$
with empty refinement
is well-formed with star kind provided that 
$\stype$ is well-formed with some kind $\skind$.
%
Similar to rule \wtRefn, our rule \wtListR 
stipulates that a refined list type $\breft{\tlist{\stype}}{x}{\spred}$
is well-formed with star kind in some environment
if the trivally refined type $\breft{\tlist{\stype}}{x}{\ttrue}$
has star kind in the same environment and if
the refinement predicate $\spred$ has type $\tbool$
in the environment augmented by binding a fresh variable to type $\tlist{\stype}$.

\begin{figure}%[t!]
%
%{\small
\begin{mathpar}
\judgementHead{Well-formed Type (ext. Figure \ref{fig:wf})}{\isWellFormed{\tcenv}{\stype}{\skind}}

%%%%%%%%%%%% WELL-FORMEDNESS %%%%%%%%%%%%%
%
\inferrule% *[Right=\wtList]
    {\isWellFormed{\tcenv}{\stype}{\skind}}
    {\isWellFormed{\tcenv}{\tlist{\stype}\greybox{\!\breft{}{x}{\ttrue}}\!}{\skstar}}
    {\wtList}
\quad
%    
\greybox{\inferrule% *[Right=\;\;\wtListR]
    { \isWellFormed{\tcenv}{\breft{\tlist{\stype}}{x}{\ttrue}}{\skstar} \\\\
      \forall\notmem{y}{\tcenv}.
      \hasftype{\bind{y}{\tlist{\stype}}, \forgetreft{\tcenv}}{\subst{p}{x}{y}}{\tbool}
    }
    {\isWellFormed{\tcenv}{\breft{\tlist{\stype}}{x}{p}}{\skbase}}
    {\wtRefn} }
%
\end{mathpar}
%}
\vspace{-0.00cm}
\caption{Well-formedness of $\sysrfd$ types. The rules for
  $\sysf$ exclude the grey boxes.}
\label{fig:wfD}
\vspace{-0.00cm}
\end{figure}

%The judgment $\hastype{\tcenv}{\sexpr}{\stype}$ states
%that the term $\sexpr$ has type $\stype$ in the context of
%environment $\tcenv$.
\Cref{fig:typingD} summarizes
the rules that establish typing for both $\sysf$ and
$\sysrf$, with grey %boxes
%indicating extensions needed
for the $\sysrf$ extensions.
%
Rule $\tNil$ states that whenever $\stype$ is a well-formed
type in some environment, then  $\nil{\stype}$ has the type
$\breft{\tlist{\stype}}{x}{\len x = 0}$ of lists of elements
of type $\stype$ of length zero.
%
The rule $\tCons$ is slightly more complex: whenever $\sexpr_h$
can be given type $\stype$ in some environment, and whenver
$\sexpr_t$ can be given type
$\breft{\tlist{\stype}}{x}{\spred}$ in the same environment,
then the list $\cons{\stype}{\sexpr_h}{\sexpr_t}$ can be given
the type $\tlist{\stype}$ with the refinement that says that
this list has length one more than some list of type
$\breft{\tlist{\stype}}{x}{\spred}$.
%
The purpose of this refinement is to embed the information about
the specific length of a list at the refinement level.
%
Our typing rule for the list destructor 
$\switch{\sexpr}{\sexpr_n}{\sexpr_c}$ is 
best thought of as analogous to our rule $\tIf$
because it enables path-sensitive reasoning about lists.
%
This rule $\tSwitch$ says that whenever the match scrutinee
$\sexpr$ can be given a list type 
$\breft{\tlist{\stype}}{x}{\spred}$ in some environment,
whenever $\stype'$ is well-typed in the same environment,
and whenver each branch can be given this type
$\stype'$ in this environment augmented by the knowledge 
that the about the scrutinee and its length, then the full
term $\switch{\sexpr}{\sexpr_n}{\sexpr_c}$ can be given type
$\stype'$. 
%
Note that, per the semantics in \ref{fig:eD}, the cons-branch
$\sexpr_c$ is a function expecting two arguments: the head
of the scrutinee and the tail (which we know has length
one less than the scrutinee).
%
In our formalism, it is neccessary for us to augment the 
environment with \emph{two} dummy variables here. We need
to preserve $\spred$,
the knowledge obtained from the match scrutinee,
and the fact that the scrutinee is one element longer than
the tail to which $\sexpr_c$ is being applied.
%
We could express the antecedent judgment as
\[
\forall\notmem{z}{\tcenv}. 
  \hastype{\bind{z}{\breft{\tlist{\stype}}{x}{p},\tcenv}}
     {\sexpr_c}
     {\funcftype{\stype}{\funcftype{\tlist{\stype}\greybox{\breft{}{\vv}{\suc \len \vv = \len y}}}{\stype'}}}
\]
but this would pose a problem for proving type soundness 
for \sysrfd (specifically preservation)
from a minimal interace of axioms (\ref{lem:implicationD}) 
for implication.
%
These axioms are syntactic, and so slight variations



\begin{figure}
%  {\small
  \begin{mathpar}             %%%%%%%%%%%%% TYPING %%%%%%%%%%%%%%%%%%
  \judgementHead{Typing (ext. Figure \ref{fig:t})}{\hastype{\tcenv}{\sexpr}{\stype}} \\

    %
        \inferrule
        {\isWellFormed{\tcenv}{\stype}{\skind}}
        {\hastype{\tcenv}{\nil{\stype}}{\tlist{\stype}\greybox{\breft{}{x}{\len x = 0}}}}
        {\tNil}
    %
    \quad  
    %
        \inferrule
        {\hastype{\tcenv}{\sexpr_h}{\stype} \\
         \hastype{\tcenv}{\sexpr_t}{\tlist{\stype}\greybox{\breft{}{x}{\spred}}}
        }
        {\hastype{\tcenv}{\cons{\stype}{\sexpr_h}{\sexpr_t}}
                 {\greybox{\existype{y}{\tlist{\stype}\greybox{\breft{}{x}{\spred}}}}
                  \tlist{\stype}\greybox{\breft{}{\vv}{\len \vv = \suc \len y}}}}
        {\tCons}
    \\    
    %
        \inferrule
        {\hastype{\tcenv}{\sexpr}{\tlist{\stype}\greybox{\breft{}{x}{\spred}}} \\
         \isWellFormed{\tcenv}{\stype'}{\skind} \\\\
          %
         \greybox{\forall\notmem{y}{\tcenv}.}
         \hastype{\greybox{\bind{y}{\breft{\tlist{\stype}}{x}{p \wedge \len x = 0}},}\tcenv}
             {\sexpr_n}{\stype'}\\\\
          %
         \greybox{\forall\notmem{y, z}{\tcenv}.}
         \hastype{\greybox{\bind{z}{\breft{\tlist{\stype}}{x}{p \wedge \suc \len y = \len x}},
                          \bind{y}{\breft{\tlist{\stype}}{}{\ttrue}},}\tcenv}
            {\sexpr_c}{\funcftype{\stype}{\funcftype{\tlist{\stype}\greybox{\breft{}{\vv}{\len y = \len \vv}}}{\stype'}}}
        }
        {\hastype{\tcenv}{\eswitch{\sexpr}{\sexpr_n}{\sexpr_c}}{\stype'}}
        {\tSwitch}        
    \end{mathpar}
%  }
\vspace{-0.00cm}
\caption{Typing rules.
The judgment $\hasftype{\tcenv}{\sexpr}{\sftype}$ is extended by excluding the grey boxes.}
\label{fig:tD}\label{fig:typingD}
\vspace{-0.00cm}
\end{figure}

\begin{comment}

\mypara{Typing Primitives}
%
The type of a built-in primitive $\sconst$ is given by
the function $\ty{\sconst}$, which is defined for every
constant of our system. Below we present essential
examples of the $\ty{\sconst}$ definition.
%
{\small
$$\begin{array}{rclrcl}
\ty{\ttrue} & \defeq & \breft{\tbool}{x}{x = \ttrue} &
%&  \ty{\tfalse} & \defeq& \breft{\tbool}{x}{x = \tfalse} \\
%\ty{3} & \defeq& \breft{\tint}{x}{x = 3} \\
%&  \ty{n} & \defeq& \breft{\tint}{x}{x = n} \\
 \ty{\wedge} & \defeq & \functype{x}{\tbool}{\functype{y}{\tbool}{\breft{\tbool}{v}{v = x \wedge y}}} \\
%  \ty{\neg} & \defeq & \functype{x}{\tbool}{\breft{\tbool}{y}{y = \neg x}} \\
\ty{3} & \defeq& \breft{\tint}{x}{x = 3} &
\ty{\leq} & \defeq & \polytype{\al}{\skbase}{\functype{x}{\al}{\functype{y}{\al}{\breft{\tbool}{v}{v = (x \leq y)}}}} \\
\ty{m\!\!\leq} & \defeq & \functype{y}{\tint}{\breft{\tbool}{v}{v = (m \leq y)}} & 
% \ty{m\!\!\leq} \defeq&\; \functype{n}{\tint}{\breft{\tbool}{v}{v = (m \leq n)}} \\
\ty{=} & \defeq & \polytype{\tvar}{\skbase}{\functype{x}{\al}{\functype{y}{\al}{\breft{\tbool}{v}{v = (x = y)}}}}
\end{array}$$
}
We note that the $=$ used in the refinements is the polymorphic
equals with type applications elided.
%
Further, we use $m\!\!\leq$ to represent
an arbitrary member of the infinite family
of primitives $0\!\!\leq,\, 1\!\!\leq,\, 2\!\!\leq,\ldots$.
%
For \sysf we erase the refinements
using $\forgetreft{\ty{\sconst}}$.
%
The rest of the definition is similar.

Our choice to make the typing and reduction
of constants external to our language,
\ie given by the functions
$\ty{\sconst}$ and $\tc{\sconst}$,
makes our system easily extensible with further constants, 
including a terminating \texttt{fix} constant to encode induction.
%
The requirement, for soundness, is that
these two functions % on constants
together satisfy the following four conditions.
%
\begin{requirement}(Primitives) \label{lem:prim-typing}
For every primitive $c$,
\begin{enumerate}
\item If $\ty{\sconst} = \breft{\sbase}{x}{\spred}$, then
  $\isWellFormed{\varnothing}{\ty{\sconst}}{\skbase}$ and
  $\imply{\varnothing}{\ttrue}{\subst{\spred}{x}{\sconst}}$.
\item If $\ty{\sconst} = \functype{x}{\stype_x}{\stype}$ or
         $\ty{\sconst} = \polytype{\al}{\skind}{\stype}$, then
         $\isWellFormed{\varnothing}{\ty{\sconst}}{\skstar}$.
\item If ${\ty{\sconst}} = \functype{x}{\stype_x}{\stype}$,
      then for all $v_x$ such that
     $\hastype{\varnothing}{v_x}{\stype_x}$,
     % $\delta(\sconst,v)$ is defined and
    $\hastype{\varnothing}{\delta(\sconst,v_x)}{\subst{\stype}{x}{v_x}}$.
\item If ${\ty{\sconst}} = \polytype{\al}{\skind}{\stype}$,
      then for all $\stype_\al$ such that
     $\isWellFormed{\varnothing}{\stype_\al}{\skind}$,
     % $\delta_T(\sconst,\stype)$ is defined and we have
    $\hastype{\varnothing}{\delta_T(\sconst,\stype_\al)}{\subst{\stype}{\al}{\stype_\al}}$.
\end{enumerate}
\end{requirement}

Theorem 3 of~\cite{Vazou14} proves that 
a terminating \texttt{fix} constant satisfies
requirement~\ref{lem:prim-typing}. 

To type constants, rule \tPrim gives the type
$ty(\sconst)$ to any built-in
primitive $\sconst$, in any context.
%
\begin{fullversion}
The typing rules for boolean and integer
constants are included in \tPrim.
\end{fullversion}

\mypara{Typing Variables with Selfification}
%
Rule \tVar establishes that any variable $x$ that
appears as $\bind{x}{\stype}$ in environment $\tcenv$
can be given the \emph{selfified} type \cite{Ou2004}
$\self{\stype}{x}{\skind}$ provided that
$\isWellFormed{\tcenv}{\stype}{\skind}$.
%
This rule is crucial in practice,
to enable path-sensitive ``occurrence'' typing \cite{Tob08},
where the types of variables are refined by control-flow guards.
%
\NV{Can we use an example from the overview?}
\NV{go prime require self for if condition}
For example, suppose we want to establish
$\hastype{\bind{\al}{\skbase}}{(\vabs{x}{x})}{\functype{x}{\al}{\breft{\al}{y}{x=y}}}$,
and not just $\hastype{\bind{\al}{\skbase}}{(\vabs{x}{x})}{\funcftype{\al}{\al}}$.
%
The latter would result
if \tVar merely stated that
$\hastype{\tcenv}{x}{\stype}$
whenever $\bind{x}{\stype} \in \tcenv$.
%
Instead, we strengthen the \tVar rule to
be \emph{selfified}.
%:
%
Informally, to get information about $x$
into the refinement level, we need to say
that $x$ is constrained to elements of
type $\al$ that are equal to $x$ itself.
%
In order to express the exact type of
variables, below we define the ``selfification''
function that strengthens a refinement
with the condition that a value is equal
to itself.
%
Since abstractions do not admit equality,
we only selfify the base types and the existential
quantifications of them.
%
$$\begin{array}{r@{\hskip 0.03in}c@{\hskip 0.03in}l@{\hskip 0.2in}r@{\hskip 0.03in}c@{\hskip 0.03in}l}
  \self{\existype{z}{t_z}{t}}{x}{\skind} & \defeq & \existype{z}{t_z}{\self{t}{x}{\skind}} & 
  \self{\breft{\sbase}{z}{p}}{x}{\skbase} & \defeq & \breft{\sbase}{z}{p \wedge z = x} \\
  \self{\functype{x}{t_x}{t}}{\_}{\_} & \defeq & \functype{x}{t_x}{t} &
  \self{\breft{\sbase}{z}{p}}{x}{\skstar} & \defeq & \breft{\sbase}{z}{p} \\ 
  \self{\polytype{\al}{k}{t}}{\_}{\_} & \defeq & \polytype{\al}{k}{t}
% \self{\stype}{x}{\skstar}               & \defeq & \stype &
%  & &
\end{array}$$

\mypara{Typing Applications with Existentials}
%
Our rule \tApp states the conditions for typing
a term application $\app{\sexpr}{\sexpr_x}$.
%
Under the same environment,
we must be able to type $\sexpr$ at
some function type $\functype{x}{\stype_x}{\stype}$ and
$\sexpr_x$ at $\stype_x$. Then we can give $\app{\sexpr}{\sexpr_x}$
the existential type $\existype{x}{\stype_x}{\stype}$.
%
The use of existential types in rule \tApp
is one of the distinctive features of our language
and was introduced by~\citet{Knowles09}.
%
As overviewed in \S~\ref{overview:exists},
we chose this form of \tApp over the conventional
form of $\hastype{\tcenv}{\app{\sexpr}{\sexpr_x}}{\subst{\stype}{x}{\sexpr_x}}$
because our version prevents
the substitution of arbitrary expressions (\eg functions and type abstractions)
into refinements.
%
As an alternative, we could have used ANF (A-Normal Form~\cite{Flanagan93}),
but our metatheory would be more complex
since ANF is not preserved under the small step operational semantics.

\mypara{Other Typing Rules}
%
\begin{fullversion}
    Rule \tAbs says that we can type a lambda abstraction
    $\vabs{x}{\sexpr}$ at a function type $\functype{x}{\stype_x}{\stype}$
    whenever $\stype_x$ is well-formed and the body $\sexpr$
    can be typed at $\stype$ in the environment augmented by binding
    a fresh variable to $\stype_x$.  
\end{fullversion}
%
Our rule \tTApp states that whenever a term $\sexpr$ has polymorphic
type $\polytype{\al}{\skind}{s}$, then for any well-formed type $\stype$
with kind $\skind$, %in the same environment, 
we can give the type
$\subst{s}{\al}{\stype}$ to the type application $\tyapp{\sexpr}{\stype}$.
%
For the \sysf variant of \tTApp, we erase the refinements (via $\forgetreft{\stype}$)
before checking well-formedness and performing the substitution.
%
\begin{fullversion}
    The rule \tTAbs establishes that a type-abstraction $\tabs{\al}{\skind}{\sexpr}$
    can be given a polymorphic type $\polytype{\al}{\skind}{\stype}$ in some % environment
    $\tcenv$ whenever $\sexpr$ can be given the well-formed type $\stype$
    in %the environment 
    $\tcenv$ augmented by binding a fresh type variable
    to kind $\skind$.
    %
    Next, rule $\tLet$ states that an expression $\eletin{x}{\sexpr_x}{e}$ has type
    $\stype$ in some environment whenever $\stype$ is well-formed, $\sexpr_x$ can
    be given some type $\stype_x$, and the body $\sexpr$ can be given type $\stype$
    in the environment augmented by binding a fresh variable to $\stype_x$.
\end{fullversion}
%
Rule $\tAnn$ establishes that an explicit annotation $\tyann{\sexpr}{\stype}$
indeed has type $\stype$ when the underlying %expression 
$\sexpr$ has type $\stype$ and $\stype$ is well-formed.
The \sysf version of the rule erases the refinements and uses
$\forgetreft{\stype}$.
%
Rule $\tIf$ states that a conditional expression $\eif{\sexpr}{\sexpr_1}{\sexpr_2}$
has the type $\stype$ when the guard $\sexpr$ can be 
given type $\tbool$ refined by $p$ and $\sexpr_1$ (\resp $\sexpr_2$)
can be given type $\stype$ in the environment $\tcenv$ 
augmented by the knowledge we have about the type 
and semantics of the guard $\sexpr$.
The extension of the environment $\tcenv$
with a fresh variable that captures the semantics of the guard
when checking the two paths is critical to 
permit path-sensitive reasoning.
%
Finally, rule \tSub tells us that we can exchange a subtype $s$
for a supertype $t$ in a judgment $\hastype{\tcenv}{\sexpr}{t}$
provided 
$t$ is well-formed and
$\isSubType{\tcenv}{s}{t}$, which we present next.
%
\subsection{Subtyping}
\label{sec:typing:sub}

The \emph{subtyping} judgment ${\isSubType{\tcenv}{s}{t}}$, 
defined in \Cref{fig:s}, 
stipulates that the type $s$ is a subtype of
the type $t$ in the environment $\tcenv$
and is used in the subsumption typing rule \tSub (of~\Cref{fig:typing}).

\begin{figure}
\judgementHead{Subtyping}{\isSubType{\tcenv}{s}{t}}

\begin{mathpar}   %%%%%%%%%%%%%%%%%% SUBTYPING %%%%%%%%%%%%%%%%%%
  \inferrule*[Right=\sFunc]
  { \isSubType{\tcenv}{t_{x2}}{t_{x1}} \quad
    \forall\notmem{y}{\tcenv}. \quad
    \isSubType{\bind{y}{t_{x2}},\tcenv}{\subst{t_1}{x}{y}}{\subst{t_2}{x}{y}} }
  {\isSubType{\tcenv}{\functype{x}{t_{x1}}{t_1}}{\functype{x}{t_{x2}}{t_2}}} \\
%
\and
%
  \inferrule*[Right=\sWitn]
  { \hastype{\tcenv}{\sval_x}{\stype_x} \quad
    \isSubType{\tcenv}{\stype}{\subst{\stype'}{x}{\sval_x}}}
  {\isSubType{\tcenv}{\stype}{\existype{x}{\stype_x}{\stype'}}}
%
\and
%
  \inferrule*[Right=\sBind]
  { \forall\notmem{y}{\free{\stype}\cup \tcenv}. \quad
    \isSubType{\bind{y}{\stype_x},\tcenv}{\subst{\stype}{x}{y}}{\stype'}
%    \quad
%    \lc{\stype'}
  }
  {\isSubType{\tcenv}{\existype{x}{\stype_x}{\stype}}{\stype'}} \\
%
\and
%
  \inferrule*[Right=\sPoly]
  { \forall\notmem{\al'}{\tcenv}. \quad
    \isSubType{\bind{\al'}{\skind},\tcenv}{\subst{\stype_1}{\al}{\al'}}{\subst{\stype_2}{\al}{\al'}} }
  {\isSubType{\tcenv}{\polytype{\al}{\skind}{\stype_1}}{\polytype{\al}{\skind}{\stype_2}}}
%
\and
%
  \inferrule*[Right=\sBase]
  {\forall\notmem{y}{\tcenv}. \quad
    \imply{\bind{y}{{\sbase}},\tcenv}{\subst{p_1}{x}{y}}{\subst{p_2}{x}{y}} }
  {\isSubType{\tcenv}{\breft{\sbase}{x}{\spred_1}}{\breft{\sbase}{x}{\spred_2}}}
\end{mathpar}
\vspace{-0.00cm}
\caption{Subtyping Rules.}
\label{fig:s}
\label{fig:subtyping}
\label{fig:ent}
\vspace{-0.00cm}
\end{figure}
\mypara{Subtyping Rules}
\begin{fullversion}  
    The rule \sFunc states that one function type $\functype{x_1}{t_{x1}}{t_1}$
    is a subtype of another function type $\functype{x_2}{t_{x2}}{t_2}$ in a
    given environment $\tcenv$ when both $t_{x2}$ is a subtype of $t_{x1}$
    and $t_1$ is a subtype of $t_2$ when we augment $\tcenv$ by
    binding a fresh variable to type $t_{x2}$.
    As usual, function subtyping is contravariant
    in the input type and covariant in the outputs.
\end{fullversion}
%
Rules \sBind and \sWitn establish subtyping for existential
types \cite{Knowles09}, \resp when the existential
appears on the left or right.
%
Rule \sBind allows us to exchange a universal quantifier
(a variable bound to some type $\stype_x$ in the environment)
for an existential quantifier.
%
If we have a judgment of the form
$\isSubType{\bind{y}{\stype_x},\tcenv}{\subst{\stype}{x}{y}}{\stype'}$
where $y$ does \emph{not} appear free in either $\stype'$ or in the context $\tcenv$,
%$\stype'$ is locally closed, which means that all its deBruijn
%indices point to binders within $\stype'$ (denoted $\lc{t'}$),
% \NV{We should not talk about locally closed since we do not model deBruin.
%     We could instead add a well formedness requirement, but I assume you have a proof
%     that everything in the right hand side of a subtyping is well formed?}
then we can conclude that $\existype{x}{\stype_x}{\stype}$
is a subtype of $\stype'$.
%
Rule \sWitn states that if type $\stype$ is a subtype of
$\subst{\stype'}{x}{\sval_x}$ for some value $\sval_x$
of type $\stype_x$, then we can discard the specific
\emph{witness} for $x$ and quantify existentially to
obtain that $\stype$ is a subtype of $\existype{x}{\stype_x}{\stype'}$.
%
\begin{fullversion} 
    Rule \sPoly states that one polymorphic type
    $\polytype{\al}{\skind}{\stype_1}$
    is a subtype of another polymorphic type
    $\polytype{\al}{\skind}{\stype_2}$
    in some environment $\tcenv$, when
    $\stype_1$ is a subtype of $\stype_2$ in the
    environment $\tcenv$ augmented  by binding
    a fresh type variable to kind $\skind$.
\end{fullversion}

Refinements enter the scene in the rule \sBase which
specifies
%uses implication to specify
that a refined base type $\breft{\sbase}{x}{p_1}$
is a subtype of another $\breft{\sbase}{x}{p_2}$
in context $\tcenv$ when $p_1$ \emph{implies} $\spred_2$
in the environment $\tcenv$ augmented by binding
a fresh variable to the unrefined type $\sbase$.
%
%Next, we describe how we formalized implication.

\subsection{Implication}
\label{sec:typing:implication}

The \emph{implication} judgment $\imply{\tcenv}{\spred_1}{\spred_2}$
states that the implication $\spred_1 \Rightarrow \spred_2$
holds under the assumptions captured by the context $\tcenv$.
%
In refinement type implementations~\cite{newfstar,Seidel14}, this relation
is implemented as an external automated (usually SMT) solver.
%
Since external solvers are not easy to encode in mechanized proofs,
we follow an approach that decouples the mechanization from the implementation. 
Concretely, 
first we define the interface of the implication (\S~\ref{sec:typing:implication:interface})
that precisely captures all the requirements that the implication 
judgement should satisfy to establish the soundness of $\sysrf$. 
Then, we define two alternative implementations of the interface: 
a logical implementation (\S~\ref{sec:typing:implication:logical})
that is used in refinement type implementations 
and a denotational implementation (\S~\ref{sec:typing:implication:denotational})
that we used to complete our mechanized proof. 


\subsubsection{Implication's Interface}
\label{sec:typing:implication:interface}
In our mechanization, 
following~\citet{LehmannTanter}, 
we encode implication
as an axiomatized judgment that satisfies the
requirements below.


\begin{requirement}[Implication Interface]\label{lem:implication}
  The implication relation satisfies the below statements:
  \begin{enumerate}
      \item (Reflexivity) $\imply{\tcenv}{\spred}{\spred}$.
      \item (Transitivity) If $\imply{\tcenv}{\spred_1}{\spred_2}$
        and $\imply{\tcenv}{\spred_2}{\spred_3}$, then
        $\imply{\tcenv}{\spred_1}{\spred_3}$.
      \item (Faithfulness) $\imply{\tcenv}{\spred}{\ttrue}$.
      \item (Introduction) If $\imply{\tcenv}{\spred_1}{\spred_2}$
        and $\imply{\tcenv}{\spred_1}{\spred_3}$, then
        $\imply{\tcenv}{\spred_1}{\csand{\spred_2}{\spred_3}}$.
      \item (Conjunction)
      $\imply{\tcenv}{\csand{\spred_1}{\spred_2}}{\spred_1}$ and
      $\imply{\tcenv}{\csand{\spred_1}{\spred_2}}{\spred_2}$.
      \item (Repetition)
      $\imply{\tcenv}{\csand{\spred_1}{\spred_2}}
                     {\csand{\spred_1}{\csand{\spred_1}{\spred_2}}}$.
      \item (Evaluation) If $\evalsTo{\spred_1}{\spred_2}$, 
        then $\imply{\tcenv}{\spred_1}{\spred_2}$
        and  $\imply{\tcenv}{\spred_2}{\spred_1}$.
      \item (Narrowing) If
      $\imply{\tcenv_1,\bind{x}{t_x},\tcenv_2}{\spred_1}{\spred_2}$
      and $\isSubType{\tcenv_2}{s_x}{t_x}$, then
      $\imply{\tcenv_1,\bind{x}{s_x},\tcenv_2}{\spred_1}{\spred_2}$.
      \item (Weaken) If
      $\imply{\tcenv_1,\tcenv_2}{\spred_1}{\spred_2}$,
       $a,x\not\in\tcenv$, then
      $\imply{\tcenv_1,\bind{x}{t_x},\tcenv_2}{\spred_1}{\spred_2}$ and
      $\imply{\tcenv_1,\bind{a}{k},\tcenv_2}{\spred_1}{\spred_2}$.
      \item (Subst I) If
      $\imply{\tcenv_1,\bind{x}{t_x},\tcenv_2}{\spred_1}{\spred_2}$
      and $\hastype{\tcenv_2}{v_x}{t_x}$, then
      $\imply{\subst{\tcenv_1}{x}{v_x},\tcenv_2}{\subst{\spred_1}{x}{v_x}}
                                             {\subst{\spred_2}{x}{v_x}}$.
      \item (Subst II) If $\imply{\tcenv_1,\bind{a}{k},\tcenv_2}{\spred_1}{\spred_2}$
      and $\isWellFormed{\tcenv_2}{t}{k}$, then
      $\imply{\subst{\tcenv_1}{a}{t},\tcenv_2}{\subst{\spred_1}{a}{t}}
      {\subst{\spred_2}{a}{t}}$.
      \item (Strengthening) If
      $\imply{\bind{y}{\breft{\sbase}{x}{q},\tcenv}}{\spred_1}{\spred_2}$,
      then $\imply{\bind{y}{\sbase},\tcenv}{\csand{\subst{q}{x}{y}}{\spred_1}}
                  {\csand{\subst{q}{x}{y}}{\spred_2}}$.
  \end{enumerate}
\end{requirement}

\noindent
This interface  precisely explicates
the requirements of
the implication checker 
to establish the soundness of the
entire refinement type system.
%
The first six statements are standard properties of implication. 
Evaluation is used to prove that built-in constants satisfy the~\Cref{lem:prim-typing} 
and the rest, as captured by their name, are required to prove  
the narrowing (\ref{lem:narrowing}), weakening (\ref{lem:weakening}), 
substitution (\ref{lem:subst}) lemmas hold in \sysrf. 

Our requirements are very similar to Assumption 1 of~\cite{Knowles09}. 
Our Strengthening and Subst II cases are required for polymorphism, thus 
they do not appear in~\citet{Knowles09}'s assumption.  
Instead they require Consistency and Exact Quantification. 
We do not require Exact Quantification since our relation captures 
the minimum requirements to prove soundness. 
Instead of explicitly requiring Consistency, 
in \S~\ref{sec:typing:implication:denotational} we define (and mechanize) an implementation, \ie inhabitant, 
of the interface thus show our assumptions are consistent. 


\subsubsection{Logical Implementation (non mechanized)}
\label{sec:typing:implication:logical}
The logical implementation of \imply{\tcenv}{\spred_1}{\spred_2}
checks that the logical implication $\spred_1 \Rightarrow \spred_2$
is valid assuming the refinements of the base types in $\tcenv$:
%
$$
  \inferrule
  {\models_{\texttt{LOGIC}} \wedge\{ \subst{\spred}{\vv}{x} \mid \bind{x}{\breft{\sbase}{\vv}{\spred}} \in \tcenv \} \Rightarrow \spred_1 \Rightarrow \spred_2 }
  {\imply{\tcenv}{\spred_1}{\spred_2}}
  {\iLog}
$$
This encoding is imprecise, since some information is ignored from the environment $\tcenv$, 
but when the language of refinements is decidable, implication checking 
is also decidable and can be efficiently checked by an SMT solver.
\lh, for example, uses this encoding to reduce type checking to decidable 
implications checked by Z3~\cite{z3}, while the soundness 
of this implementation (concretely statement 7 of~\Cref{lem:implication}) 
is hinted by Theorem 2 of~\cite{Vazou14}.
\citet{10.1145/3546196.3550162} defines a mechanization of a 
refinement type system in Agda that uses a similar encoding of implication
where logical implications are checked using Agda's logic. 


\subsubsection{Denotational Implementation (mechanized)}
\label{sec:typing:implication:denotational}
The denotational implementation of \imply{\tcenv}{\spred_1}{\spred_2}
checks that if $\spred_1$ evaluates to \ttrue, so does $\spred_2$. 
%
$$
  \inferrule
  { \forall \clsub\!\in\!\denote{\tcenv}. \
      \evalsTo{\applysubst{\clsub}{\spred_1}}{\ttrue}
      \Rightarrow 
      \evalsTo{\applysubst{\clsub}{\spred_2}}{\ttrue}
  }
  {\imply{\tcenv}{\spred_1}{\spred_2}}
  {\iDen}
$$
The refinements $\spred_1$ and $\spred_2$ are boolean expressions, 
so evaluation uses the operational semantics of~\Cref{fig:opsem}. 
But, they are open expressions with variables bound in $\tcenv$,
so before evaluation we apply the closing substitution $\clsub$
that belongs to the denotation of $\tcenv$, as defined next. 

\begin{figure}
$$\begin{array}{r@{\hskip 0.03in}c@{\hskip 0.03in}l}
\denote{\breft{\sbase}{x}{p}} & \defeq &
  \setcomp{\sval}{\hasftype{\varnothing}{\sval}{\sbase} \,\wedge\, \evalsTo{\subst{p}{x}{\sval}}{\ttrue}} \\
\denote{\functype{x}{\stype_x}{\stype}} & \defeq &
  \setcomp{\sval}{\hasftype{\varnothing}{\sval}{\funcftype{\forgetreft{\stype_x}}{\forgetreft{\stype}}} \,\wedge\, (\forall\, \sval_x \in \denote{\stype_x}.\, \evalsTo{\app{\sval}{\sval_x}}{\sval'} \,{\rm s.t.}\, \sval' \in \denote{\subst{\stype}{x}{\sval_x}} }  \\
\denote{\existype{x}{\stype_x}{\stype}} & \defeq &
  \setcomp{\sval}{(\hasftype{\varnothing}{\sval}{\forgetreft{\stype}}) \,\wedge\,
    ( \exists\, \sval_x \in \denote{\stype_x}.\, \sval\in \denote{\subst{\stype}{x}{\sval_x}} }   \\
\denote{\polytype{\al}{\skind}{\stype}} & \defeq &
  \setcomp{\sval}{(\hasftype{\varnothing}{\sval}{\polytype{\al}{\skind}{\forgetreft{\stype}}}) \,\wedge\,
    (\forall\, \stype_\al.\, (\isWellFormed{\varnothing}{\stype_\al}{\skind}) \Rightarrow
    \evalsTo{\tyapp{\sval}{\stype_\al}}{\sval'}
    \,{\rm s.t.}\, \sval' \in \denote{\subst{\stype}{\al}{\stype_\al}}} \\ \\ 
\denote{\tcenv} & \defeq &
\setcomp{\clsub} %= (x_1 \mapsto v_1,\ldots, x_n \mapsto v_n,  \al_1 \mapsto t_1, \,\ldots,\, \al_m\mapsto t_m)}{
{\forall\, (x:t) \in \tcenv.\, \clsub(x) \in \denote{\applysubst{\clsub}{t}} \;\wedge\;
\forall\, (\al:k) \in \tcenv.\, \isWellFormed{\varnothing}{\clsub(\al)}{k}}.
\end{array}$$
\vspace{-0.2cm}
\caption{Denotations of Types and Environments.}
\label{fig:den}
\vspace{-0.4cm}
\end{figure}

\paragraph{Closing Substitutions}
A \emph{closing substitution} is a sequence
of value bindings to variables:
$
    \clsub = (x_1\mapsto v_1,\,\ldots,\, x_n\mapsto v_n, \al_1 \mapsto t_1, \,\ldots,\, \al_m\mapsto t_m)
    \;\; {\rm with\;\; all}\; x_i,\; \al_j\; {\rm distinct.}
$
%
We write $\clsub(x)$ to refer to $v_i$ if $x = x_i$
and we use $\clsub(\al)$ to refer to $t_j$ if $\al = \al_j$.
%
We define $\applysubst{\clsub}{t}$ to be the type derived from $t$
by substituting for all variables in $\clsub$:
%
$
  \applysubst{\clsub}{t} \defeq \subst{t}{x_1}{v_1}\cdots\subst{}{x_n}{v_n}\subst{}{\al_1}{t_1}\cdots\subst{}{\al_m}{t_m}.
$
%


\paragraph{Denotational Semantics}
\Cref{fig:den} defines the denotations of types and environments.
Following~\citet{flanagan06}, 
each closed type has a denotation $\denote{\stype}$
containing the set of closed values
of the appropriate base type
which satisfy the type's refinement
predicate.
(The denotation of a type variable $\al$
is not defined as we only require
denotations for closed types.)
%
We lift the notion of denotations to environments $\denote{\tcenv}$
as the set of closing substitutions, \ie value
and type bindings for the variables in $\tcenv$, such that the
values respect the denotations of the respective $\tcenv$-bound
types and the types are well formed with respect
to the corresponding kinds.


\paragraph{Revisiting rule~\iDen}
The premise of the rule~\iDen quantifies 
over all closing substitutions in the denotations of the 
typing environment 
(\ie $\forall \clsub\!\in\!\denote{\tcenv}$). 
This quantification has two consequences. 

First, the environment denotation appears in a negative position 
on the premise of the rule. Inspecting~\Cref{fig:den}, 
the environment denotation uses the type denotation, which in turn uses 
type checking, thus rendering a \textit{potential circularity
between type and implication checking} (arrow 6 of~\Cref{fig:dependencies}). 
Because of the negative occurrence, this mutual dependency 
would lead to a non-monotonic and thus non-well defined system. 
To break this circularity, we use $\sysf$'s type checking in the definition 
of type denotations. 

Second, the quantification is over all closing substitutions
which are infinite. For example, 
a typing environment that binds $x$ to an integer 
(\ie $\bind{x}{\tint}\in \tcenv$) 
has infinitely many closing substitutions mapping $x$ to a different 
integer. Thus, the denotational implementation 
cannot be used to implement a decidable type checker. 
%
On the positive side, the denotational implementation
connects implication checking to the operational semantics
thus it is amicable to mechanization.
Concretely, we proved (\S~\ref{sec:coq})
that the denotational implementations satisfies 
the statements of~\Cref{lem:implication}. 

\end{comment}