\section{Static Semantics}
\label{sec:lang:static}

The static semantics of our calculi comprise
four main judgment forms:
%
(\S~\ref{sec:typing:wf}) {\emph{well-formedness}} judgments that determine when a type
or environment is syntactically well-formed (in $\sysf$ and $\sysrf$);
%
(\S~\ref{sec:typing:typ}) {\emph{typing}} judgments that stipulate that a term has
a particular type in a given context (in $\sysf$ and $\sysrf$);
%
(\S~\ref{sec:typing:sub}) {\emph{subtyping}} judgments that establish when one type can
be viewed as a subtype of another (in $\sysrf$); and
%
(\S~\ref{sec:typing:implication}) {\emph{implication}} judgments that establish when one predicate
implies another (in $\sysrf$).
%
Next, we present the static semantics of \sysrf by describing
the rules that establish 
each of these judgments.
%
We use $\greybox{\mbox{grey}}$ to highlight the antecedents and rules
%needed for refinements in $\sysrf$.
specific to $\sysrf$.
%

\begin{fullversion}

\mypara{Co-finite Quantification}
%
We define our rules using the co-finite quantification technique
of~\citet{AydemirCPPW08}.
%
This technique enforces a small (but critical) restriction
in the way fresh names are introduced in the antecedents of rules.
%
For example, below we present the standard (on the left)
and our (on the right) rules for type abstraction.
$$
\inferrule*[Right=\tAbsEx]
    {\notmem{\al'}{\tcenv} \;\;
      \hastype{\bind{\al'}{\skind},\tcenv}{\subst{\sexpr}{\al}{\al'}}{\subst{\stype}{\al}{\al'}} }
    {\hastype{\tcenv}{\tabs{\al}{\skind}{\sexpr}}{\polytype{\al}{\skind}{\stype}}}
\qquad\qquad
\inferrule*[Right=\tTAbs]
    {\forall\notmem{\al'}{L}. \;\;
      \hastype{\bind{\al'}{\skind},\tcenv}{\subst{\sexpr}{\al}{\al'}}{\subst{\stype}{\al}{\al'}} }
    {\hastype{\tcenv}{\tabs{\al}{\skind}{\sexpr}}{\polytype{\al}{\skind}{\stype}}}
$$
%
%\NV{Make the rule consistent with the implementation?}
The standard rule \tAbsEx requires the \text{existence} of a
fresh type variable name $\al'$.
%
Instead our co-finite quantification rule
states that the rule holds for any name excluding
a finite set of names $L$. %(here the ones that already appear in $\tcenv$).
%
As observed by~\citet{AydemirCPPW08} this rephrasing
simplifies the mechanization of metatheory
by eliminating the need for renaming lemmas.

\end{fullversion}

\subsection{Well-formedness}
\label{sec:typing:wf}

\mypara{Judgments}
%
The judgment $\isWellFormed{\tcenv}{\stype}{\skind}$
says that the type $\stype$ is well-formed in the environment
$\tcenv$ and has kind $\skind$.
%
The judgment $\isWellFormedE{\tcenv}$ says that the
environment $\tcenv$ is well formed, meaning
that it only binds to well-formed
types.
%
Well-formedness is also used in the (unrefined) system $\sysf$, where
$\isWFFT{\tcenv}{\sftype}{\skind}$
means that the (unrefined) $\sysf$ type
$\sftype$ is well-formed in environment
$\tcenv$ and has kind $\skind$
and $\isWFFE{\tcenv}$ means
that the free type variables
of the
environment $\tcenv$ are bound earlier in the environment.
%
\begin{fullversion}
Well-formedness is not strictly
required for \sysf, but it 
simplifies the mechanization \cite{Remy21}.
\end{fullversion}


\mypara{Rules}
%
\Cref{fig:wf} summarizes the rules
that establish the well-formedness of types
and environments. 
%, with the grey highlighting
%the parts relevant for refinements.
%
Rule \wtBase states that the two closed base
types (\tint and \tbool, 
refined with \ttrue in \sysrf)
are well-formed and have base kind.
%
Similarly, rule \wtVar says that a %n unrefined or trivially refined
type variable $\tvar$ is well-formed with kind $\skind$
so long as $\bind{\tvar}{\skind}$ is bound in the environment.
%
The rule \wtRefn stipulates that a refined base type $\breft{\sbase}{x}{\spred}$
is well-formed with base kind in some environment
if the unrefined base type $\sbase$
has base kind in the same environment and if
the refinement predicate $\spred$ has type $\tbool$
in the environment augmented by binding a fresh variable to type $\sbase$.
Note that if $\sbase \equiv \tvar$ then we can only form the antecedent
$\isWellFormed{\tcenv}{\breft{\tvar}{x}{\ttrue}}{\skbase}$
when $\bind{\tvar}{\skbase} \in \tcenv$ (rule \wtVar),
which prevents us from refining star-kinded type variables.
%\NV{Do you mean that otherwise the rule var will be used?}
%
\textit{To break a circularity}\NV{check ref in the link in the overview pic}
in which well-formedness judgments 
appear in the antecedents  of
typing judgments and a typing judgment
appears in the antecedents
of \wtRefn, we use the $\sysf$
judgment to check that $\spred$ has
type $\tbool$.
%
\begin{fullversion}
  Our rule \wtFunc states that a function
  type $\functype{x}{t_x}{t}$ is well-formed
  with star kind in some environment $\tcenv$
  if both type $t_x$ is well-formed (with any kind)
  in the same environment and type $t$
  is well-formed (with any kind) in the
  environment $\tcenv$ augmented by binding
  a fresh variable to $t_x$.
  %
  Rule \wtExis states that an existential
  type $\existype{x}{t_x}{t}$ is well-formed
  with some kind $\skind$ in some environment
  $\tcenv$ if both type $t_x$ is well-formed
  (with any kind) in the same environment
  and type $t$ is well-formed with kind
  $\skind$ in the environment $\tcenv$
  augmented by binding a fresh variable
  to $t_x$.
  %
  Rule \wtPoly establishes that a
  polymorphic type $\polytype{\tvar}{\skind}{\stype}$
  has star kind in environment $\tcenv$ if the
  inner type $\stype$ is well-formed (with any kind)
  in environment $\tcenv$ augmented by binding a fresh
  type variable $\tvar$ to kind $\skind$.
\end{fullversion}
%
Finally, rule \wtKind simply states that if a type $\stype$
is well-formed with base kind in some environment, then
it is also well-formed with star kind. % in that environment.
This rule is required by our metatheory to convert base to star
kinds in type variables.
%Rule \wtKind is needed in our formalism because we chose
%that a star kinded variable may only be instantiated with
%a star kinded type in order to simplify other aspects of our
%metatheory.
%\NV{Why would we need to convert from base to star kind?}
%\NV{This rules does the opposite, check again}

\begin{conference}
As for environments, the empty environment
is well-formed.
A well-formed environment remains well-formed 
after binding a fresh term or type variable to \resp any well-formed type or kind. 
\end{conference}
\begin{fullversion}
As for environments, rule \wfeEmp states that the empty environment
is well-formed. Rule \wfeBind says that a well-formed environment
$\tcenv$ remains well-formed after binding a fresh variable $x$ to any
type $t_x$ that is well-formed in $\tcenv$.
%
Finally rule \wfeTBind states that a well-formed environment remains
well-formed after binding a fresh type variable to any kind.
\end{fullversion}


\begin{figure}%[t!]
%
%{\small
\begin{mathpar}
\judgementHead{Well-formed Type}{\isWellFormed{\tcenv}{\stype}{\skind}}

%%%%%%%%%%%% WELL-FORMEDNESS %%%%%%%%%%%%%
%
\inferrule% *[Right=\wtBase]
    {\sbase \in \{\tbool, \tint\}}
    {\isWellFormed{\tcenv}{\sbase\greybox{\!\breft{}{x}{\ttrue}}\!}{\skbase}}
    {\wtBase}
\quad
\inferrule% *[Right=]
    {\bind{\al}{\skind} \in \tcenv}
    {\isWellFormed{\tcenv}{\al\greybox{\!\breft{}{x}{\ttrue}}\!}{\skind}}
    {\wtVar} 
\quad
\inferrule% *[Right=]
    { \isWellFormed{\tcenv}{t}{\skbase} }
    { \isWellFormed{\tcenv}{t}{\skstar} }
    {\wtKind} 

%    
\greybox{\inferrule% *[Right=\;\;\wtRefn]
    { \isWellFormed{\tcenv}{\breft{b}{x}{\ttrue}}{\skbase} \\\\
      \forall\notmem{y}{\tcenv}.
      \hasftype{\bind{y}{b}, \forgetreft{\tcenv}}{\subst{p}{x}{y}}{\tbool}
    }
    {\isWellFormed{\tcenv}{\breft{b}{x}{p}}{\skbase}}
    {\wtRefn} }

\inferrule% *[Right=\wtFunc]
    { \isWellFormed{\tcenv}{t_x}{\skind_x} \\\\
      \greybox{\forall\notmem{y}{\tcenv}.} 
      \isWellFormed{\greybox{\bind{y}{t_x},}\tcenv}{t\greybox{\!\subst{}{x}{y}}}{\skind}
    }
    { \isWellFormed{\tcenv}{\greybox{x\!}\!\functype{}{t_x}{t}}{\skstar} }
    {\wtFunc} 

\greybox{\inferrule% *[Right=\;\;\wtExis]
    { \isWellFormed{\tcenv}{t_x}{\skind_x} \;\;
      \forall\notmem{y}{\tcenv}. \;
      \isWellFormed{\bind{y}{t_x},\tcenv}{\subst{t}{x}{y}}{\skind}
    }
    { \isWellFormed{\tcenv}{\existype{x}{t_x}{t}}{\skind} }
    {\wtExis} }

\inferrule% *[Right=]
    { \forall\notmem{\al'}{\tcenv}. \;\;
      \isWellFormed{\bind{\al'}{k}, \tcenv}{\subst{t}{\al}{\al'}}{\skind_t}
    }
    { \isWellFormed{\tcenv}{\polytype{\al}{\skind}{t}}{\skstar} }
    {\wtPoly}
%
\end{mathpar}
%}


%{\small
\begin{mathpar}
\judgementHead{Well-formed Environment}{\isWellFormedE{\tcenv}}

%
\inferrule% *[Right=\wfeEmp]
    { }
    { \isWellFormedE{\varnothing} }
    {\wfeEmp}
%
\quad
%
\inferrule% *[Right=\wfeBind]
    {\isWellFormed{\tcenv}{\stype_x}{\skind_x} \quad
     \isWellFormedE{\tcenv} \quad
     \notmem{x}{\tcenv}
    }
    { \isWellFormedE{\bind{x}{\stype_x}, \tcenv} }
    {\wfeBind}
%
\quad
%
\inferrule% *[Right=\wfeTBind]
    {\isWellFormedE{\tcenv} \quad
     \notmem{\al}{\tcenv}
    }
    { \isWellFormedE{\bind{\al}{\skind}, \tcenv} }
    {\wfeTBind}
\end{mathpar}
%}
\vspace{-0.00cm}
\caption{Well-formedness of types and environments. The rules for
  $\sysf$ exclude the grey boxes.}
\label{fig:wf}
\label{fig:wfe}
\vspace{-0.00cm}
\end{figure}


% \NV{Commented out kinded polymorphism as is already explained in the overview}
\begin{comment}
\mypara{Kinded (Refinement) Polymorphism}
%
\RJ{fix with OVERVIEW}\NV{all this is pushed to overview now}
%
Recall that \sysrf features two kinds:
\emph{base} ($\skbase$) and \emph{star} ($\skstar$).
%
Only base types \tbool and \tint (and existential
quantifications of them) can be refined in our syntax;
we do not permit refinements for function types
and polymorphic types.
%
This kind system allows us to keep track of types
eligible for refinement in the presence of type variables.
%
A type variable of base kind is eligible for refinements,
but one with star kind could be instantiated to any type
and so may not be refined.
%
For example, consider the polymorphic function definition in code
\ha{let max = \x y -> if x > y then x else y}.
What postconditions can we state about this function?
If $\vv$ is the return value, then we know that
$x \leq \vv$ and $y \leq \vv$. The type for the function is then
$\polytype{\al}{\skbase}{\functype{x}{\al}{\functype{y}{\al}
  {\breft{\al}{\vv}{x \leq \vv \wedge y \leq \vv}}}}$.
This type is well-formed due \wtRefn: we can form the judgment
$\isWellFormed{\bind{y}{\al},\bind{x}{\al}, \bind{\al}{\skbase}}
  {\breft{\al}{\vv}{x\leq\vv \wedge y\leq\vv}}{\skbase}$
only because $\bind{\al}{\skbase}$ appears in the environment.
%
\wtPoly then requires that we quantify over
$\bind{\al}{\skbase}$.
%
The type system will then reject usage like \ha{max f1 f2}
on functions because \ha{max} is only polymorphic over base types.
\end{comment}

% For example, by the typing rule \tVar below
% we can establish the judgment
% $\hastype{\bind{x}{(\polytype{\al}{\skstar}{\al})} }
%          {x}{(\polytype{\al}{\skstar}{\al})}$
% because by rule \wtPoly we know that
% $\isWellFormed{\varnothing}{(\polytype{\al}{\skstar}{\al})}{\skstar}$
% %
% Then by rule \tTApp below,
% for \emph{any} well-formed type
% $\isWellFormed{\bind{x}{(\polytype{\al}{\skstar}{\al})}}{\stype}{\skstar}$
% we have the judgment
% $\hastype{\bind{x}{(\polytype{\al}{\skstar}{\al})}}
%          {\tyapp{x}{\stype}}{\subst{\al}{\al}{\stype}}$.
%
% In particular, $\stype$ could a function type or polymorphic type,
% and so any refinement allowed on $\al$ would have to apply to such
% a type $\stype$.
% %
% Note, however, note that we cannot establish
% $\isWellFormed{\varnothing}{(\polytype{\al}{\skstar}{\breft{\al}{\vv}{\spred}})}{\skstar}$
% for any non-trivial $\spred$,
% and so we cannot establish
% $\hastype{\bind{x}{(\polytype{\al}{\skstar}{\breft{\al}{\vv}{\spred}})}}
%          {\tyapp{x}{t}}{\subst{\breft{\al}{\vv}{\spred}}{\al}{\stype}}$
% for types $\stype$ with arbitrary kind.
% This is prevented by our keeping track of kinds for type variables
% in the binding environment.
% %\RJ{explain how rules establish the above}

\subsection{Typing}
\label{sec:typing:typ}

The judgment $\hastype{\tcenv}{\sexpr}{\stype}$ states
that the term $\sexpr$ has type $\stype$ in the context of
environment $\tcenv$.
%
We write $\hasftype{\tcenv}{\sexpr}{\sftype}$
to indicate that term $\sexpr$ has the (unrefined)
$\sysf$ type $\sftype$ in the (unrefined) context
$\tcenv$.
%
\Cref{fig:typing} summarizes
the rules that establish typing for both $\sysf$ and
$\sysrf$, with grey %boxes
%indicating extensions needed
for the $\sysrf$ extensions.

\begin{figure}
%  {\small
  \begin{mathpar}             %%%%%%%%%%%%% TYPING %%%%%%%%%%%%%%%%%%
  \judgementHead{Typing}{\hastype{\tcenv}{\sexpr}{\stype}} \\
        \inferrule% *[Right=\tPrim]
        { % \ty{\sconst} = \stype
        }{\hastype{\tcenv}{\sconst}{\ty{\sconst}}}
        {\tPrim}
    \quad
        \inferrule% *[Right=\tVar]
        {\bind{x}{\stype} \in \tcenv  \\\\
         \greybox{\isWellFormed{\tcenv}{\stype}{\skind}}}
        {\hastype{\tcenv}{x}{\greybox{\self{\whitebox{\stype}}{x}{\skind}}}}
        {\tVar}
    \quad
    \inferrule%*[Right=\tAnn]
    {\hastype{\tcenv}{\sexpr}{\stype} \\\\
     \greybox{\isWellFormed{\tcenv}{\stype}{\skind}}}
    {\hastype{\tcenv}{\tyann{\sexpr}{\stype}}{\stype}}
    {\tAnn}
        \quad
        \greybox{
          \inferrule% *[Right=\;\;\tSub]
          {
            \isWellFormed{\tcenv}{t}{\skind}\\\\
            \hastype{\tcenv}{\sexpr}{s} \\
           \isSubType{\tcenv}{s}{t}
           }
          {\hastype{\tcenv}{\sexpr}{t}}
          {\tSub}
        }
        \\
    %
    \inferrule% *[Right=\tApp]
    {
      \hastype{\tcenv}{\sexpr_x}{\stype_x}\\\\
      \hastype{\tcenv}{\sexpr}{\greybox{x\!}\!\functype{}{\stype_x}{\stype}}
    }
    {\hastype{\tcenv}{\app{\sexpr}{\sexpr_x}}{\greybox{\existype{x}{\stype_x}{\whitebox{\stype}}}}}
    {\tApp} 
    \quad
    \inferrule%*[Right=\tAbs]
        { 
          \isWellFormed{\tcenv}{\stype_x}{\skind_x}\\\\
          \forall\notmem{y}{\tcenv}.
          \hastype{\bind{y}{\stype_x},\tcenv}{\subst{\sexpr}{x}{y}}{\greybox{\subst{\whitebox{\stype}}{x}{y}}} 
         }
        {\hastype{\tcenv}{\vabs{x}{\sexpr}}{\greybox{x\!}\!\functype{}{\stype_x}{\stype}}}
        {\tAbs}
    %
    \quad
    %
        \inferrule% *[Right=\tTAbs]
        {\forall\notmem{\al'}{\tcenv}.\\\\
        \hastype{\bind{\al'}{\skind},\tcenv}{\subst{\sexpr}{\al}{\al'}}{\subst{\stype}{\al}{\al'}}
        }
        {\hastype{\tcenv}{\tabs{\al}{\skind}{\sexpr}}{\polytype{\al}{\skind}{\stype}}}
        {\tTAbs}
        \and
    %
        \inferrule% *[Right=\tTApp]
        {
          \isWellFormed{\tcenv}{t}{\skind}\\\\
          \hastype{\tcenv}{\sexpr}{\polytype{\al}{\skind}{s}}
        }
        {\hastype{\tcenv}{\tyapp{\sexpr}{t}}{\subst{s}{\al}{t}}}
        {\tTApp}
    \quad    
    %
        \inferrule% *[Right=\tLet]
        {\hastype{\tcenv}{\sexpr_x}{\stype_x} \\
        \greybox{\isWellFormed{\tcenv}{\stype}{\skind}}\\\\
        \forall\notmem{y}{\tcenv}.
        \hastype{\bind{y}{\stype_x},\tcenv}{\subst{\sexpr}{x}{y}}
             {\stype\greybox{\!\subst{}{x}{y}}}
         }
        {\hastype{\tcenv}{\eletin{x}{\sexpr_x}{\sexpr}}{\stype}}
        {\tLet}
    \quad    
    %
        \inferrule% *[Right=\tIf]
        {\hastype{\tcenv}{\sexpr}{\greybox{\breft{\whitebox{\tbool}}{x}{p}}} \\
        \greybox{\isWellFormed{\tcenv}{\stype}{\skind}}\\\\
        \greybox{\forall\notmem{y}{\tcenv}.}
        \hastype{\greybox{\bind{y}{\breft{\tbool}{x}{p \wedge x}},}\tcenv}
             {\sexpr_1}{\stype}\\\\
        \greybox{\forall\notmem{y}{\tcenv}.}
        \hastype{\greybox{\bind{y}{\breft{\tbool}{x}{p \wedge \neg x}},}\tcenv}
             {\sexpr_2}{\stype}
         }
        {\hastype{\tcenv}{\eif{\sexpr}{\sexpr_1}{\sexpr_2}}{\stype}}
        {\tIf}        
    \end{mathpar}
%  }
\vspace{-0.00cm}
\caption{Typing rules.
The judgment $\hasftype{\tcenv}{\sexpr}{\sftype}$ is defined by excluding the grey boxes.}\label{fig:t}\label{fig:typing}
\vspace{-0.00cm}
\end{figure}


\mypara{Typing Primitives}
%
The type of a built-in primitive $\sconst$ is given by
the function $\ty{\sconst}$, which is defined for every
constant of our system. Below we present essential
examples of the $\ty{\sconst}$ definition.
%
{\small
$$\begin{array}{rclrcl}
\ty{\ttrue} & \defeq & \breft{\tbool}{x}{x = \ttrue} &
%&  \ty{\tfalse} & \defeq& \breft{\tbool}{x}{x = \tfalse} \\
%\ty{3} & \defeq& \breft{\tint}{x}{x = 3} \\
%&  \ty{n} & \defeq& \breft{\tint}{x}{x = n} \\
 \ty{\wedge} & \defeq & \functype{x}{\tbool}{\functype{y}{\tbool}{\breft{\tbool}{v}{v = x \wedge y}}} \\
%  \ty{\neg} & \defeq & \functype{x}{\tbool}{\breft{\tbool}{y}{y = \neg x}} \\
\ty{3} & \defeq& \breft{\tint}{x}{x = 3} &
\ty{\leq} & \defeq & \polytype{\al}{\skbase}{\functype{x}{\al}{\functype{y}{\al}{\breft{\tbool}{v}{v = (x \leq y)}}}} \\
\ty{m\!\!\leq} & \defeq & \functype{y}{\tint}{\breft{\tbool}{v}{v = (m \leq y)}} & 
% \ty{m\!\!\leq} \defeq&\; \functype{n}{\tint}{\breft{\tbool}{v}{v = (m \leq n)}} \\
\ty{=} & \defeq & \polytype{\tvar}{\skbase}{\functype{x}{\al}{\functype{y}{\al}{\breft{\tbool}{v}{v = (x = y)}}}}
\end{array}$$
}
We note that the $=$ used in the refinements is the polymorphic
equals with type applications elided.
%
Further, we use $m\!\!\leq$ to represent
an arbitrary member of the infinite family
of primitives $0\!\!\leq,\, 1\!\!\leq,\, 2\!\!\leq,\ldots$.
%
For \sysf we erase the refinements
using $\forgetreft{\ty{\sconst}}$.
%
The rest of the definition is similar.

Our choice to make the typing and reduction
of constants external to our language,
\ie given by the functions
$\ty{\sconst}$ and $\tc{\sconst}$,
makes our system easily extensible with further constants, 
including a terminating \texttt{fix} constant to encode induction.
%
The requirement, for soundness, is that
these two functions % on constants
together satisfy the following four conditions.
%
\begin{requirement}(Primitives) \label{lem:prim-typing}
For every primitive $c$,
\begin{enumerate}
\item If $\ty{\sconst} = \breft{\sbase}{x}{\spred}$, then
  $\isWellFormed{\varnothing}{\ty{\sconst}}{\skbase}$ and
  $\imply{\varnothing}{\ttrue}{\subst{\spred}{x}{\sconst}}$.
\item If $\ty{\sconst} = \functype{x}{\stype_x}{\stype}$ or
         $\ty{\sconst} = \polytype{\al}{\skind}{\stype}$, then
         $\isWellFormed{\varnothing}{\ty{\sconst}}{\skstar}$.
\item If ${\ty{\sconst}} = \functype{x}{\stype_x}{\stype}$,
      then for all $v_x$ such that
     $\hastype{\varnothing}{v_x}{\stype_x}$,
     % $\delta(\sconst,v)$ is defined and
    $\hastype{\varnothing}{\delta(\sconst,v_x)}{\subst{\stype}{x}{v_x}}$.
\item If ${\ty{\sconst}} = \polytype{\al}{\skind}{\stype}$,
      then for all $\stype_\al$ such that
     $\isWellFormed{\varnothing}{\stype_\al}{\skind}$,
     % $\delta_T(\sconst,\stype)$ is defined and we have
    $\hastype{\varnothing}{\delta_T(\sconst,\stype_\al)}{\subst{\stype}{\al}{\stype_\al}}$.
\end{enumerate}
\end{requirement}

Theorem 3 of~\cite{Vazou14} proves that 
a terminating \texttt{fix} constant satisfies
requirement~\ref{lem:prim-typing}. 

To type constants, rule \tPrim gives the type
$ty(\sconst)$ to any built-in
primitive $\sconst$, in any context.
%
\begin{fullversion}
The typing rules for boolean and integer
constants are included in \tPrim.
\end{fullversion}

\mypara{Typing Variables with Selfification}
%
Rule \tVar establishes that any variable $x$ that
appears as $\bind{x}{\stype}$ in environment $\tcenv$
can be given the \emph{selfified} type \cite{Ou2004}
$\self{\stype}{x}{\skind}$ provided that
$\isWellFormed{\tcenv}{\stype}{\skind}$.
%
This rule is crucial in practice,
to enable path-sensitive ``occurrence'' typing \cite{Tob08},
where the types of variables are refined by control-flow guards.
%
\NV{Can we use an example from the overview?}
\NV{go prime require self for if condition}
For example, suppose we want to establish
$\hastype{\bind{\al}{\skbase}}{(\vabs{x}{x})}{\functype{x}{\al}{\breft{\al}{y}{x=y}}}$,
and not just $\hastype{\bind{\al}{\skbase}}{(\vabs{x}{x})}{\funcftype{\al}{\al}}$.
%
The latter would result
if \tVar merely stated that
$\hastype{\tcenv}{x}{\stype}$
whenever $\bind{x}{\stype} \in \tcenv$.
%
Instead, we strengthen the \tVar rule to
be \emph{selfified}.
%:
%
Informally, to get information about $x$
into the refinement level, we need to say
that $x$ is constrained to elements of
type $\al$ that are equal to $x$ itself.
%
In order to express the exact type of
variables, below we define the ``selfification''
function that strengthens a refinement
with the condition that a value is equal
to itself.
%
Since abstractions do not admit equality,
we only selfify the base types and the existential
quantifications of them.
%
$$\begin{array}{r@{\hskip 0.03in}c@{\hskip 0.03in}l@{\hskip 0.2in}r@{\hskip 0.03in}c@{\hskip 0.03in}l}
  \self{\existype{z}{t_z}{t}}{x}{\skind} & \defeq & \existype{z}{t_z}{\self{t}{x}{\skind}} & 
  \self{\breft{\sbase}{z}{p}}{x}{\skbase} & \defeq & \breft{\sbase}{z}{p \wedge z = x} \\
  \self{\functype{x}{t_x}{t}}{\_}{\_} & \defeq & \functype{x}{t_x}{t} &
  \self{\breft{\sbase}{z}{p}}{x}{\skstar} & \defeq & \breft{\sbase}{z}{p} \\ 
  \self{\polytype{\al}{k}{t}}{\_}{\_} & \defeq & \polytype{\al}{k}{t}
% \self{\stype}{x}{\skstar}               & \defeq & \stype &
%  & &
\end{array}$$

\mypara{Typing Applications with Existentials}
%
Our rule \tApp states the conditions for typing
a term application $\app{\sexpr}{\sexpr_x}$.
%
Under the same environment,
we must be able to type $\sexpr$ at
some function type $\functype{x}{\stype_x}{\stype}$ and
$\sexpr_x$ at $\stype_x$. Then we can give $\app{\sexpr}{\sexpr_x}$
the existential type $\existype{x}{\stype_x}{\stype}$.
%
The use of existential types in rule \tApp
is one of the distinctive features of our language
and was introduced by~\citet{Knowles09}.
%
As overviewed in \S~\ref{overview:exists},
we chose this form of \tApp over the conventional
form of $\hastype{\tcenv}{\app{\sexpr}{\sexpr_x}}{\subst{\stype}{x}{\sexpr_x}}$
because our version prevents
the substitution of arbitrary expressions (\eg functions and type abstractions)
into refinements.
%
As an alternative, we could have used ANF (A-Normal Form~\cite{Flanagan93}),
but our metatheory would be more complex
since ANF is not preserved under the small step operational semantics.

\mypara{Other Typing Rules}
%
\begin{fullversion}
    Rule \tAbs says that we can type a lambda abstraction
    $\vabs{x}{\sexpr}$ at a function type $\functype{x}{\stype_x}{\stype}$
    whenever $\stype_x$ is well-formed and the body $\sexpr$
    can be typed at $\stype$ in the environment augmented by binding
    a fresh variable to $\stype_x$.  
\end{fullversion}
%
Our rule \tTApp states that whenever a term $\sexpr$ has polymorphic
type $\polytype{\al}{\skind}{s}$, then for any well-formed type $\stype$
with kind $\skind$, %in the same environment, 
we can give the type
$\subst{s}{\al}{\stype}$ to the type application $\tyapp{\sexpr}{\stype}$.
%
For the \sysf variant of \tTApp, we erase the refinements (via $\forgetreft{\stype}$)
before checking well-formedness and performing the substitution.
%
\begin{fullversion}
    The rule \tTAbs establishes that a type-abstraction $\tabs{\al}{\skind}{\sexpr}$
    can be given a polymorphic type $\polytype{\al}{\skind}{\stype}$ in some % environment
    $\tcenv$ whenever $\sexpr$ can be given the well-formed type $\stype$
    in %the environment 
    $\tcenv$ augmented by binding a fresh type variable
    to kind $\skind$.
    %
    Next, rule $\tLet$ states that an expression $\eletin{x}{\sexpr_x}{e}$ has type
    $\stype$ in some environment whenever $\stype$ is well-formed, $\sexpr_x$ can
    be given some type $\stype_x$, and the body $\sexpr$ can be given type $\stype$
    in the environment augmented by binding a fresh variable to $\stype_x$.
\end{fullversion}
%
Rule $\tAnn$ establishes that an explicit annotation $\tyann{\sexpr}{\stype}$
indeed has type $\stype$ when the underlying %expression 
$\sexpr$ has type $\stype$ and $\stype$ is well-formed.
The \sysf version of the rule erases the refinements and uses
$\forgetreft{\stype}$.
%
Rule $\tIf$ states that a conditional expression $\eif{\sexpr}{\sexpr_1}{\sexpr_2}$
has the type $\stype$ when the guard $\sexpr$ can be 
given type $\tbool$ refined by $p$ and $\sexpr_1$ (\resp $\sexpr_2$)
can be given type $\stype$ in the environment $\tcenv$ 
augmented by the knowledge we have about the type 
and semantics of the guard $\sexpr$.
The extension of the environment $\tcenv$
with a fresh variable that captures the semantics of the guard
when checking the two paths is critical to 
permit path-sensitive reasoning.
%
Finally, rule \tSub tells us that we can exchange a subtype $s$
for a supertype $t$ in a judgment $\hastype{\tcenv}{\sexpr}{t}$
provided 
$t$ is well-formed and
$\isSubType{\tcenv}{s}{t}$, which we present next.
%
\subsection{Subtyping}
\label{sec:typing:sub}

The \emph{subtyping} judgment ${\isSubType{\tcenv}{s}{t}}$, 
defined in \Cref{fig:s}, 
stipulates that the type $s$ is a subtype of
the type $t$ in the environment $\tcenv$
and is used in the subsumption typing rule \tSub (of~\Cref{fig:typing}).

\begin{figure}
\judgementHead{Subtyping}{\isSubType{\tcenv}{s}{t}}

\begin{mathpar}   %%%%%%%%%%%%%%%%%% SUBTYPING %%%%%%%%%%%%%%%%%%
  \inferrule*[Right=\sFunc]
  { \isSubType{\tcenv}{t_{x2}}{t_{x1}} \quad
    \forall\notmem{y}{\tcenv}. \quad
    \isSubType{\bind{y}{t_{x2}},\tcenv}{\subst{t_1}{x}{y}}{\subst{t_2}{x}{y}} }
  {\isSubType{\tcenv}{\functype{x}{t_{x1}}{t_1}}{\functype{x}{t_{x2}}{t_2}}} \\
%
\and
%
  \inferrule*[Right=\sWitn]
  { \hastype{\tcenv}{\sval_x}{\stype_x} \quad
    \isSubType{\tcenv}{\stype}{\subst{\stype'}{x}{\sval_x}}}
  {\isSubType{\tcenv}{\stype}{\existype{x}{\stype_x}{\stype'}}}
%
\and
%
  \inferrule*[Right=\sBind]
  { \forall\notmem{y}{\free{\stype}\cup \tcenv}. \quad
    \isSubType{\bind{y}{\stype_x},\tcenv}{\subst{\stype}{x}{y}}{\stype'}
%    \quad
%    \lc{\stype'}
  }
  {\isSubType{\tcenv}{\existype{x}{\stype_x}{\stype}}{\stype'}} \\
%
\and
%
  \inferrule*[Right=\sPoly]
  { \forall\notmem{\al'}{\tcenv}. \quad
    \isSubType{\bind{\al'}{\skind},\tcenv}{\subst{\stype_1}{\al}{\al'}}{\subst{\stype_2}{\al}{\al'}} }
  {\isSubType{\tcenv}{\polytype{\al}{\skind}{\stype_1}}{\polytype{\al}{\skind}{\stype_2}}}
%
\and
%
  \inferrule*[Right=\sBase]
  {\forall\notmem{y}{\tcenv}. \quad
    \imply{\bind{y}{{\sbase}},\tcenv}{\subst{p_1}{x}{y}}{\subst{p_2}{x}{y}} }
  {\isSubType{\tcenv}{\breft{\sbase}{x}{\spred_1}}{\breft{\sbase}{x}{\spred_2}}}
\end{mathpar}
\vspace{-0.00cm}
\caption{Subtyping Rules.}
\label{fig:s}
\label{fig:subtyping}
\label{fig:ent}
\vspace{-0.00cm}
\end{figure}
\mypara{Subtyping Rules}
\begin{fullversion}  
    The rule \sFunc states that one function type $\functype{x_1}{t_{x1}}{t_1}$
    is a subtype of another function type $\functype{x_2}{t_{x2}}{t_2}$ in a
    given environment $\tcenv$ when both $t_{x2}$ is a subtype of $t_{x1}$
    and $t_1$ is a subtype of $t_2$ when we augment $\tcenv$ by
    binding a fresh variable to type $t_{x2}$.
    As usual, function subtyping is contravariant
    in the input type and covariant in the outputs.
\end{fullversion}
%
Rules \sBind and \sWitn establish subtyping for existential
types \cite{Knowles09}, \resp when the existential
appears on the left or right.
%
Rule \sBind allows us to exchange a universal quantifier
(a variable bound to some type $\stype_x$ in the environment)
for an existential quantifier.
%
If we have a judgment of the form
$\isSubType{\bind{y}{\stype_x},\tcenv}{\subst{\stype}{x}{y}}{\stype'}$
where $y$ does \emph{not} appear free in either $\stype'$ or in the context $\tcenv$,
%$\stype'$ is locally closed, which means that all its deBruijn
%indices point to binders within $\stype'$ (denoted $\lc{t'}$),
% \NV{We should not talk about locally closed since we do not model deBruin.
%     We could instead add a well formedness requirement, but I assume you have a proof
%     that everything in the right hand side of a subtyping is well formed?}
then we can conclude that $\existype{x}{\stype_x}{\stype}$
is a subtype of $\stype'$.
%
Rule \sWitn states that if type $\stype$ is a subtype of
$\subst{\stype'}{x}{\sval_x}$ for some value $\sval_x$
of type $\stype_x$, then we can discard the specific
\emph{witness} for $x$ and quantify existentially to
obtain that $\stype$ is a subtype of $\existype{x}{\stype_x}{\stype'}$.
%
\begin{fullversion} 
    Rule \sPoly states that one polymorphic type
    $\polytype{\al}{\skind}{\stype_1}$
    is a subtype of another polymorphic type
    $\polytype{\al}{\skind}{\stype_2}$
    in some environment $\tcenv$, when
    $\stype_1$ is a subtype of $\stype_2$ in the
    environment $\tcenv$ augmented  by binding
    a fresh type variable to kind $\skind$.
\end{fullversion}

Refinements enter the scene in the rule \sBase which
specifies
%uses implication to specify
that a refined base type $\breft{\sbase}{x}{p_1}$
is a subtype of another $\breft{\sbase}{x}{p_2}$
in context $\tcenv$ when $p_1$ \emph{implies} $\spred_2$
in the environment $\tcenv$ augmented by binding
a fresh variable to the unrefined type $\sbase$.
%
%Next, we describe how we formalized implication.

\subsection{Implication}
\label{sec:typing:implication}

The \emph{implication} judgment $\imply{\tcenv}{\spred_1}{\spred_2}$
states that the implication $\spred_1 \Rightarrow \spred_2$
holds under the assumptions captured by the context $\tcenv$.
%
In refinement type implementations~\cite{newfstar,Seidel14}, this relation
is implemented as an external automated (usually SMT) solver.
%
Since external solvers are not easy to encode in mechanized proofs,
we follow an approach that decouples the mechanization from the implementation. 
Concretely, 
first we define the interface of the implication (\S~\ref{sec:typing:implication:interface})
that precisely captures all the requirements that the implication 
judgement should satisfy to establish the soundness of $\sysrf$. 
Then, we define two alternative implementations of the interface: 
a logical implementation (\S~\ref{sec:typing:implication:logical})
that is used in refinement type implementations 
and a denotational implementation (\S~\ref{sec:typing:implication:denotational})
that we used to complete our mechanized proof. 


\subsubsection{Implication's Interface}
\label{sec:typing:implication:interface}
In our mechanization, 
following~\citet{LehmannTanter}, 
we encode implication
as an axiomatized judgment that satisfies the
requirements below.


\begin{requirement}[Implication Interface]\label{lem:implication}
  The implication relation satisfies the below statements:
  \begin{enumerate}
      \item (Reflexivity) $\imply{\tcenv}{\spred}{\spred}$.
      \item (Transitivity) If $\imply{\tcenv}{\spred_1}{\spred_2}$
        and $\imply{\tcenv}{\spred_2}{\spred_3}$, then
        $\imply{\tcenv}{\spred_1}{\spred_3}$.
      \item (Faithfulness) $\imply{\tcenv}{\spred}{\ttrue}$.
      \item (Introduction) If $\imply{\tcenv}{\spred_1}{\spred_2}$
        and $\imply{\tcenv}{\spred_1}{\spred_3}$, then
        $\imply{\tcenv}{\spred_1}{\csand{\spred_2}{\spred_3}}$.
      \item (Conjunction)
      $\imply{\tcenv}{\csand{\spred_1}{\spred_2}}{\spred_1}$ and
      $\imply{\tcenv}{\csand{\spred_1}{\spred_2}}{\spred_2}$.
      \item (Repetition)
      $\imply{\tcenv}{\csand{\spred_1}{\spred_2}}
                     {\csand{\spred_1}{\csand{\spred_1}{\spred_2}}}$.
      \item (Evaluation) If $\evalsTo{\spred_1}{\spred_2}$, 
        then $\imply{\tcenv}{\spred_1}{\spred_2}$
        and  $\imply{\tcenv}{\spred_2}{\spred_1}$.
      \item (Narrowing) If
      $\imply{\tcenv_1,\bind{x}{t_x},\tcenv_2}{\spred_1}{\spred_2}$
      and $\isSubType{\tcenv_2}{s_x}{t_x}$, then
      $\imply{\tcenv_1,\bind{x}{s_x},\tcenv_2}{\spred_1}{\spred_2}$.
      \item (Weaken) If
      $\imply{\tcenv_1,\tcenv_2}{\spred_1}{\spred_2}$,
       $a,x\not\in\tcenv$, then
      $\imply{\tcenv_1,\bind{x}{t_x},\tcenv_2}{\spred_1}{\spred_2}$ and
      $\imply{\tcenv_1,\bind{a}{k},\tcenv_2}{\spred_1}{\spred_2}$.
      \item (Subst I) If
      $\imply{\tcenv_1,\bind{x}{t_x},\tcenv_2}{\spred_1}{\spred_2}$
      and $\hastype{\tcenv_2}{v_x}{t_x}$, then
      $\imply{\subst{\tcenv_1}{x}{v_x},\tcenv_2}{\subst{\spred_1}{x}{v_x}}
                                             {\subst{\spred_2}{x}{v_x}}$.
      \item (Subst II) If $\imply{\tcenv_1,\bind{a}{k},\tcenv_2}{\spred_1}{\spred_2}$
      and $\isWellFormed{\tcenv_2}{t}{k}$, then
      $\imply{\subst{\tcenv_1}{a}{t},\tcenv_2}{\subst{\spred_1}{a}{t}}
      {\subst{\spred_2}{a}{t}}$.
      \item (Strengthening) If
      $\imply{\bind{y}{\breft{\sbase}{x}{q},\tcenv}}{\spred_1}{\spred_2}$,
      then $\imply{\bind{y}{\sbase},\tcenv}{\csand{\subst{q}{x}{y}}{\spred_1}}
                  {\csand{\subst{q}{x}{y}}{\spred_2}}$.
  \end{enumerate}
\end{requirement}

\noindent
This interface  precisely explicates
the requirements of
the implication checker 
to establish the soundness of the
entire refinement type system.
%
The first six statements are standard properties of implication. 
Evaluation is used to prove that built-in constants satisfy the~\Cref{lem:prim-typing} 
and the rest, as captured by their name, are required to prove  
the narrowing (\ref{lem:narrowing}), weakening (\ref{lem:weakening}), 
substitution (\ref{lem:subst}) lemmas hold in \sysrf. 

Our requirements are very similar to Assumption 1 of~\cite{Knowles09}. 
Our Strengthening and Subst II cases are required for polymorphism, thus 
they do not appear in~\citet{Knowles09}'s assumption.  
Instead they require Consistency and Exact Quantification. 
We do not require Exact Quantification since our relation captures 
the minimum requirements to prove soundness. 
Instead of explicitly requiring Consistency, 
in \S~\ref{sec:typing:implication:denotational} we define (and mechanize) an implementation, \ie inhabitant, 
of the interface thus show our assumptions are consistent. 


\subsubsection{Logical Implementation (non mechanized)}
\label{sec:typing:implication:logical}
The logical implementation of \imply{\tcenv}{\spred_1}{\spred_2}
checks that the logical implication $\spred_1 \Rightarrow \spred_2$
is valid assuming the refinements of the base types in $\tcenv$:
%
$$
  \inferrule
  {\models_{\texttt{LOGIC}} \wedge\{ \subst{\spred}{\vv}{x} \mid \bind{x}{\breft{\sbase}{\vv}{\spred}} \in \tcenv \} \Rightarrow \spred_1 \Rightarrow \spred_2 }
  {\imply{\tcenv}{\spred_1}{\spred_2}}
  {\iLog}
$$
This encoding is imprecise, since some information is ignored from the environment $\tcenv$, 
but when the language of refinements is decidable, implication checking 
is also decidable and can be efficiently checked by an SMT solver.
\lh, for example, uses this encoding to reduce type checking to decidable 
implications checked by Z3~\cite{z3}, while the soundness 
of this implementation (concretely statement 7 of~\Cref{lem:implication}) 
is hinted by Theorem 2 of~\cite{Vazou14}.
\citet{10.1145/3546196.3550162} defines a mechanization of a 
refinement type system in Agda that uses a similar encoding of implication
where logical implications are checked using Agda's logic. 


\subsubsection{Denotational Implementation (mechanized)}
\label{sec:typing:implication:denotational}
The denotational implementation of \imply{\tcenv}{\spred_1}{\spred_2}
checks that if $\spred_1$ evaluates to \ttrue, so does $\spred_2$. 
%
$$
  \inferrule
  { \forall \clsub\!\in\!\denote{\tcenv}. \
      \evalsTo{\applysubst{\clsub}{\spred_1}}{\ttrue}
      \Rightarrow 
      \evalsTo{\applysubst{\clsub}{\spred_2}}{\ttrue}
  }
  {\imply{\tcenv}{\spred_1}{\spred_2}}
  {\iDen}
$$
The refinements $\spred_1$ and $\spred_2$ are boolean expressions, 
so evaluation uses the operational semantics of~\Cref{fig:opsem}. 
But, they are open expressions with variables bound in $\tcenv$,
so before evaluation we apply the closing substitution $\clsub$
that belongs to the denotation of $\tcenv$, as defined next. 

\begin{figure}
$$\begin{array}{r@{\hskip 0.03in}c@{\hskip 0.03in}l}
\denote{\breft{\sbase}{x}{p}} & \defeq &
  \setcomp{\sval}{\hasftype{\varnothing}{\sval}{\sbase} \,\wedge\, \evalsTo{\subst{p}{x}{\sval}}{\ttrue}} \\
\denote{\functype{x}{\stype_x}{\stype}} & \defeq &
  \setcomp{\sval}{\hasftype{\varnothing}{\sval}{\funcftype{\forgetreft{\stype_x}}{\forgetreft{\stype}}} \,\wedge\, (\forall\, \sval_x \in \denote{\stype_x}.\, \evalsTo{\app{\sval}{\sval_x}}{\sval'} \,{\rm s.t.}\, \sval' \in \denote{\subst{\stype}{x}{\sval_x}} }  \\
\denote{\existype{x}{\stype_x}{\stype}} & \defeq &
  \setcomp{\sval}{(\hasftype{\varnothing}{\sval}{\forgetreft{\stype}}) \,\wedge\,
    ( \exists\, \sval_x \in \denote{\stype_x}.\, \sval\in \denote{\subst{\stype}{x}{\sval_x}} }   \\
\denote{\polytype{\al}{\skind}{\stype}} & \defeq &
  \setcomp{\sval}{(\hasftype{\varnothing}{\sval}{\polytype{\al}{\skind}{\forgetreft{\stype}}}) \,\wedge\,
    (\forall\, \stype_\al.\, (\isWellFormed{\varnothing}{\stype_\al}{\skind}) \Rightarrow
    \evalsTo{\tyapp{\sval}{\stype_\al}}{\sval'}
    \,{\rm s.t.}\, \sval' \in \denote{\subst{\stype}{\al}{\stype_\al}}} \\ \\ 
\denote{\tcenv} & \defeq &
\setcomp{\clsub} %= (x_1 \mapsto v_1,\ldots, x_n \mapsto v_n,  \al_1 \mapsto t_1, \,\ldots,\, \al_m\mapsto t_m)}{
{\forall\, (x:t) \in \tcenv.\, \clsub(x) \in \denote{\applysubst{\clsub}{t}} \;\wedge\;
\forall\, (\al:k) \in \tcenv.\, \isWellFormed{\varnothing}{\clsub(\al)}{k}}.
\end{array}$$
\vspace{-0.2cm}
\caption{Denotations of Types and Environments.}
\label{fig:den}
\vspace{-0.4cm}
\end{figure}

\paragraph{Closing Substitutions}
A \emph{closing substitution} is a sequence
of value bindings to variables:
$
    \clsub = (x_1\mapsto v_1,\,\ldots,\, x_n\mapsto v_n, \al_1 \mapsto t_1, \,\ldots,\, \al_m\mapsto t_m)
    \;\; {\rm with\;\; all}\; x_i,\; \al_j\; {\rm distinct.}
$
%
We write $\clsub(x)$ to refer to $v_i$ if $x = x_i$
and we use $\clsub(\al)$ to refer to $t_j$ if $\al = \al_j$.
%
We define $\applysubst{\clsub}{t}$ to be the type derived from $t$
by substituting for all variables in $\clsub$:
%
$
  \applysubst{\clsub}{t} \defeq \subst{t}{x_1}{v_1}\cdots\subst{}{x_n}{v_n}\subst{}{\al_1}{t_1}\cdots\subst{}{\al_m}{t_m}.
$
%


\paragraph{Denotational Semantics}
\Cref{fig:den} defines the denotations of types and environments.
Following~\citet{flanagan06}, 
each closed type has a denotation $\denote{\stype}$
containing the set of closed values
of the appropriate base type
which satisfy the type's refinement
predicate.
(The denotation of a type variable $\al$
is not defined as we only require
denotations for closed types.)
%
We lift the notion of denotations to environments $\denote{\tcenv}$
as the set of closing substitutions, \ie value
and type bindings for the variables in $\tcenv$, such that the
values respect the denotations of the respective $\tcenv$-bound
types and the types are well formed with respect
to the corresponding kinds.


\paragraph{Revisiting rule~\iDen}
The premise of the rule~\iDen quantifies 
over all closing substitutions in the denotations of the 
typing environment 
(\ie $\forall \clsub\!\in\!\denote{\tcenv}$). 
This quantification has two consequences. 

First, the environment denotation appears in a negative position 
on the premise of the rule. Inspecting~\Cref{fig:den}, 
the environment denotation uses the type denotation, which in turn uses 
type checking, thus rendering a \textit{potential circularity
between type and implication checking} (arrow 6 of~\Cref{fig:dependencies}). 
Because of the negative occurrence, this mutual dependency 
would lead to a non-monotonic and thus non-well defined system. 
To break this circularity, we use $\sysf$'s type checking in the definition 
of type denotations. 

Second, the quantification is over all closing substitutions
which are infinite. For example, 
a typing environment that binds $x$ to an integer 
(\ie $\bind{x}{\tint}\in \tcenv$) 
has infinitely many closing substitutions mapping $x$ to a different 
integer. Thus, the denotational implementation 
cannot be used to implement a decidable type checker. 
%
On the positive side, the denotational implementation
connects implication checking to the operational semantics
thus it is amicable to mechanization.
Concretely, we proved (\S~\ref{sec:coq})
that the denotational implementations satisfies 
the statements of~\Cref{lem:implication}. 
