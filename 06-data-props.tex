\chapter{\lh \& Refined Data Propositions}
\label{ch:data-props}

In Chapter~\ref{ch:implementation} we will present how 
we proved 
\sysrf soundness in \lh. 
%
To do so, we developed \textit{refined data propositions}, 
a novel feature of \lh that made such 
a meta-theoretic proof possible. 
%% NEW
Although \lh had been previously used to prove theorems
inductively about provably terminating user defined 
functions, these extensions were needed to 
reason about potentially non-terminating properties, 
such as the typing judgment of \sysrf.
%%


\section{\lh}
\lh's core proof system is \sysrf, 
that is, it is using the typing judgement 
presented in Figure~\ref{fig:t} to check whether a Haskell program 
satisfies its refinement type annotations.
%
The expression language checked by \lh is GHC's intermediate language 
(\coresyn~\cite{coresyn}) which is a superset of \sysrf that also includes 
literals, datatypes, and coercions. Thus, \lh's typing judgement is extended to 
include these constructs. 
%
To guess the unknown types of Figure~\ref{fig:t} (\ie in the rules \tSub and \tLet)
and make the typing judgment algorithmic,
\lh  implements the refinement type inference algorithm of 
liquid types~\cite{LiquidPLDI08}.
%
To check the implications \lh uses the \iLog 
rule of~\S~\ref{sec:typing:implication:logical}
which are automatically discharged by an SMT solver.
% }

% \newtext{
\paragraph{\lh as a Theorem Prover}
Equipped with the SMT solver, \lh can be used to prove theorems 
over theories known to the SMT solver.
For example, that addition over integers is associative 
and that for every integer there exists a larger one, 
as encoded by the below functions: 
% }
\begin{mcode}
  assoc :: x:Int -> y:Int -> {v:() | x + y == y + x } 
  assoc _ _ = () 
  
  exLg :: x:Int -> (y::Int,{v:() | y > x })
  exLg x = (x+1, ())   
\end{mcode}
%
% \newtext{
These definitions use lambda abstraction and dependent pairs 
to respectively encode the universal and existential quantifiers.
To encode logical terms, such as \texttt{y > x}, they refine the unit type 
with such terms. 
Building upon this idea, \lh has been extensively used 
to prove theorems, using recursive Haskell definitions to 
encode inductive proofs and refinement reflection~\cite{Vazou18}
to allow user-defined terminating functions into the refinement logic. 
Yet, the proving power of \lh was limited because 
only provably terminating functions can be used in the refinement logic
and the proofs were implicitly performed by the SMT solver.  
Thus, the programmer could not inspect the proof terms. 
% }   

\section{Refined Data Propositions}
\label{sec:data-props}
%
Refined data propositions encode \coq-style
inductive predicates in \lh by refining Haskell's 
data types, allowing the programmer to write plain
Haskell functions to provide constructive proofs 
for user-defined propositions.
Here, for exposition, we present the four steps we followed  
in the mechanization of \sysrf to 
define the ``has-type'' proposition
and then use it to type the primitive one. 

\mypara{Step 1: Reifying Propositions as Data}
%
Our first step is to represent the propositions
of interest as plain Haskell data.
%
For example, we can define the following types 
(suffixed \ha{Pr} for ``proposition''):
%
\begin{mcode}
    data HasTyPr   = HasTyPr    Env  Expr Type
    data IsSubTyPr = IsSubTyPr  Env  Type Type
\end{mcode}
%
Thus, \ha{HasTyPr $\Gamma$ e t} and \ha{IsSubTyPr $\Gamma$ s t}
respectively represent the \emph{propositions} 
$\hastype{\Gamma}{e}{t}$ and 
$\isSubType{\Gamma}{s}{t}$,
which say
that \ha{e} can be typed as \ha{t} under environment $\Gamma$ and
that \ha{t} is a subtype of \ha{t'} under $\Gamma$.

\mypara{Step 2: Reifying Evidence as Data}
%
Next, we reify evidence, \ie \emph{derivation trees}
as data by defining Haskell data types with a
\emph{single constructor per derivation rule}.
%
For example, we define the data type \ha{HasTyEv}
to encode the typing rules of~\Cref{fig:t}, 
with constructors that match the names of each rule.
%
\begin{mcode}
  data HasTyEv where
    TPrim :: Env ->Prim ->HasTyEv
    TSub  :: Env ->Expr ->Type ->Type ->HasTyEv ->IsSubTyEv ->HasTyEv
    ...
\end{mcode}
Using these data one can construct derivation trees. 
For instance,  
\ha{TPrim Empty (PInt 1) :: HasTyEv} is the tree that types 
the primitive one under the empty environment. 

\mypara{Step 3: Relating Evidence to its Propositions}
%
Next, we specify the relationship between the evidence and the proposition
that it establishes, via a refinement-level \emph{uninterpreted function}:
%that maps evidence to its corresponding proposition:
%
\begin{mcode}
  measure hasTyEvPr   :: HasTyEv ->HasTyPr
  measure isSubTyEvPr :: IsSubTyEv ->IsSubTyPr
\end{mcode}
%
The above signatures declare that \ha{hasTyEvPr} (resp. \ha{isSubTyEvPr})
is a refinement-level function that maps has-type (resp. is-subtype)
evidence to its corresponding proposition.
%
We can now use these uninterpreted functions to define \emph{type aliases}
that denote well-formed evidence that establishes a proposition.
%
For example, consider the (refined) type aliases
%
\begin{mcode}
  type HasTy   $\Gamma$ e t = {ev:HasTyEv   | hasTyEvPr ev == HasTyPr $\Gamma$ e t }
  type IsSubTy $\Gamma$ s t = {ev:IsSubTyEv | isSubTyEvPr ev == IsSubTyPr $\Gamma$ s t }
\end{mcode}
%
The definition stipulates that the type \ha{HasTy $\Gamma$ e t}
is inhabited by evidence (of type \ha{HasTyEv}) that
establishes the typing proposition \ha{HasTyPr $\Gamma$ e t}.
%
Similarly \ha{IsSubTy $\Gamma$ s t} is inhabited by evidence
(of type \ha{IsSubTyEv}) that establishes the subtyping
proposition \ha{IsSubTyPr $\Gamma$ s t}.
%
Note that the first three steps have only defined separate data types
for propositions and evidence, and \emph{specified} the relationship
between them via uninterpreted functions in the refinement logic.

\mypara{Step 4: Refining Evidence to Establish Propositions}
%
Finally, we \emph{implement} the relationship between
evidence and propositions \emph{refining} the types of the evidence
data constructors (rules) with pre-conditions that require
the rules' premises and post-conditions that ensure
the rules' conclusions.
%
For example, we connect the evidence and proposition for the
typing relation by refining the data constructors for \ha{HasTyEv}
using their respecting typing rule
from ~\Cref{fig:t}.
%
\begin{mcode}
  data HasTyEv where
    TPrim :: $\Gamma$:Env ->c:Prim ->HasTy $\Gamma$ (Prim c) (ty c)
    TSub  :: $\Gamma$:Env ->e:Expr ->s:Type ->t:Type 
          ->HasTy $\Gamma$ e s ->IsSubTy $\Gamma$ s t ->HasTy $\Gamma$ e t
    ...
\end{mcode}
%
The constructors \ha{TPrim} and \ha{TSub} respectively
encode the rules \tPrim and \tSub (with 
well-formedness elided for simplicity).
%
The refinements on the input types, 
which encode the premises of the rules,
are checked whenever these constructors are used. 
The refinement on the 
output type (being evidence of a specific proposition)
is axiomatized to encode the conclusion of the rules.
%
For example, the type for \ha{TSub}
says that ``for all $\Gamma, e, s, t$,
given evidence that $\hastype{\Gamma}{e}{s}$ and $\isSubType{\Gamma}{s}{t}$'',
the constructor returns ``evidence that $\hastype{\Gamma}{e}{t}$''.

\begin{fullversion}

\mypara{Programs as Constructive Proofs}
%
Thus, the constructor refinements crucially ensure that only well-formed pieces of evidence
can be constructed, and simultaneously, precisely track the proposition established
by the evidence.
%
This lets the programmer write plain Haskell terms as constructive proofs, and \lh ensures
that those terms indeed establish the proposition stipulated by their type.
%
For example, the below Haskell term is proof that the literal \ha{1} has the type
$\breft{\Int}{\vv}{\vv = 1}$
%
\begin{mcode}
  oneTy :: HasTy Empty (EPrim (PInt 1)) {v:Int | v == 1}
  oneTy = TPrim Empty (PInt 1)
\end{mcode}
%
If instead, the programmer wrote
\ha{oneTy = TPrim Empty (PInt 2)}, \lh would reject  
this
%the program 
as the modified evidence does not establish
the proposition described in the type.
%
\end{fullversion}

% As an example, below we use the primitive and subtyping constructors
% to construct the witness (equivalently derivation tree) that \ha{1}
% is a positive integer.
% %
% \begin{mcode}
%     onePos :: HasTy Empty (EPrim (PInt 1)) {v:Int | 0 < v}
%     onePos = TSub Empty (EPrim (PInt 1))
%                   {v:Int | 1 = v} (TPrim Empty (PInt 1)) -- $\cmt{\hastype{\varnothing}{1}{\breft{\tint}{\vv}{\vv = 1}}}$
%                   {v:Int | 0 < v} (SBase ...) -- $\cmt{\isSubType{\varnothing}{\breft{\tint}{\vv}{\vv = 1}}{\breft{\tint}{\vv}{0 < \vv }}}$
% \end{mcode}

\mypara{Implementation of Data Propositions}
Data propositions are a novel feature required to encode 
inductive propositions in
the mechanization of \sysrf. 
(\citet{lweb} developed a \lh metatheoretic proof  
but before data propositions and thus had to axiomatize a terminating 
evaluation relation; see~\S~\ref{ch:related}.)
%
% \newtext{
Refined data propositions are implemented as part of \lh's existing 
refined data types that already supported subtyping on constructor arguments 
using variant and contravariant rules, as described but not formalized in~\cite{sprite}. 
The essential extension to support data propositions is that by refining the output types 
of inductive data types, \lh can support constructive derivation-tree-style proofs.
% }                                              
%
To use this feature in practice, we had to extend the refinement logic 
of \lh to use existing SMT support to make data constructors \emph{injective}, 
\ie if $C$ is a constructor then $\forall x, y.\, C(x) = C(y) => x = y$.
Thus, refined data types and injectivity are the two required 
components to implement data propositions.

