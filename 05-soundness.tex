\chapter{Soundness of \sysrf} 
\label{ch:soundness}


% Node fill colors
\definecolor{palegrey}{rgb}{0.72, 0.72, 0.72}%{0.90, 0.90, 0.90}
\definecolor{lightgreen}{rgb}{0.92, 0.92, 0.92}%{0.56, 0.93, 0.56}
% Arrow colors
\definecolor{medgreen}{rgb}{0.02, 0.02, 0.02}%{0.24, 0.39, 0.24}
\definecolor{darkgreen}{rgb}{0.02, 0.02, 0.02}%{0.19, 0.31, 0.19}
\definecolor{sysfcolor}{rgb}{0.02, 0.02, 0.02}%{0.52, 0.52, 0.52}
\begin{figure}[t]
\begin{center}
\scalebox{0.77}{
    \begin{tikzpicture}[
        > = stealth, %arrowhead=1.25cm,  % arrow head style
        %shorten > = 1pt,               % don't touch arrow head to node
        auto,
        node distance = 1.0cm,         % distance between nodes
        semithick                      % line style
    ]

    \tikzstyle{every state}=[
        rectangle,
        minimum width=3.70cm,          % make most nodes the same size
        draw=black, rounded corners
    ]
    %\tikzstyle{every box}=[
    %    rectangle,
    %    opacity=0.00
    %    minimum width=3.85cm,         % make most nodes the same size
    %    draw=thin grey, rounded corners
    %]

    %% ORANGE REGION: Substitution Lemmas and friends
    \node[state, fill=palegrey]   (we1)                    {Weaken: tv in sub};
    \node[state, fill=lightgreen] (we2) [above of=we1]     {Weaken: tv in typ};
    \node[state, fill=palegrey]   (we3) [above of=we2]     {Weaken: var in sub};
    \node[state, fill=lightgreen] (we4) [above of=we3]     {Weaken: var in typ};

    \node[state, fill=lightgreen] (we5) [right=5.40cm of we1] {Weaken: tv in wf};
    \node[state, fill=lightgreen] (we6) [right=5.40cm of we2] {Weaken: var in wf};

    \node[state, fill=palegrey]   (su1) [above right=-0.25cm and 0.75cm of we2] {Substitute: tv in sub};% [250]};
    \node[state, fill=lightgreen] (su2) [above of=su1]     {Substitute: tv in typ};% [300]};
    \node[state, fill=palegrey]   (su3) [above of=su2]     {Substitute: var in sub};% [220]};
    \node[state, fill=lightgreen] (su4) [above of=su3]     {Substitute: var in typ};% [270]};

    \node[state, fill=lightgreen] (su5) [above=1.00cm of we6] {Substitute: tv in wf};
    \node[state, fill=lightgreen] (su6) [above of=su5] {Substitute: var in wf};

%%%    \begin{pgfonlayer}{background}
%%%        \path (su4.west |- su4.north)+(-0.25,0.60) node (oa) {};
%%%        \path (su1.south -| su1.east)+(0.25,-0.25) node (oc) {};
        %\path[fill=orange!50,rounded corners, draw=black!50, dashed]
        %    (oa) rectangle (oc);
%%%        \path[rounded corners, draw=black!50, dashed]
%%%            (oa) rectangle (oc) node (orangefill) {};
%%%        \fill[pattern=custom north west lines,hatchspread=6pt,hatchthickness=1pt,hatchcolor=gray!40]
%%%            (oa) rectangle (oc) ;
%%%    \end{pgfonlayer}
%%%    \begin{pgfonlayer}{background}
%%%        \path (we4.west |- we4.north)+(-0.25,0.60) node (oa1) {};
%%%        \path (we1.south -| we1.east)+(0.25,-0.25) node (oc1) {};
        %\path[fill=orange!50,rounded corners, draw=black!50, dashed]
        %    (oa) rectangle (oc);
%%%        \path[rounded corners, draw=black!50, dashed]
%%%            (oa1) rectangle (oc1) node (orangefill) {};
%%%        \fill[pattern=custom north west lines,hatchspread=6pt,hatchthickness=1pt,hatchcolor=gray!40]
%%%            (oa1) rectangle (oc1) ;
%%%    \end{pgfonlayer}

    \begin{pgfonlayer}{middleground}
        \path (we4.west |- we4.north)+(-0.10,0.45) node (b3a) {};
        \path (we1.south -| we1.east)+(0.10,-0.10) node (b3b) {};
        \path (we4.north) +(0,0.20) node (b3cap) {Weakening Lemma (\ref{lem:weakening})};
        \path[draw=black!70]
            (b3a) rectangle (b3b) node (b3) {};
    \end{pgfonlayer}
    \begin{pgfonlayer}{middleground}
        \path (we6.west |- we6.north)+(-0.10,0.45) node (b4a) {};
        \path (we5.south -| we5.east)+(0.10,-0.10) node (b4b) {};
        \path (we6.north) +(0,0.20) node (b4cap) {Weakening Lemma};
        \path[draw=black!70]
            (b4a) rectangle (b4b) node (b4) {};
    \end{pgfonlayer}
    \begin{pgfonlayer}{middleground}
        \path (su4.west |- su4.north)+(-0.10,0.45) node (b5a) {};
        \path (su1.south -| su1.east)+(0.10,-0.10) node (b5b) {};
        \path (su4.north) +(0,0.20) node (b5cap) {Substitution Lemma (\ref{lem:subst})};
        \path[draw=black!70]
            (b5a) rectangle (b5b) node (b5) {};
    \end{pgfonlayer}
    \begin{pgfonlayer}{middleground}
        \path (su6.west |- su6.north)+(-0.10,0.45) node (b6a) {};
        \path (su5.south -| su5.east)+(0.10,-0.10) node (b6b) {};
        \path (su6.north) +(0,0.20) node (b6cap) {Substitution Lemma};
        \path[draw=black!70]
            (b6a) rectangle (b6b) node (b6) {};
    \end{pgfonlayer}

    \begin{pgfonlayer}{foreground}
        \path[every edge/.style={draw, ->, >={Stealth[width = 5pt, length = 7pt, inset=1pt,sep]}}]
            (we5.north east) edge[sysfcolor, dashed, bend right] node {} (su6.south east)
            (we2.north east) edge[sysfcolor, dashed, bend left] node {} (su3.south west)
            (we5.north west) edge[sysfcolor, dashed, bend left] node {} (we3.south east)
            (su5.north west) edge[sysfcolor, dashed, bend right] node {} (su3.south east);
    \end{pgfonlayer}
    \begin{pgfonlayer}{foreground}
        \path[every edge/.style={draw, >-<, >={Stealth[width = 6pt, length = -5pt, inset=-8pt]}}]
            (we1) edge[darkgreen, bend right=15] node {} (we2)
            (we3) edge[darkgreen, bend right=15] node {} (we4)
            (su1) edge[darkgreen, bend right=15] node {} (su2)
            (su3) edge[darkgreen, bend right=15] node {} (su4);
    \end{pgfonlayer}

    % NEW REGION: Implication Interface 
    \node[state, fill=palegrey]   (imp) [left=0.75cm of we1]   {Implication Interf. (Rq.~\ref{lem:implication})};% [55]};

%%%    \begin{pgfonlayer}{background}
%%%        \path (imp.west |- imp.north)+(-0.25,0.25) node (ia) {};
%%%        \path (imp.south -| imp.east)+(0.25,-0.25) node (ic) {};
        %\path[fill=red!50,rounded corners, draw=black!50, dashed]
        %    (ra) rectangle (rc);
%%%        \path[rounded corners, draw=black!50, dashed]
%%%            (ia) rectangle (ic) node (redfill) {};
%%%        \fill[pattern=dots, pattern color=gray!25]
%%%            (ia) rectangle (ic);
%%%    \end{pgfonlayer}

    \begin{pgfonlayer}{foreground}
        \path[every edge/.style={draw, ->, >={Stealth[width = 5pt, length = 7pt, inset=1pt,sep]}}]
            (imp.east) edge[medgreen] node {} (we2.north west)
            (imp.south east) edge[medgreen, bend right=60] node {} (su1.south west);
    \end{pgfonlayer}

    % PURPLE REGION: Denotational Soundness
    \node[state, fill=palegrey]   (de3) [below=1.1cm of imp] {Den. Sound: subtyping};% [320]};
    \node[state, fill=palegrey]   (de2) [right=0.75cm of de3] {Denot. Sound: typing};% [375]};
    \node[state, fill=palegrey]   (de1) [right=0.75cm of de2] {Selfified Den. (\ref{denote-selfification})};% [115]};

%%%    \begin{pgfonlayer}{background}
%%%        \path (de2.west |- de2.south)+(-0.25,-0.25) node (pa) {};
%%%        \path (de1.north -| de1.east)+(0.25,0.60) node (pc) {};
        %\path[fill=violet!50,rounded corners, draw=black!50, dashed]
        %   (pa) rectangle (pc);
%%%        \path[rounded corners, draw=black!50, dashed]
%%%            (pa) rectangle (pc) node (purplefill) {};
%%%        \fill[pattern=custom horizontal lines, hatchspread=5pt, hatchthickness=0.75pt, hatchcolor=gray!35]
%%%            (pa) rectangle (pc);
%%%    \end{pgfonlayer}
    \begin{pgfonlayer}{middleground}
        \path (de3.west |- de3.north)+(-0.10,0.45) node (b7a) {};
        \path (de2.south -| de2.east)+(0.10,-0.10) node (b7b) {};
        \path (de3.north east) +(0,0.20) node (b7cap) {Denotational Soundness (\ref{denote-sound})};
        \path[draw=black!70]
            (b7a) rectangle (b7b) node (b7) {};
    \end{pgfonlayer}

    \begin{pgfonlayer}{foreground}
        \path[every edge/.style={draw, ->, >={Stealth[width = 5pt, length = 7pt, inset=1pt,sep]}}]
            (de1.west) edge[medgreen] node {} (de2.east);
    \end{pgfonlayer}
    \begin{pgfonlayer}{foreground}
        \path[every edge/.style={draw, >-<, >={Stealth[width = 6pt, length = -5pt, inset=-8pt]}}]
            (de2.west) edge[darkgreen, bend left=15] node {} (de3.east);
    \end{pgfonlayer}
    \begin{pgfonlayer}{foreground}
        \path[every edge/.style={draw, ->, >={Stealth[width = 6pt, length = 5pt, inset=1pt,sep]}}]
            (de2.north west) edge[darkgreen, bend right=15] node {} (imp);
    \end{pgfonlayer}    

    % RED REGION: Narrowing Lemma (and Exactness)
    \node[state, fill=palegrey]   (na1) [left=5.25cm of su2]   {Exact Subtypes (\ref{lem:exact})};% [55]};
    \node[state, fill=palegrey]   (na2) [above=0.70cm of na1]   {Exact Types (\ref{lem:exact})};% [100]};
    \node[state, fill=palegrey]   (na3) [above=0.70cm of na2]   {Narrowing Lemma (\ref{lem:narrowing})};% [390]};

%%%    \begin{pgfonlayer}{background}
%%%        \path (na3.west |- na3.north)+(-0.25,0.25) node (ra) {};
%%%        \path (na1.south -| na1.east)+(0.25,-0.25) node (rc) {};
        %\path[fill=red!50,rounded corners, draw=black!50, dashed]
        %    (ra) rectangle (rc);
%%%        \path[rounded corners, draw=black!50, dashed]
%%%            (ra) rectangle (rc) node (redfill) {};
%%%        \fill[pattern=custom checkerboard, hatchcolor=gray!25]
%%%            (ra) rectangle (rc);
%%%    \end{pgfonlayer}

    \begin{pgfonlayer}{foreground}
        \path[every edge/.style={draw, ->, >={Stealth[width = 5pt, length = 7pt, inset=1pt,sep]}}]
            (na1.north) edge[medgreen] node {} (na2.south)
            (na2.north) edge[medgreen] node {} (na3.south)
            (imp.north west) edge[medgreen, bend left=30] node {} (na3.west)
            (na2.east) edge[medgreen, bend left=25] node {} (su2.north west);
    \end{pgfonlayer}

    % YELLOW REGION: Inversion of Typing
    \node[state, fill=palegrey]   (in1) [above left=-0.25cm and 0.80cm of su4]   {Transitivity (\ref{lem:transitivity})};% [295]};
    \node[state, fill=lightgreen] (in2) [above=0.60cm of in1]   {Inversion of Typing (\ref{lem:inversion})};% [80]};

%%%    \begin{pgfonlayer}{background}
%%%        \path (in2.west |- in2.north)+(-0.25,0.25) node (ya) {};
%%%        \path (in1.south -| in1.east)+(0.25,-0.25) node (yc) {};
        %\path[fill=yellow!50,rounded corners, draw=black!50, dashed]
        %    (ya) rectangle (yc);
%%%        \path[rounded corners, draw=black!50, dashed]
%%%            (ya) rectangle (yc) node (yellowfill) {};
%%%        \fill[pattern=custom vertical lines, hatchspread=5pt, hatchthickness=1pt, hatchcolor=gray!35]
%%%            (ya) rectangle (yc);
%%%    \end{pgfonlayer}
    \begin{pgfonlayer}{middleground}
        \path (in2.west |- in2.north)+(-0.10,0.10) node (in6a) {};
        \path (in1.south -| in1.east)+(0.10,-0.45) node (in6b) {};
        \path (in1.south) +(0,-0.20) node (in6cap) {Inversion};
        \path[draw=black!70]
            (in6a) rectangle (in6b) node (in6) {};
    \end{pgfonlayer}
    \begin{pgfonlayer}{foreground}
        \path[every edge/.style={draw, ->, >={Stealth[width = 5pt, length = 7pt, inset=1pt,sep]}}]
            (in1.north) edge[medgreen] node {} (in2.south)
            (na3.east) edge[medgreen] node {} (in2.west)
            (na3.east) edge[medgreen] node {} (in1.west)
            (su4.west) edge[medgreen] node {} (in1.east);
    \end{pgfonlayer}

    % TOP REGION: Buildup to Progress and Preservation
    %%\node[state, fill=lightgreen] (pr1) [left=0.75cm of in2]   {Canonical Forms};
    \node[state, fill=lightgreen] (pr2) [right=0.75cm of in2]   {Primitives (Req. \ref{lem:prim-typing})};% [30]};
    \node[state, fill=lightgreen] (pr3) [right=0.75cm of pr2]   {Polym. Prim. (Req. \ref{lem:prim-typing})};% [30]};
    \node[state, fill=lightgreen] (pr4) [above left=0.70cm and 0.75cm of in2]   {Progress (\ref{lem:progressF})};% [55]};
    \node[state, fill=lightgreen] (pr5) [right=0.75cm of pr4]   {Preservation (\ref{lem:preservationF})};% [125]};
    \node[state, fill=lightgreen] (pr6) [right=0.75cm of pr5]   {Values Stuck};% [60]};
    \node[state, fill=lightgreen] (pr7) [right=0.75cm of pr6]   {Det. Semantics};% [85]};

    \begin{pgfonlayer}{foreground}
        \path[every edge/.style={draw, ->, >={Stealth[width = 5pt, length = 7pt, inset=1pt,sep]}}]
            (su4.north) edge[medgreen] node {} (pr2.south)          %% TO SECOND ROW
            (su4.east) edge[medgreen] node {} (pr3.south)
            (in2.north west) edge[sysfcolor, dashed] node {} (pr4.south)    %% TO FIRST ROW
            (in2.north) edge[sysfcolor, dashed] node {} (pr5.south)
            (pr2.north west) edge[sysfcolor, dashed] node {} (pr4.south)
            (pr2.north west) edge[sysfcolor, dashed] node {} (pr5.south)
            (pr3.north west) edge[sysfcolor, dashed] node {} (pr4.south east)
            (pr3.north west) edge[sysfcolor, dashed] node {} (pr5.south east)
            (su4.north west) edge[sysfcolor, dashed, bend right=0] node {} (pr5.south east)
            (pr6.west) edge[sysfcolor, dashed] node {} (pr5.east)
            (pr7.north west) edge[sysfcolor, dashed, bend right=20] node {} (pr5.north east);
    %%        (pr1.north) edge[sysfcolor, bend left] node {} (pr4.south west)
    %%        (pr1.north east) edge[sysfcolor] node {} (pr5.south west)
    %%        (pr4.east) edge[sysfcolor] node {} (pr5.west)  %% REMOVED DEPENDENCY FROM PROOF
    \end{pgfonlayer}

    \end{tikzpicture}
}
\end{center}
\vspace{-0.00cm}
\caption{Dependencies in the metatheory. We write
``var'' and ``tv'' to resp. abbreviate term and type variables.}
%The number in brackets indicates lines of code.}
\label{fig:graph}
\vspace{-0.00cm}
\end{figure}

    % BLUE REGION: Substitution Lemma for Entailment Judgments
    %\node[state, fill=palegrey]   (se1) [above=1.5cm of de3]   {Substitute: var in entail};% [165]};
    %\node[state, fill=palegrey]   (se2) [above=0.4cm of se1] {Substitute: tv in entail};% [175]};

    %\begin{pgfonlayer}{background}
    %    \path (se2.west |- se2.north)+(-0.25,0.60) node (ba) {};
    %    \path (se1.south -| se1.east)+(0.25,-0.25) node (bc) {};
    %    %\path[fill=blue!50,rounded corners, draw=black!50, dashed]
    %    %    (ba) rectangle (bc);
    %    \path[rounded corners, draw=black!50, dashed]
    %        (ba) rectangle (bc) node (bluefill) {};
    %    \fill[pattern=custom crosshatch dots, hatchspread=2.5pt, hatchthickness=0.25pt, hatchcolor=gray!35]
    %        (ba) rectangle (bc);
    %\end{pgfonlayer}
    %\begin{pgfonlayer}{middleground}
    %    \path (se2.west |- se2.north)+(-0.10,0.45) node (b8a) {};
    %    \path (se1.south -| se1.east)+(0.10,-0.10) node (b8b) {};
    %    \path (se2.north) +(0,0.20) node (b8cap) {Substitution Lemma};
    %    \path[draw=black!70]
    %        (b8a) rectangle (b8b) node (b8) {};
    %\end{pgfonlayer}

    %\begin{pgfonlayer}{foreground}
    %    \path[every edge/.style={draw, ->, >={Stealth[width = 5pt, length = 7pt, inset=1pt,sep]}}]
    %        (de3.north) edge[medgreen, bend right=15] node {} (se1.south)
    %        (se1.east) edge[medgreen] node {} (su1.west)
    %        (se2.east) edge[medgreen] node {} (su3.west);
    %\end{pgfonlayer}
    %\begin{pgfonlayer}{foreground}
    %    \path[every edge/.style={draw, >-<, >={Stealth[width = 6pt, length = -5pt, inset=-8pt]}}]
    %        (se1) edge[darkgreen, bend right=15] node {} (se2);
    %\end{pgfonlayer}


Our development for \sysf (\S~\ref{sec:soundness})
follows the standard presentation of System F's
metatheory by \citet{TAPL}.
%
The main difference 
is that ours includes well-formedness of types and
environments, which help with mechanization \cite{Remy21}
and are crucial in \sysrf when formalizing refinements.
%

%
%We proceed to the metatheory
%of the two systems $\sysf$ and $\sysrf$
%and concretely, 
%type safety for both systems (\Cref{lem:soundness})
%and denotational soundness for $\sysrf$ (\Cref{lem:denote-sound-first}).
%
\Cref{fig:graph} charts the overall
landscape of our formal development as a
dependency graph of the main lemmas which
establish meta-theoretic properties of the
different judgments for \sysf and \sysrf.
%
\NV{I think RJ says to remove colors?}
Nodes shaded \colboth represent lemmas
in the metatheories for both \sysf and \sysrf.
%
The \colref nodes denote lemmas
that only appear in %the metatheory for 
\sysrf.
%
An arrow shows a dependency: the lemma at the
\emph{tail} is used in the proof of the lemma at
the \emph{head}. %A double-headed arrow indicates
%a mutual dependency, \ie mutually recursive proofs.
Solid arrows are dependencies in \sysrf only.
The chart already shows that the metatheory of 
the refined calculus \sysrf is much more complex 
that the one of the unrefined system \sysf, 
as also shown by the summary 
of our mechanization (\Cref{fig:empirical}).

In the rest of this chapter
we establish 
denotational soundness (\S~\ref{sec:soundness:denotational}) and 
type safety (\S~\ref{sec:soundness:safety})
and we flesh out the skeleton of~\Cref{fig:graph}, \ie  
the inversion (\S~\ref{sec:soundness:inversion}),
substitution (\S~\ref{sec:soundness:substitution}),
and narrowing (\S~\ref{sec:soundness:narrowing}) lemmas.





\section{Denotational Soundness}
\label{sec:soundness:denotational}
\label{sec:denot:soundness}

Denotational soundness connects syntactic typing and subtyping 
with the type denotations (of~\Cref{fig:den}).
For typing it states that if
$\hastype{\tcenv}{e}{t}$,
then when $e$ is closed by any closing substitution of $\tcenv$
it evaluates to a value that belongs in the denotation of the closed $t$.
For subtyping,
if $\isSubType{\tcenv}{s}{t}$, then
under all closing substitutions, the denotation of the
former type is contained in the latter:

\begin{theorem}\label{lem:denote-sound-first}
\label{denote-sound} (Denotational Soundness)
\begin{enumerate}
\item If $\hastype{\tcenv}{e}{t}$
      and $\isWellFormedE{\tcenv}$
      and $\theta \in \denote{\tcenv}$
      then $\evalsTo{\theta(e)}{v} \in \denote{\theta(t)}$ for some value $v$.
\item If $\isSubType{\tcenv}{t_1}{t_2}$
      and $\isWellFormedE{\tcenv}$
      and $\isWellFormed{\tcenv}{t_1}{k_1}$
      and $\isWellFormed{\tcenv}{t_2}{k_2}$ and $\theta \in \denote{\tcenv}$ then $\denote{\theta(t_1)} \subseteq\denote{\theta(t_2)}$.
\end{enumerate}
\end{theorem}

The proof is by mutual induction
on the structure of the judgments
$\hastype{\tcenv}{e}{t}$ and
$\isSubType{\tcenv}{t_1}{t_2}$ respectively.
%
Our rule \tVar mentions selfification,
so we use \Cref{denote-selfification}
for that case.

\begin{lemma}\label{denote-selfification} (Selfified Denotations)
    If $\isWellFormed{\varnothing}{\stype}{\skind}$,
       $\hastype{\varnothing}{\sexpr}{\stype}$,
       $\evalsTo{\sexpr}{\sval}$ for some $\sval \in \denote{\stype}$
    then $\sval \in \denote{\self{t}{e}{k}}$.
    %(DenotationsSelfify.hs) Depends on: ${\tt lem\_typing\_wf}$
\end{lemma}

This lemma captures the intuition
that if $v \in \denote{\breft{\sbase}{x}{p}}$
(\ie if $v$ has base type $\sbase$
and $\evalsTo{\subst{p}{x}{v}}{\ttrue}$),
then we have $v \in \denote{\breft{\sbase}{x}{ p \wedge x = v}}$
as $\subst{(p \wedge x=v)}{x}{v}$ certainly evaluates to $\ttrue$.
%
The full proof also handles the case that $t$
is an existential type as well as selfification
by an expression $e$ that evaluates to $v$.



\section{Type Safety}
\label{sec:soundness:safety}

The type safety theorem  states that a well-typed 
term does not get stuck: \ie either 
evaluates to a value or can step 
to another term (progress) 
of the same type (preservation).
%

\begin{theorem} (Type Safety of \sysrf) 
\label{lem:soundness} 
\begin{enumerate}
    \item (Type Safety)
    If $\hastype{\varnothing}{\sexpr}{\stype}$ and $\evalsTo{\sexpr}{\sexpr'}$,
    then $\sexpr'$ is a value or $\sexpr' \step \sexpr''$
    %and $\hastype{\varnothing}{\sexpr''}{\stype}$
    for some $\sexpr''$.
    \item (No Error)
    If $\hastype{\varnothing}{\sexpr}{\stype}$ and $\evalsTo{\sexpr}{\sexpr'}$,
    then $\sexpr' \not = \eerr$.
\end{enumerate}
\end{theorem}
The No Error property explicitly states that well-typed terms 
cannot evaluate to the term $\eerr$ (that encodes stuck terms)
and is a direct implication of type safety.
We prove type safety by induction on the 
length of the sequence of steps  
$\evalsTo{\sexpr}{\sexpr'}$, using  
preservation and progress.

\mypara{Progress} \label{sec:sysf:progress}
%
The progress lemma says a well-typed term is a value 
or steps to some other term.
%
\begin{lemma} (Progress) \label{lem:progressF} 
If $\hastype{\varnothing}{\sexpr}{\stype}$, 
then $\sexpr$ is a value or $\sexpr \step \sexpr'$ for some $\sexpr'$.
\end{lemma}

The proof is by induction on the typing derivation 
using the primitives~\Cref{lem:prim-typing}, that we proved 
for our built-in primitives, and the inversion of typing lemma.



\mypara{Preservation} \label{sec:sysf:preservation}
%
The preservation lemma states that typing is preserved
by evaluation.
%
\begin{lemma} (Preservation) \label{lem:preservationF} 
If $\hastype{\varnothing}{\sexpr}{\stype}$ and $\sexpr \step \sexpr'$, 
then $\hastype{\varnothing}{\sexpr'}{\stype}$.
\end{lemma}    

The proof is by structural induction on the 
derivation of the typing judgment and implicitly uses the 
inversion lemma. 
We use the determinism of the operational 
semantics (lemma~\ref{lem:step-determ}) and 
the canonical forms lemma to case split 
on $\sexpr$ to determine $\sexpr'$.
%
The interesting cases are for \fApp and \fTApp that require 
a substitution~\Cref{lem:subst}.
%
Next, let's see the three main lemmas used in 
the preservation and progress proofs.

\section{Inversion of Typing Judgments}
\label{sec:soundness:inversion}

The region of~\Cref{fig:graph} labelled ``Inversion''
accounts for the fact that, due to subtyping
chains, the typing judgment in \sysrf is not
syntax-directed.
%
% \mypara{Transitivity of Subtyping}
%
First, we establish that subtyping is transitive:

\begin{lemma} (Transitivity) \label{lem:transitivity} 
    If\; $\isWellFormed{\tcenv}{t_1\!}{\! k_1}$,
       %$\isWellFormed{\tcenv}{t_2}{k_2}$,
       $\isWellFormed{\tcenv}{t_3\!}{\! k_3}$,
       $\isWellFormedE{\tcenv}$,
       $\isSubType{\tcenv}{t_1}{t_2}$, 
       $\isSubType{\tcenv}{t_2}{t_3}$, then
       $\isSubType{\tcenv}{t_1}{t_3}$.
    % (LemmasTransitive.hs) Our ${\tt lem\_sub\_trans}$ depends on ${\tt lem\_subst\_sub}$.
\end{lemma}
%
The proof consists of a case-split
on the possible rules for $\isSubType{\tcenv}{t_1}{t_2}$
and $\isSubType{\tcenv}{t_2}{t_3}$.
%
When the last rule used
in the former is \sWitn and the
latter is \sBind, we require the
substitution~\Cref{lem:subst}.
%
As \citet{Aydemir05},
we use the narrowing~\Cref{lem:narrowing}
for the transitivity for function types.

\mypara{Inverting Typing Judgments}
%
We use the transitivity of subtyping to prove
some non-trivial lemmas that let us ``invert''
the typing judgments to recover information
about the underlying terms and types.
%
We describe the non-trivial case which
pertains to type and value abstractions:

\begin{lemma} (Inversion of $\tAbs$, $\tTAbs$) \label{lem:inversion} 
    \begin{enumerate}
    \item If $\hastype{\tcenv}{(\vabs{w}{\sexpr})}{\functype{x}{\stype_x}{\stype}}$
    and $\isWellFormedE{\tcenv}$,
    then for all $\notmem{y}{{\tcenv}}$,
    $\hastype{\bind{y}{\stype_x},\tcenv}{\subst{\sexpr}{w}{y}}{\subst{\stype}{x}{y}}$.
    \item If $\hastype{\tcenv}{(\tabs{\al_1}{\skind_1}{\sexpr})}{\polytype{\al}{\skind}{\stype}}$
    and $\isWellFormedE{\tcenv}$,
    then for all $\notmem{\al'}{{\tcenv}}$, 
    $\hastype{\bind{\al'}{\skind},\tcenv}{\subst{\sexpr}{\al_1}{\al'}}{\subst{\stype}{\al}{\al'}}$. %(LemmasInvertLambda.hs) Both of these depend on ${\tt lem_sub_trans}$. Our ${\tt lem\_invert\_tabs}$ depends on ${\tt lem\_narrow\_typ}$, which in turn depends on ${\tt lem\_exact\_type}$.
    \end{enumerate}
\end{lemma}
%
If $\hastype{\tcenv}{(\vabs{w}{e})}{\functype{x}{t_x}{t}}$,
then we cannot directly invert the typing judgment
to get a judgment for the body $e$ of $\vabs{w}{e}$.
%
Perhaps the last rule used was \tSub,
and inversion only tells us that there
exists a type $t_1$ such that
$\hastype{\tcenv}{(\vabs{w}{e})}{t_1}$
and $\isSubType{\tcenv}{t_1}{\functype{x}{t_x}{t}}$.
%
Inverting again, we may in fact find a chain
of types $t_{i+1} \subt t_i \subt \cdots \subt t_2 \subt t_1$
which can be arbitrarily long.
%
But the proof tree must be finite so eventually
we find a type $\functype{w}{s_w}{s}$ such that
$\hastype{\tcenv}{(\vabs{w}{e})}{\functype{w}{s_w}{s}}$
and $\isSubType{\tcenv}{\functype{w}{s_w}{s}}{\functype{x}{t_x}{t}}$
(by transitivity)
and the last rule was $\tAbs$.
%
Then inversion gives us that for any $\notmem{y}{\tcenv}$
we have
$\hastype{\bind{y}{s_w}, \tcenv}{e}{\subst{s}{w}{y}}$.
To get the desired typing judgment, we must
use the narrowing~\Cref{lem:narrowing} to obtain
$\hastype{\bind{y}{t_x}, \tcenv}{e}{\subst{s}{w}{y}}$.
Finally, we use $\tSub$ to derive
$\hastype{\bind{y}{t_x}, \tcenv}{e}{\subst{t}{w}{y}}$.

\section{Substitution Lemma}
\label{sec:soundness:substitution}

%The main result in the \colsubtyping
%region of~\Cref{fig:graph} is
%the substitution lemma.
%
In $\sysrf$, unlike unrefined calculi such as $\sysf$, 
typing and subtyping are mutual dependent. 
%
Due to this dependency, both the substitution
the weakening lemmas
must now be proven
in a mutually recursive form: %  for both judgments: 

\begin{lemma}\label{substitution} \label{lem:subst} (Substitution)
    \begin{enumerate}
        %If $\entails{\tcenv',\bind{\al}{\skind}\tcenv}{p}$ and $\vdash_w \tcenv',\bind{\al}{\skind}\tcenv$ and $\hastype{\tcenv}{t_a}{k}$ then $\entails{\subst{\tcenv'}{\al}{t_\al},\tcenv}{p[t_a/a]}$. (SubstitutionLemmaEntTV.hs) \\
        \item If $\isSubType{\tcenv_1,\bind{x}{t_x},\tcenv_2}{s}{t}$,
                 $\isWellFormedE{\tcenv_2}$,
            and $\hastype{\tcenv_2}{v_x}{t_x}$,
            then $\isSubType{\subst{\tcenv_1}{x}{v_x},\tcenv_2}{\subst{s}{x}{v_x}}{\subst{t}{x}{v_x}}$.
            %(SubstitutionLemmaTyp.hs) Our ${\tt lem\_subst\_sub}$ depends on ${\tt lem\_subst\_ent}$. \\
        \item If $\hastype{\tcenv_1,\bind{x}{t_x},\tcenv_2}{e}{t}$,
                 $\isWellFormedE{\tcenv_2}$,
            and $\hastype{\tcenv_2}{v_x}{t_x}$,
            then $\hastype{\subst{\tcenv_1}{x}{v_x},\tcenv_2}{\subst{e}{x}{v_x}}{\subst{t}{x}{v_x}}$.
            % (SubstitutionLemmaTyp.hs) Mutually recursive with ${\tt lem\_subst\_sub}$. \\
        \item If $\isSubType{\tcenv_1,\bind{\al}{\skind},\tcenv_2}{s}{t}$,
                $\isWellFormedE{\tcenv_2}$,
            and $\isWellFormed{\tcenv_2}{t_\al}{k}$,
            then $\isSubType{\subst{\tcenv_1}{\al}{t_\al},\tcenv_2}{\subst{s}{\al}{t_\al}}{\subst{t}{\al}{t_\al}}$.
            %(SubstitutionLemmaTypTV.hs) Our ${\tt lem\_subst\_tv\_sub}$ depends on ${\tt lem\_subst\_tv\_ent}$.\\
        \item If $\hastype{\tcenv_1,\bind{\al}{\skind},\tcenv_2}{e}{t}$,
                 $\isWellFormedE{\tcenv_2}$,
            and $\isWellFormed{\tcenv_2}{t_\al}{\skind}$,
            then $\hastype{\subst{\tcenv_1}{\al}{t_\al},\tcenv_2}{\subst{e}{\al}{t_\al}}{\subst{t}{\al}{t_\al}}$.
        %(SubstitutionLemmaTypTV.hs) Mutually recursive with ${\tt lem\_subst\_tv\_sub}$.
    \end{enumerate}
\end{lemma}

The proof goes by induction on the derivation trees. 
The main difficulty arises
in substituting some type
$t_\al$ for variable $\al$
in $\isSubType{\tcenv_1,\bind{\al}{\skind},\tcenv_2}{\breft{\al}{x_1}{p}}{\breft{\al}{x_2}{q}}$
because $t_\al$  must be
strengthened  by
the refinements $p$ and
$q$ respectively.
%
Because we encoded our typing rules using 
cofinite quantification~\cite{AydemirCPPW08}
the proof does not require a renaming lemma, but 
the rules that lookup environments 
(rules \tVar and \wtVar) do need a \emph{weakening Lemma}:
%\Cref{lem:weakeningF}. 
% and \emph{rename} free variables 
%in typing and well-formedness judgments (\Cref{lem:freevarsF}).

%As with the $\sysf$ version, the proof 
%requires the \emph{Weakening}~\Cref{lem:weakening}
%but now both for typing and subtyping.

%\RJ{where is "weakening" used? the label is unused in the text. CUT or REFER}
\begin{lemma}\label{lem:weakening} (Weakening)
If $\notmem{x,\al}{{\tcenv_1, \tcenv_2}}$, then
    \begin{enumerate}
        \item if $\hastype{\tcenv_1, \tcenv_2}{\sexpr}{\stype}$
        then $\hastype{\tcenv_1, \bind{x}{\stype_x}, \tcenv_2}{\sexpr}{\stype}$ and $\hastype{\tcenv_1, \bind{\al}{\skind}, \tcenv_2}{\sexpr}{\stype}$.
        \item if $\isSubType{\tcenv_1, \tcenv_2}{s}{t}$
        then $\isSubType{\tcenv_1, \bind{x}{\stype_x}, \tcenv_2}{s}{t}$
        and $\isSubType{\tcenv_1, \bind{\al}{\skind}, \tcenv_2}{s}{t}$.
            %\item If $\tcenv', \tcenv \vdash_e p$ then $\tcenv', \bind{x}{t_x}, \tcenv \vdash_e p$ and $\tcenv', \al\bindt k, \tcenv \vdash_e p$.
            %\item If $\tcenv', \tcenv \vdash_w t:k$ then $\tcenv', \bind{x}{t_x}, \tcenv     \vdash_w t:k$ and $\tcenv', \al\bindt k, \tcenv \vdash_w t:k$.
    \end{enumerate}
\end{lemma}

The proof is by mutual induction
on the derivation of the typing and subtyping judgments.

\section{Narrowing} 
\label{sec:soundness:narrowing}

The narrowing lemma says that whenever
we have a judgment where a binding
$x\bindt t_x$ appears in the binding
environment, we can replace $t_x$
by any subtype $s_x$.
%
The intuition here is that the judgment
holds under the replacement because we
are making the context more specific.

\begin{lemma} \label{subtype-env} \label{lem:narrowing} (Narrowing)
    If $\tcenv_2 \vdash s_x <: t_x$,  $\isWellFormed{\tcenv_2}{s_x}{k_x}$,
    and $\isWellFormedE{\tcenv_2}$ then
    \begin{enumerate}
        \item if $\tcenv_1, \bind{x}{t_x}, \tcenv_2 \vdash_w t : k$, then
                 $\tcenv_1, x\bindt s_x, \tcenv_2 \vdash_w t : k $.
        \item if $\tcenv_1, \bind{x}{t_x}, \tcenv_2 \vdash t_1 <: t_2$, then
                 $\tcenv_1, x\bindt s_x, \tcenv_2 \vdash t_1 <: t_2$.
        \item if $\tcenv_1, \bind{x}{t_x}, \tcenv_2 \vdash e : t$, then
                 $\tcenv_1, x\bindt s_x, \tcenv_2 \vdash e : t$.
    \end{enumerate}
\end{lemma}

\begin{fullversion}
The narrowing proof
requires an exact typing~\Cref{lem:exact}
which says that a subtyping
judgment $\isSubType{\tcenv}{s}{t}$
is preserved after
selfification on both types.
%
Similarly whenever we can type
a value $v$ at type $t$ then we
also type $v$ at the type $t$
selfified by $v$.
\end{fullversion}
\begin{conference}
    The narrowing proof
    requires an exact typing~\Cref{lem:exact}
    which says that both subtyping and typing is preserved 
    after selfification.
\end{conference}
\begin{lemma} (Exact Typing) \label{lem:exact}
\begin{enumerate}
  \item If $\hastype{\tcenv}{e}{t}$, $\isWellFormedE{\tcenv}$, $\isWellFormed{\tcenv}{t}{k}$, and $\isSubType{\tcenv}{s}{t}$, then $\isSubType{\tcenv}{{\rm self}(s, v, k)}{{\rm self}(t, v, k)}$.
  \item If $\hastype{\tcenv}{v}{t}$, $\isWellFormedE{\tcenv}$, and $\isWellFormed{\tcenv}{t}{k}$, then $\hastype{\tcenv}{v}{{\rm self}(t, v, k)}$. %(LemmasExactness.hs) Our ${\tt lem\_exact\_type}$ depends on ${\tt lem\_exact\_subtype}$.
\end{enumerate}
\end{lemma}
%\NV{Put these two in one line if we need space}



% REMOVED LEMMAS

%We can also prove analogous statements for the well-formedness judgments for environments. Next, we have the weakening lemma for well-formedness judgments. As with the free variables lemma, it even suffices for the erased environments to be well-formed.

%\begin{lemma}\label{WF-weaken}
%    For any environments $\tcenv$, $\tcenv'$ and $x,\al, \not\in dom(\tcenv', \tcenv)$:
%    If $\tcenv', \tcenv \vdash_w t:k$ and $\vdash_{w} \tcenv', \tcenv$, then $\tcenv', \bind{x}{t_x}, \tcenv \vdash_w t:k$ and $\tcenv', \al\bindt k_{\al}, \tcenv \vdash_w t:k$.
%\end{lemma}

%Next, we have the substitution lemma for well-formedness judgments.

%\begin{lemma}\label{WF-substitution}
%    For any environments $\tcenv$, $\tcenv'$ and $x,\al \not\in dom(\tcenv', \tcenv)$:
%    \item If $\tcenv', \bind{x}{t_x}, \tcenv \vdash_w t:k$ and
%    $\hastype{\tcenv}{v_x}{t_x}$, then
%    then $\tcenv'[v_x/x], \tcenv \vdash_w \subst{t}{x}{v_x}:k$.
%    \item If $\tcenv', \al\bindt k_{\al}, \tcenv \vdash_w t:k$ and
%    $\isWFFT{\tcenv}{t_{\al}}{k_{\al}}$,
%    then $\tcenv'[t_{\al}/\al], \tcenv \vdash_w t[t_{\al}/\al]:k$.
%\end{lemma}

%\begin{lemma}
%    {\em Typing implies Well-Formedness}: If $\hastype{\tcenv}{e}{t}$ and $\vdash \tcenv$ then $\isWellFormed{\tcenv}{t}{{}^{*}}$. (LemmasTyping.hs) Used in many, many places.
%\end{lemma}

%    \begin{lemma}
%        {\em ``Assumption 1'' / Typing of $\delta$}. In the mechanization we  assume something more conservative and then show:  If $\hastype{\varnothing}{c}{ x\bindt t_x->t'}$ and $\hastype{\varnothing}{v}{t_x}$ then $\hastype{\varnothing}{\delta(c,v)}{t'[v/x]}$ (PrimitivesDeltaTyping.hs) (and similarly for the polymorphic Eql). The latter one, our ${\tt lem\_deltaT\_typ}$ depends on ${\tt lem\_subst\_tv\_sub}$.
%    \end{lemma}

%    \begin{theorem}
%        Progress Theorem: If $\varnothing \vdash e : t$ then either $e$ is a value or there exists a term $e'$ such that $e \hookrightarrow$ e'. (MainTheorems.hs) No significant dependencies.
%    \end{theorem}

%    \begin{theorem}
%        Preservation Theorem: If $\varnothing \vdash e : t$ and $e \hookrightarrow e'$, then $\varnothing \vdash e' : t$. (MainTheorems.hs)  Our ${\tt thm\_preservation}$ depends on ${\tt lem\_sem\_det}$, ${\tt lem\_value\_stuck}$, ${\tt lem\_subst\_typ}$, ${\tt lem\_invert\_tabs}$, ${\tt lem\_invert\_tabst}$, ${\tt lem\_delta\_typ}$, ${\tt lem\_detaT\_typ}$, and ${\tt thm\_progress}$.
%    \end{theorem}

\begin{comment}
\section{Subtyping and Entailments}

In order to prove the substitution lemma (\Cref{lem:subst})
for the subtyping judgments, the case of \sBase requires
us to prove a corresponding lemma for entailment judgments.

\begin{lemma} (Entailment Substitution) \label{lem:entail-base}
    \begin{enumerate}
        \item If $\entails{\tcenv',\bind{x}{t_x},\tcenv}{p}$,
                 $\isWellFormedE{\tcenv',\bind{x}{t_x},\tcenv}$,
            and $\hastype{\tcenv}{v_x}{t_x}$,
            then $\entails{\subst{\tcenv'}{x}{v_x},\tcenv}{\subst{p}{x}{v_x}}$.
            %(SubstitutionLemmaEnt.hs) This one uses ${\tt lem\_denote\_sound\_typ}$ \\
        \item If $\entails{\tcenv',\bind{\al}{\skind},\tcenv}{p}$,
                 $\isWellFormedE{\tcenv',\bind{\al}{\skind},\tcenv}$,
            and $\hastype{\tcenv}{t_\al}{\skind}$,
            then $\entails{\subst{\tcenv'}{\al}{t_\al},\tcenv}{\subst{p}{\al}{t_\al}]}$.
            %(SubstitutionLemmaEntTV.hs) \\
    \end{enumerate}
\end{lemma}

The proof of the above lemma depends
on Denotational Soundness (\Cref{lem:denote-sound-first})
to prove the particular statement about denotations
in the antecedent of rule \entPred holds under substitution.
%
Specifically, we are given the statement
$\foralltheta{\theta}{\tcenv',\bind{x}{t_x},\tcenv}
    \Rightarrow \evalsTo{\theta(p)}{\ttrue}$
and we need to show the statement
$\foralltheta{\theta'}{\subst{\tcenv'}{x}{v_x},\tcenv}
    \Rightarrow \evalsTo{\theta'(\subst{p}{x}{v_x})}{\ttrue}$ holds.
The second statement is a special case of the first;
for any appropriate $\theta'$ and $v_x$, the closing substitution
$\theta \defeq (\theta', x\mapsto v_x)$ is appropriate to the first statement.
%
The proof hinges on Denotational Soundness,
described below, which gives us $v_x \in \denote{\theta'(t_x)}$
and on the commutativity of substitution
($v_x$ is a closed value and so has no variables appearing in it).

\end{comment}