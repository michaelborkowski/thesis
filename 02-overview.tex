\chapter{Refinement Types}
\label{ch:overview}
We start with an informal overview of
the usefulness of refinement types and of our
refined core calculus \sysrf,
which we later present formally (\Cref{ch:language})
and prove sound (\Cref{ch:soundness}).
%
Concretely,
we present the goals of refinement types (\S~\ref{sec:overview:goal})
and how they are achieved via three essential features:
semantic subtyping, existential types, and polymorphism
(\S~\ref{sec:overview:essence}).
We explain how the typing judgements are designed to accommodate these features
(\S~\ref{sec:overview:design})
and how we addressed the challenges these features impose in the mechanization
of the soundness proof (\S~\ref{sec:overview:soundness}).
Our examples here are presented with the syntax of \lh,
but can be encoded in \sysrf.

\section{The goal of Refinement Types} 
\label{sec:overview:goal}

Refinement types refine the types of an existing programming language
with logical predicates to define abstractions not expressible by the
underlying type system, which can then be used for static
%
(1) error prevention and (2) functional correctness.


\begin{figure}[t]
% \begin{code}
%   type Array a = Int -> a
%   {-@ type ArrayN a N = {i:Nat | i < N} -> a @-}

%   {-@ new :: n:Nat -> a -> ArrayN a n @-}
%   new :: Int -> a -> Array a
%   new n x = \i -> if 0 <= i && i < n then x else error "Out of Bounds"

%   {-@ set :: n:Nat -> i:{Nat | i < n} -> a -> ArrayN a n -> ArrayN a n @-}
%   set :: Int -> Int -> a -> Array a  -> Array a
%   set n i x a = \j -> if i == j then x else a j

%   {-@ get :: n:Nat -> i:{Nat | i < n} -> ArrayN a n -> a @-}
%   get :: Int -> Int -> Array a -> a
%   get n i a = a i
% \end{code}
\begin{code}
  type ArrayN a N = {i:Nat | i < N} -> a

  new :: n:Nat -> a -> ArrayN a n
  new n x = \i -> if 0 <= i && i < n then x else error "Out of Bounds"

  set :: n:Nat -> i:{Nat | i < n} -> a -> ArrayN a n -> ArrayN a n
  set n i x a = \j -> if i == j then x else a j

  get :: n:Nat -> i:{Nat | i < n} -> ArrayN a n -> a
  get n i a = a i
\end{code}
\vspace{0.0cm}
\caption{Functional Arrays with refinement types that ensure safe indexing.}
\vspace{0.0cm}
\label{fig:array}
\end{figure}


\mypara{Error Prevention}
Figure \ref{fig:array} presents the interface of a fixed size array
that is encoded in the core calculus \sysrf as a function.
%
The function @new n x@ returns an array that contains
@x@ when indexed with an integer between @0@ and @n@
and otherwise throws an ``out of bounds'' error.
To statically ensure that this error will never occur,
@new@ returns the refined array @ArrayN a n@, \ie
a function whose domain is restricted to integers less than @n@.
%
The @set@ and @get@ operators manipulate the refined arrays
on the index @i:{Nat | i < n}@, \ie refined to be in-bounds of the array.
%
With this refined interface, out-of-bounds indexing is statically ruled out:
\begin{code}
  array10 :: ArrayN Int 10
  array10 = new 10 0

  good = get 10 4  array10 -- OK
  bad  = get 10 42 array10 -- Refinement Type Error
\end{code}

\mypara{Functional Correctness}
\label{sec:overview:primes}
%
Refinement types are also used to ensure that the program has the intended behavior.
To achieve this, we use uninterpreted functions to \textit{specify} behaviors
and rely on the type system to propagate them.
For example, below using the uninterpreted function @isPrime@ we \textit{specify}
that some integers are primes, as denoted by the \emph{uninterpreted} predicate @isPrime@.
%
\begin{code}
  measure isPrime :: Int -> Bool
  type Prime = {v:Int | isPrime v}
\end{code}
%
Refinement types are not ideally suited to verifying properties
like primality checking, which requires reasoning beyond SMT
decidable fragments.
%
However, \emph{assuming} that a function establishes primality,
refinements can be used to easily track and propagate the invariant:
%
\begin{code}
  assume checkPrime :: x:Int -> {v:Bool | v <=> isPrime x}

  nextPrime :: Nat -> Prime
  nextPrime x = if checkPrime x then x else nextPrime (x+1)
\end{code}
%  nextPrime x = if checkPrime x then x else nextPrime (x+1)
% fix f = f (fix f)
%
The path-sensitivity of refinement types (Rule~\tIf of \Cref{fig:typing})
ensures that the function @nextPrime@ returns only
values that pass the primality check.

\mypara{Note on recursion}
%
Our core calculus does not explicitly support recursion.
But it can be extended with primitive constants (as long as they satisfy 
the consistency condition in ~\Cref{lem:prim-typing} below).
So, to encode inductive definitions, like @nextPrime@ in our system, we
use the fixpoint constant @fix@:
\begin{mcode}
  fix :: (a -> a) -> a
  nextPrime = fix @(Nat->Prime) (\f x->if checkPrime x then x else f (x+1))
\end{mcode}
Importantly, our calculus is fully polymorphic, in the sense that
type variables can be instantiated with refined types.
So, the type variable of @fix@ can be instantiated with
the refined type @Nat ->Prime@ to get the desired type of @nextPrime@.
Here, for emphasis, we make this instantiation explicit,
but in real systems, like
\href{http://goto.ucsd.edu:8090/index.html#?demo=permalink%2F1687971696_2572.hs}{\lh},
the refined type application is inferred.

\mypara{Primes Array Example}
%
As a bigger example, consider an example where refinements are used for both
error prevention and functional correctness.
%
The function @primes n@ generates an array with the first @n@ prime numbers:
%
\begin{code}
  primes :: n:Nat -> ArrayN Prime n
  primes n = (fix go) 1 0 (new n (nextPrime 1))
    where go f i p acc = if i < n
                               then let p' = nextPrime (p+1) in
                                      go f (i+1) p' (set n i p' acc)
                               else acc
\end{code}
%
Since @primes@ typechecks under the safe array interface of \Cref{fig:array},
no out-of-bounds errors will occur.
%
At the same time, all elements of the array are @set@ by a result @nextPrime@
and thus @primes@ returns an array of prime numbers.

\section{The Essence of Refinement Types} \label{sec:overview:essence}
%
The practicality of refinement types, as illustrated in the examples above,
is due to the combination of three essential features:
%
\begin{enumerate}[leftmargin=*]

\item \textbf{\textit{Semantic Subtyping:}}
The user does not need to provide any explicit type casts, because
subtyping is implicit and semantic.
%
For example, to type check @get 10 4 array10@ (from~\S~\ref{sec:overview:goal}),
the type of @4 :: {v:Int | v == 4}@ is implicitly converted to @{v:Int | 0 <= v < 10}@

\item \textbf{\textit{Decidability:}}
%
The semantic casts are reduced to logical implications checked by an SMT solver.
Refinement types are designed to generate decidable logical implications,
to ensure predictable verification and also permit type inference~\cite{LiquidPLDI08}
that makes verification practical, \eg the @primes@ definition requires zero annotations.

\item \textbf{\textit{Polymorphism:}}
Polymorphism on refinement types permits instantiation of type variables with any refined type.
For example, the same array interface can be used to describe primes,
functions with positive domains, and any other concept encoded as a refinement type.
\end{enumerate}

\section{The Design of Refinement Types}
\label{sec:overview:design}

Next, we develop a minimal calculus \sysrf that shows how
Refinement type systems enjoy the three essential features
of \S~\ref{subsec:overview:essence}.
%
\sysrf has four judgements that relate expressions ($e$),
types ($t$), kinds ($k$), predicates ($p$), and environments ($\tcenv$):
%
(1)~typing (\hastype{\tcenv}{e}{t}),
%
(2)~subtyping (\isSubType{\tcenv}{t_1}{t_2}),
%
(3)~well-formedness (\isWellFormed{\tcenv}{\stype}{\skind}),
and
(4)~implication checking (\imply{\tcenv}{p_1}{p_2}).
%
In \S~\ref{sec:lang:static} we define the judgements in detail.
%
Here, we present the design decisions that ensure the three
essential features of refinement types.


\subsection{Semantic Subtyping}
%
\label{subsec:overview:subtyping}
Refinement types rely on implicit semantic subtyping,
that is, type conversion (from subtypes) happens without
any explicit casts and is checked semantically via logical
validity.
For example, in the application @get 10 4 array10@ (of Fig.~\ref{fig:array}),
the type of @4@ was implicitly converted.
%
To see how, consider an environment $\tcenv$ that contains
the array interface.
%
Let
%
$
    \tcenv \subseteq \{\texttt{get}:n:\tint \rightarrow i:\breft{\tint}{\vv}{ \vv < n} \rightarrow \texttt{ArrayN}\ a\ n \rightarrow a\}
$
%
For brevity, we ignore the requirement that $i$ and $n$ are natural numbers
and, as in Fig.~\ref{fig:array}, we use $\texttt{ArrayN}\ a\ n$ as shorthand for
$\breft{\tint}{\vv}{ \vv < n} \rightarrow a$.
%
The application $(\texttt{get}\ 10)\ 4$ will type check as below,
using the \tSub rule to implicitly convert the
type of the argument and the \sBase rule to
check that $4$ is a valid index
by checking the validity of the formula $\forall \vv.\ \vv = 4 \Rightarrow \vv < 10$.

$${\footnotesize
\inferrule%*[Right=\tApp]
{
    \inferrule%*[Right=\tVar]
    { \dots }{
    \hastype{\tcenv}{\texttt{get}\ 10}{{\color{olive}\breft{\tint}{\vv}{ \vv < 10}} \rightarrow
    \dots
    % \texttt{ArrayN}\ a\ 10 \rightarrow a
    }
    }
%    \qquad
%    \qquad
   \
   \inferrule%*[Right=\tSub]
    {
        \inferrule%*[Right=\tVar]
        { }
        {
            \hastype{\tcenv}{4}{ {\color{teal} \breft{\tint}{\vv}{ \vv = 4 }}}
        }% {\tPrim}
        \
        \inferrule%*[Right=\sBase]
        { \forall \vv. {\color{teal} \vv = 4} \Rightarrow  {\color{olive} \vv  < 10 }}{
              \isSubType{\tcenv}{{\color{teal}\breft{\tint}{\vv}{ \vv = 4 }}}{{\color{olive}\breft{\tint}{\vv}{ \vv < 10}}}
        }{\sBase}
     }{
        \hastype{\tcenv}{4}{{\color{olive}\breft{\tint}{\vv}{ \vv < 10}}}
    }{\tSub}
 }{
    \hastype{\tcenv}{\app{\texttt{get}\ 10}{\;\;4}}{\texttt{ArrayN}\ a\ 10 \rightarrow a}
}% {\tApp}
}
$$

Importantly, most refinement type systems
use syntax-directed rules to destruct
subtyping obligations into basic (semantic)
implications.
%
For example, in~\Cref{fig:subtyping}
the rule \sFunc states that functions
are covariant on the result and contravariant
on the arguments.
%
Thus, a refinement type system can,
without any %annotations or
casts,
decide that $a_{20} : \texttt{ArrayN}\ a\ 20$
is a suitable argument for the higher order function
$\texttt{get}\ 10\ 4 : \funcftype{\texttt{ArrayN}\ a\ 10}{a}$
and type check the expression $\texttt{get}\ 10\ 4\ a_{20}$.


\subsection{Decidability}
\label{subsec:overview:exists}
%
As illustrated in the previous type derivation,
refinement type checking essentially
generates a set of verification conditions (VCs)
whose validity implies type safety.
Importantly, the refinement type checking rules
are designed to generate VCs in the logical language 
used by the user-provided specifications.
%
In general, let $\mathcal{L}$ be a logical language that contains
equality and conjunction. If all the user-specified predicates 
belong to $\mathcal{L}$,
then the VCs will be in $\mathcal{L}$ as well.
In practice (\eg in Liquid Haskell~\cite{Seidel14} and Flux~\cite{Flux}), 
$\mathcal{L}$ is the
qualifier-free logic of equality, uninterpreted functions, 
and linear arithmetic (QF-EUFLIA).

To achieve this logical-language preservation, special care is taken
in type checking function declarations and applications.


\mypara{Function Declarations}
Function declarations are checked using the refinement type rule 
for let bindings (Rule~\tLet also in~\Cref{fig:typing}).
%(We note that recursive definitions are encoded using a
%fixpoint primitive that we assume satisfies~\Cref{lem:prim-typing}
%and do not require a special rule.)
$$
\inferrule
{\hastype{\tcenv}{\sexpr_f}{\stype_f} \\
{\isWellFormed{\tcenv}{\stype}{\skind}}\\
\hastype{\bind{f}{\stype_f},\tcenv}{{\sexpr}}
     {\stype}
 }
{\hastype{\tcenv}{\eletin{f}{\sexpr_f}{\sexpr}}{\stype}}
{\tLet}
$$
The type checking must infer the type $\stype_f$ of the function,
but that could be user-annotated
(\eg $\sexpr_f$ could be $\tyann{\sexpr_f'}{\stype_f}$).

Importantly, the body $e$ is checked without knowledge of the definition of $f$.
%
The exact encoding of the body of the function definitions
(for example, as done in Dafny~\cite{Dafny} or Prusti~\cite{Prusti})
requires the use of $\forall$-quantifiers in the SMT solver,
thus potentially leading to undecidability.
%
Instead, refinement types only use the refinement
type specifications of functions, providing a fast
but incomplete verification technique.
%
For example, given only the specifications of @get@ and @set@,
and not their exact definitions, it is \emph{not} possible to show
that @get@ after @set@ returns the value that was set.

\begin{code}
  getSet :: n:Int -> i:{Nat|i<n} -> x:a -> ArrayN a n -> {v:a|x == v}
  getSet n i x a = get n i (set n i x a) -- Refinement Type Error
\end{code}

\mypara{Function Application}
% \NV{this is mostly as before, can we instead use an array related example?}
For decidable type checking, refinement types use an existential type~\cite{Knowles09} to check
dependent function application, \ie the \rulename{TApp-Exists} rule below, instead
of the standard type-theoretic \rulename{TApp-Exact} rule.
%
$$
{
\inferrule%*[Right=\rulename{TApp-Exact}]
{
    \hastype{\tcenv}{f}{\functype{x}{\stype_x}{t}} \qquad \hastype{\tcenv}{e}{\stype_x}
 }{
    \hastype{\tcenv}{\app{f}{e}}{\stype\subst{}{x}{e}}
}{\rulename{TApp-Exact}} \qquad\quad
\inferrule%*[Right=\rulename{TApp-Exists}]
{
    \hastype{\tcenv}{f}{\functype{x}{\stype_x}{t}} \qquad \hastype{\tcenv}{e}{\stype_x}
 }{
    \hastype{\tcenv}{\app{f}{e}}{\existype{x}{\stype_x}{\stype}}
}{\rulename{TApp-Exists}}
}
$$

To understand the difference, consider
some expression $e$ of type $\tpos$ and the identity function $f$
%
\begin{align*}
    e&:\tpos & f&:\functype{x}{\tint}{\breft{\tint}{v}{v = x}}
\end{align*}
%
The application $f\ e$ is typed as $\breft{\tint}{v}{v = e}$ with the \rulename{TApp-Exact} rule,
which has two problems.
%
First, the information that $e$ is positive is lost.
%
To regain this information the system needs to re-analyze the expression $e$
breaking compositional reasoning.
%
Second, the arbitrary expression $e$ enters the refinement logic
potentially breaking decidability.
%
Using the \rulename{TApp-Exists} rule, both of these problems are addressed.
Typing first uses subtyping on $f$ to track the actual type of the argument,
thus weakening the type of $f$ to $f:\functype{x}{\tpos}{\breft{\tint}{v}{v = x}}$.
%
With this, the type of $f\ e$ becomes $\existype{x}{\tpos}{\breft{\tint}{v}{v = x}}$
preserving the information that the application argument is positive,
while the variable $x$ cannot break any carefully crafted decidability guarantees.

\citet{Knowles09} introduce the existential application rule
and show that it preserves the decidability and completeness of the refinement type system.
%
An alternative approach for decidable and compositional type checking
is to ensure that all the application arguments
are variables by ANF transforming the original
program~\cite{Flanagan93}.
%
ANF is more amicable to \emph{implementation}
as it does not require the definition of one
more type form.
%
However, ANF is more problematic for the
\emph{metatheory}, as ANF is not preserved
by evaluation.
%
Additionally, existentials let us \emph{synthesize}
types for let-binders in a bidirectional style: when
typing $\eletin{x}{e_1}{e_2}$, the existential lets
us eliminate $x$ from the type synthesized for
$e_2$, yielding a precise, algorithmic
system \cite{CosmanICFP17}.
%
Thus, we choose to use existential types in \sysrf.


\subsection{Polymorphism}
\label{subsec:overview:polymorphism}

Polymorphism is a precious type
abstraction~\cite{10.1145/99370.99404},
but combined with refinements, it can lead to
imprecise or, worse, unsound systems.
%
As an example, below we present the
function \ha{max} with four potential
type signatures.
%
$$\begin{array}{rrrcl}
& \text{Definition} &  \texttt{max} &  = & \lambda x\ y . \texttt{if } x < y \texttt{ then } y \texttt{ else } x \\[0.05in]
\text{Attempt 1:}& \textit{Monomorphism} & \texttt{max} & :: &
\functype{x}{\tint}{\functype{y}{\tint}{\breft{\tint}{\vv}{x \leq \vv \wedge y \leq \vv}}}\\
\text{Attempt 2:}& \textit{Unrefined Polymorphism} & \texttt{max} &  :: &
\functype{x}{\al}{\functype{y}{\al}{\al}}\\
\text{Attempt 3:}& \textit{Refined Polymorphism} & \texttt{max} &  :: &
\functype{x}{\al}{\functype{y}{\al}{\breft{\al}{\vv}{x \leq \vv \wedge y \leq \vv}}}\\
\sysrf\text{:} & \textit{Kinded Polymorphism} & \texttt{max} &  :: &
\polytype{\al}{\skbase}\functype{x}{\al}{\functype{y}{\al}{\breft{\al}{\vv}{x \leq \vv \wedge y \leq \vv}}}
\end{array}$$
As a first attempt, we give \ha{max} a monomorphic type,
stating that the result of  \ha{max} is an integer greater than
or equal to each of its arguments.
This type is insufficient because it forgets any information known for
\ha{max}'s arguments. For example, if both arguments are positive,
the system cannot decide that \ha{max x y} is also positive.
%
To preserve the argument information we give \ha{max} a polymorphic
type, as a second attempt. Now the system can deduce that
\ha{max x y} is positive, but forgets that it is also
greater than or equal to both \ha{x} and \ha{y}.
In a third attempt, we naively combine the benefits of
polymorphism with refinements to give \ha{max} a very precise type
that is sufficient to propagate the arguments' properties (positivity)
and \ha{max} behavior (inequality).

Unfortunately, refinements on arbitrary type
variables are dangerous for two reasons.
%
First, the type of \ha{max} implies
that the system allows comparison
of any values (including functions).
%
Second, if refinements on type variables
are allowed, then, for soundness~\cite{Belo11},
all the types that substitute variables should be refined.
%
For example, as detailed in~\S 6~of~\cite{sprite}, if a type variable
is refined with \tfalse (\ie $\breft{\al}{\vv}{\tfalse}$)
and gets instantiated with an unrefined function
type ($\functype{x}{t_x}{t}$),
then the \tfalse refinement
is lost and the system becomes unsound.

\mypara{Base Kind when Refined}
%
To preserve the benefits of refinements
on type variables, without the complications
of refining function types, we introduce
a kind system that separates the type
variables that can be refined from the
ones that cannot.
%
To do so, we extend the standard well-formedness rule of refinement types
to also perform kind checking (\isWellFormed{\tcenv}{\stype}{\skind}).
%
Variables with the base kind $\skbase$
can be refined, compared, and only
substituted by base, refined types.
%
The other type variables have kind $\skstar$
and can only be trivially refined with \ttrue.
With this kind system, we have a simple
and convenient way to encode comparable values
and we can give \ha{max}
a polymorphic and precise type that
naturally rejects non-comparable
(\eg function) arguments.


This simple kind system could be further stratified,
\ie if some base types did not support comparison,
and it could be implemented via typeclass constraints,
if our system contained data types.
%% NEW sentence:
A first step towards data types is presented in Chapter
\Cref{ch:lists}, where we add polymorphic lists to \sysrf,
which can be refined (for instance, to restrict the length
of the list), but cannot be compared like base types. This 
latter restriction is needed because lists themselves can
contain incomparable terms like functions.
%%


\section{The Soundness of Refinement Types}
\label{sec:overview:soundness}

In this work we establish two soundness theorems for refinement types
that precisely relate typing judgments $\hastype{\tcenv}{e}{t}$
with the high-level goals of error prevention (type safety) and
functional correctness (denotational soundness).

\mypara{1. Type Safety} ensures that well-typed programs 
do not get stuck at runtime.
It says that  if an expression has a type ($\hastype{\emptyset}{e}{t}$)
and evaluates to another expression ($\evalsTo{\sexpr}{\sexpr'}$),
then either evaluation reached a value 
or it can take another step (${\sexpr'}\step{\sexpr''}$).
In \sysrf, we use the primitive \eerr to denote program errors 
(such as out-of-bounds indexing of~\Cref{fig:array}).
The \eerr primitive neither is a value nor takes a step.
Thus, if an expression type checks, via type safety, 
we know that \eerr will not be reached at runtime.
\Cref{lem:soundness} formally defines type safety 
and it is proved via the preservation and progress lemmas.
%
Type safety ensures that programs will not get stuck, but does not ensure that they
satisfy their functional specifications. This is ensured by the second soundness theorem.
\NV{But how to we know that evaluation preserves semantics? Revisit after we add the denotations}
\NV{CHECK RJ's comment}

\mypara{2. Denotational Soundness} states that if an expression has a type ($\hastype{\emptyset}{e}{t}$),
then it belongs in the denotations of this type ($e \in \denote{t}$).
For example, the denotation of the type @{i:Nat | i <= 42}@ is the set of integers
between 0 and 42.
%
In~\S~\ref{sec:lang:static} we inductively define the denotations of each type and
\Cref{lem:denote-sound-first} formally encodes denotational soundness.
%% NEW 
Combining Lemmas \Cref{lem:soundness} and \Cref{lem:denote-sound-first},
we see that the a refined \sysrf type that can be checked for a specific program
is preserved under evaluation, and thus the semantics of the program is 
preserved under evaluation.
%%

This dissertation, for the first time, mechanizes the soundness of refinement types with
semantic subtyping, existential types, and polymorphism.
This mechanization turned out to be challenging for three main reasons:

\tikzstyle{block} = [rectangle, draw,
    text width=6em, text centered, rounded corners, minimum height=3em]
\begin{figure}
    \begin{tikzpicture}[
        colored/.style={fill=#1!20,draw=#1!50!black!50}
      ]
        \node [block] (init) (T){\hastype{\tcenv}{e}{t}\\ {\tiny TYPING}};
        \node [block] (S) [right=1.5cm  of T]{\isSubType{\tcenv}{t}{t} {\tiny SUBTYPING}};
        \node [block] (W) [left=1.5cm  of T]{\isWellFormed{\tcenv}{\stype}{\skind} {\tiny WELL-FORMEDNESS}};
        \node [block] (I) [below=1cm  of T]{\imply{\tcenv}{p}{p} {\tiny IMPLICATION}};
        \draw[-latex]
           ([yshift=-3mm]T.north east) -- node[right,above]{1}([yshift=-3mm]S.north west);
        \draw[-latex]
           ([yshift=+3mm]S.south west) -- node[right,below]{2}([yshift=+3mm]T.south east);
        \draw[-latex]
           ([yshift=-3mm]T.north west) -- node[above]{3}([yshift=-3mm]W.north east);
        \draw[-latex, dashed, blue]
           ([yshift=+3mm]W.south east) -- node[below]{4}([yshift=+3mm]T.south west);
        \draw[-latex, dashed, red]
           (I) -- node[right]{6} (T);
        \draw[-latex]
           (S) |- node[midway,left, yshift=1.7ex, xshift=-7ex]{5}  (I);
    \end{tikzpicture}
\caption{Dependencies of Typing Judgements in Refinement Types. (Dashed lines do not exist in our formalism.) }
\label{fig:dependencies}
\end{figure}

\mypara{Challenge 1: Circularities}
%
\Cref{fig:dependencies} presents the dependencies of the four typing judgements in refinement types.
As we saw in the example of~\S~\ref{subsec:overview:subtyping} 
(and can be confirmed in the rules defined in~\S~\ref{sec:lang:static}),
typing depends on subtyping (arrow 1) which in turn depends on implication checking (arrow 5).
Subtyping depends on typing (arrow 2; because of rule \sWitn of \Cref{fig:subtyping}),
so typing and subtyping have a circular dependency we cannot break.
%
Typing also depends on well-formedness (arrow 3) that checks that
types, especially the ones inferred by the system, are well-formed:
all the variables appearing in the refinements are bound in the type environment
and refinements are of boolean type.
To check the type of the refinements the system could use typing
thus introducing one more dependency (arrow 4) and yet another circle.
We break this dependency by using an unrefined calculus (system \sysf)
that erases refinements, to check that refinements are well-typed booleans.
%
The final potential circle is introduced when implication depends on typing (arrow 6).
In~\S~\ref{sec:typing:implication:denotational} we define implication via type denotations,
but as observed by~\citet{Greenberg13}, in this case, special care should be taken
so that the system is monotonic and thus well-defined.
To avoid this dangerous circularity we again use typing of \sysf (and not \sysrf) to define
denotations and thus implication.

% \newtext{
In summary, circularities in typing judgements are problematic for two reasons:
\begin{enumerate}[leftmargin=*]
\item Circularities increase the complexity of proof mechanization. 
Concretely, because typing and subtyping have a circular dependency, 
the metatheoretical lemmas (substitution, weakening, narrowing, \etc) 
require versions for both typing and subtyping, which are proved by mutual induction. 
If well-formedness was also included in this circularity (arrow 4),
 then the complexity of the proofs would greatly increase, 
 but would not necessarily be impossible.
\item Second, circularities are problematic because they can lead to non well-defined systems. 
Concretely, \citet{Greenberg13} describes an older refinement type system 
in which typing appeared in the left hand side of subtyping and, as such, 
it was non-monotonic and thus not well-defined. 
This situation corresponds to the red arrow 6 in \cref{fig:dependencies}, 
which would make the proof impossible due to the typing judgment occurring in 
a negative position in the implication judgment.  
\end{enumerate}
% }


\mypara{Challenge 2: Implications}
%
The second mechanization challenge was the encoding of implication.
In the bibliography of refinement types, implication has been defined in three ways:
\begin{enumerate}[leftmargin=*]
\item Using \emph{denotations} (of types as sets of terms) defined via operational semantics~\cite{Vazou18, flanagan06}.
This encoding is more convenient when proving the soundness of the system, since implication
and thus subtyping and typing, directly connect with operational semantics, making the proof of soundness more direct.
However, the implementation of this encoding of implication is not realistic, since it is not decidable.
\item Using \emph{logical implication}~\cite{LiquidPLDI08, Gordon2010PrinciplesAA}.
The encoding of the implication as a logical implication is the closest to the implementation of a refinement system,
where an SMT is used to check logical implications. Yet, to prove soundness, a claim should be made that
logical implication checked by the SMT correctly approximates the runtime semantics of the system
(\ie presented in rule~\iLog of~\S~\ref{sec:typing:implication:logical}) which has never been mechanized.
\item By \emph{axiomatization}~\cite{LehmannTanter}.
A final approach is to leave the implication uninterpreted and axiomatize it with all the properties required to prove soundness.
This approach is the easiest to mechanize, but it is dangerous, since in the past the axioms assumed for implication
were inconsistent, thus soundness was ``proved with flawed premises'' (as quoted from Table 1 of~\cite{SekiyamaIG17}).
\end{enumerate}
Our mechanization follows a combination of the first and the third approach.
We specify the interface of implication (via \Cref{lem:implication} of~\S~\ref{sec:typing:implication:interface} which is encoded as
an inductive data type in the proof mechanization)
to articulate the exact properties required by the soundness proof.
Then, in~\S~\ref{sec:typing:implication:denotational}, we implement the implication interface using
the denotational semantics of the system.
This encoding has two major benefits.
First, the denotational implementation ensures that our interface is consistent.
Second, the development of the interface leaves room for the implementation
of alternative implication ``oracles'', \eg closer to SMT solvers.
Even though we did not mechanize this alternative implementation, in~\S~\ref{sec:typing:implication:logical} we present how
logical implications are derived from the implication judgement.


\mypara{Challenge 3: Proof Complexity}
%
All the three essential features of refinement types add complexity 
to the mechanization of the soundness proof.
Polymorphism requires the extension of well-formedness to kind checking.
%
Semantic subtyping makes type checking not syntax-directed
(thus inversion is not trivial, cf.~\S~\ref{sec:soundness:inversion}) and dependent upon subtyping.
%
In turn, the existential types required for decidability make subtyping dependent upon type checking.
%
Due to this mutual dependency, the standard metatheoretical lemmas (substitution, weakening, narrowing, \etc)
require versions for both typing and subtyping, which are proved by mutual induction.
Thus, the combination of the three essential for refinement types features
makes the metatheoretical development more complex and prone to unsoundness.
%
Once, we have carefully broken the various circularities
and eliminated potential sources of unsoundness, we get
unsurprising, albeit strenuous, proofs of the soundness
of refinement typing.
