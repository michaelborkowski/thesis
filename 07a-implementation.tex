\chapter{Implementation and Mechanization}
\label{ch:implementation}

\section{\lh Mechanization}
\label{sec:implementation:lh}

We mechanized type safety (\Cref{lem:soundness}) 
of \sysrf in both \coq 8.15.1 and \lh 8.10.7.1
(available online as supplementary
material).
%
In \lh we use refined data propositions 
(\S~\ref{sec:data-props}) to specify the
static (\eg typing, subtyping, well-formedness)
and dynamic (\ie small-step transitions and their closure)
semantics of \sysrf.
%
\begin{fullversion}
  The \lh mechanization is simplified by 
  SMT-automation (\S~\ref{impl:settheory}),
  uses a co-finite encoding for reasoning about
  variables (\S~\ref{subsec:implementation:co-finite}),
\end{fullversion}
\begin{conference}
  The \lh mechanization is simplified by 
  SMT-automation (\S~\ref{impl:settheory})
\end{conference}
and consists of proofs implemented as recursive functions that
construct  evidence to establish propositions by
induction (\S~\ref{impl:proofs}).
%


  Other that the development of data propositions, we extended \lh 
  with two more features during the development of this proof. 
  First, we implemented an interpreter that 
  critically dropped the verification time from 10 hours 
  to only 29 minutes (\S\ref{subsec:quantitiative}). 
  Second, we implemented a (\coq-style) strictly-positive-occurrence checker
  to ensure data propositions are well-defined, 
  since early versions of our proof used negative occurrences. 

%
Note that while Haskell types are inhabited
by diverging $\bot$ values, \lh's totality,
termination, and type checks ensure that all
cases are handled, the induction (recursion)
is well-founded, and that the proofs (programs)
indeed inhabit the propositions (types).



\subsection{Quantitative Results}
\label{subsec:quantitiative}

We provide a mechanically checked
proof of the type safety in~\S~\ref{sec:soundness:safety}, 
that only assumes the requirements
\ref{lem:prim-typing} and \ref{lem:implication}. 
%
%The only facts that are assumed are 
%Req. \ref{lem:implication}, \ie the implication interface
%which we encoded as a data proposition, 
%and 
%Req. \ref{lem:prim-typing}, \ie assumptions about built-in primitives. 
Concretely, we assumed the primitives \Cref{lem:prim-typing} for some 
constants of \sysrf because 
it was too strenuous to mechanically prove 
without interactive aid. 
%
In \lh type denotations (of \Cref{fig:den}) 
cannot be currently encoded: 
since they include $\forall$-quantification
they could only be encoded as data propositions, 
but the strictly-positive-occurrence checker 
rejects the definition of the function denotation. 
Due to this limitation, 
we can neither define the denotational implementation
of the implication (\S~\ref{sec:typing:implication:denotational})
nor prove 
the denotational soundness (\Cref{lem:denote-sound-first}). \NV{CHECK}


% \newtext{
  \mypara{Representing Binders}
One main challenge in the mechanized
metatheory is the syntactic representation
of variables and binders~\cite{Aydemir05}.
%
The \emph{named} representation
has severe difficulties because
of variable capturing substitutions
and the \emph{nameless} (\aka de Bruijn)
requires heavy index shifting.
%
The variable representation
of $\sysrf$ is
\emph{locally nameless representation}~\cite{Pollack93,AydemirCPPW08},
where free variables are named, but
bound variables are represented by
%syntactically distinct 
de Bruijn indices.
%
Our mechanization still resembles
the paper and pencil proofs (performed
before  mechanization),
yet it clearly addresses
the following two problems with named bound variables.
%
First, when different refinements are strengthened
(as in \Cref{fig:type-subst}) the variable
capturing problem reappears
because we are substituting underneath a binder.
%
Second, subtyping usually permits
alpha-renaming of binders,
which breaks a required invariant
that each \sysrf derivation tree
is a valid \sysf tree after erasure. \\
% }

\begin{fullversion}
\mypara{Representing Binders}
%
In our mechanization, we use the
\emph{locally-nameless representation} \cite{AydemirCPPW08,Chargueraud12}.
%
Free variables and bound variables
are taken to be separate syntactic
objects, so we do not need to worry
about alpha renaming of free variables
to avoid capture in substitutions.
%
We also use de Bruijn indices only
for bound variables. This enables us to avoid
taking binder names into account in the
$\mathsf{strengthen}$ function used to define
substitution (\Cref{fig:type-subst}).
%
% While the uniqueness of names in the environment
% is convenient for the mechanized metatheory,
% it comes at the cost of renaming lemmas that
% let us change the name of bound variables.
% %
% In \S~\ref{sec:related} we discuss another
% approach to the mechanization \cite{AydemirCPPW08}
% that avoids the frequent use of renaming.
%
\end{fullversion}

Table \ref{fig:empirical} summarizes the development 
of our metatheory,
which was checked using \lh 8.10.7.1
and a Lenovo ThinkPad T15p laptop with
an Intel Core i7-11800H processor.
%with 8 physical cores and 32 GB of RAM.
%
Our mechanized proofs are substantial. The entire
\lh development comprises over 9,300 lines 
across about 35 files. 
%
Currently, the whole \lh proof can be checked
in 29 minutes, which makes interactive
development difficult, especially compared to 
the \coq proof (\S~\ref{sec:coq}) that 
is checked in about 60 seconds.
%
While incremental modular checking provides
a modicum of interactivity, improving the
ergonomics of \lh, 
\ie verification time and 
actionable error messages, remains an important
direction for future work.

\begin{table}
\caption{Quantitative mechanization details. We split each development into sets of
modules pertaining to regions of \Cref{fig:graph} and for each we
count lines of specification (definitions, lemma statements)
and of proof.}
\label{fig:empirical}
\vspace{0.00cm}
\begin{tabular}{l}
\end{tabular}
\end{table}
\begin{table}
% \begin{center}
\setlength{\tabcolsep}{10pt}
\scalebox{0.90}{
\begin{tabular}{ lrrrr|rrr  }
  \multicolumn{1}{l}{ } & 
  \multicolumn{4}{c|}{\lh Mechanization} & 
  \multicolumn{3}{l}{\coq Mechanization} \\
  \toprule
  \textbf{Subject} & \textbf{Files} & \textbf{Time (m)} &
  \textbf{Spec} & \textbf{Proof} &  
  \textbf{Files} & \textbf{Spec} & \textbf{Proof} \\
  \hline
  Definitions             & 6      &  1     & 1805  &  374  & 7      &  941  &  190 \\
  Basic Properties        & 8      &  4     &  646  & 2117  & 8      & 1201  & 2360 \\
  $\sysf$ Soundness       & 4      &  3     &  138  &  685  & 4      &  173  &  773 \\
  Weakening               & 4      &  1     &  379  &  467  & 4      &  110  &  568 \\
  Substitution            & 4      &  7     &  458  &  846  & 4      &  158  &  859 \\
  Exact Typing            & 2      &  4     &   70  &  230  & 2      &   33  &  182 \\
  Narrowing               & 1      &  1     &   88  &  166  & 1      &   54  &  262 \\
  Inversion               & 1      &  1     &  124  &  206  & 1      &   57  &  258 \\
  Primitives              & 3      &  4     &  120  &  277  & 3      &   89  &  508 \\
  $\sysrf$ Soundness      & 1      &  1     &   14  &  181  & 1      &   12  &  233 \\
  Denotational Soundness  & -      &  -     &    -  &    -  & 13     &  815  & 3010 \\
  \midrule
  \textbf{Total}          & 35     & 29     & 3842  & 5549  & 49     & 3643  & 9203 \\
  \bottomrule
\end{tabular}
}
% \end{center}
\end{table}
