\section{\coq Mechanization}
\label{sec:coq}
\label{sec:implementation:coq}

Our \coq mechanization 
proves both type safety and
denotation soundness, \ie all the statements of \S~\ref{sec:soundness:denotational}
and \S~\ref{sec:soundness:safety}
and serves as a comparison for the metatheoretical 
development abilities of the two theorem provers. 
%is a translation from \lh
%and was built to compare the two developments. 
%
%All theorems from~\S~\ref{sec:soundness} are proven in \coq. %(\ie the proof has no \ha{Admitted}.)% and zero trusted code base. 
%
In \coq, 
Req. \ref{lem:prim-typing} 
is proved (using \coq's interactive development)
and type denotations (of \Cref{fig:den})
are defined as recursive functions using 
Equations~\cite{10.1145/3341690}, 
which make both the 
definition the denotational implementation
of the implication (\S~\ref{sec:typing:implication:denotational})
and the proof  
the denotational soundness (\Cref{lem:denote-sound-first})
possible. 
\begin{fullversion}
    The implication judgment
    is  axiomatized per Requirement \ref{lem:implication}.
\end{fullversion}
%
To fairly compare the two developments
%In order to better understand the relative strengths 
%and tradeoffs of \lh vs. \coq 
in terms of effort and ergonomics,
we did not use external \coq libraries 
%and implemented our own infrastructure 
because no such libraries exist yet for \lh.
%
\citet{Vazou17} previously compared \lh and \coq 
as theorem provers, but their mechanizations were an order of magnitude
smaller than ours and did not use data propositions (\S~\ref{sec:data-props}),
which permit constructive \lh proofs. 

The source code for our mechanizations in \coq and \lh, 
together with instructions on how to replicate the results, 
are available on Zenodo \cite{michael_h_borkowski_2023_8425960}. 
%
Additionally, a virtual appliance for Oracle VM VirtualBox 
is available on Zenodo \cite{michael_h_borkowski_2023_8425176}
to assist with replication.

\section*{Acknowledgements for Chapter 7}
%
This chapter is adapted from 
``Mechanizing Refinement Types'' in the proceedings of the 
$51^{st}$ ACM SIGPLAN Symposium on Principles of Programming
Languages (POPL 2024), by Michael Borkowski, Niki Vazou, and
Ranjit Jhala.
%
The dissertation author was the primary investigator 
and author of this material.